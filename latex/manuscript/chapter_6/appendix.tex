\chapter{Appendix of Chapter~\ref{chap: wpb-practical}}
\label{ap: chapter-6}

\begin{noaddcontents}
    The supplementary material is organized as follows:
    \begin{enumerate}
        \item We provide more discussion about \Cref{theorem:supervised-ht,theorem:supervised-nnl} in \Cref{sec:discussion-supervised};
        \item The proofs of \Cref{theorem:supervised-ht,theorem:supervised-nnl,theorem:online-ht,theorem:online} are presented in \Cref{sec:proofs};
        \item We present in \Cref{sec:supplementary-expes} additional information about the experiments.
    \end{enumerate}
    
    \section{Additional insights on \Cref{sec:wasserstein-batch}}
    \label{sec:discussion-supervised}
    
    In \Cref{sec:discussion-supervised-ht}, we provide additional discussion about \Cref{theorem:supervised-ht} while \Cref{sec:discussion-supervised-nnl} discuss about the convergence rates for \Cref{theorem:supervised-nnl}.
    
    \subsection{Supplementary discussion about \Cref{theorem:supervised-ht}}
    \label{sec:discussion-supervised-ht}
    
     \cite[Corollary 10]{haddouche2023wasserstein} proposed PAC-Bayes bounds with Wasserstein distances on a Euclidean predictor space with Gaussian prior and posteriors. 
     The bounds have an explicit convergence rate of $\mathcal{O}({\sqrt{\frac{d W_1(\Q,\P)}{m}}})$ where the predictor space is Euclidean with dimension $d$.
     While our bound does not propose such an explicit convergence rate, it allows us to derive learning algorithms as described in \Cref{sec:experiments}. 
     A broader discussion about the role of $K$ is detailed in \Cref{theorem:supervised-nnl}.
     Furthermore, our bound holds for any Polish predictor space and does not require Gaussian distributions.
     Furthermore, our result exploits data-dependent priors and deals with the dimension only through the Wasserstein distance, which can attenuate the impact of the dimension. 
    
    \subsection{Convergence rates for \Cref{theorem:supervised-nnl}}
    \label{sec:discussion-supervised-nnl}
    
    In this section, we discuss more deeply the values of $K$ in \Cref{theorem:supervised-nnl}. 
    This implies a tradeoff between the number of sets $K$ and the cardinal of each $\S_i$.
    The tightness of the bound depends highly on the sets $\S_1, \dots, \S_K$.
    
    \textbf{Full batch setting K=1.} When $\S_1=\S$ with $K=1$, the bound of \Cref{theorem:supervised-nnl} becomes, with probability $1-\delta$, for any $\Q\in\Mcal(\H)$
    \begin{align*}
     \EE_{h\sim\Q}\Big[ \Risk_{\D}(h) - \Riskhat_{\S}(h) \Big] \le 2L\W(\Q, \P) + 2\sqrt{\frac{\ln\frac{1}{\delta}}{m}}\ ,
    \end{align*}
    where $\P=\P_1$ is data-free.
    This bound can be seen as the high-probability (PAC-Bayesian) version of the expected bound of \cite{wang2019information}.
    Furthermore, in this setting, we are able, through our proof technique, to recover an explicit convergence rate similar to the one of \cite[][Theorem 12]{amit2022integral}. It is stated below.
    
    \begin{corollary}
    For any distribution $\D$ on $\Z$, for any finite hypothesis space $\H$ equipped with a distance $d$, for any $L$-Lipschitz loss function $\loss: \H\times\Z\to[0,1]$, for any $\delta\in(0,1]$, we have, with probability $1-\delta$ over the sample $\S$, for any $\Q\in\Mcal(\H)$:
    \begin{align*}
    \ \EE_{h\sim\Q}\Big[ \Risk_{\D}(h) - \Riskhat_{\S}(h) \Big] \le L\sqrt{\frac{2\ln\left(\frac{4|\H|^2}{\delta}\right)}{m}} \W(\Q, \P) +  2\sqrt{\frac{\ln\left(\frac{2}{\delta}\right)}{m}} 
     \end{align*}
    where $\P$ is a data-free prior.
    \end{corollary}
    \begin{proof}
        We exploit \cite[][Equation 35]{amit2022integral} to state that with probability at least $1-\frac{\delta}{2}$, for any $(h,h')\in \H^2$: 
        \[\left|\frac{1}{m} \sum_{i=1}^m\left[\loss\left(h^{\prime}, \z_i\right)-\loss\left(h, \z_i\right)\right]-\EE_{\z \sim \D}\left[\loss\left(h^{\prime}, \z\right)-\loss(h, \z)\right]\right| \leq  L \sqrt{\frac{2 \ln \left(\frac{4|\H|^2}{\delta}\right)}{m}}d\left(h, h^{\prime}\right) .\]
    So, with high probability, we can exploit the Kantorovich-Rubinstein duality with this new Lipschitz constant: with probability at least $1-\delta/2$:
    \begin{align*}
        \EE_{h\sim\Q}\Big[ \Risk_{\D}(h) - \Riskhat_{\S}(h) \Big] & \le L \sqrt{\frac{2 \ln \left(\frac{4|\H|^2}{\delta}\right)}{m}}\W(\Q, \P) +\EE_{h\sim\P} \frac{1}{m}\LB  \sum_{i=1}^m \Risk_{\D}(h) -\loss(h, \z_i) \RB,
    \end{align*}
    To conclude, we control the quantity on the right-hand side the same way as in \Cref{theorem:supervised-ht} and \Cref{theorem:supervised-nnl}. We then have, with probability at least $1-\delta/2$, for a loss function in $[0,1]$:
    \[\frac{1}{m} \sum_{i=1}^m \Risk_{\D}(h) -\loss(h, \z_i) \leq 2\sqrt{\frac{\ln\frac{K}{\delta}}{m}}.\]
    Taking the union bound concludes the proof.
    \end{proof}
    
    \textbf{Mini-batch setting $K=\sqrt{m}$.} When a tradeoff is desired between the quantity of data we want to infuse in our priors and an explicit convergence rate, a meaningful candidate is when $K=\sqrt{m}$.  \Cref{theorem:supervised-nnl}'s bound becomes, in this particular case:
    \begin{align}
    \label{eq:supervised-with-tuned-K}
     \EE_{h\sim\Q}\Big[ \Risk_{\D}(h) - \Riskhat_{\S}(h) \Big] \le \frac{2L}{\sqrt{m}}\sum_{i=1}^{\sqrt{m}}\W(\Q, \P_i) + 2\sqrt{\frac{\ln\frac{\sqrt{m}}{\delta}}{\sqrt{m}}}.
    \end{align}
    
    \textbf{Towards online learning: $K=m$.} When $K=m$, the sets $\S_i$ contain only one example. 
    More precisely, we have for all $i\in\{1,\dots,m\}$ the set $\S_i=\{\z_i\}$.
    In this case, the bound becomes:
    \begin{align*}
    \ \EE_{h\sim\Q}\Big[ \Risk_{\D}(h) - \Riskhat_{\S}(h) \Big] \le \frac{2L}{m}\sum_{i=1}^{m}\W(\Q, \P_i) + 2\sqrt{\ln\frac{m}{\delta}}.
    \end{align*}
    This bound is vacuous since the last term is incompressible, hence the need for a new technique detailed in \Cref{sec:wasserstein-online} to deal with it.
    
    \section{Proofs}
    \label{sec:proofs}
    
    The proof of \Cref{theorem:supervised-ht} is presented in \Cref{sec:proof-supervised-ht}. 
    \Cref{sec:proof-supervised,sec:alt-proof-supervised} introduce two proofs of \Cref{theorem:supervised-nnl}.
    \Cref{theorem:online-ht}'s proof is presented in \Cref{sec:proof-online-ht}.
    \Cref{sec:proof-online} provides the proof of \Cref{theorem:online-ht}.
    
    \subsection{Proof of \Cref{theorem:supervised-ht}}
    \label{sec:proof-supervised-ht}
    
    \theoremsupervisedht*
    \begin{proof}
    For the sake of readability, we identify, for any $i$, $\P_i $ and $\P_{i,\S}$.
    \paragraph{Step 1: Exploit the Kantorovich duality \cite[Remark 6.5]{villani2009optimal}.}
    First of all, note that for a $L$-Lipschitz loss function $\loss: \H\times\Z\to[0,1]$, we have
    \begin{align}
    \LN\LP |\S_i|\Risk_{\D}(h_1){-}\sum_{\z\in\S_i}\loss(h_1, \z)\RP - \LP |\S_i|\Risk_{\D}(h_2){-}\sum_{\z\in\S_i}\loss(h_2, \z)\RP\right\vert \le 2|\S_i|Ld(h_1, h_2).\label{eq:proof-supervised-ht-1}
    \end{align}
    Indeed, we can deduce \Cref{eq:proof-supervised-ht-1} from Jensen inequality, the triangle inequality, and by definition that we have
    \begin{align*}
    &\LN\LP |\S_i|\Risk_{\D}(h_1){-}\sum_{\z\in\S_i}\loss(h_1, \z)\RP - \LP |\S_i|\Risk_{\D}(h_2){-}\sum_{\z\in\S_i}\loss(h_2, \z)\RP\right\vert\\
    &= \LN\LP\sum_{\z\in\S_i}\Risk_{\D}(h_1){-}\sum_{\z\in\S_i}\loss(h_1, \z)\RP - \LP\sum_{\z\in\S_i}\Risk_{\D}(h_2){-}\sum_{\z\in\S_i}\loss(h_2, \z)\RP\right\vert\\
    &\le \sum_{\z\in\S_i}\EE_{\z'\sim\D}\Big[ |\loss(h_1, \z')-\loss(h_2, \z')|+|\loss(h_2, \z)-\loss(h_1, \z)|\Big]\\
    &\le \EE_{\z'\sim\D}\sum_{\z\in\S_i} 2Ld(h_1, h_2)\\
    &= 2|\S_i|Ld(h_1, h_2).
    \end{align*}
    We are now able to upper-bound $\EE_{h\sim\Q}[ \Risk_{\D}(h) - \Riskhat_{\S}(h) ]$. 
    Indeed, we have
    \begin{align}
    \EE_{h\sim\Q}\Big[ \Risk_{\D}(h) - \Riskhat_{\S}(h) \Big] &= \frac{1}{m}\sum_{i=1}^{K} \EE_{h\sim\Q}\LB |\S_i|\Risk_{\D}(h) - \sum_{\z\in\S_i}\loss(h, \z) \RB\nonumber\\
    &\le \sum_{i=1}^{K} \frac{2|\S_i|L}{m}\W(\Q, \P_i) + \sum_{i=1}^{K} \EE_{h\sim\P_i} \frac{1}{m}\LB |\S_i|\Risk_{\D}(h) - \sum_{\z\in\S_i} \loss(h, \z) \RB,\label{eq:proof-supervised-ht-2}
    \end{align}
    where the inequality comes from the Kantorovich-Rubinstein duality theorem.
    
    \paragraph{Step 2: Define an adapted supermartingale.}
        For any $1\leq i \leq K$, we fix $\lambda_i>0$ and we provide an arbitrary order to the elements of $\S_i\defeq \{\z_{i,1},\cdots,\z_{i,|S_i|}\}$. Then we define for any $h$: 
        \begin{align*}
        M_{|\S_i|}(h)\defeq   |\S_i|\Risk_{\D}(h) - \sum_{\z\in\S_i} \loss(h, \z) = \sum_{j=1}^{|\S_i|} R_\D(h) - \loss(h,\z_{i,j}).
        \end{align*}
        Remark that, because our data are \iid, $(M_{|\S_i|})_{|\S_i|\geq 1}$ is a martingale.
        We then exploit the technique \cite{haddouche2023pac} to define a supermartingale.
        More precisely, we exploit a result from \cite{bercu2008exponential} cited in Lemma 1.3 of \cite{haddouche2023pac} coupled with Lemma 2.2 of \cite{haddouche2023pac} to ensure that the process
        \begin{align*}
        SM_{|\S_i|} \defeq \EE_{h\sim \P_i} \LB\exp\left(\lambda_i M_{|\S_i|}(h) - \frac{\lambda_i^2}{2}\left( \Vhat_{|\S_i|}(h) + V_{|\S_i|}(h) \right)\right)\RB,
        \end{align*}
        is a supermartingale, where $\Vhat_{|\S_i|}(h)= \sum_{j=1}^{|\S_i|} \left(\loss(h,\z_{i,j}) - R_\D(h)\right)^2$ and $V_{|\S_i|}(h) = \EE_{\S_i} \LB \Vhat_{|\S_i|}(h) \RB $.
    
    
    
    \paragraph{Step 3. Combine steps 1 and 2.}
        
        We restart from \Cref{eq:proof-supervised-ht-2} to exploit again the Kantorovich-Rubinstein duality.
    
        \begin{align*}
            \EE_{h\sim\Q}\Big[ \Risk_{\D}(h) - \Riskhat_{\S}(h) \Big] &=  \frac{1}{m}\sum_{i=1}^{K} \EE_{h\sim\Q}\LB |\S_i|\Risk_{\D}(h) - \sum_{\z\in\S_i}\loss(h, \z) \RB\nonumber\\
            &\le \sum_{i=1}^{K} \frac{2|\S_i|L}{m}\W(\Q, \P_i) + \sum_{i=1}^{K} \frac{1}{m\lambda_i}\lambda_i\EE_{h\sim\P_i} \LB |\S_i|\Risk_{\D}(h) - \sum_{\z\in\S_i} \loss(h, \z) \RB, \\
            & = \sum_{i=1}^{K} \frac{2|\S_i|L}{m}\W(\Q, \P_i) + \sum_{i=1}^{K} \frac{1}{m\lambda_i}\EE_{h\sim\P_i} \LB \lambda_i M_{|\S_i|}| \RB, \\
            &\le   \sum_{i=1}^{K} \frac{2|\S_i|L}{m}\W(\Q, \P_i) + \sum_{i=1}^{K} \frac{1}{m\lambda_i} \ln\left( SM_{|\S_i|}  \right) \\
            & + \frac{1}{m}\sum_{i=1}^K \EE_{h\sim\P_i}\LB\frac{\lambda_i}{2}\left( \Vhat_{|\S_i|}(h) + V_{|\S_i|}(h) \right) \RB.
        \end{align*}
        The last line holds thanks to Jensen's inequality.
        We now apply Ville's inequality (see \eg, Section 1.2 of \cite{haddouche2023pac}).
        We have for any $i$: 
        \begin{align*}
        \PP_{\S_i \sim \D^{|\S_i|}}\left( \forall |S_i|\geq 1, SM_{|S_i|} \leq \frac{1}{\delta}  \right) \geq 1-\delta.
        \end{align*}
        Applying an union bound and authorising $\lambda_i$ to be a function of $|S_i|$ (thus the inequality does not hold for all $|\S_i|$ simultaneously) finally gives with probability at least $1-\delta$, for all $\Q\in \Mcal(\H)$ :
        \begin{multline*}
             \EE_{h\sim\Q}\Big[ \Risk_{\D}(h) - \Riskhat_{\S}(h) \Big] \leq    \sum_{i=1}^{K} \frac{2|\S_i|L}{m}\W(\Q, \P_i) + \sum_{i=1}^{K}   \frac{\ln\left( \frac{K}{\delta}  \right)}{\lambda_i m} + \frac{\lambda_i}{2m}\EE_{h\sim\P_i}\LB \Vhat_{|\S_i|}(h) + V_{|\S_i|}(h) \RB.
        \end{multline*}    
    \end{proof}
    
    \subsection{Proof of \Cref{theorem:supervised-nnl}}
    \label{sec:proof-supervised}
    
    \theoremsupervisednnl*
    \begin{proof}
    For the sake of readability, we identify, for any $i$, $\P_i $ and $\P_{i,\S}$.
    \paragraph{Step 1: Exploit the Kantorovich duality \cite[Remark 6.5]{villani2009optimal}.}
    First of all, note that for a $L$-Lipschitz loss function $\loss: \H\times\Z\to[0,1]$, we have
    \begin{align}
    \LN\LP |\S_i|\Risk_{\D}(h_1){-}\sum_{\z\in\S_i}\loss(h_1, \z)\RP - \LP |\S_i|\Risk_{\D}(h_2){-}\sum_{\z\in\S_i}\loss(h_2, \z)\RP\right\vert \le 2|\S_i|Ld(h_1, h_2).\label{eq:proof-supervised-1}
    \end{align}
    Indeed, we can deduce \Cref{eq:proof-supervised-1} from Jensen inequality, the triangle inequality, and by definition that we have
    \begin{align*}
    &\LN\LP |\S_i|\Risk_{\D}(h_1){-}\sum_{\z\in\S_i}\loss(h_1, \z)\RP - \LP |\S_i|\Risk_{\D}(h_2){-}\sum_{\z\in\S_i}\loss(h_2, \z)\RP\right\vert\\
    &= \LN\LP\sum_{\z\in\S_i}\Risk_{\D}(h_1){-}\sum_{\z\in\S_i}\loss(h_1, \z)\RP - \LP\sum_{\z\in\S_i}\Risk_{\D}(h_2){-}\sum_{\z\in\S_i}\loss(h_2, \z)\RP\right\vert\\
    &\le \sum_{\z\in\S_i}\EE_{\z'\sim\D}\Big[ |\loss(h_1, \z')-\loss(h_2, \z')|+|\loss(h_2, \z)-\loss(h_1, \z)|\Big]\\
    &\le \EE_{\z'\sim\D}\sum_{\z\in\S_i} 2Ld(h_1, h_2)\\
    &= 2|\S_i|Ld(h_1, h_2).
    \end{align*}
    We are now able to upper-bound $\EE_{h\sim\Q}[ \Risk_{\D}(h) - \Riskhat_{\S}(h) ]$. 
    Indeed, we have
    \begin{align}
    \EE_{h\sim\Q}\Big[ \Risk_{\D}(h) - \Riskhat_{\S}(h) \Big] &= \frac{1}{m}\sum_{i=1}^{K} \EE_{h\sim\Q}\LB |\S_i|\Risk_{\D}(h) - \sum_{\z\in\S_i}\loss(h, \z) \RB\nonumber\\
    &\le \sum_{i=1}^{K} \frac{2|\S_i|L}{m}\W(\Q, \P_i) + \sum_{i=1}^{K} \EE_{h\sim\P_i} \frac{1}{m}\LB |\S_i|\Risk_{\D}(h) - \sum_{\z\in\S_i} \loss(h, \z) \RB,\label{eq:proof-supervised-2}
    \end{align}
    where the inequality comes from the Kantorovich-Rubinstein duality theorem.
    
    \paragraph{Step 2: Define an adapted supermartingale. }
        For any $1\leq i \leq K$, we fix $\lambda_i>0$ and we provide an arbitrary order to the elements of $\S_i\defeq \{\z_{i,1},\cdots,\z_{i,|S_i|}\}$. Then we define for any $h$: 
        \begin{align*}
        M_{|\S_i|}(h)\defeq   |\S_i|\Risk_{\D}(h) - \sum_{\z\in\S_i} \loss(h, \z) = \sum_{j=1}^{|\S_i|} R_\D(h) - \loss(h,\z_{i,j}).
        \end{align*}
        Remark that, because our data are \iid, $(M_{|\S_i|})_{|\S_i|\geq 1}$ is a martingale. 
        We then exploit the technique \cite{chugg2023unified} to define a supermartingale. 
        More precisely, we exploit \cite[][Lemma A.2 and Lemma B.1]{chugg2023unified} to ensure that the process
        \begin{align*}
        SM_{|\S_i|} \defeq \EE_{h\sim \P_i} \LB\exp\left(\lambda_i M_{|\S_i|}(h) - \frac{\lambda_i^2}{2} L_{|\S_i|}(h) \right)\RB,
        \end{align*}
        is a supermartingale, where, because our data are \iid,  $L_{|\S_i|}(h) = \EE_{\S} \LB \sum_{j=1}^{|S_i|}\loss(h,\z_{i,j})^2 \RB = |\S_i| \EE_{z\sim \D}[\loss(h,z)^2]$. 
    
    \paragraph{Step 3. Combine steps 1 and 2.}
        We restart from \Cref{eq:proof-supervised-2} to exploit the Kantorovich-Rubinstein duality again.
        \begin{align*}
            \EE_{h\sim\Q}\Big[ \Risk_{\D}(h) - \Riskhat_{\S}(h) \Big] &=  \frac{1}{m}\sum_{i=1}^{K} \EE_{h\sim\Q}\LB |\S_i|\Risk_{\D}(h) - \sum_{\z\in\S_i}\loss(h, \z) \RB\nonumber\\
            &\le \sum_{i=1}^{K} \frac{2|\S_i|L}{m}\W(\Q, \P_i) + \sum_{i=1}^{K} \frac{1}{m\lambda_i}\lambda_i\EE_{h\sim\P_i} \LB |\S_i|\Risk_{\D}(h) - \sum_{\z\in\S_i} \loss(h, \z) \RB, \\
            & = \sum_{i=1}^{K} \frac{2|\S_i|L}{m}\W(\Q, \P_i) + \sum_{i=1}^{K} \frac{1}{m\lambda_i}\EE_{h\sim\P_i} \LB \lambda_i M_{|\S_i|}| \RB, \\
            &\le   \sum_{i=1}^{K} \frac{2|\S_i|L}{m}\W(\Q, \P_i) + \sum_{i=1}^{K} \frac{1}{m\lambda_i} \ln\left( SM_{|\S_i|}  \right) \\
            & + \frac{1}{m}\sum_{i=1}^K \EE_{h\sim\P_i}\LB\frac{\lambda_i}{2} L_{|\S_i|}(h) \RB.
        \end{align*}
        The last line holds thanks to Jensen's inequality.
        We now apply Ville's inequality (see \eg, section 1.2 of \cite{haddouche2023pac}). We have for any $i$: 
        \begin{align*}
        \PP_{\S_i \sim \D^{|\S_i|}}\left( \forall |S_i|\geq 1, SM_{|S_i|} \leq \frac{1}{\delta}  \right) \geq 1-\delta.
        \end{align*}
        Applying an union bound and authorising $\lambda_i$ to be a function of $|S_i|$ (thus the inequality does not hold for all $|\S_i|$ simultaneously) finally gives with probability at least $1-\delta$, for all $\Q\in \Mcal(\H)$ :
        \begin{align*}
             \EE_{h\sim\Q}\Big[ \Risk_{\D}(h) - \Riskhat_{\S}(h) \Big] \leq    \sum_{i=1}^{K} \frac{2|\S_i|L}{m}\W(\Q, \P_i) + \sum_{i=1}^{K}   \frac{\ln\left( \frac{K}{\delta}  \right)}{\lambda_i m} + \frac{\lambda_i}{2m}\EE_{h\sim\P_i}\LB L{|\S_i|}(h) \RB.
        \end{align*}
        Finally, using the assumption $\EE_{h\sim \P_i}\EE_{z\sim \D}[\loss(h,z)^2] \leq 1$ gives, with probability at least $1-\delta$, for all $\Q\in \Mcal(\H)$:
        \begin{align*}
             \EE_{h\sim\Q}\Big[ \Risk_{\D}(h) - \Riskhat_{\S}(h) \Big] \leq    \sum_{i=1}^{K} \frac{2|\S_i|L}{m}\W(\Q, \P_i) + \sum_{i=1}^{K}   \frac{\ln\left( \frac{K}{\delta}  \right)}{\lambda_i m} + \frac{\lambda_i[\S_i|]}{2m}.
        \end{align*}
        Taking for each $i$, $\lambda_i= \sqrt{\frac{2\ln(K/\delta)}{|\S_i|}}$ concludes the proof.
    \end{proof}
        
    
    \subsection{Alternative proof of \Cref{theorem:supervised-nnl}}
    \label{sec:alt-proof-supervised}
    
    
    We state here a slightly tighter version of \Cref{theorem:supervised-nnl} for bounded losses, which relies on an application of McDiarmid's inequality instead of supermartingale techniques.
    This is useful for the numerical evaluations of our bound.
    
    \begin{theorem}\label{theorem:supervised_tight}
    We assume our loss $\loss$ to be in $[0,1]$ and $L$-Lipschitz.
    Then, for any $\delta\in(0,1]$, with probability at least $1-\delta$ over the sample $\S$, the following holds for the distributions $\P_{i,\S}\defeq \P_i(\S,.)$ and for any $\Q\in\Mcal(\H)$:
    \begin{align*}
    \ \EE_{h\sim\Q}\Big[ \Risk_{\D}(h) - \Riskhat_{\S}(h) \Big] \le \sum_{i=1}^{K} \frac{2|\S_i|L}{m} \W(\Q, \P_{i,\S}) + \sum_{i=1}^{K} \sqrt{\frac{|\S_i|\ln\frac{K}{\delta}}{2m^2}} 
    \end{align*}
    where $\P_i$ {\it does not} depend on $\S_i$.
    \end{theorem}
    \begin{proof}
    For the sake of readability, we identify, for any $i$, $\P_i $ and $\P_{i,\S}$.
    
    First of all, note that for a $L$-Lipschitz loss function $\loss: \H\times\Z\to[0,1]$, we have
    \begin{align}
    \LN\LP |\S_i|\Risk_{\D}(h_1){-}\sum_{\z\in\S_i}\loss(h_1, \z)\RP - \LP |\S_i|\Risk_{\D}(h_2){-}\sum_{\z\in\S_i}\loss(h_2, \z)\RP\right\vert \le 2|\S_i|Ld(h_1, h_2).\label{eq:alt-proof-supervised-1}
    \end{align}
    Indeed, we can deduce \Cref{eq:alt-proof-supervised-1} from Jensen's inequality, the triangle inequality, and by definition that we have
    \begin{align*}
    &\LN\LP |\S_i|\Risk_{\D}(h_1){-}\sum_{\z\in\S_i}\loss(h_1, \z)\RP - \LP |\S_i|\Risk_{\D}(h_2){-}\sum_{\z\in\S_i}\loss(h_2, \z)\RP\right\vert\\
    &= \LN\LP\sum_{\z\in\S_i}\Risk_{\D}(h_1){-}\sum_{\z\in\S_i}\loss(h_1, \z)\RP - \LP\sum_{\z\in\S_i}\Risk_{\D}(h_2){-}\sum_{\z\in\S_i}\loss(h_2, \z)\RP\right\vert\\
    &\le \sum_{\z\in\S_i}\EE_{\z'\sim\D}\Big[ |\loss(h_1, \z')-\loss(h_2, \z')|+|\loss(h_2, \z)-\loss(h_1, \z)|\Big]\\
    &\le \EE_{\z'\sim\D}\sum_{\z\in\S_i} 2Ld(h_1, h_2)\\
    &= 2|\S_i|Ld(h_1, h_2).
    \end{align*}
    We are now able to upper-bound $\EE_{h\sim\Q}[ \Risk_{\D}(h) - \Riskhat_{\S}(h) ]$. 
    Indeed, we have
    \begin{align}
    \EE_{h\sim\Q}\Big[ \Risk_{\D}(h) - \Riskhat_{\S}(h) \Big] &= \frac{1}{m}\sum_{i=1}^{K} \EE_{h\sim\Q}\LB |\S_i|\Risk_{\D}(h) - \sum_{\z\in\S_i}\loss(h, \z) \RB\nonumber\\
    &\le \sum_{i=1}^{K} \frac{2|\S_i|L}{m}\W(\Q, \P_i) + \sum_{i=1}^{K} \EE_{h\sim\P_i} \frac{1}{m}\LB |\S_i|\Risk_{\D}(h) - \sum_{\z\in\S_i} \loss(h, \z) \RB,\label{eq:alt-proof-supervised-2}
    \end{align}
    where the inequality comes from the Kantorovich-Rubinstein duality theorem.
    Let $f(\S_i)=\EE_{h\sim\P_i} \frac{1}{m}\LB |\S_i|\Risk_{\D}(h) - \sum_{\z\in\S_i} \loss(h, \z_i)\RB$, the function has the bounded difference inequality, \ie, for two datasets $\S_i$ and $\S'_i$ that differs from one example (the $k$-th example, without loss of generality), we have
    \begin{align*}
    \LN f(\S_i) - f(\S'_i) \right\vert &= \LN \EE_{h\sim\P_i} \frac{1}{m}\LB |\S_i|\Risk_{\D}(h) - \sum_{\z\in\S_i} \loss(h, \z)\RB - \EE_{h\sim\P_i} \frac{1}{m}\LB |\S_i|\Risk_{\D}(h) - \sum_{\z'\in\S'_i} \loss(h, \z')\RB \right\vert\\
    &= \LN \EE_{h\sim\P_i} \LB \frac{1}{m}|\S_i|\Risk_{\D}(h) - \frac{1}{m}\sum_{\z\in\S_i} \loss(h, \z) - \frac{1}{m}|\S_i|\Risk_{\D}(h) + \frac{1}{m}\sum_{\z'\in\S'_i} \loss(h, \z')\RB \right\vert\\
    &= \LN \EE_{h\sim\P_i} \LB \frac{1}{m}\sum_{\z'\in\S'_i} \loss(h, \z') - \frac{1}{m}\sum_{\z\in\S_i} \loss(h, \z)\RB \right\vert\\
    &= \LN \EE_{h\sim\P_i} \LB \frac{1}{m}\loss(h, \z'_k) - \frac{1}{m}\loss(h, \z_k)\RB \right\vert\\
    &\le \frac{1}{m}.
    \end{align*}
    
    Hence, from Mcdiarmid's inequality, we have with probability at least $1-\frac{\delta}{K}$ over $\S\sim\D^{m}$
    \begin{align*}
    &\EE_{h\sim\P_i} \frac{1}{m}\LB |\S_i|\Risk_{\D}(h) - \sum_{\z\in\S_i} \loss(h, \z) \RB \\ &\le \EE_{\S\sim\D^{m}}\EE_{h\sim\P_i} \frac{1}{m}\LB |\S_i|\Risk_{\D}(h) - \sum_{\z\in\S_i} \loss(h, \z) \RB + \sqrt{\frac{|\S_i|\ln\frac{K}{\delta}}{2m^2}}\\
    &= \EE_{\S^c_i\sim\D^{m-|\S_i|}}\EE_{\S_i\sim\D^{|\S_i|}}\EE_{h\sim\P_i} \frac{1}{m}\LB |\S_i|\Risk_{\D}(h) - \sum_{\z\in\S_i} \loss(h, \z) \RB + \sqrt{\frac{|\S_i|\ln\frac{K}{\delta}}{2m^2}}\\
    &= \EE_{\S^c_i\sim\D^{m-|\S_i|}}\EE_{h\sim\P_i} \frac{1}{m}\LB |\S_i|\Risk_{\D}(h) - \EE_{\S_i\sim\D^{|\S_i|}}\sum_{\z\in\S_i} \loss(h, \z) \RB + \sqrt{\frac{|\S_i|\ln\frac{K}{\delta}}{2m^2}}\\
    &= \EE_{\S^c_i\sim\D^{m-|\S_i|}}\EE_{h\sim\P_i} \frac{1}{m}\Big[ |\S_i|\Risk_{\D}(h) - |\S_i|\Risk_{\D}(h)\Big] + \sqrt{\frac{|\S_i|\ln\frac{K}{\delta}}{2m^2}}\\
    &= \sqrt{\frac{|\S_i|\ln\frac{K}{\delta}}{2m^2}}.
    \end{align*}
    
    From the union bound, we have with probability at least $1-\delta$ over $\S\sim\D^m$, for any $\Q\in\Mcal(\H)$,
    \begin{align*}
    \EE_{h\sim\Q}\Big[ \Risk_{\D}(h) - \Riskhat_{\S}(h) \Big] \le \sum_{i=1}^{K} \frac{2|\S_i|L}{m}\W(\Q, \P_i) + \sum_{i=1}^{K} \sqrt{\frac{|\S_i|\ln\frac{K}{\delta}}{2m^2}},
    \end{align*}
    which is the claimed result.
    \end{proof}
    
    We are now able to give a corollary of \Cref{theorem:supervised_tight}.
    
    \begin{corollary}\label{corollary:supervised-nnl} We assume our loss $\loss$ to be in $[0,1]$ and $L$-Lipschitz. 
    Then, for any $\delta\in(0,1]$, with probability at least $1-\delta$ over the sample $\S$, the following holds for the hypotheses $h_{i,\S}\in\H$ associated with the Dirac distributions $\P_{i,\S}$ and for any $h\in\H$: 
    \begin{align*}
    \Risk_{\D}(h) \le \Riskhat_{\S}(h) + \sum_{i=1}^{K} \frac{2|\S_i|L}{m} d(h, h_{i,\S}) + \sum_{i=1}^{K} \sqrt{\frac{|\S_i|\ln\frac{K}{\delta}}{2m^2}}.
    \end{align*}
    \end{corollary}
    
    Such a bound was impossible to obtain from the PAC-Bayesian bounds based on a KL divergence.
    Indeed, the KL divergence is infinite for two distributions with disjoint supports. 
    Hence, the PAC-Bayesian framework based on the Wasserstein distance allows us to provide uniform-convergence bounds from a proof technique different from the ones based on the Rademacher complexity~\cite{koltchinskii2000rademacher,bartlett2001rademacher,bartlett2002rademacher} or the VC-dimension~\cite{vapnik1968uniform,vapnik1974theory}.
    In \Cref{sec:experiments}, we provide an algorithm minimising such a bound.
    
    \subsection{Proof of \Cref{theorem:online-ht}}
    \label{sec:proof-online-ht}
    
    \theoremonlineht*
    \begin{proof}
    First of all, note that for a $L$-Lipschitz loss function $\loss: \H\times\Z\to\R$, we have
    \begin{align}
    \LN\LP \EE_{i-1}[\loss(h_i,\z_i)]{-}\loss(h_i, \z_i)\RP - \LP \EE_{i-1}[\loss(h'_i,\z_i)]{-}\loss(h'_i, \z_i)\RP\right\vert \le 2Ld(h_i, h'_i).\label{eq:proof-online-ht-1}
    \end{align}
    Indeed, we can deduce \Cref{eq:proof-online-ht-1} from Jensen inequality, the triangle inequality, and by definition that we have
    \begin{multline*}
    \LN\LP\EE_{i-1}[\loss(h_i,\z_i)]{-}\loss(h_i, \z_i)\RP - \LP\EE_{i-1}[\loss(h'_i,\z_i)]{-}\loss(h'_i, \z_i)\RP\right\vert \\ 
    \le \EE_{i-1}\Big[ |\loss(h_i, \z'_i)-\loss(h'_i, \z'_i)|+|\loss(h_i, \z_i)-\loss(h'_i, \z_i)|\Big]\\
     \le \EE_{i-1} 2Ld(h_i, h'_i) = 2Ld(h_i, h'_i).
    \end{multline*}
    From the Kantorovich-Rubinstein duality theorem \cite[Remark 6.5]{villani2009optimal}, we have
    \begin{align*}
    \sum_{i=1}^{m}\EE_{h_i\sim\Q_{i}}\LB\EE_{i-1}[\loss(h_i,\z_i)]-\loss(h_i, \z_i) \RB \le 2L\sum_{i=1}^{m}W_1(\Q_{i}, \P_{i,\S}) + \sum_{i=1}^{m}\EE_{h\sim\P_{i,\S}}\LB\Risk_{\D}(h_i)-\loss(h_i, \z_i) \RB.
    \end{align*}
    
    Now, we define $X_i(h_i,\z_i)\defeq \EE_{i-1}[\loss(h_i,\z_i)]- \loss(h_i, \z_i)$. We also recall that for any $i$, we have $\Vhat_i(h_i,\z_i)= (\loss(h_i,\z_i)-\EE_{i-1}[\loss(h_i,\z_i)])^2$ and $V_i(h_i)= \EE_{i-1}[\Vhat(h_i,\z_i)]$.
    To apply the supermartingales techniques of \cite{haddouche2023pac}, we define the following function:
    \begin{align*}
       f_m(S,h_1,...,h_m) & \defeq \sum_{i=1}^m \lambda X_i(h_i,\z_i)  - \frac{\lambda^2}{2}\sum_{i=1}^m(\Vhat_i(h_i,\z_i) + V_i(h_i)).
       \end{align*}
        Now, Lemma 3.2 of \cite{haddouche2023pac} state that the sequence $(SM_m)_{m\geq 1}$ defined for any $m$ as:
        \begin{align*}
         SM_m \defeq \EE_{(h_1,\cdots,h_m) \sim \P_{1,\S} \otimes \cdots \otimes \P_{m,\S}} \LB \exp \Big( f_m(\S,h_1,...,h_m) \Big)  \RB,
         \end{align*}
        is a supermartingale. 
        We exploit this fact as follows: 
        \begin{align*}
            \sum_{i=1}^{m}\EE_{h\sim\Q_{i-1}}\LB\EE_{i-1}[\loss(h_i,\z_i)]-\loss(h_i, \z_i)\RB &= \EE_{(h_1,\cdots,h_m) \sim \P_{1,\S} \otimes \cdots \otimes \P_{m,\S}} \LB\sum_{i=1}^m X_i(h_i,\z_i)\RB \\
            & = \frac{1}{\lambda}\EE_{(h_1,\cdots,h_m) \sim \P_{1,\S} \otimes \cdots \otimes \P_{m,\S}} \LB f_m (\S,h_1,\cdots,h_m)\RB \\
            & + \frac{\lambda}{2} \sum_{i=1}^m \EE_{h_i\sim \P_{i,\S}}\left[ \Vhat_i(h_i,\z_i) + V_i(h_i) \right] \\
            & \leq \frac{\ln\left( SM_m \right)}{\lambda} + \frac{\lambda}{2} \sum_{i=1}^m \EE_{h_i\sim \P_{i,\S}}\left[ \Vhat_i(h_i,\z_i) + V_i(h_i) \right]
        \end{align*}
        The last line holds thanks to Jensen's inequality.
        Now using Ville's inequality ensures us that:
        \begin{align*}
        \PP_{\S}\left( \forall m, SM_m \leq \frac{1}{\delta} \right) \geq \frac{1}{\delta}.
        \end{align*}
        Thus, with probability $1-\delta$, for any $m$ we have $\ln\left( SM_m \right) \leq \ln\left( \frac{1}{\delta} \right)$.
        This concludes the proof.
    \end{proof}
    
    \subsection{Proof of \Cref{theorem:online}}
    \label{sec:proof-online}
    \theoremonline*
    \begin{proof}
    The proof starts similarly to the one of \Cref{theorem:online-ht}.
    Indeed, note that for a $L$-Lipschitz loss function $\loss: \H\times\Z\to\R$, we have
    \begin{align}
    \LN\LP \EE_{i-1}[\loss(h_i,\z_i)]{-}\loss(h_i, \z_i)\RP - \LP \EE_{i-1}[\loss(h'_i,\z_i)]{-}\loss(h'_i, \z_i)\RP\right\vert \le 2Ld(h_i, h'_i).\label{eq:proof-online-1}
    \end{align}
    Indeed, we can deduce \Cref{eq:proof-online-1} from Jensen inequality, the triangle inequality, and by definition that we have
    \begin{multline*}
    \LN\LP\EE_{i-1}[\loss(h_i,\z_i)]{-}\loss(h_i, \z_i)\RP - \LP\EE_{i-1}[\loss(h'_i,\z_i)]{-}\loss(h'_i, \z_i)\RP\right\vert \\ 
    \le \EE_{i-1}\Big[ |\loss(h_i, \z'_i)-\loss(h'_i, \z'_i)|+|\loss(h_i, \z_i)-\loss(h'_i, \z_i)|\Big]\\
     \le \EE_{i-1} 2Ld(h_i, h'_i) = 2Ld(h_i, h'_i).
    \end{multline*}
    From the Kantorovich-Rubinstein duality theorem \cite[Remark 6.5]{villani2009optimal}, we have
    \begin{align*}
    \sum_{i=1}^{m}\EE_{h_i\sim\Q_{i}}\LB\EE_{i-1}[\loss(h_i,\z_i)]-\loss(h_i, \z_i) \RB \le 2L\sum_{i=1}^{m}W_1(\Q_{i}, \P_{i,\S}) + \sum_{i=1}^{m}\EE_{h\sim\P_{i,\S}}\LB\Risk_{\D}(h_i)-\loss(h_i, \z_i) \RB.
    \end{align*}
    Now, we define $X_i(h_i,\z_i)\defeq \EE_{i-1}[\loss(h_i,\z_i)]- \loss(h_i, \z_i)$. 
    To apply the supermartingales techniques of \cite{chugg2023unified}, we define the following function:
    \begin{align*}
       f_m(S,h_1,...,h_m) & \defeq \sum_{i=1}^m \lambda X_i(h_i,\z_i)  - \frac{\lambda^2}{2}\sum_{i=1}^m \EE_{i-1}[\loss(h_i, \z_i)^2].
       \end{align*}
        Now, because our loss is nonnegative, \cite[][Lemma A.2 and Lemma B.1]{chugg2023unified} state that the sequence $(SM_m)_{m\geq 1}$ defined for any $m$ as:
        \begin{align*}
        SM_m \defeq \EE_{(h_1,\cdots,h_m) \sim \P_{1,\S} \otimes \cdots \otimes \P_{m,\S}} \LB \exp \Big( f_m(\S,h_1,...,h_m) \Big)  \RB,
        \end{align*}
        is a supermartingale. 
        We exploit this fact as follows:
        \begin{align*}
            \sum_{i=1}^{m}\EE_{h\sim\Q_{i-1}}\LB\EE_{i-1}[\loss(h_i,\z_i)]-\loss(h_i, \z_i)\RB &= \EE_{(h_1,\cdots,h_m) \sim \P_{1,\S} \otimes \cdots \otimes \P_{m,\S}} \LB\sum_{i=1}^m X_i(h_i,\z_i)\RB \\
            & = \frac{1}{\lambda}\EE_{(h_1,\cdots,h_m) \sim \P_{1,\S} \otimes \cdots \otimes \P_{m,\S}} \LB f_m (\S,h_1,\cdots,h_m)\RB \\
            & + \frac{\lambda}{2} \sum_{i=1}^m \EE_{h_i\sim \P_{i,\S}}\left[ \EE_{i-1}[\loss(h_i, \z_i)^2] \right] \\
            & \leq \frac{\ln\left( SM_m \right)}{\lambda} + \frac{\lambda}{2} \sum_{i=1}^m \EE_{h_i\sim \P_{i,\S}}\left[ \EE_{i-1}[\loss(h_i, \z_i)^2] \right]
        \end{align*}
        The last line holds thanks to Jensen's inequality.
        Now using Ville's inequality ensures us that:
        \begin{align*}
        \PP_{\S}\left( \forall m, SM_m \leq \frac{1}{\delta} \right) \geq \frac{1}{\delta}
        \end{align*}
        Thus, with probability $1{-}\delta$, for any $m$ we have $\ln(SM_m){\leq}\ln\frac{1}{\delta}$. 
        We conclude the proof by exploiting the boundedness assumption on conditional order 2 moments and optimising the bound in $\lambda$. 
    \end{proof}
    
    \section{Supplementary insights on experiments}
    \label{sec:supplementary-expes}
    
    In this section, \Cref{sec:alg-batch} presents the learning algorithm for the \iid setting.
    We also introduce the online algorithm in \Cref{sec:alg-online}.
    We prove the Lipschitz constant of the loss for the linear models in \Cref{sec:lip-linear}.
    Finally, we provide more experiments in \Cref{sec:experiments-supp}.
    
    \subsection{Batch algorithm for the \iid setting}
    \label{sec:alg-batch}
    
    The pseudocode of our batch algorithm is presented in \Cref{alg:batch}.
    
    \begin{algorithm}[H]
    \caption{(Mini-)Batch Learning Algorithm with Wasserstein distances}\label{alg:batch}
    \begin{algorithmic}[1]
    \Procedure{Priors Learning}{}
    \State{$h_1,\dots, h_K \leftarrow$ initialize the hypotheses}\\
    \For{$t\leftarrow 1,\dots, T$}{\For{\textbf{each} mini-batch $\mathcal{U}\subseteq \S$}{\For{$i \leftarrow 1,\dots, K$}{$\mathcal{U}_i \leftarrow \mathcal{U}\setminus \S_i$\newline
    $h_i\leftarrow$ perform a gradient descent step with $\nabla\Risk_{\mathcal{U}_i}(h_i)$}}}
    \State{\textbf{return} hypotheses $h_1,\dots, h_K$}
    \EndProcedure
    \newline
    \Procedure{Posterior Learning}{}
    \State{$h \leftarrow$ initialize the hypothesis}\\
    \For{$t\leftarrow 1,\dots, T'$}{\For{\textbf{each} mini-batch $\mathcal{U}\subseteq \S$}{$h \leftarrow$ perform a gradient descent step with $\nabla[ \Risk_{\mathcal{U}}(h) + \varepsilon\sum_{i=1}^{K}\frac{|\S_i|}{m}d(h, h_i) ]$}}
    \State{\textbf{return} hypothesis $h$}
    \EndProcedure
    \end{algorithmic}
    \end{algorithm}
    
    \textsc{Priors Learning} minimises the empirical risk through mini-batches $\mathcal{U}\subseteq\S$ for $T$ epochs. More precisely, for each epoch, we {\it (a)} sample a mini-batch $\mathcal{U}$ (line 4) by excluding the set $\S_i$ from $\mathcal{U}$ for each $h_i\in\H$ (line 5-6), then {\it (b)} the hypotheses $h_1,\dots,h_K\in\H$ are updated (line 7).
    In \textsc{Posterior Learning}, we perform a gradient descent step (line 14) on the objective function associated with \Cref{eq:batch-alg-exp} for $T'$ epochs in a mini-batch fashion.
    
    \subsection{Learning algorithm for the online setting}
    \label{sec:alg-online}
    
    \Cref{alg:online} presents the pseudocode of our online algorithm.
    
    \begin{algorithm}[H]
    \caption{Online Learning Algorithm with Wasserstein distances}\label{alg:online}
    \begin{algorithmic}[1]
    \State{Initialize the hypothesis $h_0\in\H$}\\
    \For{$i\leftarrow 1,\dots, m$}{\For{$t\leftarrow 1,\dots, T$}{$h_i \leftarrow $ perform a gradient step with $\nabla[ \loss(h_i, \z_i) + \Bhat(d(h_i, h_{i-1}){-}1) ]$~(\cref{eq:online-alg-exp-2} with $\Bhat$)}}
    \State{{\bf return} hypotheses $h_1,\dots, h_m$}
    \end{algorithmic}
    \end{algorithm}
    
    For each time step $i$, we perform $T$ gradient descent steps on the objective associated with \Cref{eq:online-alg-exp} (line 4).
    Note that we can retrieve OGD from \Cref{alg:online} by {\it (a)} setting $T=1$ and {\it (b)} removing the regularisation term $\Bhat(d(h_i, h_{i-1}){-}1)$.
    
    \subsection{Lipschitzness for the linear model}
    \label{sec:lip-linear}
    
    Recall that we use, in our experiments, the multi-margin loss function from the Pytorch module defined for any linear model with weights $W\in\R^{|\Y|\times d}$ and biases $b\in\R^{|\Y|}$, any data point $\z\in \X\times\Y$
    
    \[\loss(W,b,\z)= \frac{1}{|\Y|-1}\sum_{y^\prime \neq y} \max\left(0, f(W,b,\z,y') \right), \]
    
    where $f(W,b,\z,y')= 1+ \langle W[y']-W[y],\x \rangle+b[y']-b[y]$, and $W[y]\in \R^{d}$ and $b[y]\in\R$ are respectively the vector and the scalar for the $y$-th output.
    
    To apply our theorems, we must ensure that our loss function is Lipschitz with respect to the linear model, hence the following lemma.
    
    \begin{lemma}\label{lemma:lipschitz-linear}
        For any $\z=(\x,y)\in\X\times\Y$ with the norm of $\x$ bounded by $1$, the function $W,b\mapsto \loss(W,b,\z)$ is $2$-Lipschitz.
    \end{lemma}
    
    \begin{proof}
        Let $(W,b),(W',b')$ both in $ \R^{|\Y|\times d}\times \R^{|\Y|}$, we have
        \begin{align*}
            |\loss(W,b,\z)- \loss(W',b',\z)| & \leq \frac{1}{|\Y|-1}\sum_{y^\prime \neq y} |\max\left(0, f(W,b,\z,y') \right) - \max\left(0, f(W',b',\z,y') \right)|.
        \end{align*}
        Note that because $\alpha\mapsto\max(0,\alpha)$ is $1$-Lipschitz, we have: 
        \begin{align*}
            |\loss(W,b,\z)- \loss(W',b',\z)| & \leq \frac{1}{|\Y|-1}\sum_{y^\prime \neq y} | f(W,b,\z,y') - f(W',b',\z,y')|.
        \end{align*}
        Finally, notice that: 
        \begin{align*}
            \frac{1}{|\Y|-1}\sum_{y^\prime \neq y} | f(W,b,\z,y')  -f(W',b',\z,y')| &\le \frac{1}{|\Y|-1}\sum_{y^\prime \neq y} \left|\left\langle (W-W')[y'] - (W-W')[y],\x\right\rangle   \right| \\
            & + \frac{1}{|\Y|-1}\sum_{y^\prime \neq y} |(b-b')[y'] - (b-b')[y]|\\
            & \leq \frac{1}{|\Y|-1}\sum_{y^\prime \neq y} \left\| (W-W')[y'] - (W-W')[y]  \right\|\|\x\| \\
            & + \frac{1}{|\Y|-1}\sum_{y^\prime \neq y} |(b-b')[y'] - (b-b')[y]|.
        \end{align*}
    
    
       Because we consider the Euclidean norm, we have for any $y'\in\Y$:
    
        \begin{align*}
            \left\| (W-W')[y'] - (W-W')[y]  \right\| & = \sqrt{\| (W-W')[y'] - (W-W')[y]\|^2} \\
            & \leq \sqrt{2 \left(\|(W-W')[y']\|^2 + \|(W-W')[y]\|^2 \right)} \\
            & \leq \sqrt{2}\|W-W'\|.
        \end{align*}
    
        The second line holding because for any scalars $a,b$, we have $(a-b)^2 \leq 2(a^2 + b^2)$ and the last line holding because $\|W-W'\|^2= \sum_{y\in\Y} \|(W-W')[y]\|^2$.
        A similar argument gives
    
        \[\frac{1}{|\Y|-1}\sum_{y^\prime \neq y} |(b-b')[y'] - (b-b')[y]| \leq \sqrt{2}||b-b'||.\]
    
        Then, using that $\|x\| \leq 1$ and summing on all $y'$ gives: 
    
        \[|\loss(W,b,\z)- \loss(W',b',\z)| \leq \sqrt{2}\left(\|W-W'\| + \|b-b'\|\right). \]
    
        Finally, notice that $(\|W-W'\| + \|b-b'\|)^2 \leq 2(\|W-W'\|^2 + \|b-b'\|^2) = 2\|(W,b)-(W',b')\|^2$. 
    
        Thus $\|W-W'\| + \|b-b'\| \leq \sqrt{2}\|(W,b)-(W',b')\|$. 
        This concludes the proof.
    \end{proof}
    
    \subsection{Lipschitzness for neural networks}
    \label{sec:lip-nn}
    
    Recall that we use, in our experiments, the multi-margin loss function from the Pytorch module defined we consider the loss $\loss(h, (\x, y)) = \frac{1}{|\Y|}\sum_{y'\ne y} \max(0, 1{-} \eta(h[y]{-}h[y']))$, which is $\eta$-Lipschitz \wrt the outputs $h[1],\dots, h[|\Y|]$.
    For neural networks, $h$ is the output of the neural network with input $\x$. Note that this loss is $\eta$-lipschitz with respect to the outputs.
    To apply our theorems, we must ensure that our loss function is Lipschitz with respect to the weights of the neural networks, hence the following lemma with associated background.
    
    We define a FCN recursively as follows: for a vector $\Wbf_1= \vect(\{W_1,b\})$, (\ie, the vectorisation of a weight matrix $W_1$ and a bias $b$) and an input datum $\x$, $\text{FCN}_1(\Wbf_1, \x)= \sigma_1\left(W_1\x +b_1  \right)$, where $\sigma_1$ is the activation function.
    Also, for any $i\geq 2$ we define for a vector $\Wbf_i= (W_{i},b_i,\Wbf_{i-1})$ (defined recursively as well), $\text{FCN}_i(\Wbf_{i},\x)= \sigma_i\left(W_i\text{FCN}_{i-1}(\Wbf_{i-1},\x) +b_i  \right)$.
    Then, setting $\z=(\x,y)$ a datum and  $h_i(\x) \defeq \text{FCN}_i(\Wbf_{i},\x)$ we can rewrite our loss as a function of $(\Wbf_{i},\z)$. 
    
    \begin{lemma}
        \label{l:lip-nn}
        Assume that all the weight matrices of $\Wbf_i$ are bounded and that the activation functions are Lipschitz continuous with constant bounded by $K_\sigma$.
        Then for any datum $\z= (\x,y)$, any $i$, $\Wbf_i\rightarrow\loss(\Wbf_i,\z)$ is Lipschitz continuous.
    \end{lemma}
    \begin{proof}
        We consider the Frobenius norm on matrices as $\Wbf_2$ is a vector as we consider the L2-norm on the vector.
        We prove the result for $i=2$, assuming it is true for $i=1$. We then explain how this proof generalises the case $i=1$ and works recursively.
        Let $\z,\Wbf_2,\Wbf_2'$, for clarity we write $\text{FCN}_2(\x)\defeq \text{FCN}(\Wbf_2,\x)$ and $\text{FCN}_2'(\x)\defeq \text{FCN}(\Wbf_2',\x)$. 
        As $\loss$ is Lipschitz on the outputs $\text{FCN}_2(\x), \text{FCN}_2'(\x)$. 
        We have
        
        \begin{multline*}
            |\loss(\Wbf_2, \z) - \loss(\Wbf'_2, \z) |  \le \eta \left\|\text{FCN}_2(\x)-\text{FCN}_2'(\x) \right\| \\
            \le \eta \left\|\sigma_2\left(W_2\text{FCN}_{1}(\x) +b_2 \right) - \sigma_2\left(W_2'\text{FCN}_{1}'(\x) +b_2' \right) \right\| \\
             \le \eta K_{\sigma}\| W_2\text{FCN}_{1}(\x) +b_2 - W_2'\text{FCN}_{1}'(\x) -b_2'\| \\
             \le \eta K_\sigma\left( ||(W_2- W_2') \text{FCN}_{1}(\x) || + ||W_2'(\text{FCN}_{1}(\x)-\text{FCN}_{1}'(\x))\| + \|b_2-b_2'\|  \right).
        \end{multline*}
        
        Then, we have $||(W_2- W_2') \text{FCN}_{1}(\x) || \le ||(W_2- W_2')||_F ||\text{FCN}_1(\x)|| \le  K_\x ||(W_2- W_2')||_F$. The second inequality holding as $\text{FCN}_1(\x)$ is a continuous function of the weights. Indeed, as on a compact space, a continuous function reaches its maximum, then its norm is bounded by a certain $K_\x$. 
        Also, as the weights are bounded, any weight matrix has its norm bounded by a certain $K_{W}$ thus $\|W_2'(\text{FCN}_{1}(\x)-\text{FCN}_{1}'(\x)\| \le \|W_2'\|_F\|(\text{FCN}_{1}(\x)-\text{FCN}_{1}'(\x)\| \le K_{W}\|\text{FCN}_{1}(\x)-\text{FCN}_{1}'(\x)\|$. Finally, taking $K_{\text{temp}}= \eta K_{\sigma}\max(K_{\x},K_W,1)$ gives: 
        \begin{align*}
            |\loss(\Wbf_2, \z) - \loss(\Wbf'_2, \z) |
            &\le K_{\text{temp}} \left(\|(W_2- W_2')\|_F + \|b_2-b_2'\| + \|\text{FCN}_{1}(\x)-\text{FCN}_{1}'(\x)\|\right).
        \end{align*}  
    
        Exploiting the recursive assumption that $\text{FCN}_1$ is Lipschitz with respect to its weights $\Wbf_1$ gives  $\|\text{FCN}_{1}(\x)-\text{FCN}_{1}'(\x)\| \le K_1 ||\Wbf_1- \Wbf_1'|| $.
    
        If we denote by $(W_2,b_2)$ the vector of all concatenated weights, notice that $\|(W_2- W_2')\|_F + \|b_2-b_2'\| = \sqrt{(\|(W_2- W_2')\|_F + \|b_2-b_2'\|)^2} \le \sqrt{2(\|(W_2- W_2')\|_F^2 + \|b_2-b_2'\|^2)} = \sqrt{2}\|(W_2,b_2)-(W_2',b_2')\|$ (we used that for any real numbers $a,b, (a+b)^2\le 2(a^2 + b^2)$). We then have: 
        \begin{align*}
             |\loss(\Wbf_2, \z) - \loss(\Wbf'_2, \z) | & \le K_{\text{temp}} \max(\sqrt{2},K_1) \left( \|(W_2,b_2)-(W_2',b_2')\| + ||\Wbf_1- \Wbf_1'|| \right) \\
             & \le \sqrt{2} K_{\text{temp}} \max(\sqrt{2},K_1) ||\Wbf_2- \Wbf_2'||.
         \end{align*} 
        The last line holds by reusing the same calculation trick. This concludes the proof for $i=2$. Then for $i=1$ the same proof holds by replacing $W_2, b_2, \text{FCN}_2$ by $W_1, b_1, \text{FCN}_1$ and replacing $\text{FCN}_1(\x),  \text{FCN}_1'(\x)$ by $\x$ (we then do not need to assume a recursive Lipschitz behaviour). Therefore the result holds for $i=1$. 
    
        We then properly apply a recursive argument by assuming the result at rank $i-1$ reusing the same proof at any rank $i$ by replacing $W_2, b_2, \text{FCN}_2$ by $W_i, b_i, \text{FCN}_i$ and $\text{FCN}_1(\x), \text{FCN}_1'(\x)$ by $\text{FCN}_{i-1}(\x), \text{FCN}_{i-1}'(\x)$. This concludes the proof.
    \end{proof}
    
    \subsection{Experiments with varying number of priors}
    \label{sec:experiments-supp}
    
    The experiments of \Cref{sec:experiments} rely on data-dependent priors constructed through the procedure \textsc{Priors Learning}.
    We fixed a number of priors $K$ equal to $0.2\sqrt{m}$.
    This number is an empirical tradeoff between the informativeness of our priors and time-efficient computation.
    However, there is no theoretical intuition for the value of this parameter (the discussion of \Cref{sec:wasserstein-batch} considered $K=\sqrt{m}$ as a potential tradeoff; see \Cref{sec:discussion-supervised}). 
    Thus, we gather below the performance of our learning procedures for $K=\alpha\sqrt{m}$, where $\alpha\in\{0,0.4,0.6,0.8,1\}$ (the case $\alpha =0$ being a convention to denote $K=1$).
    The experiments are gathered below, and all remaining hyperparameters (except $K$) are identical to those described in \Cref{sec:experiments}.
    
    
    \textbf{Analysis of our results.}
    First, when considering neural networks, note that for any dataset except \textsc{segmentation}, \textsc{letter}, the performances of our methods are similar or better when considering data-dependent priors (\ie, when $\alpha >0$).
    A similar remark holds for the linear models for all datasets except for \textsc{satimage}, \textsc{segmentation}, and \textsc{tictactoe}. This illustrates the relevance of data-dependent priors.
    We also remark that there is no value of $\alpha$, which provides a better performance on all datasets.
    For instance, considering neural networks, note that $\alpha=1$ gives the better performance (\ie, the smallest $\mathfrak{R}_\D(h)$) for \Cref{alg:batch}~($\frac{1}{\sqrt{m}}$) for the \textsc{satimage} dataset while, for the same algorithm, the better performance on the \textsc{segmentation} dataset is attained for $\alpha=0.8$. 
    Sometimes, the number $K$ does not have a clear influence: on \textsc{mnist} with NNs, for \Cref{alg:batch}~($\frac{1}{\sqrt{m}}$), our performances are similar, whatever the value of $K$, but still significantly better than ERM.
    In any case, note that for every dataset, there exists a value of $K$ and such that our algorithm attains either similar or significantly better performances than ERM on every dataset, which shows the relevance of our learning algorithm to ensure a good generalisation ability.
    Moreover, there is no obvious choice for the parameters $\varepsilon$.
    For instance, in \Cref{tab:expe-2,tab:expe-3}, for the \textsc{segmentation} dataset, the parameters $K=1,\varepsilon=\frac{1}{m}$ are optimal (in terms of test risks) for both models. 
    As $K=1$ means that our single prior is data-free, this shows that the intrinsic structure of \textsc{segmentation} makes it less sensitive to both the information contained in the prior ($K=1$ meaning data-free prior) and the place of the prior itself ($\varepsilon=1/m$ meaning that we give less weight to the regularisation within our optimisation procedure).
    On the contrary, in Table \Cref{tab:expe}, the \textsc{yeast} dataset performs significantly better when $\varepsilon=1/\sqrt{m} (K=0.2\sqrt{m})$, exhibiting a positive impact of our data-dependent priors.
    
    \subsection{Experiments on classical regularisation methods}
    
    We perform additional experiments to see the performance of the weight decay, \ie, the L2 regularisation on the weights; the results are presented in \Cref{tab:expe-4}.
    Moreover, notice that the 'distance to initialisation' $\|\wbf-\wbf_0\|$ (where $\wbf_0$ is the weights initialized randomly) is a particular case of \Cref{alg:batch} when $K=1$ (\ie, we treat the data as a single batch, and the prior is the data-free initialisation); the results are in \Cref{tab:expe-2,tab:expe-3}.
    
    \textbf{Analysis of our results.}
    This experiment on the weight decay demonstrates that on a few datasets (namely \textsc{sensorless} and \textsc{yeast}), when our predictors are neural nets, the weight decay regularisation fails to learn while ours succeeds, as shown in \Cref{tab:expe}.
    In general, this table shows that, on most of the datasets, considering data-dependent priors leads to sharper results.
    This shows the efficiency of our method compared to the 'distance to initialisation' regularisation.
    
    %%%%%%%%%%%%%%%%%%%%%%%%%%%%%%%%%%%%%%%%%%%%%%%%%%%%%%%%%%%%%%%%%%%%%%%%%%%%%%%
    
    \begin{table}[ht]
        \caption{Performance of \Cref{alg:batch} compared to ERM on different datasets for neural network models.
    We consider $\varepsilon=\frac{1}{m}$ and $\varepsilon=\frac{1}{\sqrt{m}}$, with $K=\alpha\sqrt{m}$ and $\alpha\in\{0,0.4,0.6,0.8,1\}$.
    We plot the empirical risk $\Rfrak_{\S}(h)$ with its associated test risk $\Rfrak_{\D}(h)$.}
        \begin{adjustbox}{width=0.45\columnwidth}
        \begin{subtable}{0.5\textwidth}
            \centering
            \caption{$K=1$}
            \begin{tabular}{c|cc|cc}
\toprule
 & \multicolumn{2}{c}{\cref{alg:batch} {\small ($\frac{1}{m}$)}} & \multicolumn{2}{c}{\cref{alg:batch} {\small ($\frac{1}{\sqrt{m}}$)}} \\
Dataset & {\scriptsize $\Rfrak_{\S}(h)$} & {\scriptsize $\Rfrak_{\D}(h)$} & {\scriptsize $\Rfrak_{\S}(h)$} & {\scriptsize $\Rfrak_{\D}(h)$} \\
\midrule
\textsc{\footnotesize adult} & 0.207 & 0.207 & 0.248 & 0.248 \\
\textsc{\footnotesize fashionmnist} & 0.160 & 0.164 & 0.158 & 0.164 \\
\textsc{\footnotesize letter} & 0.258 & 0.269 & 0.268 & 0.280 \\
\textsc{\footnotesize mnist} & 0.116 & 0.123 & 0.085 & 0.096 \\
\textsc{\footnotesize mushrooms} & 0.000 & 0.000 & 0.000 & 0.001 \\
\textsc{\footnotesize nursery} & 0.705 & 0.720 & 0.720 & 0.736 \\
\textsc{\footnotesize pendigits} & 0.704 & 0.724 & 0.021 & 0.037 \\
\textsc{\footnotesize phishing} & 0.048 & 0.052 & 0.038 & 0.055 \\
\textsc{\footnotesize satimage} & 0.148 & 0.208 & 0.147 & 0.207 \\
\textsc{\footnotesize segmentation} & 0.141 & 0.176 & 0.248 & 0.385 \\
\textsc{\footnotesize sensorless} & 0.907 & 0.911 & 0.907 & 0.911 \\
\textsc{\footnotesize tictactoe} & 0.000 & 0.042 & 0.000 & 0.033 \\
\textsc{\footnotesize yeast} & 0.695 & 0.712 & 0.677 & 0.658 \\
\bottomrule
\end{tabular}

           \label{tab:sup_nn_batch_0}
        \end{subtable}\end{adjustbox}%
        \hspace{1.0cm}\begin{adjustbox}{width=0.45\columnwidth}
        \begin{subtable}{0.5\textwidth}
            \centering
            \caption{$K=0.4\sqrt{m}$}
            \begin{tabular}{c|cc|cc}
\toprule
 & \multicolumn{2}{c}{\cref{alg:batch} {\small ($\frac{1}{m}$)}} & \multicolumn{2}{c}{\cref{alg:batch} {\small ($\frac{1}{\sqrt{m}}$)}} \\
Dataset & {\scriptsize $\Rfrak_{\S}(h)$} & {\scriptsize $\Rfrak_{\D}(h)$} & {\scriptsize $\Rfrak_{\S}(h)$} & {\scriptsize $\Rfrak_{\D}(h)$} \\
\midrule
\textsc{\footnotesize adult} & 0.167 & 0.166 & 0.164 & 0.164 \\
\textsc{\footnotesize fashionmnist} & 0.160 & 0.164 & 0.156 & 0.160 \\
\textsc{\footnotesize letter} & 0.263 & 0.275 & 0.252 & 0.263 \\
\textsc{\footnotesize mnist} & 0.112 & 0.120 & 0.085 & 0.096 \\
\textsc{\footnotesize mushrooms} & 0.000 & 0.000 & 0.000 & 0.000 \\
\textsc{\footnotesize nursery} & 0.705 & 0.720 & 0.706 & 0.719 \\
\textsc{\footnotesize pendigits} & 0.011 & 0.025 & 0.010 & 0.022 \\
\textsc{\footnotesize phishing} & 0.043 & 0.053 & 0.041 & 0.052 \\
\textsc{\footnotesize satimage} & 0.147 & 0.178 & 0.145 & 0.174 \\
\textsc{\footnotesize segmentation} & 0.345 & 0.408 & 0.225 & 0.416 \\
\textsc{\footnotesize sensorless} & 0.075 & 0.078 & 0.074 & 0.077 \\
\textsc{\footnotesize tictactoe} & 0.000 & 0.031 & 0.000 & 0.019 \\
\textsc{\footnotesize yeast} & 0.450 & 0.480 & 0.695 & 0.712 \\
\bottomrule
\end{tabular}

            \label{tab:sup_nn_batch_4}
        \end{subtable}\end{adjustbox}
        \begin{adjustbox}{width=0.45\columnwidth}
        \begin{subtable}{0.5\textwidth}
            \centering
            \caption{$K=0.6\sqrt{m}$}
            \begin{tabular}{c|cc|cc}
\toprule
 & \multicolumn{2}{c}{\cref{alg:batch} {\small ($\frac{1}{m}$)}} & \multicolumn{2}{c}{\cref{alg:batch} {\small ($\frac{1}{\sqrt{m}}$)}} \\
Dataset & {\scriptsize $\Rfrak_{\S}(h)$} & {\scriptsize $\Rfrak_{\D}(h)$} & {\scriptsize $\Rfrak_{\S}(h)$} & {\scriptsize $\Rfrak_{\D}(h)$} \\
\midrule
\textsc{\footnotesize adult} & 0.165 & 0.163 & 0.165 & 0.164 \\
\textsc{\footnotesize fashionmnist} & 0.158 & 0.164 & 0.156 & 0.160 \\
\textsc{\footnotesize letter} & 0.259 & 0.275 & 0.260 & 0.267 \\
\textsc{\footnotesize mnist} & 0.112 & 0.121 & 0.084 & 0.094 \\
\textsc{\footnotesize mushrooms} & 0.000 & 0.000 & 0.000 & 0.000 \\
\textsc{\footnotesize nursery} & 0.706 & 0.719 & 0.706 & 0.719 \\
\textsc{\footnotesize pendigits} & 0.008 & 0.023 & 0.009 & 0.022 \\
\textsc{\footnotesize phishing} & 0.043 & 0.055 & 0.040 & 0.050 \\
\textsc{\footnotesize satimage} & 0.138 & 0.184 & 0.141 & 0.174 \\
\textsc{\footnotesize segmentation} & 0.577 & 0.845 & 0.145 & 0.309 \\
\textsc{\footnotesize sensorless} & 0.073 & 0.076 & 0.073 & 0.076 \\
\textsc{\footnotesize tictactoe} & 0.000 & 0.023 & 0.000 & 0.013 \\
\textsc{\footnotesize yeast} & 0.461 & 0.449 & 0.410 & 0.426 \\
\bottomrule
\end{tabular}

            \label{tab:sup_nn_batch_6}
        \end{subtable}\end{adjustbox}%
        \hspace{1.0cm}\begin{adjustbox}{width=0.45\columnwidth}
        \begin{subtable}{0.5\textwidth}
            \centering
            \caption{$K=0.8\sqrt{m}$}
            \begin{tabular}{c|cc|cc}
\toprule
 & \multicolumn{2}{c}{\cref{alg:batch} {\small ($\frac{1}{m}$)}} & \multicolumn{2}{c}{\cref{alg:batch} {\small ($\frac{1}{\sqrt{m}}$)}} \\
Dataset & {\scriptsize $\Rfrak_{\S}(h)$} & {\scriptsize $\Rfrak_{\D}(h)$} & {\scriptsize $\Rfrak_{\S}(h)$} & {\scriptsize $\Rfrak_{\D}(h)$} \\
\midrule
\textsc{\footnotesize adult} & 0.165 & 0.164 & 0.164 & 0.164 \\
\textsc{\footnotesize fashionmnist} & 0.158 & 0.163 & 0.156 & 0.161 \\
\textsc{\footnotesize letter} & 0.260 & 0.274 & 0.260 & 0.267 \\
\textsc{\footnotesize mnist} & 0.113 & 0.121 & 0.083 & 0.093 \\
\textsc{\footnotesize mushrooms} & 0.000 & 0.000 & 0.000 & 0.000 \\
\textsc{\footnotesize nursery} & 0.706 & 0.719 & 0.704 & 0.721 \\
\textsc{\footnotesize pendigits} & 0.011 & 0.026 & 0.008 & 0.020 \\
\textsc{\footnotesize phishing} & 0.042 & 0.054 & 0.048 & 0.063 \\
\textsc{\footnotesize satimage} & 0.136 & 0.174 & 0.128 & 0.183 \\
\textsc{\footnotesize segmentation} & 0.140 & 0.463 & 0.121 & 0.249 \\
\textsc{\footnotesize sensorless} & 0.075 & 0.079 & 0.074 & 0.077 \\
\textsc{\footnotesize tictactoe} & 0.392 & 0.301 & 0.000 & 0.008 \\
\textsc{\footnotesize yeast} & 0.394 & 0.422 & 0.686 & 0.671 \\
\bottomrule
\end{tabular}

            \label{tab:sup_nn_batch_8}
        \end{subtable}\end{adjustbox}
        \begin{adjustbox}{width=0.45\columnwidth}
        \begin{subtable}{0.5\textwidth}
            \centering
            \caption{$K=\sqrt{m}$}
            \begin{tabular}{c|cc|cc}
\toprule
 & \multicolumn{2}{c}{\cref{alg:batch} {\small ($\frac{1}{m}$)}} & \multicolumn{2}{c}{\cref{alg:batch} {\small ($\frac{1}{\sqrt{m}}$)}} \\
Dataset & {\scriptsize $\Rfrak_{\S}(h)$} & {\scriptsize $\Rfrak_{\D}(h)$} & {\scriptsize $\Rfrak_{\S}(h)$} & {\scriptsize $\Rfrak_{\D}(h)$} \\
\midrule
\textsc{\footnotesize adult} & 0.167 & 0.166 & 0.167 & 0.167 \\
\textsc{\footnotesize fashionmnist} & 0.159 & 0.164 & 0.157 & 0.162 \\
\textsc{\footnotesize letter} & 0.253 & 0.269 & 0.255 & 0.267 \\
\textsc{\footnotesize mnist} & 0.110 & 0.119 & 0.084 & 0.094 \\
\textsc{\footnotesize mushrooms} & 0.000 & 0.000 & 0.000 & 0.000 \\
\textsc{\footnotesize nursery} & 0.706 & 0.719 & 0.703 & 0.722 \\
\textsc{\footnotesize pendigits} & 0.012 & 0.026 & 0.012 & 0.024 \\
\textsc{\footnotesize phishing} & 0.042 & 0.050 & 0.039 & 0.053 \\
\textsc{\footnotesize satimage} & 0.137 & 0.176 & 0.125 & 0.169 \\
\textsc{\footnotesize segmentation} & 0.162 & 0.398 & 0.153 & 0.387 \\
\textsc{\footnotesize sensorless} & 0.075 & 0.077 & 0.073 & 0.076 \\
\textsc{\footnotesize tictactoe} & 0.000 & 0.035 & 0.000 & 0.008 \\
\textsc{\footnotesize yeast} & 0.415 & 0.435 & 0.418 & 0.450 \\
\bottomrule
\end{tabular}

            \label{tab:sup_nn_batch_10}
        \end{subtable}\end{adjustbox}%
        \hspace{1.0cm}\begin{adjustbox}{width=0.48\columnwidth}
        \begin{subtable}{0.5\textwidth}
            \centering
            \caption{ERM}
            \begin{tabular}{c|cc}
\toprule
Dataset & {\scriptsize $\Rfrak_{\S}(h)$} & {\scriptsize $\Rfrak_{\D}(h)$} \\
\midrule
\textsc{\footnotesize adult} & 0.165 & 0.163 \\
\textsc{\footnotesize fashionmnist} & 0.163 & 0.167 \\
\textsc{\footnotesize letter} & 0.258 & 0.270 \\
\textsc{\footnotesize mnist} & 0.119 & 0.127 \\
\textsc{\footnotesize mushrooms} & 0.000 & 0.000 \\
\textsc{\footnotesize nursery} & 0.706 & 0.719 \\
\textsc{\footnotesize pendigits} & 0.009 & 0.022 \\
\textsc{\footnotesize phishing} & 0.046 & 0.055 \\
\textsc{\footnotesize satimage} & 0.141 & 0.189 \\
\textsc{\footnotesize segmentation} & 0.174 & 0.389 \\
\textsc{\footnotesize sensorless} & 0.075 & 0.078 \\
\textsc{\footnotesize tictactoe} & 0.000 & 0.023 \\
\textsc{\footnotesize yeast} & 0.644 & 0.682 \\
\bottomrule
\end{tabular}

            \label{tab:sup_nn_batch_ERM}
        \end{subtable}\end{adjustbox}%
        \label{tab:expe-2}
    \end{table}
    
    %%%%%%%%%%%%%%%%%%%%%%%%%%%%%%%%%%%%%%%%%%%%%%%%%%%%%%%%%%%%%%%%%%%%%%%%%%%%%%%
    
    \begin{table}[ht]
        \caption{Performance of \Cref{alg:batch} compared to ERM on different datasets for linear models.
    We consider $\varepsilon=\frac{1}{m}$ and $\varepsilon=\frac{1}{\sqrt{m}}$, with $K=\alpha\sqrt{m}$ and $\alpha\in\{0,0.4,0.6,0.8,1\}$.
    We plot the empirical risk $\Rfrak_{\S}(h)$ with its associated test risk $\Rfrak_{\D}(h)$.}
        \begin{adjustbox}{width=0.45\columnwidth}
        \begin{subtable}{0.5\textwidth}
            \centering
            \caption{$K=1$}
            \begin{tabular}{c|cc|cc}
\toprule
 & \multicolumn{2}{c}{\cref{alg:batch} {\small ($\frac{1}{m}$)}} & \multicolumn{2}{c}{\cref{alg:batch} {\small ($\frac{1}{\sqrt{m}}$)}} \\
Dataset & {\scriptsize $\Rfrak_{\S}(h)$} & {\scriptsize $\Rfrak_{\D}(h)$} & {\scriptsize $\Rfrak_{\S}(h)$} & {\scriptsize $\Rfrak_{\D}(h)$} \\
\midrule
\textsc{\footnotesize adult} & 0.207 & 0.207 & 0.248 & 0.248 \\
\textsc{\footnotesize fashionmnist} & 0.142 & 0.155 & 0.126 & 0.149 \\
\textsc{\footnotesize letter} & 0.286 & 0.296 & 0.286 & 0.295 \\
\textsc{\footnotesize mnist} & 0.067 & 0.092 & 0.069 & 0.094 \\
\textsc{\footnotesize mushrooms} & 0.001 & 0.001 & 0.000 & 0.000 \\
\textsc{\footnotesize nursery} & 0.788 & 0.799 & 0.796 & 0.804 \\
\textsc{\footnotesize pendigits} & 0.049 & 0.060 & 0.047 & 0.057 \\
\textsc{\footnotesize phishing} & 0.063 & 0.065 & 0.057 & 0.062 \\
\textsc{\footnotesize satimage} & 0.142 & 0.202 & 0.136 & 0.199 \\
\textsc{\footnotesize segmentation} & 0.053 & 0.151 & 0.079 & 0.176 \\
\textsc{\footnotesize sensorless} & 0.907 & 0.911 & 0.907 & 0.911 \\
\textsc{\footnotesize tictactoe} & 0.013 & 0.021 & 0.013 & 0.021 \\
\textsc{\footnotesize yeast} & 0.702 & 0.720 & 0.693 & 0.687 \\
\bottomrule
\end{tabular}

           \label{tab:sup_linear_batch_0}
        \end{subtable}\end{adjustbox}%
        \hspace{1.0cm}\begin{adjustbox}{width=0.45\columnwidth}
        \begin{subtable}{0.5\textwidth}
            \centering
            \caption{$K=0.4\sqrt{m}$}
            \begin{tabular}{c|cc|cc}
\toprule
 & \multicolumn{2}{c}{\cref{alg:batch} {\small ($\frac{1}{m}$)}} & \multicolumn{2}{c}{\cref{alg:batch} {\small ($\frac{1}{\sqrt{m}}$)}} \\
Dataset & {\scriptsize $\Rfrak_{\S}(h)$} & {\scriptsize $\Rfrak_{\D}(h)$} & {\scriptsize $\Rfrak_{\S}(h)$} & {\scriptsize $\Rfrak_{\D}(h)$} \\
\midrule
\textsc{\footnotesize adult} & 0.166 & 0.167 & 0.166 & 0.167 \\
\textsc{\footnotesize fashionmnist} & 0.128 & 0.150 & 0.126 & 0.150 \\
\textsc{\footnotesize letter} & 0.285 & 0.296 & 0.286 & 0.297 \\
\textsc{\footnotesize mnist} & 0.069 & 0.089 & 0.067 & 0.093 \\
\textsc{\footnotesize mushrooms} & 0.001 & 0.001 & 0.001 & 0.001 \\
\textsc{\footnotesize nursery} & 0.760 & 0.778 & 0.769 & 0.781 \\
\textsc{\footnotesize pendigits} & 0.050 & 0.061 & 0.048 & 0.061 \\
\textsc{\footnotesize phishing} & 0.062 & 0.067 & 0.065 & 0.068 \\
\textsc{\footnotesize satimage} & 0.565 & 0.773 & 0.137 & 0.200 \\
\textsc{\footnotesize segmentation} & 0.058 & 0.212 & 0.177 & 0.382 \\
\textsc{\footnotesize sensorless} & 0.220 & 0.220 & 0.133 & 0.134 \\
\textsc{\footnotesize tictactoe} & 0.378 & 0.290 & 0.013 & 0.021 \\
\textsc{\footnotesize yeast} & 0.488 & 0.478 & 0.492 & 0.478 \\
\bottomrule
\end{tabular}

            \label{tab:sup_linear_batch_4}
        \end{subtable}\end{adjustbox}
        \begin{adjustbox}{width=0.45\columnwidth}
        \begin{subtable}{0.5\textwidth}
            \centering
            \caption{$K=0.6\sqrt{m}$}
            \begin{tabular}{c|cc|cc}
\toprule
 & \multicolumn{2}{c}{\cref{alg:batch} {\small ($\frac{1}{m}$)}} & \multicolumn{2}{c}{\cref{alg:batch} {\small ($\frac{1}{\sqrt{m}}$)}} \\
Dataset & {\scriptsize $\Rfrak_{\S}(h)$} & {\scriptsize $\Rfrak_{\D}(h)$} & {\scriptsize $\Rfrak_{\S}(h)$} & {\scriptsize $\Rfrak_{\D}(h)$} \\
\midrule
\textsc{\footnotesize adult} & 0.166 & 0.167 & 0.166 & 0.167 \\
\textsc{\footnotesize fashionmnist} & 0.127 & 0.147 & 0.127 & 0.150 \\
\textsc{\footnotesize letter} & 0.288 & 0.296 & 0.286 & 0.296 \\
\textsc{\footnotesize mnist} & 0.067 & 0.092 & 0.067 & 0.093 \\
\textsc{\footnotesize mushrooms} & 0.001 & 0.001 & 0.001 & 0.001 \\
\textsc{\footnotesize nursery} & 0.791 & 0.802 & 0.759 & 0.779 \\
\textsc{\footnotesize pendigits} & 0.048 & 0.061 & 0.047 & 0.059 \\
\textsc{\footnotesize phishing} & 0.062 & 0.067 & 0.064 & 0.068 \\
\textsc{\footnotesize satimage} & 0.146 & 0.202 & 0.137 & 0.199 \\
\textsc{\footnotesize segmentation} & 0.058 & 0.215 & 0.058 & 0.204 \\
\textsc{\footnotesize sensorless} & 0.129 & 0.130 & 0.130 & 0.130 \\
\textsc{\footnotesize tictactoe} & 0.013 & 0.021 & 0.013 & 0.021 \\
\textsc{\footnotesize yeast} & 0.477 & 0.461 & 0.478 & 0.464 \\
\bottomrule
\end{tabular}

            \label{tab:sup_linear_batch_6}
        \end{subtable}\end{adjustbox}%
        \hspace{1.0cm}\begin{adjustbox}{width=0.45\columnwidth}
        \begin{subtable}{0.5\textwidth}
            \centering
            \caption{$K=0.8\sqrt{m}$}
            \begin{tabular}{c|cc|cc}
\toprule
 & \multicolumn{2}{c}{\cref{alg:batch} {\small ($\frac{1}{m}$)}} & \multicolumn{2}{c}{\cref{alg:batch} {\small ($\frac{1}{\sqrt{m}}$)}} \\
Dataset & {\scriptsize $\Rfrak_{\S}(h)$} & {\scriptsize $\Rfrak_{\D}(h)$} & {\scriptsize $\Rfrak_{\S}(h)$} & {\scriptsize $\Rfrak_{\D}(h)$} \\
\midrule
\textsc{\footnotesize adult} & 0.166 & 0.167 & 0.166 & 0.167 \\
\textsc{\footnotesize fashionmnist} & 0.130 & 0.149 & 0.128 & 0.151 \\
\textsc{\footnotesize letter} & 0.285 & 0.296 & 0.288 & 0.297 \\
\textsc{\footnotesize mnist} & 0.067 & 0.091 & 0.067 & 0.093 \\
\textsc{\footnotesize mushrooms} & 0.001 & 0.001 & 0.001 & 0.001 \\
\textsc{\footnotesize nursery} & 0.771 & 0.787 & 0.758 & 0.778 \\
\textsc{\footnotesize pendigits} & 0.047 & 0.060 & 0.047 & 0.059 \\
\textsc{\footnotesize phishing} & 0.062 & 0.066 & 0.065 & 0.068 \\
\textsc{\footnotesize satimage} & 0.168 & 0.216 & 0.137 & 0.199 \\
\textsc{\footnotesize segmentation} & 0.053 & 0.212 & 0.052 & 0.204 \\
\textsc{\footnotesize sensorless} & 0.129 & 0.130 & 0.132 & 0.132 \\
\textsc{\footnotesize tictactoe} & 0.013 & 0.021 & 0.013 & 0.021 \\
\textsc{\footnotesize yeast} & 0.476 & 0.461 & 0.477 & 0.460 \\
\bottomrule
\end{tabular}

            \label{tab:sup_linear_batch_8}
        \end{subtable}\end{adjustbox}
        \begin{adjustbox}{width=0.45\columnwidth}
        \begin{subtable}{0.5\textwidth}
            \centering
            \caption{$K=\sqrt{m}$}
            \begin{tabular}{c|cc|cc}
\toprule
 & \multicolumn{2}{c}{\cref{alg:batch} {\small ($\frac{1}{m}$)}} & \multicolumn{2}{c}{\cref{alg:batch} {\small ($\frac{1}{\sqrt{m}}$)}} \\
Dataset & {\scriptsize $\Rfrak_{\S}(h)$} & {\scriptsize $\Rfrak_{\D}(h)$} & {\scriptsize $\Rfrak_{\S}(h)$} & {\scriptsize $\Rfrak_{\D}(h)$} \\
\midrule
\textsc{\footnotesize adult} & 0.166 & 0.167 & 0.166 & 0.167 \\
\textsc{\footnotesize fashionmnist} & 0.354 & 0.361 & 0.127 & 0.151 \\
\textsc{\footnotesize letter} & 0.287 & 0.296 & 0.288 & 0.298 \\
\textsc{\footnotesize mnist} & 0.068 & 0.092 & 0.065 & 0.092 \\
\textsc{\footnotesize mushrooms} & 0.001 & 0.001 & 0.001 & 0.001 \\
\textsc{\footnotesize nursery} & 0.795 & 0.805 & 0.796 & 0.805 \\
\textsc{\footnotesize pendigits} & 0.050 & 0.062 & 0.047 & 0.059 \\
\textsc{\footnotesize phishing} & 0.062 & 0.067 & 0.065 & 0.067 \\
\textsc{\footnotesize satimage} & 0.143 & 0.200 & 0.137 & 0.201 \\
\textsc{\footnotesize segmentation} & 0.055 & 0.210 & 0.055 & 0.212 \\
\textsc{\footnotesize sensorless} & 0.130 & 0.130 & 0.131 & 0.132 \\
\textsc{\footnotesize tictactoe} & 0.013 & 0.021 & 0.392 & 0.301 \\
\textsc{\footnotesize yeast} & 0.476 & 0.456 & 0.476 & 0.457 \\
\bottomrule
\end{tabular}

            \label{tab:sup_linear_batch_10}
        \end{subtable}\end{adjustbox}%
        \hspace{1.0cm}\begin{adjustbox}{width=0.48\columnwidth}
        \begin{subtable}{0.5\textwidth}
            \centering
            \caption{ERM}
            \begin{tabular}{c|cc}
\toprule
Dataset & {\scriptsize $\Rfrak_{\S}(h)$} & {\scriptsize $\Rfrak_{\D}(h)$} \\
\midrule
\textsc{\footnotesize adult} & 0.166 & 0.167 \\
\textsc{\footnotesize fashionmnist} & 0.139 & 0.153 \\
\textsc{\footnotesize letter} & 0.287 & 0.297 \\
\textsc{\footnotesize mnist} & 0.065 & 0.091 \\
\textsc{\footnotesize mushrooms} & 0.001 & 0.001 \\
\textsc{\footnotesize nursery} & 0.794 & 0.807 \\
\textsc{\footnotesize pendigits} & 0.052 & 0.064 \\
\textsc{\footnotesize phishing} & 0.064 & 0.067 \\
\textsc{\footnotesize satimage} & 0.148 & 0.209 \\
\textsc{\footnotesize segmentation} & 0.087 & 0.232 \\
\textsc{\footnotesize sensorless} & 0.134 & 0.136 \\
\textsc{\footnotesize tictactoe} & 0.228 & 0.238 \\
\textsc{\footnotesize yeast} & 0.470 & 0.427 \\
\bottomrule
\end{tabular}

            \label{tab:sup_linear_batch_ERM}
        \end{subtable}\end{adjustbox}%
        \label{tab:expe-3}
    \end{table}
    
    %%%%%%%%%%%%%%%%%%%%%%%%%%%%%%%%%%%%%%%%%%%%%%%%%%%%%%%%%%%%%%%%%%%%%%%%%%%%%%%
    
    \begin{table}[ht]
        \caption{Performance of ERM with weight decay (with the L2 regularisation) for linear and neural network models.}
        \centering
        \begin{adjustbox}{width=0.48\columnwidth}
    \begin{subtable}{0.5\textwidth}
        \centering
        \caption{Linear}
        \begin{tabular}{c|cc|cc}
\toprule
  & \multicolumn{2}{c}{L2 Reg. {\small ($\frac{1}{m}$)}} & \multicolumn{2}{c}{L2 Reg. {\small ($\frac{1}{\sqrt{m}}$)}} \\
Dataset & {\scriptsize $\Rfrak_{\S}(h)$} & {\scriptsize $\Rfrak_{\D}(h)$} & {\scriptsize $\Rfrak_{\S}(h)$} & {\scriptsize $\Rfrak_{\D}(h)$} \\
\midrule
\textsc{\footnotesize adult} & 0.207 & 0.207 & 0.248 & 0.248 \\
\textsc{\footnotesize fashionmnist} & 0.141 & 0.149 & 0.127 & 0.150 \\
\textsc{\footnotesize letter} & 0.285 & 0.295 & 0.285 & 0.296 \\
\textsc{\footnotesize mnist} & 0.067 & 0.092 & 0.066 & 0.092 \\
\textsc{\footnotesize mushrooms} & 0.001 & 0.001 & 0.000 & 0.000 \\
\textsc{\footnotesize nursery} & 0.788 & 0.799 & 0.796 & 0.804 \\
\textsc{\footnotesize pendigits} & 0.049 & 0.060 & 0.047 & 0.057 \\
\textsc{\footnotesize phishing} & 0.063 & 0.065 & 0.057 & 0.062 \\
\textsc{\footnotesize satimage} & 0.144 & 0.203 & 0.138 & 0.200 \\
\textsc{\footnotesize segmentation} & 0.058 & 0.157 & 0.075 & 0.177 \\
\textsc{\footnotesize sensorless} & 0.907 & 0.911 & 0.907 & 0.911 \\
\textsc{\footnotesize tictactoe} & 0.013 & 0.021 & 0.013 & 0.021 \\
\textsc{\footnotesize yeast} & 0.702 & 0.720 & 0.693 & 0.687 \\
\bottomrule
\end{tabular}
        \label{tab:sup_linear_batch_L2}
    \end{subtable}\end{adjustbox}
    \hfill
    \begin{adjustbox}{width=0.48\columnwidth}
    \begin{subtable}{0.5\textwidth}
        \centering
        \caption{NN}
        \begin{tabular}{c|cc|cc}
\toprule
  & \multicolumn{2}{c}{L2 Reg. {\small ($\frac{1}{m}$)}} & \multicolumn{2}{c}{L2 Reg. {\small ($\frac{1}{\sqrt{m}}$)}} \\
Dataset & {\scriptsize $\Rfrak_{\S}(h)$} & {\scriptsize $\Rfrak_{\D}(h)$} & {\scriptsize $\Rfrak_{\S}(h)$} & {\scriptsize $\Rfrak_{\D}(h)$} \\
\midrule
\textsc{\footnotesize adult} & 0.207 & 0.207 & 0.248 & 0.248 \\
\textsc{\footnotesize fashionmnist} & 0.160 & 0.166 & 0.159 & 0.164 \\
\textsc{\footnotesize letter} & 0.261 & 0.275 & 0.256 & 0.269 \\
\textsc{\footnotesize mnist} & 0.116 & 0.125 & 0.084 & 0.095 \\
\textsc{\footnotesize mushrooms} & 0.000 & 0.000 & 0.000 & 0.000 \\
\textsc{\footnotesize nursery} & 0.704 & 0.721 & 0.770 & 0.788 \\
\textsc{\footnotesize pendigits} & 0.009 & 0.022 & 0.012 & 0.026 \\
\textsc{\footnotesize phishing} & 0.042 & 0.050 & 0.054 & 0.059 \\
\textsc{\footnotesize satimage} & 0.150 & 0.215 & 0.143 & 0.205 \\
\textsc{\footnotesize segmentation} & 0.141 & 0.216 & 0.198 & 0.371 \\
\textsc{\footnotesize sensorless} & 0.907 & 0.911 & 0.907 & 0.911 \\
\textsc{\footnotesize tictactoe} & 0.000 & 0.046 & 0.000 & 0.021 \\
\textsc{\footnotesize yeast} & 0.662 & 0.674 & 0.693 & 0.683 \\
\bottomrule
\end{tabular}
        \label{tab:sup_nn_batch_L2}
    \end{subtable}\end{adjustbox}
    \label{tab:expe-4}
    \end{table}
\end{noaddcontents}