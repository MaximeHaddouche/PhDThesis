\chapter{Appendix of Chapter~\ref{chap:dis-pra}}
\label{ap:dis-pra}

\begin{noaddcontents}
\section{Proof of \Cref{chap:dis-pra:theorem:disintegrated}}
\label{chap:dis-pra:sec:proof-disintegrated}

\theoremdisintegrated*
\begin{proof}
For any sample $\S\in(\X{\times}\Y)^\m$, prior $\P\in\M^{*}(\H)$ and deterministic algorithm $A$ fixed apriori, let $\AQ=A(\S, \P)$ the distribution obtained from the algorithm $A$.
Note that $\varphi(\h,\!\S)$ is a strictly-positive random variable.
Hence, from \textsc{Markov}'s inequality (\Cref{ap:tools:theorem:first-markov}), we have
\begin{align*}
&\PP_{\h\sim \AQ}\LB\varphi(\h,\!\S)\le \frac{2}{\delta}\EE_{\h'{\sim} \AQ}\varphi(\h'\!, \S)\RB\ge 1{-}\tfrac{\delta}{2}\\
\iff &\EE_{\h\sim \AQ}\indic\LB \varphi(\h,\!\S)\le \frac{2}{\delta}\EE_{\h'{\sim} \AQ}\varphi(\h'\!, \S) \RB\ge 1{-}\tfrac{\delta}{2}.
\end{align*}
Taking the expectation over $\S\sim\D^{\m}$ to both sides of the inequality gives 
\begin{align*}
&\EE_{\S\sim\D^{\m}}\EE_{\h\sim \AQ}\indic\LB \varphi(\h,\!\S)\le \frac{2}{\delta}\EE_{\h'{\sim} \AQ}\varphi(\h'\!, \S) \RB\ge 1-\tfrac{\delta}{2}\\
\iff &\PP_{\S\sim\D^{\m},\h\sim \AQ}\LB \varphi(\h,\!\S)\le \frac{2}{\delta}\EE_{\h'{\sim} \AQ}\varphi(\h'\!, \S) \RB\ge 1-\tfrac{\delta}{2}.
\end{align*}
Since both sides of the inequality are strictly positive, we can take the logarithm and multiply by $\frac{\lambda}{\lambda-1}>0$ to obtain
\begin{align*}
\PP_{\S\sim\D^{\m},\h\sim \AQ}\LB\frac{\lambda}{\lambda-1}\ln\LP\varphi(\h,\!\S)\RP \le \frac{\lambda}{\lambda-1}\ln\LP\frac{2}{\delta}\EE_{\h'{\sim} \AQ}\varphi(\h'\!, \S)\RP\RB\ge 1-\tfrac{\delta}{2}.
\end{align*}
We develop the right-hand side of the inequality and take the expectation of the hypothesis over the prior distribution $\P$. 
We have for all prior $\P\in\M^{*}(\H)$,
\begin{align*}
 \frac{\lambda}{\lambda-1}\ln\LP\frac{2}{\delta}\EE_{\h'{\sim} \AQ}\varphi(\h'\!, \S)\RP &= \frac{\lambda}{\lambda-1}\ln\LP\frac{2}{\delta}\EE_{\h'{\sim} \AQ}\frac{\AQ(\h')\P(\h')}{\P(\h')\AQ(\h')}\varphi(\h'\!, \S)\RP\\
&= \frac{\lambda}{\lambda-1}\ln\LP\frac{2}{\delta}\EE_{\h'{\sim}\P}\frac{\AQ(\h')}{\P(\h')}\varphi(\h'\!, \S)\RP,
\end{align*}
Remark that $\frac{1}{r}+\frac{1}{s}=1$ with $r=\lambda$ and $s=\frac{\lambda}{\lambda-1}$. Hence, we can apply \textsc{Hölder}'s inequality (\Cref{ap:tools:theorem:holder}):
\begin{align*}
    \EE_{\h'{\sim}\P}\frac{\AQ(\h')}{\P(\h')}\varphi(\h'\!, \S) \le \LB\EE_{\h'{\sim}\P}\Bigg(\Bigg[\frac{\AQ(\h')}{\P(\h')}\Bigg]^{\lambda}\Bigg)\RB^{\frac{1}{\lambda}}\LB\EE_{\h'{\sim}\P}\LP\varphi(\h'\!, \S)^{\frac{\lambda}{\lambda-1}}\RP\RB^{\frac{\lambda-1}{\lambda}}.
\end{align*}
Then, since both sides of the inequality are strictly positive, we take the logarithm; add $\ln(\tfrac{2}{\delta})$ and multiply by $\frac{\lambda}{\lambda-1}>0$ to both sides of the inequality, to obtain
\begin{align*}
    &\frac{\lambda}{\lambda{-}1}\ln\!\LP\frac{2}{\delta}\EE_{\h'{\sim}\P}\frac{\AQ(\h')}{\P(\h')}\varphi(\h'\!, \S)\RP\\
    \le &\frac{\lambda}{\lambda{-}1}\ln\!\LP\frac{2}{\delta}\!\LB\EE_{\h'{\sim}\P}\LP\Bigg[\frac{\AQ(\h')}{\P(\h')}\Bigg]^{\lambda}\RP\RB^{\frac{1}{\lambda}}\!\LB\EE_{\h'{\sim}\P}\LP\varphi(\h'\!, \S)^{\frac{\lambda}{\lambda-1}}\RP\RB^{\frac{\lambda-1}{\lambda}}\RP\\
    = &\frac{1}{\lambda{-}1}\ln\!\LP\EE_{\h'{\sim}\P}\!\LP\Bigg[\frac{\AQ(\h')}{\P(\h')}\Bigg]^{\lambda}\RP\RP+
    \frac{\lambda}{\lambda{-}1}\ln\frac{2}{\delta} +\ln\!\LP\EE_{\h'{\sim}\P}\LP\varphi(\h'\!, \S)^{\frac{\lambda}{\lambda-1}}\RP\!\RP\\
    = &\Renyi_{\lambda}(\AQ\|\P) + \frac{\lambda}{\lambda{-}1}\ln\frac{2}{\delta} + \ln\LP\EE_{\h'{\sim}\P}\LP\varphi(\h'\!, \S)^{\frac{\lambda}{\lambda-1}}\RP\RP\!.
\end{align*}
From this inequality, we can deduce that
\begin{align}
    \PP_{\S\sim\D^{\m},\h\sim \AQ}\Big[\forall \P\in\M^{*}(\H), &\frac{\lambda}{\lambda-1}\ln\LP\varphi(\h,\!\S)\RP \le
    \Renyi_{\lambda}(\AQ\|\P)\nonumber\\
    &+ \frac{\lambda}{\lambda{-}1}\ln\frac{2}{\delta} +\ln\LP\EE_{\h'{\sim}\P}\LP\varphi(\h'\!, \S)^{\frac{\lambda}{\lambda-1}}\RP\RP\Big]\ge 1-\tfrac{\delta}{2}.
    \label{chap:dis-pra:eq:disintegrated-proof-1}
\end{align}
Given a prior $\P\in\M^{*}(\H)$, note that $\EE_{\h'{\sim}\P}\varphi(\h'\!, \S)^{\frac{\lambda}{\lambda-1}}$ is a strictly positive random variable. 
Hence, we apply \textsc{Markov}'s inequality (\Cref{ap:tools:theorem:first-markov}) to have
\begin{align*}
    \PP_{\S\sim\D^{\m}}\LB\EE_{\h'{\sim}\P}\LP\varphi(\h'\!, \S)^{\frac{\lambda}{\lambda-1}}\RP \le \frac{2}{\delta}\EE_{\S'{\sim}\D^{\m}}\EE_{\h'{\sim}\P}\LP\varphi(\h'\!, \S')^{\frac{\lambda}{\lambda-1}}\RP\RB\ge 1-\tfrac{\delta}{2}.
\end{align*}
Since the inequality does not depend on the random variable $\h\sim \AQ$, we have
\begin{align*}
    &\PP_{\S\sim\D^{\m}}\!\LB\EE_{\h'{\sim}\P}\LP\varphi(\h'\!, \S)^{\!\frac{\lambda}{\lambda-1}}\RP \!\le\! \frac{2}{\delta}\EE_{\S'{\sim}\D^{\m}}\EE_{\h'{\sim}\P}\LP\varphi(\h'\!, \S')^{\!\frac{\lambda}{\lambda-1}}\RP\!\RB\\
    = &\EE_{\S\sim\D^{\m}}\!\indic\!\LB\EE_{\h'{\sim}\P}\!\LP\varphi(\h'\!, \S)^{\!\frac{\lambda}{\lambda-1}}\RP \!\le\! \frac{2}{\delta}\EE_{\S'{\sim}\D^{\m}}\EE_{\h'{\sim}\P}\LP\varphi(\h'\!, \S')^{\!\frac{\lambda}{\lambda-1}}\RP\RB\\
    = &\EE_{\S\sim\D^{\m}}\EE_{\h\sim \AQ}\!\indic\!\LB\EE_{\h'{\sim}\P}\!\LP\varphi(\h'\!, \S)^{\!\frac{\lambda}{\lambda-1}}\RP \!\le\! \frac{2}{\delta}\EE_{\S'{\sim}\D^{\m}}\EE_{\h'{\sim}\P}\LP\varphi(\h'\!, \S')^{\!\frac{\lambda}{\lambda-1}}\RP\RB\\
    = &\PP_{\S\sim\D^{\m},\h\sim \AQ}\!\LB\EE_{\h'{\sim}\P}\LP\varphi(\h'\!, \S)^{\!\frac{\lambda}{\lambda-1}}\RP \!\le\! \frac{2}{\delta}\EE_{\S'{\sim}\D^{\m}}\EE_{\h'{\sim}\P}\LP\varphi(\h'\!, \S')^{\!\frac{\lambda}{\lambda-1}}\RP\RB\!.
\end{align*}
Since both sides of the inequality are strictly positive, we take the logarithm to both sides of the inequality, and we add $\frac{\lambda}{\lambda-1}\ln\frac{2}{\delta}$ to have
\begin{align}
    &\PP_{\S\sim\D^{\m},\h\sim \AQ}\LB\EE_{\h'{\sim}\P}\LP\varphi(\h'\!, \S)^{\frac{\lambda}{\lambda-1}}\RP \le \frac{2}{\delta}\EE_{\S'{\sim}\D^{\m}}\EE_{\h'{\sim}\P}\LP\varphi(\h'\!, \S')^{\frac{\lambda}{\lambda-1}}\RP\RB\ge 1-\tfrac{\delta}{2} \hspace{0.1cm}\iff\nonumber\\
    &\PP_{\S\sim\D^{\m},\h\sim \AQ}\Bigg[\frac{\lambda}{\lambda{-}1}\ln\frac{2}{\delta}+\ln\LP\EE_{\h'{\sim}\P}\LP\varphi(\h'\!, \S)^{\frac{\lambda}{\lambda-1}}\RP\RP
    \le\frac{2\lambda-1}{\lambda{-}1}\ln\frac{2}{\delta}\nonumber\\
    &\hspace{2cm}+\ln\!\LP\EE_{\S'{\sim}\D^{\m}}\EE_{\h'{\sim}\P}\LP\varphi(\h'\!, \S')^{\frac{\lambda}{\lambda-1}}\RP\RP\Bigg]\ge 1{-}\tfrac{\delta}{2}.
    \label{chap:dis-pra:eq:disintegrated-proof-2}
\end{align}
Combining \Cref{chap:dis-pra:eq:disintegrated-proof-1,chap:dis-pra:eq:disintegrated-proof-2} with a union bound gives us the desired result.
\end{proof}

\section{Proof of \Cref{chap:dis-pra:corollary:disintegrated}}
\label{chap:dis-pra:sec:proof-corollary-disintegrated}

\corollarydisintegrated*
\begin{proof} 
Starting from \Cref{chap:dis-pra:theorem:disintegrated} and rearranging, we have
\begin{align*}
    \PP_{\S\sim\D^{\m},\h\sim \AQ}\!\Bigg[\! &\ln\LP\varphi(\h,\!\S)\RP \le {\frac{2\lambda{-}1}{\lambda}}\ln\frac{2}{\delta}\, +\frac{\lambda{-}1}{\lambda}\Renyi_{\lambda}(\AQ\|\P)\\
    &+ \ln\LP\LB\EE_{\S'{\sim}\D^{\m}}\EE_{\h'{\sim}\P}\LP\varphi(\h'\!, \S')^{\frac{\lambda}{\lambda{-}1}}\RP\RB^{\frac{\lambda{-}1}{\lambda}}\RP \!\Bigg]\!\!\ge\! 1{-}\delta.
\end{align*}
Then, we will prove the case when $\lambda\rightarrow 1$ and $\lambda\rightarrow +\infty$ separately.

\paragraph{When $\lambda\rightarrow 1$.}
We have $\lim_{\lambda\rightarrow 1^+}\frac{2\lambda{-}1}{\lambda}\ln\!\frac{2}{\delta}{=}\ln\frac{2}{\delta}$ and $\lim_{\lambda\rightarrow 1^+}\frac{\lambda-1}{\lambda}\Renyi_{\lambda}(\AQ\|\P){=}0$.\\
Furthermore, note that 
\begin{align*}
    \|\varphi\|_{\frac{\lambda}{\lambda{-}1}} = \LB\EE_{\S'{\sim}\D^{\m}}\EE_{\h'{\sim}\P}\LP\vert\varphi(\h'\!, \S')\vert^{\frac{\lambda}{\lambda{-}1}}\RP\RB^{\frac{\lambda{-}1}{\lambda}} = \LB\EE_{\S'{\sim}\D^{\m}}\EE_{\h'{\sim}\P}\LP\varphi(\h'\!, \S')^{\frac{\lambda}{\lambda{-}1}}\RP\RB^{\frac{\lambda{-}1}{\lambda}}
\end{align*}
is the $L^{\frac{\lambda}{\lambda{-}1}}$-norm of the function $\varphi: \H\times(\X{\times}\Y)^\m \rightarrow \Rbb_{+}^*$, where $\lim_{\lambda\rightarrow 1} \|\varphi\|_{\frac{\lambda}{\lambda{-}1}} = \lim_{\lambda'\rightarrow +\infty} \|\varphi\|_{\lambda'}$ (since we have $\lim_{\lambda\rightarrow 1^+}\frac{\lambda}{\lambda{-}1} = (\lim_{\lambda\rightarrow1}\lambda)(\lim_{\lambda\rightarrow1}\frac{1}{\lambda{-}1}) = +\infty$).
Then, it is well known that
\begin{align*}
    \|\varphi\|_{\infty}= \lim_{\lambda'\rightarrow+\infty}\|\varphi\|_{\lambda'} = \esssup_{\S'\in(\X{\times}\Y), \h'\in\H}\varphi(\h'{,} \S').
\end{align*}
Hence, we have 
\begin{align*}
    &\lim_{\lambda\rightarrow1} \ln\LP\LB\EE_{\S'{\sim}\D^{\m}}\EE_{\h'{\sim}\P}\LP\varphi(\h'\!, \S')^{\frac{\lambda}{\lambda{-}1}}\RP\RB^{\frac{\lambda{-}1}{\lambda}}\RP\\
    = &\ln\LP \lim_{\lambda\rightarrow1} \LB\EE_{\S'{\sim}\D^{\m}}\EE_{\h'{\sim}\P}\LP\varphi(\h'\!, \S')^{\frac{\lambda}{\lambda{-}1}}\RP\RB^{\frac{\lambda{-}1}{\lambda}}\RP\\
    = &\ln\LP \lim_{\lambda\rightarrow1} \| \varphi\|_{\frac{\lambda}{\lambda-1}}\RP = \ln\LP \lim_{\lambda'\rightarrow+\infty} \| \varphi\|_{\lambda'}\RP \\
    = &\ln\LP \| \varphi\|_{\infty}\RP = \ln\LP \esssup_{\S'\in(\X{\times}\Y), \h'\in\H}\varphi(\h'{,} \S') \RP.
\end{align*}
Finally, we can deduce that 
\begin{align*}
    &\lim_{\lambda\rightarrow 1}\LB{\frac{2\lambda{-}1}{\lambda}}\ln\frac{2}{\delta}\, +\frac{\lambda{-}1}{\lambda}\Renyi_{\lambda}(\AQ\|\P){+} \ln\LP\LB\EE_{\S'{\sim}\D^{\m}}\EE_{\h'{\sim}\P}\LP\varphi(\h'\!, \S')^{\frac{\lambda}{\lambda{-}1}}\RP\RB^{\frac{\lambda{-}1}{\lambda}}\RP\RB\\
    = &\ln\frac{2}{\delta} + \ln\left[\esssup_{\S'\in(\X{\times}\Y), \h'\in\H}\varphi(\h'{,} \S')\right].
\end{align*}

\paragraph{When \mbox{$\lambda\rightarrow +\infty$}.} First, we have  
$\lim_{\lambda\rightarrow+\infty}{\frac{2\lambda{-}1}{\lambda}}\ln\frac{2}{\delta} = \ln\frac{2}{\delta}\LB2 -\lim_{\lambda\rightarrow+\infty}\frac{1}{\lambda} \RB = 2\ln\frac{2}{\delta}= \ln\frac{4}{\delta^2}$ and $\lim_{\lambda\rightarrow +\infty} \|\varphi\|_{\frac{\lambda}{\lambda{-}1}} = \lim_{\lambda'\rightarrow 1} \|\varphi\|_{\lambda'} = \|\varphi\|_1$ (since $\lim_{\lambda\rightarrow +\infty}\frac{\lambda}{\lambda-1}\!= \lim_{\lambda\rightarrow +\infty} \frac{1}{1-\frac{1}{\lambda}}=1$).
Hence, we have 
\begin{align*}
    &\lim_{\lambda\rightarrow+\infty} \ln\LP\LB\EE_{\S'{\sim}\D^{\m}}\EE_{\h'{\sim}\P}\LP\varphi(\h'\!, \S')^{\frac{\lambda}{\lambda{-}1}}\RP\RB^{\frac{\lambda{-}1}{\lambda}}\RP\\
    = &\ln\LP \lim_{\lambda\rightarrow+\infty} \LB\EE_{\S'{\sim}\D^{\m}}\EE_{\h'{\sim}\P}\LP\varphi(\h'\!, \S')^{\frac{\lambda}{\lambda{-}1}}\RP\RB^{\frac{\lambda{-}1}{\lambda}}\RP\\
    = &\ln\LP \lim_{\lambda\rightarrow+\infty} \| \varphi\|_{\frac{\lambda}{\lambda-1}}\RP = \ln\LP \lim_{\lambda'\rightarrow1} \| \varphi\|_{\lambda'}\RP\\
    = &\ln\LP \| \varphi\|_{1}\RP = \ln\LP \EE_{\S'{\sim}\D^{\m}}\EE_{\h'{\sim}\P}\varphi(\h'\!, \S') \RP.
\end{align*}
Moreover, by rearranging the terms in $\frac{\lambda{-}1}{\lambda}\Renyi_{\lambda}(\AQ\|\P)$, we have
\begin{align*}
\frac{\lambda{-}1}{\lambda}\Renyi_{\lambda}(\AQ\|\P) = &\frac{1}{\lambda}\ln\!\LP \EE_{\h{\sim}\P}\!\LP\LB\!\frac{ \AQ(\h)}{\P(\h)}\RB^{\!\lambda}\RP\RP = \ln\!\LP \LB\EE_{\h{\sim}\P}\!\LP\LB\!\frac{ \AQ(\h)}{\P(\h)}\RB^{\!\lambda}\RP\RB^{\frac{1}{\lambda}}\RP\\
= &\ln\!\LP \LB\EE_{\h{\sim}\P}\LP\gamma(\h)^{\lambda}\RP\RB^{\frac{1}{\lambda}}\RP = \ln\!\LP \| \gamma\|_{\lambda}\RP,
\end{align*}
where $\| \gamma\|_{\lambda}$ is the $L^{\lambda}$-norm of the function $\gamma$ defined as $\gamma(\h)=\tfrac{\AQ(\h)}{\P(\h)}$.
We have
\begin{align*}
    \lim_{\lambda\rightarrow +\infty}  \frac{\lambda{-}1}{\lambda}\Renyi_{\lambda}(\AQ\|\P) =& \lim_{\lambda\rightarrow +\infty}  \ln\!\LP \| \gamma\|_{\lambda}\RP = \ln\LP\lim_{\lambda\rightarrow +\infty} \|\gamma\|_{\lambda}\RP\\
    =& \ln\LP \|\gamma\|_{\infty}\RP = \ln\LP\esssup_{\h\in\H}\gamma(\h)\RP = \ln\LP\esssup_{\h\in\H}\frac{\AQ(\h)}{\P(\h)}\RP.
\end{align*}
Finally, we can deduce that 
\begin{align*}
    &\lim_{\lambda\rightarrow +\infty}\LB{\frac{2\lambda{-}1}{\lambda}}\ln\frac{2}{\delta}\, +\frac{\lambda{-}1}{\lambda}\Renyi_{\lambda}(\AQ\|\P){+} \ln\LP\LB\EE_{\S'{\sim}\D^{\m}}\EE_{\h'{\sim}\P}\LP\varphi(\h'\!, \S')^{\frac{\lambda}{\lambda{-}1}}\RP\RB^{\frac{\lambda{-}1}{\lambda}}\RP\RB\\ 
    = &\ln{\displaystyle\esssup_{\h'\in\H}}\,\frac{\AQ(\h')}{\P(\h')}{+}\ln\!\Big[\frac{4}{\delta^2} {\displaystyle \EE_{\S'{\sim}\D^{\m}}\EE_{\h'{\sim}\P}\varphi(\h'{,}\S')}\Big].
\end{align*}
\end{proof}

\section{Proof of \Cref{chap:dis-pra:theorem:disintegrated-lambda}}
\label{chap:dis-pra:sec:proof-disintegrated-lambda}

For the sake of completeness, we first prove an upper bound on $\sqrt{ab}$ \citep[see, \eg,][]{ThiemannIgelWintenbergerSeldin2017}. 
\begin{lemma} For any $a>0, b>0$, we have
\begin{align*}
    \sqrt{\tfrac{a}{b}} = \argmin_{\lambda>0}\LP\frac{a}{\lambda}+\lambda b\RP,\ &\text{and}\ \ 2\sqrt{ab} = \min_{\lambda>0}\LP\frac{a}{\lambda}+\lambda b\RP,\\
    &\ \text{and}\ \ \forall\lambda>0, \sqrt{ab} \le \frac{1}{2}\LP\frac{a}{\lambda}+\lambda b\RP.
\end{align*}
\label{chap:dis-pra:lemma:sqrt}
\end{lemma}
\begin{proof}
Let $f(\lambda) = \LP \tfrac{a}{\lambda}+\lambda b \RP$. The first derivative of $f$ {\it w.r.t.} $\lambda$ is
\begin{align*}
    \frac{\partial f}{\partial\lambda}(\lambda) = \LP b-\frac{a}{\lambda^2}\RP.
\end{align*}
Moreover, from the derivative we can deduce that  we have $\frac{\partial f}{\partial\lambda}(\lambda) < 0 \iff \lambda \in (0, \sqrt{\frac{a}{b}})$, and $\frac{\partial f}{\partial\lambda}(\lambda) > 0 \iff \lambda > \sqrt{\frac{a}{b}}$ and $\frac{\partial f}{\partial\lambda}(\lambda) = 0 \iff \lambda = \sqrt{\frac{a}{b}}$.
It implies that the function is strictly decreasing on $\lambda \in (0, \sqrt{\frac{a}{b}})$, strictly increasing for $\lambda > \sqrt{\frac{a}{b}}$ and admit a unique minimum at $\lambda^* = \sqrt{\frac{a}{b}}$.
Additionally, $f(\lambda^*)=2\sqrt{ab}$ which proves the claim.
\end{proof}
We can now prove \Cref{chap:dis-pra:theorem:disintegrated-lambda} with \Cref{chap:dis-pra:lemma:sqrt}.\\

\theoremdisintegratedlambda*
\begin{proof}
The proof is similar to the one of \Cref{chap:dis-pra:theorem:disintegrated}. 
Since $\varphi(\h,\!\S)$ is a strictly positive random variable, from \textsc{Markov}'s inequality (\Cref{ap:tools:theorem:first-markov}), we have
\begin{align*}
&\PP_{\h\sim \AQ}\!\!\LB\varphi(\h,\!\S)\le \frac{2}{\delta}\EE_{\h'{\sim} \AQ}\varphi(\h'\!, \S)\RB\!\!\ge 1-\tfrac{\delta}{2}\\
\iff &\EE_{\h\sim \AQ}\!\!\indic\LB\varphi(\h,\!\S)\le \frac{2}{\delta}\EE_{\h'{\sim} \AQ}\varphi(\h'\!, \S)\RB\!\!\ge 1-\tfrac{\delta}{2}.
\end{align*}
Taking the expectation over $\S\sim\D^{\m}$ to both sides of the inequality gives
\begin{align*}
    &\EE_{\S\sim\D^{\m}}\EE_{\h\sim \AQ}\!\!\indic\LB\varphi(\h,\!\S)\le \frac{2}{\delta}\EE_{\h'{\sim} \AQ}\varphi(\h'\!, \S)\RB\!\!\ge 1-\tfrac{\delta}{2}\\
    \iff &\PP_{\S\sim\D^{\m},\h\sim \AQ}\!\!\LB\varphi(\h,\!\S) \le \frac{2}{\delta}\EE_{\h'{\sim} \AQ}\varphi(\h'\!, \S)\RB\!\!\ge 1-\tfrac{\delta}{2}.
\end{align*}
Using \Cref{chap:dis-pra:lemma:sqrt} with $a=\tfrac{4}{\delta^2}\varphi(\h'\!, \S)^2$ and $b=\tfrac{ \AQ(\h')^2}{\P(\h')^2}$, we have for all prior $\P{\in}\M^{*}(\H)$
\begin{align*}
    \forall\lambda{>}0,\quad \frac{2}{\delta}\!\EE_{\h'{\sim} \AQ}\varphi(\h'\!, \S) &= \EE_{\h'{\sim}\P}\sqrt{\frac{ \AQ(\h')^2}{\P(\h')^2}\frac{4}{\delta^2}\varphi(\h'\!, \S)^2}\\
    &\le \frac{1}{2}\!\LB\lambda\EE_{\h'{\sim}\P}\!\!\LP\frac{ \AQ(\h')}{\P(\h')}\RP^2\!\!{+} \frac{4}{\lambda\delta^2}\EE_{\h'{\sim}\P}\LP\varphi(\h'\!, \S)^2\RP\RB\!.
\end{align*}
Then, since both sides of the inequality are strictly positive, we take the logarithm to obtain
\begin{align*}
    \forall\lambda{>}0, \ln\!\LP\frac{2}{\delta}\EE_{\h'{\sim} \AQ}\varphi(\h'\!, \S)\RP &\le \ln\!\LP\frac{1}{2}\!\LB\lambda\EE_{\h'{\sim}\P}\LP\frac{ \AQ(\h')}{\P(\h')}\RP^2\!\!{+} \frac{4}{\lambda\delta^2}\EE_{\h'{\sim}\P}\LP\varphi(\h'\!, \S)^2\RP\RB\RP\\
    &=  \ln\!\LP\frac{1}{2}\!\LB\lambda \exp(\Renyi_{2}( \AQ\|\P)){+} \frac{4}{\lambda\delta^2}\EE_{\h'{\sim}\P}\LP\varphi(\h'\!, \S)^2\RP\RB\RP.
\end{align*}
Hence, we can deduce that
\begin{align}
    \PP_{\S\sim\D^{\m},\h\sim \AQ}\!\!\Bigg[\forall\P{\in}\M^{*}(\H),\ &\forall\lambda>0, \ln\LP\varphi(\h,\!\S)\RP\nonumber\\
    &\le \ln\LP\frac{1}{2}\LB\lambda e^{\Renyi_{2}( \AQ\|\P)}+ \frac{4}{\lambda\delta^2}\EE_{\h'{\sim}\P}\LP\varphi(\h'\!, \S)^2\RP\RB\RP\Bigg]\ge 1{-}\tfrac{\delta}{2}.
    \label{chap:dis-pra:eq:disintegrated-param-proof-1}
\end{align}
Given a prior $\P\in\M^{*}(\H)$, note that $\EE_{\h'{\sim}\P}\varphi(\h'\!, \S)^{2}$ is a strictly-positive random variable. 
Hence, we apply \textsc{Markov}'s inequality (\Cref{ap:tools:theorem:first-markov}):
\begin{align*}
    \PP_{\S\sim\D^{\m}}\LB \EE_{\h'{\sim}\P}\varphi(\h'\!, \S)^2 \le \frac{2}{\delta}\EE_{\S'{\sim}\D^{\m}}\EE_{\h'{\sim}\P}\varphi(\h'\!, \S')^2\RB \ge 1-\tfrac{\delta}{2}.
\end{align*}
Since the inequality does not depend on the random variable $\h\sim \AQ$, we have
\begin{align*}
    &\PP_{\S\sim\D^{\m}}\LB\EE_{\h'{\sim}\P}\LP\varphi(\h'\!, \S)^2\RP \le\frac{2}{\delta}\EE_{\S'{\sim}\D^{\m}}\EE_{\h'{\sim}\P}\LP\varphi(\h'\!, \S')^2\RP\RB\\
    = &\PP_{\S\sim\D^{\m},\h\sim \AQ}\LB\EE_{\h'{\sim}\P}\LP\varphi(\h'\!, \S)^2\RP \le\frac{2}{\delta}\!\EE_{\S'{\sim}\D^{\m}}\EE_{\h'{\sim}\P}\LP\varphi(\h'\!, \S')^2\RP\RB\!.
\end{align*}
Additionally, note that multiplying by $\frac{4}{2\lambda\delta^2}>0$, adding $\frac{\lambda}{2}\exp(\Renyi_{2}( \AQ\|\P))$, and taking the logarithm to both sides of the inequality results in the same indicator function. Indeed,
\allowdisplaybreaks
\begin{align*}
    &\indic\LB\EE_{\h'{\sim}\P}\LP\varphi(\h'\!, \S)^2\RP \le\frac{2}{\delta}\EE_{\S'{\sim}\D^{\m}}\EE_{\h'{\sim}\P}\LP\varphi(\h'\!, \S')^2\RP\RB\\
    =\ &\indic\LB\forall\lambda>0, \tfrac{4}{2\lambda\delta^2}\EE_{\h'{\sim}\P}\LP\varphi(\h'\!, \S)^2\RP \le\tfrac{8}{2\lambda\delta^3}\EE_{\S'{\sim}\D^{\m}}\EE_{\h'{\sim}\P}\LP\varphi(\h'\!, \S')^2\RP\RB\\
    =\ &\indic\Bigg[\forall\lambda>0, \ln\!\LP\!\tfrac{\lambda}{2}\!\exp(\Renyi_{2}( \AQ\|\P)){+}\tfrac{4}{2\lambda\delta^2}\!\EE_{\h'{\sim}\P}\LP\varphi(\h'\!, \S)^2\RP\!\RP\\
    &\hspace{0.3cm}\le \ln\!\LP\!\tfrac{\lambda}{2}\!\exp(\Renyi_{2}( \AQ\|\P)){+}\!\tfrac{8}{2\lambda\delta^3}\EE_{\S'{\sim}\D^{\m}}\EE_{\h'{\sim}\P}\LP\varphi(\h'\!, \S')^2\RP\RP\Bigg]\!.
\end{align*}
Hence, we can deduce that
\begin{align}
    &\PP_{\S\sim\D^{\m},\h\sim \AQ}\Bigg[\forall\lambda{>}0,\,\ln\LP\frac{1}{2}\LB\lambda\exp(\Renyi_{2}( \AQ\|\P)){+} \frac{4}{\lambda\delta^2}\EE_{\h'{\sim}\P}\LP\varphi(\h'\!, \S)^2\RP\RB\RP\nonumber\\
    &\hspace{0.2cm}\le \ln\LP\frac{1}{2}\LB\lambda\exp(\Renyi_{2}( \AQ\|\P)){+}\frac{8}{\lambda\delta^3}\EE_{\S'{\sim}\D^{\m}}\EE_{\h'{\sim}\P}\LP\varphi(\h'\!, \S')^2\RP\RB\RP \Bigg]\ge 1-\tfrac{\delta}{2}.
    \label{chap:dis-pra:eq:disintegrated-param-proof-2}
\end{align}
Combining \Cref{chap:dis-pra:eq:disintegrated-param-proof-1,chap:dis-pra:eq:disintegrated-param-proof-2} with a union bound gives us the desired result.
\end{proof}

\section{Proof of \Cref{chap:dis-pra:prop:lambda-min}}
\label{chap:dis-pra:sec:proof-min-lambda}

\proplambdamin*
\begin{proof}
We consider the right-hand side of the inequality of \Cref{chap:dis-pra:theorem:disintegrated-lambda} (which is strictly positive): we have 
\begin{align}
\ln\LB \frac{\lambda}{2} e^{\Renyi_{2}(\AQ\|\P)} {+} \frac{8}{2\lambda\delta^3}\EE_{\S'{\sim}\D^{\m}}\EE_{\h'{\sim}\P}\LB\varphi(\h', \S')^2\RB\RB.
\label{chap:dis-pra:eq:prop:lambda-min-1}
\end{align}
Since $\ln$ is a strictly increasing function, we have
\begin{align*}
    &\min_{\lambda>0}\LC \ln\LB \frac{\lambda}{2} e^{\Renyi_{2}(\AQ\|\P)} {+} \frac{8}{2\lambda\delta^3}\EE_{\S'{\sim}\D^{\m}}\EE_{\h'{\sim}\P}\LB\varphi(\h', \S')^2\RB\RB \RC\\
    = &\ln\LB\min_{\lambda>0}\LC \frac{\lambda}{2} e^{\Renyi_{2}(\AQ\|\P)} {+} \frac{8}{2\lambda\delta^3}\EE_{\S'{\sim}\D^{\m}}\EE_{\h'{\sim}\P}\LB\varphi(\h', \S')^2\RB\RC\RB.
\end{align*}
Then, we apply \Cref{chap:dis-pra:lemma:sqrt} by taking $a = \frac{8}{2\delta^3}\EE_{\S'{\sim}\D^{\m}}\EE_{\h'{\sim}\P}\LB\varphi(\h', \S')^2\RB$ and $b=\frac{1}{2}e^{\Renyi_{2}(\AQ\|\P)}$ to obtain $\lambda^*=\sqrt{\frac{a}{b}}= \sqrt{\frac{\EE_{\S'{\sim}\D^{\m}}{\EE}_{{\h'{\sim}\P}}\LB8\varphi(\h'\!, \S')^2\RB}{\delta^3 \exp(\Renyi_{2}(\AQ\|\P))}}$.
Finally, by substituting $\lambda^*$ into \Cref{chap:dis-pra:eq:prop:lambda-min-1}, we obtain
\begin{align*}
    &\ln\LB \frac{\lambda^*}{2} e^{\Renyi_{2}(\AQ\|\P)} {+} \frac{8}{2\lambda^*\delta^3}\EE_{\S'{\sim}\D^{\m}}\EE_{\h'{\sim}\P}\LB\varphi(\h', \S')^2\RB\RB\\
    = &\frac{1}{2}\LP \Renyi_{2}(\AQ\|\P) + \ln\LB\EE_{\S'{\sim}\D^{\m}}\EE_{\h'{\sim} \P}
\LP\frac{8\varphi(\h'\!, \S')^{2}}{\delta^3}\RP\RB\RP, 
\end{align*}
which is the desired result.
\end{proof}

\section{Proof of \Cref{chap:dis-pra:corollary:nn}}
\label{chap:dis-pra:sec:proof-corollary-nn}

We introduce \Cref{chap:dis-pra:theorem:disintegrated-union} which takes into account a set of priors $\priorset$ while \Cref{chap:dis-pra:theorem:disintegrated} handles a unique prior $\P$.

\begin{restatable}{theorem}{theoremdisintegratedunion} For any distribution $\D$ on $\X{\times}\Y$, for any hypothesis set $\H$,  for any priors set $\priorset{=}\{\P_\t\}_{\t=1}^\iter$ of $\iter$ prior $\P\in\M^{*}(\H)$, for any measurable function $\varphi\!:\! \H{\times}(\X{\times}\Y)^{\m}{\to} \Rbb_{+}^*$, for any $\lambda\!>\!1$, for any $\delta\in(0,1]$, for any algorithm $A\!:\!(\X{\times}\Y)^{\m}{\times}\M^{*}(\H){\to} \M(\H)$, we have
\begin{align*}
    \PP_{\S\sim\D^{\m},\h\sim \AQ}\!\Bigg[&\forall \P_\t\in\priorset, \frac{\lambda}{\lambda{-}1}\ln\LP\varphi(\h,\!\S)\RP \le \Renyi_{\lambda}(\AQ\|\P){+} \frac{\lambda}{\lambda{-}1}\ln\frac{2}{\delta}\\
    + &\ln\!\frac{2\iter}{\delta} + \ln\LP\EE_{\S'{\sim}\D^{\m}}\EE_{\h'{\sim}\P}\LP\varphi(\h'\!, \S')^{\frac{\lambda}{\lambda{-}1}}\RP\RP\!\Bigg]\ge 1{-}\delta,
\end{align*}
where $\AQ{\defeq}A(\S, \P)$ is output by the deterministic algorithm $A$.
\label{chap:dis-pra:theorem:disintegrated-union}
\end{restatable}

\begin{proof} The proof is mainly the same as \Cref{chap:dis-pra:theorem:disintegrated}.
Indeed, we first derive the same equation as \Cref{chap:dis-pra:eq:disintegrated-proof-1}, we have
\begin{align*}
    \PP_{\S\sim\D^{\m},\h\sim \AQ}\!\Big[&\forall\P{\in}\M^{*}(\H),\,\frac{\lambda}{\lambda{-}1}\ln\!\LP\varphi(\h,\!\S)\RP \le \Renyi_{\lambda}( \AQ\|\P)\\
    &+ \frac{\lambda}{\lambda{-}1}\ln\frac{2}{\delta}{+}\ln\!\LP\EE_{\h'{\sim}\P}\LP\varphi(\h'\!, \S)^{\frac{\lambda}{\lambda-1}}\RP\RP\Big]\!\!\ge 1{-}\tfrac{\delta}{2}.
\end{align*}
Then, we apply \textsc{Markov}'s inequality (as in \Cref{chap:dis-pra:theorem:disintegrated}) $\iter$ times with the $\iter$ priors $\P_\t$ belonging to $\priorset$, however, we set the confidence to $\frac{\delta}{2\iter}$ instead of $\tfrac{\delta}{2}$, we have 
\begin{align*}
    &\PP_{\S\sim\D^{\m},\h\sim \AQ}\Bigg[\ln\!\LP\EE_{\h'{\sim}\P_\t}\LB\varphi(\h'\!, \S)^{\frac{\lambda}{\lambda{-}1}}\RB\RP\\
    &\hspace{2cm}\le \ln\!\frac{2\iter}{\delta}{+}\ln\!\LP\EE_{\S'{\sim}\D^{\m}}\EE_{\h'{\sim}\P_\t}\LB\varphi(\h'\!, \S')^{\frac{\lambda}{\lambda-1}}\RB\RP\Bigg]\ge1{-}\tfrac{\delta}{2\iter}.
\end{align*}
Finally, combining the $\iter+1$ bounds with a union bound gives us the desired result.
\end{proof}

We now prove \Cref{chap:dis-pra:corollary:nn} from \Cref{chap:dis-pra:theorem:disintegrated-union}.\\

\corollarynn*
\begin{proof}
We instantiate \Cref{chap:dis-pra:theorem:disintegrated-union} with $\varphi(\h,\!\S)=\exp\!\LB\tfrac{\lambda-1}{\lambda}\m\kl(\RiskLoss_{\dS}(\h)\|\RiskLoss_{\D}(\h))\RB$ and $\lambda=2$. 
We have with probability at least $1-\delta$ over $\S\sim\D^\m$ and $\h\sim\AQ$, for all prior $\P_\t\!\in\! \priorset$ 
\begin{align*}
    \kl(\RiskLoss_{\dS}(\h)\|\RiskLoss_{\D}(\h))\! 
    \le \! \tfrac{1}{\m}\!\LB \Renyi_{2}(\AQ\|\P_\t)+ \ln\LP\frac{8\iter}{\delta^3}\EE_{\S'{\sim}\D^{\m}}\EE_{\h'{\sim}\P_\t}e^{\m\kl(\RiskLoss_{\dSp}(\h')\|\RiskLoss_{\D}(\h'))}\RP \RB\!.
\end{align*}
From \citet{Maurer2004} we upper-bound  $\EE_{\S'{\sim}\D^\m}\EE_{\h'{\sim}\P_\t} e^{\m\kl(\RiskLoss_{\dSp}(\h')\|\RiskLoss_{\D}(\h'))}$ by $2\sqrt{\m}$ for each prior $\P_\t$ (\Cref{ap:pac-bayes:lemma:2-sqrt-m}). 
Hence, we have, for all prior $\P_\t\!\in\! \priorset$ 
\begin{align*}
    \kl(\RiskLoss_{\dS}(\h)\|\RiskLoss_{\D}(\h))\! 
    \le \! \tfrac{1}{\m}\!\LB \Renyi_{2}(\AQ\|\P_\t)+ \ln\LP\tfrac{16\iter\sqrt{\m}}{\delta^3}\RP \RB\!.
\end{align*}
Additionally, the \textsc{Rényi} divergence $\Renyi_{2}(\AQ\|\P_\t)$ between two multivariate Gaussians $\AQ{=}\Ncal(\wbf, \sigma^2\Ibf_{D})$ and $\P_\t{=}\Ncal(\vbf_\t, \sigma^2\Ibf_{D})$ is well known: its closed-form solution is $\Renyi_{2}(\AQ\|\P_\t){=}\frac{\|\wbf{-}\vbf_\t\|_{2}^{2}}{\sigma^2}$ (see, for example, \citep{GilAlajajiLinder2013}).
\end{proof}

\section{Proof of \Cref{chap:dis-pra:corollary:nn-rbc}}
\label{chap:dis-pra:sec:proof-corollary-nn-rbc}

We first prove the following lemma in order to prove \Cref{chap:dis-pra:corollary:nn-rbc}.
\begin{lemma}
If $\AQ=\Ncal(\wbf, \sigma^2\Ibf_D)$ and $\P = \Ncal(\vbf, \sigma^2\Ibf_{D})$, we have 
\begin{align*}
    \ln\frac{\AQ(\h)}{\P(\h)} = \frac{1}{2\sigma^2}\Big[\|\wbf{+}\epsilonbf-\vbf\|_2^2-\|\epsilonbf\|_2^2\Big],
\end{align*}
where $\epsilonbf{\sim}\Ncal(\zerobf, \sigma^2\Ibf_{D})$ is a Gaussian noise 
such that  $\wbf{+}\epsilonbf$ are the weights of $\h{\sim}\AQ$ with \mbox{$\AQ{=}\Ncal(\wbf, \sigma^2\Ibf_{D})$}.
\label{chap:dis-pra:lemma:disintegrated-kl}
\end{lemma}
\begin{proof}
The probability density functions of $\AQ$ and $\P$ for $\h\sim\AQ$ (with the weights $\wbf{+}\epsilonbf$) can be rewritten as
\begin{align*}
    &\AQ(\h) = \LB\frac{1}{\sigma\sqrt{2\pi}}\RB^D\!\exp\!\LP\!-\frac{1}{2\sigma^2}\|\wbf{+}\epsilonbf-\wbf\|_2^2\RP = \LB\frac{1}{\sigma\sqrt{2\pi}}\RB^D\!\exp\!\LP\!-\frac{1}{2\sigma^2}\|\epsilonbf\|_2^2\RP\\
    \text{and}\quad &\P(\h)=\LB\frac{1}{\sigma\sqrt{2\pi}}\RB^D\!\exp\!\LP\!-\frac{1}{2\sigma^2}\|\wbf{+}\epsilonbf-\vbf\|_2^2\RP.
\end{align*}
We can derive a closed-form expression of $\ln\!\LB\frac{\AQ(\h)}{\P(\h)}\RB$. 
Indeed, we have
\begin{align*}
    \ln\!\LB\frac{\AQ(\h)}{\P(\h)}\RB &= \ln\LB\AQ(\h)\RB-\ln\LB\P(\h)\RB\\
    &= \ln\LP\LB\frac{1}{\sigma\sqrt{2\pi}}\RB^D\!\exp\!\LP\!-\frac{1}{2\sigma^2}\|\epsilonbf\|_2^2\RP\RP\\
    &\hspace{0.4cm}-\ln\LP\LB\frac{1}{\sigma\sqrt{2\pi}}\RB^D\!\exp\!\LP\!-\frac{1}{2\sigma^2}\|\wbf{+}\epsilonbf-\vbf\|_2^2\RP\RP\\
    &= -\frac{1}{2\sigma^2}\|\epsilonbf\|_2^2 + \frac{1}{2\sigma^2}\|\wbf{+}\epsilonbf-\vbf\|_2^2= \frac{1}{2\sigma^2}\Big[\|\wbf{+}\epsilonbf-\vbf\|_2^2-\|\epsilonbf\|_2^2\Big].
\end{align*}
\end{proof}

We can now prove \Cref{chap:dis-pra:corollary:nn-rbc}.

\corollarynnrbc*
\begin{proof}
We will prove the three bounds separately.\\

\textbf{\Cref{chap:dis-pra:eq:nn-rivasplata}.} We instantiate Theorem~1{\footnotesize\it (i)} of \citet{RivasplataKuzborskijSzepesvariShaweTaylor2020} (proved in \Cref{chap:pac-bayes:theorem:general-disintegrated-rivasplata}) with $\varphi(\h,\!\S)=\exp\!\LB m\kl(\RiskLoss_{\dS}(\h)\|\RiskLoss_{\D}(\h))\RB$, however, we apply the theorem $\iter$ times for each prior $\P_\t\in\priorset$ (with a confidence $\frac{\delta}{\iter}$ instead of $\delta$).
Hence, for each prior $\P_\t\in\priorset$, we have with probability at least $1-\frac{\delta}{\iter}$ over the random choice of $\S\sim\D^\m$ and $\h\sim\AQ$
\begin{align*}
\kl(\RiskLoss_{\dS}(\h)\|\RiskLoss_{\D}(\h))\le \frac{1}{\m}\!\LB\ln\!\LB\frac{\AQ(\h)}{\P_\t(\h)}\RB{+}\ln\!\LP\frac{\iter}{\delta}\!\EE_{\S'{\sim}\D^{\m}}\EE_{\h'{\sim}\P}e^{\m\kl(\RiskLoss_{\dSp}(\h')\|\RiskLoss_{\D}(\h'))}\!\RP\!\RB\!.
\end{align*}
From \citet{Maurer2004}, we upper-bound  $\EE_{\S'{\sim}\D^\m}\EE_{\h'{\sim}\P_\t} e^{\m\kl(\RiskLoss_{\dSp}(\h')\|\RiskLoss_{\D}(\h'))}$ by $2\sqrt{\m}$ (\Cref{ap:pac-bayes:lemma:2-sqrt-m}) and using \Cref{chap:dis-pra:lemma:disintegrated-kl} we rewrite the disintegrated KL divergence.
Finally, a union bound argument gives us the claim.\\

\textbf{\Cref{chap:dis-pra:eq:nn-blanchard}.}  We apply $\iter\card(\blaset)$ times Proposition~3.1 of \citet{BlanchardFleuret2007} (proved in \Cref{chap:pac-bayes:theorem:disintegrated-blanchard})
with a confidence $\frac{\delta}{\iter\card(\blaset)}$ instead of $\delta$. 
For each prior $\P_\t\in\priorset$ and hyperparameters $b\in\blaset$, we have with probability at least $1-\frac{\delta}{\iter\card(\blaset)}$ over the random choice of $\S\sim\D^\m$ and $\h\sim\AQ$
\begin{align*}
\kl(\RiskLoss_{\dS}(\h)\|\RiskLoss_{\D}(\h))\le \frac{1}{\m}\!\LB\frac{b{+}1}{b}\!\!\LB\ln\!\frac{\AQ(\h)}{\P_\t(\h)}\RB_{+}\!\!{+}\ln\!\LP\frac{\iter\card(\blaset)(b{+}1)}{\delta}\!\RP\!\RB\!.
\end{align*}
From \Cref{chap:dis-pra:lemma:disintegrated-kl} and a union bound argument, we obtain the claim.\\

\textbf{\Cref{chap:dis-pra:eq:nn-catoni}.} We apply $\iter\card(\catset)$ times Theorem 1.2.7 of \citet{Catoni2007} (proved in \Cref{chap:pac-bayes:theorem:disintegrated-catoni}) with a confidence $\tfrac{\delta}{\iter\card(\catset)}$ instead of $\delta$. 
For each prior $\P_{\t}\in\priorset$ and hyperparameter $c\in\catset$, we have with probability at least $1-\tfrac{\delta}{\iter\card(\catset)}$ over the random choice of $\S\sim\D^\m$ and $\h\sim\AQ$
\begin{align*}
\RiskLoss_{\D}(\h) \!\le\, \frac{1}{1{-}e^{{-}c}}\LB1{-}\exp\LP{-}c\RiskLoss_{\dS}(\h) {-}\frac{1}{\m}\!\!\left[\ln\LB\frac{\AQ(\h)}{\P_\t(\h)}\RB {+} \ln\!\frac{\iter\card(\catset)}{\delta}\right]\RP\RB\!.
\end{align*}
From \Cref{chap:dis-pra:lemma:disintegrated-kl} and a union bound argument, we obtain the claim.
\end{proof}

\section{Proof of \Cref{chap:dis-pra:corollary:nn-sto}}
\label{chap:dis-pra:sec:proof-nn-sto}

\corollarynnsto*
\begin{proof}
We instantiate \Cref{chap:pac-bayes:theorem:seeger} and apply \textsc{Jensen}'s inequality (\Cref{ap:tools:theorem:jensen}) on the left-hand side of the inequation for each prior $\P_\t{=}\Ncal(\vbf_\t, \sigma^2\Ibf_{D})$ with the posterior $\Q{=}\Ncal(\wbf, \sigma^2\Ibf_{D})$ with a confidence $\tfrac{\delta}{2\iter}$ instead of $\delta$.
Indeed, for each prior $\P_\t$, with probability at least $1{-}\tfrac{\delta}{2\iter}$ over the random choice of $\S\sim\D^\m$, \mbox{we have for all posterior $\Q$ on $\H$},
\begin{align*}
\kl\!\LP\EE_{\h{\sim}\Q}\!\!\RiskLoss_{\dS}(\h)\| \EE_{\h{\sim}\Q}\!\!\RiskLoss_{\D}(\h)\!\RP{\le} \frac{1}{\m}\!\LB 
\KL(\Q\|\P_\t)
{+}\ln\tfrac{4\iter\sqrt{\m}}{\delta}\RB\!.
\end{align*}
Note that the closed-form solution of the $\KL$~divergence between the Gaussian distributions $\Q$ and $\P_\t$ is well known, we have $\KL(\Q\|\P_\t){=}\frac{\|\wbf{-}\vbf_\t\|_{2}^{2}}{2\sigma^2}$.
Then, by applying a union bound argument over the $\iter$ bounds obtained with the $\iter$ priors $\P_\t$, we have with probability at least $1{-}\frac{\delta}{2}$ over the random choice of $\S\sim\D^\m$, for all prior $\P_\t\in\priorset$, for all posterior $\Q$
\begin{align*}
\kl\!\LP\EE_{\h{\sim}\Q}\!\!\RiskLoss_{\dS}(\h)\| \EE_{\h{\sim}\Q}\!\!\RiskLoss_{\D}(\h)\!\RP{\le} \frac{1}{\m}\!\LB 
\tfrac{\|\wbf{-}\vbf_\t\|_{2}^{2}}{2\sigma^2}
{+}\ln\tfrac{4\iter\sqrt{\m}}{\delta}\RB\!.\quad\text{(\Cref{chap:dis-pra:eq:nn-sto-seeger})}
\end{align*}
Additionally, we obtained \Cref{chap:dis-pra:eq:nn-sto-sample} by a direct application the Theorem~2.2 of \citet{DziugaiteRoy2017} (with confidence $\frac{\delta}{2}$ instead of $\delta$).
Finally, from a union bound of the two bounds in \Cref{chap:dis-pra:eq:nn-sto-sample,chap:dis-pra:eq:nn-sto-seeger} gives the result.
\end{proof}

\section{Evaluation and Minimization of the Bounds of \Cref{chap:dis-pra:corollary:nn,chap:dis-pra:corollary:nn-rbc,chap:dis-pra:corollary:nn-sto}}
\label{chap:dis-pra:sec:evaluation-minimization}

We optimize and evaluate the bounds of the corollaries (except \Cref{chap:dis-pra:eq:nn-catoni}) thanks to the inverting functions of $\kl()$ defined in \Cref{chap:pac-bayes:def:invert-kl}.
Indeed, for the different corollaries, the PAC-Bayesian generalization bounds become
\begin{align*}
&\RiskLoss_{\D}(\h) \le \underbrace{\klmax\!\LP\RiskLoss_{\dS}(\h) \;\middle\vert\; \frac{1}{\m}\!\!\left[ \frac{\|\wbf{-}\vbf_\t\|_{2}^{2}}{\sigma^2}{+}\ln\frac{16\iter\sqrt{\m}}{\delta^3}\right]\RP}_{\text{\Cref{chap:dis-pra:corollary:nn}}},\\
&\RiskLoss_{\D}(\h) \le \underbrace{\klmax\!\LP\RiskLoss_{\dS}(\h) \;\middle\vert\; \frac{1}{\m}\!\!\LB\!\frac{\| \wbf{+}\epsilonbf{-}\vbf_\t\|^2_{2}\!{-}\|\epsilonbf\|^2_{2}}{2\sigma^2}{+} \ln\!\tfrac{2\iter\sqrt{\m}}{\delta}\RB\RP}_{\text{\Cref{chap:dis-pra:eq:nn-rivasplata}}},\\
&\RiskLoss_{\D}(\h) \le \underbrace{\klmax\!\LP\RiskLoss_{\dS}(\h) \;\middle\vert\; \frac{1}{\m}\!\!\LB\!\frac{b{+}1}{b}\!\LB\frac{\| \wbf{+}\epsilonbf{-}\vbf_\t\|^2_{2}\!{-}\|\epsilonbf\|^2_{2}}{2\sigma^2}\RB_{+}\!\!{+} \ln\!\tfrac{(b+1)\iter\card(\blaset)}{\delta}\RB\RP}_{\text{\Cref{chap:dis-pra:eq:nn-blanchard}}},\\
\text{and }\ &\EE_{\h\sim\Q}\RiskLoss_{\D}(\h) \le \underbrace{\klmax\!\LP \spadesuit \;\middle\vert\; \frac{1}{\m}\!\LB \frac{\|\wbf{-}\vbf_\t\|_{2}^{2}}{2\sigma^2} {+}\ln\frac{4\iter\sqrt{\m}}{\delta}\RB\RP}_{\text{\Cref{chap:dis-pra:corollary:nn-sto}}},\\
\text{where }\ & \spadesuit = \klmax\!\LP \frac{1}{\K}\sum_{i=1}^{\K}\!\Risk_{\S}(\h_i) \;\middle\vert\; \frac{1}{\K} \ln\frac{4}{\delta}\RP.
\end{align*}

Based on these bounds, we can deduce some objective functions that is approximated on a mini-batch $\batch\subseteq\S$. 
Indeed, at each iteration in phase {\bf 2)}, after sampling the noise $\epsilonbf$, the algorithm updates the weights $\omegabf$ (\ie, the hypothesis $\h$) by optimizing 
\begin{align*}
&\underbrace{\klmax\LP\!\RiskLoss_{\dbatch}(\h) \middle\vert  \frac{1}{\m}\!\!\LB\frac{\|\omegabf{-}\vbf_\t\|_{2}^{2}}{\sigma^2}{+}\ln\frac{16\iter\sqrt{\m}}{\delta^3}\RB\RP}_{\text{Objective function for \Cref{chap:dis-pra:corollary:nn}}},\\
&\underbrace{\klmax\LP\!\RiskLoss_{\dbatch}(\h) \middle\vert \frac{1}{\m}\!\!\LB\frac{\| \omegabf{+}\epsilonbf{-}\vbf_\t\|^2_{2}\!{-}\|\epsilonbf\|^2_{2}}{2\sigma^2}{+}\ln\frac{2\iter\sqrt{\m}}{\delta}\RB\RP}_{\text{Objective function for \Cref{chap:dis-pra:eq:nn-rivasplata}}},\\
&\underbrace{\klmax\LP\RiskLoss_{\dbatch}(\h) \;\middle\vert\; \frac{1}{\m}\!\!\LB\!\frac{b{+}1}{b}\!\LB\frac{\| \omegabf{+}\epsilonbf{-}\vbf_\t\|^2_{2}\!{-}\|\epsilonbf\|^2_{2}}{2\sigma^2}\RB_{+}\!\!\!{+} \ln\!\tfrac{(b{+}1)\iter\card(\blaset)}{\delta}\RB\RP}_{\text{Objective function for \Cref{chap:dis-pra:eq:nn-blanchard}}},
\end{align*}
where the loss $\loss()$ is the bounded cross-entropy loss of \citet{DziugaiteRoy2018}, \ie, $\loss(\h, (\x,\y)) = -\frac{1}{Z}\ln\!\LB e^{-Z}+(1-2e^{-Z})\h[y]\RB$.\\

\looseness=-1
Concerning the optimization of the hyperparameters $c\in\catset$ and $b\in\blaset$ for \Cref{chap:dis-pra:eq:nn-blanchard,chap:dis-pra:eq:nn-catoni}, we {\it (i)} initialize $b\in\blaset$ or $c\in\catset$ with the one that performs best on the first mini-batch and {\it (ii)} optimize by gradient descent the hyperparameter.
To evaluate \Cref{chap:dis-pra:eq:nn-blanchard,chap:dis-pra:eq:nn-catoni}, we take $b\in\blaset$ and $c\in\catset$ that leads to the tightest bound.

\section{Disintegrated Information-theoretic Bounds}
\label{chap:dis-pra:sec:info-theoretic}
We discuss in this section another interpretation of the disintegration procedure through \Cref{chap:dis-pra:theorem:mutual-info-kl,chap:dis-pra:theorem:mutual-info} below.
Actually, the \textsc{Rényi} divergence between $\P$ and $\Q$ is sensitive to the choice of the learning \mbox{sample $\S$}: when the posterior $\Q$ learned from $\S$ differs greatly from the prior $\P$ the divergence is high. 
To avoid such a behavior, we consider mutual information which is a measure of dependence between the random variables $\S\!\in\!(\X{\times}\Y)^\m$ and $\h\!\in\!\H$.
More formally, the mutual information is defined as
\begin{align*}
    \MI(\h{;} \S) = \min_{\P\in\M^{*}(\H)}\EE_{\S\sim\D^\m}\KL(\AQ\|\P).
\end{align*}
From this quantity, we can derive the generalization bound introduced in the following theorem.
\begin{restatable}{theorem}{theoremmutualinfokl}
For any distribution $\D$ on $\X{\times}\Y$, for any hypothesis set $\H$, for any measurable function $\varphi:\H\times (\X{\times}\Y)^{\m}\rightarrow [1, +\infty[$, for any $\delta\in(0,1]$, for any deterministic algorithm $A:(\X{\times}\Y)^{\m}\times\M^{*}(\H){\rightarrow} \M(\H)$, we have
\begin{align*}
    \PP_{\S\sim\D^{\m}, \h\sim \AQ}\LB\ln\varphi(\h,\!\S)\le \frac{1}{\delta}\LB \MI(\h{;}\S) +\ln\LP\EE_{\S\sim\D^\m}\EE_{\h\sim\P^*}\varphi(\h, \S)\RP\RB \RB \ge 1-\delta,
\end{align*}
where $\P^*$ is defined such that $\P^*(\h)=\EE_{\S\sim\D^\m}\AQp(\h)$.
\label{chap:dis-pra:theorem:mutual-info-kl}
\end{restatable}
\begin{proof}
Deferred to~\Cref{ap:dis-pra:sec:proof-mutual-info}.
\end{proof}

As for the disintegrated bounds introduced in \Cref{chap:dis-pra:sec:contrib}, the bound on $\ln \varphi(\h,\!\S)$ depends on mainly two terms: a term (\ie, $\MI(\h{;}\S)$) that measures the dependence of $\h\in\H$ on the learning sample $\S$ and $\ln\LP\EE_{\S\sim\D^\m}\EE_{\h\sim\P^*}\varphi(\h, \S)\RP$ that must be upper-bounded to obtain a computable bound.
However, the bound has a polynomial dependence of $\delta$, \ie, we have $\frac{1}{\delta}$ instead of $\ln\frac{1}{\delta}$.
To improve such dependence, we consider Sibson's mutual information~\citep{Verdu2015}.
It involves an expectation over the learning samples of a given size $\m$ and is defined for a given $\lambda{>}1$ by
\begin{align*}
\MI_{\lambda}(\h{;}\S) 
&\defeq \min_{\P\in\M^{*}(\H)}  \frac{1}{\lambda{-}1}\!\ln\!\LB\EE_{\S\sim\D^{\m}}\EE_{\h\sim \P}\!\LB\!\frac{\AQ(\h)}{\P(\h)}\!\RB^{\lambda}\RB.
\end{align*}

The higher $\MI_{\lambda}(\h{;}\S)$, the higher the correlation is, meaning that the sampling of $\h$ is highly dependent on the choice of $\S$. 
This measure has two interesting properties: it generalizes the mutual information~\citep{Verdu2015}, and it can be related to the \textsc{Rényi} divergence.
Indeed, let $\rho(\h, \S){=} \AQ(\h)\D^{\m}(\S)$, \textit{resp.} $\pi(\h, \S){=} \P(\h)\D^{\m}(\S)$, be the probability of sampling both  $\S{\sim}\D^\m$ and  $\h{\sim}\AQ$, \textit{resp.}  $\S{\sim}\D^\m$ and $\h{\sim}\P$.
Then we can write: 
\begin{align}
\MI_{\lambda}(\h{;}\S) 
 &=\!\!\min_{\P\in\M^{*}(\H)} \frac{1}{\lambda{-}1}\!\ln\!\Bigg[\!\EE_{\S\sim\D^{\m}}\EE_{\h\sim \P}\!\LB\!\frac{\AQ(\h)\D^{\m}(\S)}{\P(\h)\D^{\m}(\S)}\!\RB^{\lambda}\!\!\Bigg]\nonumber\\
 &=\!\!\min_{\P\in\M^{*}(\H)} \Renyi_{\lambda}(\rho\|\pi).\label{chap:dis-pra:eq:mutual-info-min}
\end{align}

From~\citet{Verdu2015} the 
optimal prior $\P^*$ minimizing \Cref{chap:dis-pra:eq:mutual-info-min} is  a {\it distribution-dependent} prior: 
\begin{align*}
\displaystyle \P^*(\h)=\frac{\LB\EE_{\S'{\sim}\D^{\m}}\AQp (\h)^{\lambda}\RB^{\frac{1}{\lambda}}}{\EE_{\h'{\sim}\P}\tfrac{1}{\P(\h')}\LB\EE_{\S'{\sim}\D^{\m}}\AQp (\h')^{\lambda}\RB^{\frac{1}{\lambda}}}.
\end{align*}
This leads to an {\it Information-Theoretic generalization 
bound}.


\begin{restatable}[Disintegrated Information-Theoretic Bound]{theorem}{theoremmutualinfo}\label{chap:dis-pra:theorem:mutual-info}
For any distribution $\D$ on $\X{\times}\Y$, for any hypothesis set $\H$, for any measurable function $\varphi\!:\!\H{\times} (\X{\times}\Y)^{\m}{\to}\Rbb_{+}^*$, \mbox{for any $\lambda\!>\!1$}, for any $\delta\in(0,1]$, for any algorithm $A\!:\!(\X{\times}\Y)^{\m}\times\M^{*}(\H){\rightarrow} \M(\H)$, we have
\begin{align*}
   \PP_{\substack{\S\sim\D^{\m},\\\h\sim \AQ}} \!\left(
   \frac{\lambda}{\lambda{-}1}\!\ln\!\LP\varphi(\h,\!\S)\RP \le \MI_{\lambda}(\h'{;} \S')\!+\!  \ln\left[\frac{1}{\delta^{\frac{\lambda}{\lambda{-}1}}}\!\EE_{\S'{\sim}\D^{\m}}\EE_{\h'{\sim} \P^*}\!\!\LB\varphi(\h'\!, \S')^{\frac{\lambda}{\lambda-1}}\RB\right] 
   \right)\ge 1{-}\delta.
\end{align*}
\end{restatable}
\begin{proof}
Deferred to~\Cref{ap:dis-pra:sec:proof-mutual-info}.
\end{proof}

We can remark that \Cref{chap:dis-pra:theorem:mutual-info} is tighter than \Cref{chap:dis-pra:theorem:mutual-info-kl}. 
For example, when we instantiate \Cref{chap:dis-pra:theorem:mutual-info-kl} with $\varphi(\h,\S)=\exp\LB m\kl(\RiskLoss_{\dS}(\h)\|\RiskLoss_{\D}(\h))\RB$, the bound will be multiplied by $\frac{1}{\delta m}$, while the bound of \Cref{chap:dis-pra:theorem:mutual-info} is only multiplied by $\frac{1}{\m}$ (but we add the term $\frac{1}{\m}\ln\frac{1}{\delta}$ to the bound which is small even for small $\m$).

For the sake of comparison, we introduce the following corollary of \Cref{chap:dis-pra:theorem:mutual-info}.

\begin{restatable}{corollary}{corollarymutualinfo}\label{chap:dis-pra:corollary:mutual-info} Under the assumptions of  \Cref{chap:dis-pra:theorem:mutual-info}, when $\lambda{\to}1^+$,  with probability at least $1{-}\delta$ we have
\begin{align*}
\ln\varphi(\h{,}\S) \le \ln\frac{1}{\delta} + \ln\left[\esssup_{\S'\in(\X{\times}\Y), \h'\in\H}\varphi(\h'{,} \S')\right].
\end{align*}
\textit{When $\lambda{\to}+\infty$, with probability at least $1{-}\delta$ we have}
\begin{align*}
\ln\varphi(\h{,} \S){\le}\ln\LP\esssup_{\S\in\S, h\in\H}\frac{\AQ(\h)}{\P^*(\h)}\RP{+}\ln\!\Big[\frac{1}{\delta} {\displaystyle \EE_{\S'{\sim}\D^{\m}}\EE_{\h'{\sim}\P}\varphi(\h'{,}\S')}\Big]\!.  
\end{align*}
\end{restatable}
As for \Cref{chap:dis-pra:theorem:disintegrated}, this corollary illustrate a trade-off introduced by $\lambda$ between the Sibson's mutual information $\MI_{\lambda}(\h'; \S')$ and the term $\ln\!\LP\EE_{\S'{\sim}\D^{\m}}\EE_{\h'{\sim} \P}\LP\varphi(\h'\!, \S')^{\frac{\lambda}{\lambda-1}}\RP\!\RP$.\\

Furthermore, \citet[Cor.4]{EspositoGastparIssa2020}  introduced a bound involving Sibson's mutual information.
Their bound holds with probability at least $1{-}\delta$ over $\S\sim\D^\m$ and $\h\sim\AQ$:
\begin{align}
2(\RiskLoss_{\dS}(\h){-}\RiskLoss_{\D}(\h))^2\le \tfrac{1}{\m}\!\LB \MI_{\lambda}(\h'; \S') + \ln \tfrac{2}{\delta^{\frac{\lambda}{\lambda{-}1}}}\RB.\label{chap:dis-pra:eq:esposito}
\end{align}
Hence, we compare \Cref{chap:dis-pra:eq:esposito} with the equations of the following corollary.
\begin{corollary}
For any distribution $\D$ on $\X{\times}\Y$, for any hypothesis set $\H$, for any $\lambda\!>\!1$, for any $\delta\in(0,1]$, for any algorithm $A\!:\!(\X{\times}\Y)^{\m}\times\M^{*}(\H){\rightarrow} \M(\H)$, with probability at least $1{-}\delta$ over $\S\sim\D^\m$ and $\h\sim\AQ$, we have
\begin{align}
   & \kl(\RiskLoss_{\dS}(\h)\|\RiskLoss_{\D}(\h))\! 
    \le \! \tfrac{1}{\m}\!\LB \MI_{\lambda}(\h'; \S')\! +\! \ln\! \tfrac{2\sqrt{\m}}{\delta^{\frac{\lambda}{\lambda{-}1}}} \RB\label{chap:dis-pra:eq:mutual-info-seeger}\\
   \text{\quad and\quad}
    & 2(\RiskLoss_{\dS}(\h){-}\RiskLoss_{\D}(\h))^2\! \le\! \tfrac{1}{\m}\!\LB \MI_{\lambda}(\h'; \S')\! +\! \ln\! \tfrac{2\sqrt{\m}}{\delta^{\frac{\lambda}{\lambda{-}1}}}\RB\label{chap:dis-pra:eq:mutual-info-mcallester}\!.
\end{align}
\label{chap:dis-pra:corollary:mutual-info-seeger-mcallester}
\end{corollary}
\begin{proof}
First of all, we instantiate \Cref{chap:dis-pra:theorem:mutual-info} with the function $\varphi(\h,\!\S)=\exp\!\LB\tfrac{\lambda-1}{\lambda}\m\kl(\RiskLoss_{\dS}(\h)\|\RiskLoss_{\D}(\h))\RB$, we have (by rearranging the terms)
\begin{align*}
    \kl(\RiskLoss_{\dS}(\h)\|\RiskLoss_{\D}(\h))\! 
    \le \! \frac{1}{\m}\!\LB \MI_{\lambda}(\h'; \S')\! +\! \ln\!\LP\! \tfrac{1}{\delta^{\frac{\lambda}{\lambda{-}1}}}\EE_{\S'{\sim}\D^\m}\EE_{\h'{\sim}\P}e^{\m\kl(\RiskLoss_{\dSp}(\h')\|\RiskLoss_{\D}(\h'))}\RP \RB\!.
\end{align*}
Then, from \citet{Maurer2004}, we upper-bound  $\EE_{\S'{\sim}\D^\m}\EE_{\h'{\sim}\P} e^{\m\kl(\RiskLoss_{\dSp}(\h')\|\RiskLoss_{\D}(\h'))}$ by $2\sqrt{\m}$ (\Cref{ap:pac-bayes:lemma:2-sqrt-m}) to obtain \Cref{chap:dis-pra:eq:mutual-info-seeger}.
Finally, to obtain \Cref{chap:dis-pra:eq:mutual-info-mcallester}, we apply \textsc{Pinsker}'s inequality (\Cref{ap:pac-bayes:theorem:pinsker}), \ie, we have the inequality $2(\RiskLoss_{\dS}(\h){-}\RiskLoss_{\D}(\h))^2\le \kl(\RiskLoss_{\dS}(\h)\|\RiskLoss_{\D}(\h))$ on \Cref{chap:dis-pra:eq:mutual-info-seeger}.
\end{proof}
\Cref{chap:dis-pra:eq:mutual-info-mcallester} is slightly looser than \Cref{chap:dis-pra:eq:esposito} since it involves an extra term of $\tfrac1m\ln\sqrt{\m}$.
However, \Cref{chap:dis-pra:eq:mutual-info-seeger} is tighter than \Cref{chap:dis-pra:eq:esposito} when we have $\kl(\RiskLoss_{\dS}(\h)\|\RiskLoss_{\D}(\h)){-}2(\RiskLoss_{\dS}(\h){-}\RiskLoss_{\D}(\h))^2 \ge \tfrac1m\ln\sqrt{\m}$ (which becomes more frequent as $\m$ grows).
Moreover, from a theoretical view, \Cref{chap:dis-pra:theorem:mutual-info} brings a different philosophy than the disintegrated PAC-Bayes bounds. 
Indeed, in \Cref{chap:dis-pra:theorem:disintegrated,chap:dis-pra:theorem:disintegrated-lambda}, given $\S$, the \textsc{Rényi} divergence  $\Renyi_{\lambda}(\AQ\|\P)$ suggests that the learned posterior $\AQ$ should be close enough to the prior $\P$  to get a low bound.
While in \Cref{chap:dis-pra:theorem:mutual-info}, the Sibson's mutual information $\MI_{\lambda}(\h'; \S')$ suggests that the random variable $\h$ has to be {\it not too much correlated} to $\S$.
However, the bound of \Cref{chap:dis-pra:theorem:mutual-info} is not computable in practice due notably to the sample expectation over the unknown distribution $\D$ in $\MI_{\lambda}()$.
An exciting line of future works could be to study how we can make use of  \Cref{chap:dis-pra:theorem:mutual-info} in practice.

\section{Proof of \Cref{chap:dis-pra:theorem:mutual-info-kl}}

In order to prove \Cref{chap:dis-pra:theorem:mutual-info-kl}, we need to prove \Cref{chap:dis-pra:lemma:mutual-info-kl}.

\begin{lemma}
For any distribution $\D$ on $\X{\times}\Y$, for any hypothesis set $\H$, for any  measurable function $\varphi:\H\times (\X{\times}\Y)^{\m}\rightarrow [1, +\infty[$, for any $\delta\in(0,1]$, for any deterministic algorithm $A:(\X{\times}\Y)^{\m}\times\M^{*}(\H){\rightarrow} \M(\H)$, we have
\begin{align*}
    \PP_{\S\sim\D^{\m}, \h\sim \AQ}\Bigg[\forall\P{\in}\M^{*}(\H),\ \ln\varphi(\h,\!\S)&\le \frac{1}{\delta}\Big[\EE_{\S\sim\D^\m}\KL(\AQ\|\P)\\
    &+\ln\LP\EE_{\S\sim\D^\m}\EE_{\h\sim\P}\varphi(\h, \S)\RP\Big] \Bigg] \ge 1-\delta.
\end{align*}
\label{chap:dis-pra:lemma:mutual-info-kl}
\end{lemma}
\begin{proof}
By developing $\EE_{\S\sim\D^\m}\EE_{\h\sim\AQ}\ln\varphi(\h, \S)$, we have for all prior $\P\in\M^{*}(\H)$
\begin{align*}
\EE_{\S\sim\D^\m}\EE_{\h\sim\AQ}\ln\varphi(\h, \S) &= \EE_{\S\sim\D^\m}\EE_{\h\sim\AQ}\ln\LB\frac{\AQ(\h)\P(\h)}{\P(\h)\AQ(\h)}\varphi(\h, \S)\RB\\
&= \EE_{\S\sim\D^\m}\EE_{\h\sim\AQ}\ln\LB\frac{\AQ(\h)}{\P(\h)}\RB +\EE_{\S\sim\D^\m}\EE_{\h\sim\AQ}\ln\LB\frac{\P(\h)}{\AQ(\h)}\varphi(\h, \S)\RB\\
&= \EE_{\S\sim\D^\m}\KL(\AQ\|\P) +\EE_{\S\sim\D^\m}\EE_{\h\sim\AQ}\ln\LB\frac{\P(\h)}{\AQ(\h)}\varphi(\h, \S)\RB.
\end{align*}
From \textsc{Jensen}'s inequality (\Cref{ap:tools:theorem:jensen}), we have for all prior $\P\in\M^{*}(\H)$
\begin{align}
    &\EE_{\S\sim\D^\m}\KL(\AQ\|\P) +\EE_{\S\sim\D^\m}\EE_{\h\sim\AQ}\ln\LB\frac{\P(\h)}{\AQ(\h)}\varphi(\h, \S)\RB\nonumber\\
    \le &\EE_{\S\sim\D^\m}\KL(\AQ\|\P) +\ln\LB\EE_{\S\sim\D^\m}\EE_{\h\sim\AQ}\frac{\P(\h)}{\AQ(\h)}\varphi(\h, \S)\RB\nonumber\\
    = &\EE_{\S\sim\D^\m}\KL(\AQ\|\P) +\ln\LB\EE_{\S\sim\D^\m}\EE_{\h\sim\P}\varphi(\h, \S)\RB.\label{chap:dis-pra:eq:mutual-info-1}
\end{align}
Since we assume in this case that $\varphi(\h, \S) \ge 1$ for all $h\in\H$ and $\S\in(\X{\times}\Y)^\m$, we have $\ln\varphi(\h, \S) \ge 0$; we can apply \textsc{Markov}'s inequality (\Cref{ap:tools:theorem:first-markov}) to obtain
\begin{align}
    \PP_{\S\sim\D^{\m}, \h\sim \AQ}\LB \ln\varphi(\h,\!\S)\le \frac{1}{\delta}\EE_{\S'{\sim}\D^{\m}}\EE_{\h'{\sim} \AQ}\ln\varphi(\h, \S) \RB \ge 1-\delta.\label{chap:dis-pra:eq:mutual-info-2}
\end{align}
Then, from \Cref{chap:dis-pra:eq:mutual-info-1,chap:dis-pra:eq:mutual-info-2}, we can deduce the stated result.
\end{proof}
We are now ready to prove \Cref{chap:dis-pra:theorem:mutual-info-kl}.
\theoremmutualinfokl*
\begin{proof}
Note that the mutual information is $\MI(\h{;} \S){=} \min_{\P\in\M^{*}(\H)}\EE_{\S\sim\D^\m}\KL(\AQ\|\P)$. 
Hence, to prove \Cref{chap:dis-pra:theorem:mutual-info-kl}, we have to instantiate \Cref{chap:dis-pra:lemma:mutual-info-kl} with the optimal prior, \ie, the prior $\P$ which minimizes $\EE_{\S\sim\D^\m}\KL(\AQ\|\P)$.
The optimal prior is well-known~\citep[see, \eg,][]{Catoni2007,LeverLavioletteShaweTaylor2013}: for the sake of completeness, we derive it. First, we have
\begin{align*}
    \EE_{\S\sim\D^\m}\KL(\AQ\|\P) &= \EE_{\S\sim\D^\m}\EE_{\h\sim\AQ}\ln\frac{\AQ(\h)}{\P(\h)}\\
    &= \EE_{\S\sim\D^\m}\EE_{\h\sim\AQ}\ln\!\LB\frac{\AQ(\h)[\EE_{\S'\sim\D^\m}\Q_{\S'}(\h)]}{\P(\h)[\EE_{\S'\sim\D^\m}\Q_{\S'}(\h)]}\RB\\
    &= \EE_{\S\sim\D^\m}\EE_{\h\sim\AQ}\ln\!\LB\frac{\AQ(\h)}{\EE_{\S'\sim\D^\m}\Q_{\S'}(\h)}\RB{+}\EE_{\h\sim\AQ}\ln\!\LB\frac{\EE_{\S'\sim\D^\m}\Q_{\S'}(\h)}{\P(\h)}\RB.
\end{align*}
Hence, 

\begin{align*}
    \argmin_{\P\in\M^{*}(\H)}\EE_{\S\sim\D^\m}\KL(\AQ\|\P)= &\argmin_{\P\in\M^{*}(\H)} \Bigg[\EE_{\S\sim\D^\m}\EE_{\h\sim\AQ}\ln\LB\frac{\AQ(\h)}{\EE_{\S'\sim\D^\m}\Q_{\S'}(\h)}\RB\\
    &\hspace{1.5cm}+ \EE_{\h\sim\AQ}\ln\LB\frac{\EE_{\S'\sim\D^\m}\Q_{\S'}(\h)}{\P(\h)}\RB\Bigg]\\
    =&\argmin_{\P\in\M^{*}(\H)}\LB \EE_{\h\sim\AQ}\ln\LB\frac{\EE_{\S'\sim\D^\m}\Q_{\S'}(\h)}{\P(\h)}\RB \RB=\P^*,
\end{align*}
where $\P^*(\h) = \EE_{\S'\sim\D^\m}\Q_{\S'}(\h)$.
Note that $\P^*$ is defined from the data distribution $\D$, hence, $\P^*$ is a valid prior when instantiating \Cref{chap:dis-pra:lemma:mutual-info-kl} with $\P^*$.
Then, we have with probability at least $1{-}\delta$ over $\S\sim\D^\m$ and $\h\sim\AQ$
\begin{align*}
    \ln\varphi(\h,\!\S) &\le \frac{1}{\delta}\LB\EE_{\S\sim\D^\m}\KL(\AQ\|\P^*) +\ln\LP\EE_{\S\sim\D^\m}\EE_{\h\sim\P^*}\varphi(\h, \S)\RP\RB\\
    &= \frac{1}{\delta}\LB \MI(\h{;} \S) +\ln\LP\EE_{\S\sim\D^\m}\EE_{\h\sim\P^*}\varphi(\h, \S)\RP\RB.
\end{align*}
\end{proof}

\section{Proof of \Cref{chap:dis-pra:theorem:mutual-info}}
\label{ap:dis-pra:sec:proof-mutual-info}

We first introduce \Cref{chap:dis-pra:lemma:mutual-info} in order to prove \Cref{chap:dis-pra:theorem:mutual-info}.
\begin{lemma} For any distribution $\D$ on $\X{\times}\Y$, for any hypothesis set $\H$, for any prior distribution $\P$ on $\H$, for any  measurable function $\varphi:\H\times (\X{\times}\Y)^{\m}$, for any $\lambda>1$, for any $\delta\in(0,1]$, for any deterministic algorithm $A:(\X{\times}\Y)^{\m}\times\M^{*}(\H){\rightarrow} \M(\H)$, we have
\begin{align*}
    \PP_{\S\sim\D^{\m}, \h\sim \AQ}\!\!\Bigg[\forall\P{\in}\M^{*}(\H),& \displaystyle\frac{\lambda}{\lambda{-}1}\!\ln\!\LP\varphi(\h,\!\S)\RP \le \Renyi_{\lambda}(\rho\|\pi)\\
    &+\ln\!\LP\!\tfrac{1}{\delta^{\frac{\lambda}{\lambda{-}1}}}\EE_{\S'{\sim}\D^{\m}}\EE_{\h'{\sim} \P}\LP\varphi(\h'\!, \S')^{\frac{\lambda}{\lambda-1}}\RP\!\RP \Bigg]\!\!\ge 1{-}\delta.
\end{align*}
\textit{where $\rho(\h, \S){=} \AQ(\h)\D^{\m}(\S)$; $\pi(\h, \S){=} \P(\h)\D^{\m}(\S)$.}
\label{chap:dis-pra:lemma:mutual-info}
\end{lemma}
\begin{proof}
Note that $\varphi(\h,\!\S)$ is a non-negative random variable. 
From \textsc{Markov}'s inequality (\Cref{ap:tools:theorem:first-markov}), we have
\begin{align*}
    \PP_{\S\sim\D^{\m}, \h\sim \AQ}\LB \varphi(\h,\!\S)\le \frac{1}{\delta}\EE_{\S'{\sim}\D^{\m}}\EE_{\h'{\sim} \AQ}\varphi(\h'\!, \S') \RB \ge 1-\delta.
\end{align*}
Then, since both sides of the inequality are strictly positive, we take the logarithm to both sides of the equality and multiply by $\frac{\lambda}{\lambda-1}>0$ to obtain
\begin{align*}
\PP_{\S\sim\D^{\m},\h\sim \AQ}\LB\frac{\lambda}{\lambda-1}\ln\LP\varphi(\h,\!\S)\RP \le \frac{\lambda}{\lambda{-}1}\ln\LP\frac{1}{\delta}\EE_{\S'{\sim}\D^{\m}}\EE_{\h'{\sim} \AQp }\varphi(\h'\!, \S')\RP\RB\ge 1-\delta.
\end{align*}
We develop the right-hand side of the inequality in the indicator function and make the expectation of the hypothesis over the distribution $\P$ appear.
We have for all priors $\P{\in}\M^{*}(\H)$,
\begin{align*}
&\frac{\lambda}{\lambda{-}1}\ln\LP\frac{1}{\delta}\EE_{\S'{\sim}\D^{\m}}\EE_{\h'{\sim} \AQp }\varphi(\h'\!, \S')\RP\\
= &\frac{\lambda}{\lambda{-}1}\ln\LP\frac{1}{\delta}\EE_{\S'{\sim}\D^{\m}}\EE_{\h'{\sim} \AQp }\frac{\AQp (\h')}{\P(\h')}\frac{\P(\h')}{\AQp (\h')}\varphi(\h'\!, \S')\RP\\
= &\frac{\lambda}{\lambda{-}1}\ln\LP\frac{1}{\delta}\EE_{\S'{\sim}\D^{\m}}\EE_{\h'{\sim} \P}\frac{\AQp (\h')}{\P(\h')}\varphi(\h'\!, \S')\RP.
\end{align*}
Then, since $\tfrac{1}{r}+\tfrac{1}{s}=1$ where $r{=}\lambda$ and $s{=}\frac{\lambda}{\lambda-1}$. Hence, \textsc{Hölder}'s inequality (\Cref{ap:tools:theorem:holder}) gives
\begin{align*}
    \EE_{\S'{\sim}\D^{\m}}\EE_{\h'{\sim} \AQp}\varphi(\h'\!, \S'){\le}\!\LB\EE_{\S'{\sim}\D^{\m}}\EE_{\h'{\sim} \P}\!\LP\Bigg[\frac{\AQp (\h')}{\P(\h')}\Bigg]^{\lambda}\RP\!\RB^{\frac{1}{\lambda}}\!\!\LB\EE_{\S'{\sim}\D^{\m}}\EE_{\h'{\sim} \P}\!\LP\varphi(\h'\!, \S')^{\frac{\lambda}{\lambda-1}}\RP\RB^{\frac{\lambda-1}{\lambda}}\!\!\!.
\end{align*}
Since both sides of the inequality are positive, we take the logarithm.
Moreover, we add $\ln(\tfrac{1}{\delta})$, and we multiply by $\frac{\lambda}{\lambda-1}>0$ to both sides of the inequality.
We have
\begin{align*}
    &\frac{\lambda}{\lambda{-}1}\ln\LP\frac{1}{\delta}\EE_{\S'{\sim}\D^{\m}}\EE_{\h'{\sim} \AQp }\varphi(\h'\!, \S')\RP \\
    \le &\frac{\lambda}{\lambda{-}1}\ln\LP\frac{1}{\delta}\LB\EE_{\S'{\sim}\D^{\m}}\EE_{\h'{\sim} \P}\LP\Bigg[\frac{\AQp (\h')}{\P(\h')}\Bigg]^{\lambda}\RP\RB^{\frac{1}{\lambda}}\LB\EE_{\S'{\sim}\D^{\m}}\EE_{\h'{\sim} \P}\LP\varphi(\h'\!, \S')^{\frac{\lambda}{\lambda-1}}\RP\RB^{\frac{\lambda-1}{\lambda}}\RP\\
    = &\frac{1}{\lambda{-}1}\ln\LP\EE_{\S'{\sim}\D^{\m}}\EE_{\h'{\sim} \P}\LP\Bigg[\frac{\AQp (\h')}{\P(\h')}\Bigg]^{\lambda}\RP\RP + \ln\LP\frac{1}{\delta^{\frac{\lambda}{\lambda{-}1}}}\EE_{\S'{\sim}\D^{\m}}\EE_{\h'{\sim} \P}\LP\varphi(\h'\!, \S')^{\frac{\lambda}{\lambda-1}}\RP\RP\!.
\end{align*}
Hence, we can deduce that 
\begin{align*}
    \PP_{\S\sim\D^{\m}, \h\sim \AQ}\Bigg[&\forall\P{\in}\M^{*}(\H), \frac{\lambda}{\lambda{-}1}\!\ln\!\LP\varphi(\h,\!\S)\RP \le \frac{1}{\lambda{-}1}\!\ln\!\LP\EE_{\S'{\sim}\D^{\m}}\EE_{\h'{\sim} \P}\!\LP\Bigg[\!\frac{\AQp (\h')}{\P(\h')}\!\Bigg]^{\lambda}\RP\RP\\
    &+ \ln\!\LP\!\tfrac{1}{\delta^{\frac{\lambda}{\lambda{-}1}}}\EE_{\S'{\sim}\D^{\m}}\EE_{\h'{\sim} \P}\LP\varphi(\h'\!, \S')^{\frac{\lambda}{\lambda-1}}\RP\!\RP \Bigg]\ge 1{-}\delta,
\end{align*}
where by definition we have $\Renyi_{\lambda}(\rho\|\pi)=\frac{1}{\lambda{-}1}\!\ln\!\LP\EE_{\S'{\sim}\D^{\m}}\EE_{\h'{\sim} \P}\!\LP\LB\!\frac{\AQp (\h')}{\P(\h')}\!\RB^{\lambda}\RP\RP$.
\end{proof}
From \Cref{chap:dis-pra:lemma:mutual-info}, we prove \Cref{chap:dis-pra:theorem:mutual-info}.

\theoremmutualinfo*
\begin{proof}
Sibson's mutual information is $\MI_{\lambda}(\h{;}\S)=\min_{\P\in\M^{*}(\H)}\Renyi_{\lambda}(\rho\|\pi)$.
Hence, in order to prove \Cref{chap:dis-pra:theorem:mutual-info}, we have to instantiate \Cref{chap:dis-pra:lemma:mutual-info} with the optimal prior, \ie, the prior $\P$ which minimizes $\Renyi_{\lambda}(\rho\|\pi)$.
Actually, this optimal prior has a closed-form solution~\citep{Verdu2015}.
For the sake of completeness, we derive it. First, we have
\begin{align*}
    &\Renyi_{\lambda}(\rho\|\pi)\\
    = &\frac{1}{\lambda{-}1}\!\ln\!\LP\EE_{\S\sim\D^{\m}}\EE_{\h\sim \P}\!\LP\LB\!\frac{\AQ(\h)}{\P(\h)}\!\RB^{\lambda}\RP\RP\\
    = &\frac{1}{\lambda{-}1}\!\ln\!\LP\EE_{\h\sim \P}\LB\EE_{\S\sim\D^{\m}}\LP\AQ(\h)^{\lambda}\RP\RB\LP\P(\h)^{-\lambda}\RP\!\RP\\
    = &\frac{1}{\lambda{-}1}\!\ln\!\LP\!\EE_{\h\sim \P}\LB\EE_{\S\sim\D^{\m}}\LP\AQ (\h)^{\lambda}\RP\RB\LP\P(\h)^{-\lambda}\RP\!\!\LB\tfrac{\EE_{\h'{\sim}\P}\tfrac{1}{\P(\h')}\LB\EE_{\S'{\sim}\D^{\m}}\LP\AQp (\h')^{\lambda}\RP\RB^{\frac{1}{\lambda}}}{\EE_{\h'{\sim}\P}\tfrac{1}{\P(\h')}\LB\EE_{\S'{\sim}\D^{\m}}\LP\AQp (\h')^{\lambda}\RP\RB^{\frac{1}{\lambda}}}\RB^{\!\lambda}\RP\\
    = &\frac{\lambda}{\lambda{-}1}\!\ln\!\LP\EE_{\h'{\sim}\P}\!\tfrac{1}{\P(\h')}\!\!\LB\EE_{\S'{\sim}\D^{\m}}\LP\AQp (\h')^{\lambda}\RP\RB^{\!\frac{1}{\lambda}}\!\RP\!\\
    &\hspace{0.1cm}+\frac{1}{\lambda{-}1}\!\ln\!\LP\EE_{\h\sim\P}\tfrac{1}{\P(\h)^{\lambda}}\!\!\LB \!\tfrac{\LB\EE_{\S\sim\D^{\m}}\LP\AQ (\h)^{\lambda}\RP\RB^{\frac{1}{\lambda}}}{\EE_{\h'{\sim}\P}\!\!\tfrac{1}{\P(\h')}\LB\EE_{\S'{\sim}\D^{\m}}\LP\AQp (\h')^{\lambda}\RP\RB^{\frac{1}{\lambda}}}\!\RB^{\!\lambda} \RP\\
    = &\frac{\lambda}{\lambda{-}1}\ln\LP\EE_{\h'{\sim}\P}\tfrac{1}{\P(\h')}\LB\EE_{\S'{\sim}\D^{\m}}\LP\AQp (\h')^{\lambda}\RP\RB^{\frac{1}{\lambda}}\RP + \Renyi_{\lambda}(\P^{*}\| \P),
\end{align*}
where $\P^*(\h)=\LB \!\tfrac{\LB\EE_{\S\sim\D^{\m}}\LP\AQ (\h)^{\lambda}\RP\RB^{\frac{1}{\lambda}}}{\EE_{\h'{\sim}\P}\tfrac{1}{\P(\h')}\LB\EE_{\S'{\sim}\D^{\m}}\LP\AQp (\h')^{\lambda}\RP\RB^{\frac{1}{\lambda}}}\!\RB$.

From these equalities and using the fact that $\Renyi_{\lambda}(\P^*\| \P)$ is minimal (\ie, equal to zero) when $\P^*=\P$, we can deduce that
\begin{align*}
    &\argmin_{\P\in\M^{*}(\H)}\Renyi_{\lambda}(\rho\|\pi)\\
    {=} &\argmin_{\P\in\M^{*}(\H)} \LB\frac{\lambda}{\lambda{-}1}\ln\LP\EE_{\h'{\sim}\P}\tfrac{1}{\P(\h')}\LB\EE_{\S'{\sim}\D^{\m}}\LP\AQp (\h')^{\lambda}\RP\RB^{\frac{1}{\lambda}}\RP{+} \Renyi_{\lambda}(\P^{*}\| \P)\RB\\
    {=}&\argmin_{\P\in\M^{*}(\H)}\Renyi_{\lambda}(\P^{*}\| \P){=}\P^*.
\end{align*}
Note that $\P^*$ is defined from the data distribution $\D$, hence, $\P^*$ is a valid prior when instantiating \Cref{chap:dis-pra:lemma:mutual-info} with $\P^*$.
Then, we have with probability at least $1{-}\delta$ over $\S\sim\D^\m$ and $\h\sim\AQ$
\begin{align*}
    \frac{\lambda}{\lambda{-}1}\!\ln\!\LP\varphi(\h,\!\S)\RP &\le \Renyi_{\lambda}(\rho\|\pi^*) + \ln\!\LP\!\tfrac{1}{\delta^{\frac{\lambda}{\lambda{-}1}}}\EE_{\S'{\sim}\D^{\m}}\EE_{\h'{\sim} \P}\LP\varphi(\h'\!, \S')^{\frac{\lambda}{\lambda-1}}\RP\!\RP\\
    &= \MI_{\lambda}(\h'; \S') +\ \ln\!\LP\!\tfrac{1}{\delta^{\frac{\lambda}{\lambda{-}1}}}\EE_{\S'{\sim}\D^{\m}}\EE_{\h'{\sim} \P}\LP\varphi(\h'\!, \S')^{\frac{\lambda}{\lambda-1}}\RP\!\RP.
\end{align*}
where $\pi^*(\h, \S)=\P^*(\h)\D^{\m}(\S)$.
\end{proof}

\section{Proof of \Cref{chap:dis-pra:corollary:mutual-info}}

\corollarymutualinfo*
\begin{proof} 
The proof is similar to \Cref{chap:dis-pra:corollary:disintegrated}.
Starting from \Cref{chap:dis-pra:theorem:mutual-info} and rearranging, we have
\begin{align*}
    \PP_{\substack{\S\sim\D^{\m}\\ h\sim \AQ}}\Bigg[ &\!\ln\!\LP\varphi(\h,\!\S)\RP \le  \frac{\lambda{-}1}{\lambda}\MI_{\lambda}(\h'; \S')\\ 
    &+\ln\frac{1}{\delta} + \ln\!\LP\LB\EE_{\S'{\sim}\D^{\m}}\EE_{\h'{\sim} \P^*}\LP\varphi(\h'\!, \S')^{\frac{\lambda}{\lambda-1}}\RP\RB^{\frac{\lambda{-}1}{\lambda}}\RP \Bigg]\ge 1{-}\delta,
\end{align*}
Then, we will prove separately the case when $\lambda\rightarrow 1$ and $\lambda\rightarrow +\infty$.

\paragraph{When $\lambda\rightarrow 1$.} We have  $\lim_{\lambda\rightarrow 1^+}\frac{\lambda{-}1}{\lambda}\MI_{\lambda}(\h'; \S') = 0$.
Furthermore, note that 
\begin{align*}
    \|\varphi\|_{\frac{\lambda}{\lambda{-}1}} = \LB\EE_{\S'{\sim}\D^{\m}}\EE_{\h'{\sim}\P^*}\LP\vert\varphi(\h'\!, \S')\vert^{\frac{\lambda}{\lambda{-}1}}\RP\RB^{\frac{\lambda{-}1}{\lambda}} = \LB\EE_{\S'{\sim}\D^{\m}}\EE_{\h'{\sim}\P^*}\LP\varphi(\h'\!, \S')^{\frac{\lambda}{\lambda{-}1}}\RP\RB^{\frac{\lambda{-}1}{\lambda}}
\end{align*}
is the $L^{\frac{\lambda}{\lambda{-}1}}$-norm of the function $\varphi: \H\times(\X{\times}\Y)^\m \rightarrow \Rbb_{+}^*$, where $\lim_{\lambda\rightarrow 1} \|\varphi\|_{\frac{\lambda}{\lambda{-}1}} = \lim_{\lambda'\rightarrow +\infty} \|\varphi\|_{\lambda'}$ (since we have $\lim_{\lambda\rightarrow 1^+}\frac{\lambda}{\lambda{-}1} = (\lim_{\lambda\rightarrow1}\lambda)(\lim_{\lambda\rightarrow1}\frac{1}{\lambda{-}1}) = +\infty$).
Then, it is well known that
\begin{align*}
    \|\varphi\|_{\infty}= \lim_{\lambda'\rightarrow+\infty}\|\varphi\|_{\lambda'} = \esssup_{\S'\in(\X{\times}\Y), \h'\in\H}\varphi(\h'{,} \S').
\end{align*}
Hence, we have 
\begin{align*}
    &\lim_{\lambda\rightarrow1} \ln\LP\LB\EE_{\S'{\sim}\D^{\m}}\EE_{\h'{\sim}\P^*}\LP\varphi(\h'\!, \S')^{\frac{\lambda}{\lambda{-}1}}\RP\RB^{\frac{\lambda{-}1}{\lambda}}\RP\\
    = &\ln\LP \lim_{\lambda\rightarrow1} \LB\EE_{\S'{\sim}\D^{\m}}\EE_{\h'{\sim}\P^*}\LP\varphi(\h'\!, \S')^{\frac{\lambda}{\lambda{-}1}}\RP\RB^{\frac{\lambda{-}1}{\lambda}}\RP\\
    = &\ln\LP \lim_{\lambda\rightarrow1} \| \varphi\|_{\frac{\lambda}{\lambda-1}}\RP = \ln\LP \lim_{\lambda'\rightarrow+\infty} \| \varphi\|_{\lambda'}\RP \\
    = &\ln\LP \| \varphi\|_{\infty}\RP = \ln\LP \esssup_{\S'\in(\X{\times}\Y), \h'\in\H}\varphi(\h'{,} \S') \RP.
\end{align*}
Finally, we can deduce that 
\begin{align*}
    &\lim_{\lambda\rightarrow 1}\LB \frac{\lambda{-}1}{\lambda}\MI_{\lambda}(\h'; \S') +\ \ln\frac{1}{\delta} + \ln\!\LP\LB\EE_{\S'{\sim}\D^{\m}}\EE_{\h'{\sim} \P^*}\LP\varphi(\h'\!, \S')^{\frac{\lambda}{\lambda-1}}\RP\RB^{\frac{\lambda{-}1}{\lambda}}\RP\RB\\
    = &\ln\frac{1}{\delta} + \ln\left[\esssup_{\S'\in(\X{\times}\Y), \h'\in\H}\varphi(\h'{,} \S')\right].\\
\end{align*}

\paragraph{When \mbox{$\lambda\rightarrow +\infty$}.} First, we have $\lim_{\lambda\rightarrow +\infty} \|\varphi\|_{\frac{\lambda}{\lambda{-}1}} = \lim_{\lambda'\rightarrow 1} \|\varphi\|_{\lambda'} = \|\varphi\|_1$ 
Hence, we have 
\begin{align*}
    &\lim_{\lambda\rightarrow+\infty} \ln\LP\LB\EE_{\S'{\sim}\D^{\m}}\EE_{\h'{\sim}\P^*}\LP\varphi(\h'\!, \S')^{\frac{\lambda}{\lambda{-}1}}\RP\RB^{\frac{\lambda{-}1}{\lambda}}\RP\\
    = &\ln\LP \lim_{\lambda\rightarrow+\infty} \LB\EE_{\S'{\sim}\D^{\m}}\EE_{\h'{\sim}\P^*}\LP\varphi(\h'\!, \S')^{\frac{\lambda}{\lambda{-}1}}\RP\RB^{\frac{\lambda{-}1}{\lambda}}\RP\\
    = &\ln\LP \lim_{\lambda\rightarrow+\infty} \| \varphi\|_{\frac{\lambda}{\lambda-1}}\RP = \ln\LP \lim_{\lambda'\rightarrow1} \| \varphi\|_{\lambda'}\RP\\
    = &\ln\LP \| \varphi\|_{1}\RP = \ln\LP \EE_{\S'{\sim}\D^{\m}}\EE_{\h'{\sim}\P^*}\varphi(\h'\!, \S') \RP.
\end{align*}
Moreover, by rearranging the terms in $\frac{\lambda{-}1}{\lambda}\MI_{\lambda}(\h'; \S')$, we have
\begin{align*}
\frac{\lambda{-}1}{\lambda}\MI_{\lambda}(\h'; \S') &= \frac{1}{\lambda}\ln\!\LP \EE_{\S{\sim}\D^\m}\EE_{\h{\sim}\P^*}\!\LP\!\LB\frac{ \AQ(\h)}{\P^*(\h)}\RB^{\!\lambda}\RP\RP\\
&= \ln\!\LP \LB\EE_{\S{\sim}\D^\m}\EE_{\h{\sim}\P^*}\!\LP\LB\!\frac{ \AQ(\h)}{\P^*(\h)}\RB^{\!\lambda}\RP\RB^{\frac{1}{\lambda}}\RP\\
&= \ln\!\LP \LB\EE_{\h{\sim}\P^*}\LP\gamma(\h)^{\lambda}\RP\RB^{\frac{1}{\lambda}}\RP = \ln\!\LP \| \gamma\|_{\lambda}\RP,
\end{align*}
where $\| \gamma\|_{\lambda}$ is the $L^{\lambda}$-norm of the function $\gamma$ defined as $\gamma(\h)=\tfrac{\AQ(\h)}{\P^*(\h)}$.
We have
\begin{align*}
    \lim_{\lambda\rightarrow +\infty}  \frac{\lambda{-}1}{\lambda}\MI_{\lambda}(\h'; \S') =& \lim_{\lambda\rightarrow +\infty}  \ln\!\LP \| \gamma\|_{\lambda}\RP = \ln\LP\lim_{\lambda\rightarrow +\infty} \|\gamma\|_{\lambda}\RP\\
    =& \ln\LP \|\gamma\|_{\infty}\RP = \ln\LP\esssup_{\S\in\S, h\in\H}\gamma(\h)\RP = \ln\LP\esssup_{\S\in\S, h\in\H}\frac{\AQ(\h)}{\P^*(\h)}\RP.
\end{align*}
Finally, we can deduce that 
\begin{align*}
    & \lim_{\lambda\rightarrow 1}\LB \frac{\lambda{-}1}{\lambda}\MI_{\lambda}(\h'; \S') +\ \ln\frac{1}{\delta} + \ln\!\LP\LB\EE_{\S'{\sim}\D^{\m}}\EE_{\h'{\sim} \P^*}\LP\varphi(\h'\!, \S')^{\frac{\lambda}{\lambda-1}}\RP\RB^{\frac{\lambda{-}1}{\lambda}}\RP\RB\\ 
    =\quad &\ln\LP\esssup_{\S\in\S, h\in\H}\frac{\AQ(\h)}{\P^*(\h)}\RP{+}\ln\!\Big[\frac{1}{\delta} {\displaystyle \EE_{\S'{\sim}\D^{\m}}\EE_{\h'{\sim}\P}\varphi(\h'{,}\S')}\Big].
\end{align*}
\end{proof}

\section{Proof of \Cref{chap:dis-pra:corollary:disintegrated-riv-mv}}
\label{ap:dis-pra:sec:proof-disintegrated-riv-mv}

\corollarydisintegratedrivmv*
\begin{proof}
We apply \Cref{chap:pac-bayes:theorem:general-disintegrated-rivasplata} with $\phi(\Q,\S)=\m\kl(\RiskLoss_{\dS}(\MVQ)\|\RiskLoss_{\D}(\MVQ))$ to obtain with probability at least $1{-}\delta$ over the learning sample $\S\sim\D^\m$ and the posterior distribution $\Q\sim\hyperQ$, we have 
\begin{align}
    \kl(\RiskLoss_{\dS}(\MVQ)\|\RiskLoss_{\D}(\MVQ)){\le}\frac{1}{\m}\!\!\LB \ln\frac{\hyperQ(\Q)}{\hyperP(\Q)}{+}\ln\!\LB\frac{1}{\delta} \!\EE_{\Qp\sim\hyperP}\!e^{\m\kl(\RiskLoss_{\dS}(\MVQp)\|\RiskLoss_{\D}(\MVQp))}\RB\!\RB.\label{ap:dis-pra:eq:proof-disintegrated-riv-mv-1}
\end{align}
Moreover, the closed form solution of the disintegrated KL divergence $\ln\frac{\hyperQ(\Q)}{\hyperP(\Q)}$ is
\begin{align}
    \ln\frac{\hyperQ(\Q)}{\hyperP(\Q)} &= \ln(\hyperQ(\Q)) - \ln(\hyperP(\Q))\nonumber\\
    &= \ln\LP\frac{1}{Z(\paramDir)}\prod_{j=1}^{\card(\H)} \Big[\Q(\h_j)\Big]^{\sparamDir_j-1}\RP - \ln\LP\frac{1}{Z(\paramDirP)}\prod_{j=1}^{\card(\H)} \Big[\Q(\h_j)\Big]^{\sparamDirP_j-1}\RP\nonumber\\
    &= \ln\frac{Z(\paramDirP)}{Z(\paramDir)} + \sum_{j=1}^{\card(\H)}(\sparamDir_j-1)\ln(\Q(\h_j)) - \sum_{j=1}^{\card(\H)}(\sparamDirP_j-1)\ln(\Q(\h_j))\nonumber\\
    &= \ln\frac{Z(\paramDirP)}{Z(\paramDir)} + \sum_{j=1}^{\card(\H)}(\sparamDir_j-\sparamDirP_j)\ln(\Q(\h_j)).\label{ap:dis-pra:eq:proof-disintegrated-riv-mv-2}
\end{align}
Additionally, from \Cref{ap:pac-bayes:lemma:2-sqrt-m}, we have 
\begin{align}
    \EE_{\Qp\sim\hyperP}e^{\m\kl(\RiskLoss_{\dS}(\MVQp)\|\RiskLoss_{\D}(\MVQp))} \le 2\sqrt{\m}.\label{ap:dis-pra:eq:proof-disintegrated-riv-mv-3}
\end{align}
Lastly, by merging \Cref{ap:dis-pra:eq:proof-disintegrated-riv-mv-1,ap:dis-pra:eq:proof-disintegrated-riv-mv-2,ap:dis-pra:eq:proof-disintegrated-riv-mv-3} we obtain the claim.
\end{proof}

\section{Proof of \Cref{chap:dis-pra:corollary:disintegrated-mv}}
\label{ap:dis-pra:sec:proof-disintegrated-mv}

\corollarydisintegratedmv*
\begin{proof}
We apply \Cref{chap:dis-pra:theorem:disintegrated} with $\phi(\Q,\S)=\exp\LB\frac{\lambda-1}{\lambda}\m\kl(\RiskLoss_{\dS}(\MVQ)\|\RiskLoss_{\D}(\MVQ))\RB$ to obtain with probability at least $1{-}\delta$ over the learning sample $\S\sim\D^\m$ and the posterior distribution $\Q\sim\hyperQ$, we have 

\begin{align}
    \kl(\RiskLoss_{\dS}(\MVQ)\|\RiskLoss_{\D}(\MVQ)) \le &\frac{1}{\m}\Bigg[{\frac{2\lambda{-}1}{\lambda{-}1}}\ln\frac{2}{\delta}
 +\Renyi_{\lambda}(\hyperQ\|\hyperP)\nonumber\\
 &+ \ln\LP\EE_{\Qp\sim\hyperP}e^{\m\kl(\RiskLoss_{\dS}(\MVQp)\|\RiskLoss_{\D}(\MVQp))}\RP \Bigg].\label{ap:dis-pra:eq:proof-disintegrated-mv-1}
\end{align}

Moreover, the closed form solution of the \textsc{Rényi} divergence $\Renyi_{\lambda}(\hyperQ\|\hyperP)$~\citep{GilAlajajiLinder2013} is given by
\begin{align}
    \Renyi_{\lambda}(\hyperQ\|\hyperP) &= \frac{1}{\lambda{-}1}\ln\LP\int_{\M(\H)}\hyperQ(\Q)^\lambda\hyperP(\Q)^{1-\lambda}d\xi(\Q)\RP\nonumber\\
    &= \frac{1}{\lambda{-}1}\ln\Bigg(\int_{\M(\H)}\frac{1}{Z(\paramDir)^\lambda}\prod_{j=1}^{\card(\H)}(\Q(\h_j))^{\lambda(\sparamDir_j-1)}\nonumber\\
    &\hspace{2cm}\frac{1}{Z(\paramDirP)^{1-\lambda}}\prod_{j=1}^{\card(\H)}(\Q(\h_j))^{(1-\lambda)(\sparamDirP_j-1)} d\xi(\Q)\Bigg)\nonumber\\
    &= \frac{1}{\lambda{-}1}\ln\LP \frac{Z(\paramDirP)^{\lambda-1}}{Z(\paramDir)^{\lambda}}\RP\nonumber\\
    &\hspace{0.5cm}+ \frac{1}{\lambda{-}1}\ln\LP\int_{\M(\H)}\prod_{i=1}^{\card(\H)} (\Q(\h_j))^{\lambda\sparamDir_j + (1-\lambda)\sparamDirP_j-1} d\xi(\Q)\!\RP\nonumber\\
    &= \frac{1}{\lambda{-}1}\ln\LP \frac{Z(\paramDirP)^{\lambda-1}}{Z(\paramDir)^{\lambda}}\RP + \ln Z(\lambda\paramDir{+}(1{-}\lambda)\paramDirP)\nonumber\\
    &= \frac{1}{\lambda{-}1}\ln\LP \frac{Z(\paramDirP)^{\lambda-1}}{Z(\paramDir)^{\lambda-1+1}}\RP + \ln Z(\lambda\paramDir{+}(1{-}\lambda)\paramDirP)\nonumber\\
    &= \ln\frac{Z(\paramDirP)}{Z(\paramDir)} + \frac{1}{\lambda{-}1}\ln\frac{Z(\lambda\paramDir{+}(1{-}\lambda)\paramDirP)}{Z(\paramDir)},\label{ap:dis-pra:eq:proof-disintegrated-mv-2}
\end{align}
where $\xi$ is the reference measure on $\M(\H)$.
Additionally, from \Cref{ap:pac-bayes:lemma:2-sqrt-m}, we have 
\begin{align}
    \EE_{\Qp\sim\hyperP}e^{\m\kl(\RiskLoss_{\dS}(\MVQp)\|\RiskLoss_{\D}(\MVQp))} \le 2\sqrt{\m}.\label{ap:dis-pra:eq:proof-disintegrated-mv-3}
\end{align}
Lastly, by merging \Cref{ap:dis-pra:eq:proof-disintegrated-mv-1,ap:dis-pra:eq:proof-disintegrated-mv-2,ap:dis-pra:eq:proof-disintegrated-mv-3} we obtain the claim.
\end{proof}

\section{Details of the Results}
\Cref{chap:dis-pra:table:1_prior_0.1} to~\Cref{chap:dis-pra:table:1_prior_0.9} report empirical results for split ratios going from 0.0 to 0.9.
\Cref{chap:dis-pra:table:2_data_mnist} to~\Cref{chap:dis-pra:table:2_data_cifar10} report the performances of the prior before applying Step {\bf 2)}. 
\\

For the split 0.0, since Step {\bf 1)} is skipped, the prior distribution $\P$ is only initialized as introduced in \Cref{chap:dis-pra:sec:models}. 
Note that in this case, $\iter=1$ since we have only one prior.
To do the same number of epochs compared to the other splits, we perform 11 epochs (instead of 1) for MNIST and Fashion-MNIST and 110 epochs (instead of 10) for CIFAR-10 during Step {\bf 2)}.
The other parameters are not changed.

\begin{landscape}
\begin{table}[t]
\caption{
\looseness=-1
Comparison of \algoours, \algorivasplata, \algoblanchard and \algocatoni based on the disintegrated bounds, and \algostoNN based on the randomized bounds learned with two learning rates $\lr{\ \in}\{10^{-4}, 10^{-6}\}$ and different variances $\sigma^2{\in}\{10^{-3}, 10^{-4}, 10^{-5}, 10^{-6}\}$.
We report the test risk ($\Risk_{\dT}(\h)$), the bound value (Bnd), the empirical risk ($\Risk_{\dS}(\h)$), and the divergence (Div) associated with each bound (the \textsc{Rényi} divergence for \algoours, the KL divergence for \algostoNN, and the disintegrated KL divergence for \algorivasplata, \algoblanchard and \algocatoni).
More precisely, we report the mean $\pm$ the standard deviation for $400$ neural networks sampled from $\AQ$ for \algoours, \algorivasplata, \algoblanchard, and \algocatoni.
We consider, in this figure, that the split ratio is $0.0$.
}
\resizebox{0.63\paperheight}{!}{
\begin{tabular}{rr|clcl|clcl|clcl|clcl}
\toprule
 &  & \multicolumn{4}{c}{$\sigma^2=10^{-6}$} & \multicolumn{4}{c}{$\sigma^2=10^{-5}$} & \multicolumn{4}{c}{$\sigma^2=10^{-4}$} & \multicolumn{4}{c}{$\sigma^2=10^{-3}$} \\
\midrule
 & $\lr=10^{-6}$ & $\Risk_{\Tcal}(h)$ & Bnd & $\Risk_{\Scal}(h)$ & Div & $\Risk_{\Tcal}(h)$ & Bnd & $\Risk_{\Scal}(h)$ & Div & $\Risk_{\Tcal}(h)$ & Bnd & $\Risk_{\Scal}(h)$ & Div & $\Risk_{\Tcal}(h)$ & Bnd & $\Risk_{\Scal}(h)$ & Div \\
\midrule
\multirow[c]{5}{*}{\rotatebox[origin=c]{90}{\small{MNIST}}} & \algoours & .901 $\pm$ .002 & .908 $\pm$ .002 & .901 $\pm$ .002 & .005 & .897 $\pm$ .013 & .904 $\pm$ .012 & .897 $\pm$ .012 & .009 & .898 $\pm$ .017 & .905 $\pm$ .016 & .898 $\pm$ .016 & .027 & .902 $\pm$ .015 & .908 $\pm$ .014 & .901 $\pm$ .015 & .671 \\
 & \algoblanchard & .901 $\pm$ .002 & .926 $\pm$ .002 & .901 $\pm$ .002 & 122.846 $\pm$ 15.952 & .897 $\pm$ .013 & .912 $\pm$ .012 & .897 $\pm$ .013 & 39.350 $\pm$ 8.999 & .898 $\pm$ .017 & .907 $\pm$ .016 & .898 $\pm$ .017 & 13.023 $\pm$ 4.818 & .901 $\pm$ .015 & .907 $\pm$ .014 & .901 $\pm$ .014 & 3.041 $\pm$ 2.459 \\
 & \algocatoni & .901 $\pm$ .002 & .926 $\pm$ .003 & .901 $\pm$ .002 & 121.860 $\pm$ 15.930 & .897 $\pm$ .013 & .909 $\pm$ .012 & .897 $\pm$ .013 & 38.552 $\pm$ 8.872 & .898 $\pm$ .017 & .905 $\pm$ .016 & .898 $\pm$ .017 & 12.474 $\pm$ 4.774 & .901 $\pm$ .014 & .906 $\pm$ .013 & .901 $\pm$ .014 & 3.088 $\pm$ 2.379 \\
 & \algorivasplata & .901 $\pm$ .002 & .920 $\pm$ .002 & .901 $\pm$ .002 & 123.301 $\pm$ 15.941 & .896 $\pm$ .014 & .908 $\pm$ .012 & .896 $\pm$ .013 & 39.195 $\pm$ 8.959 & .897 $\pm$ .017 & .905 $\pm$ .016 & .897 $\pm$ .017 & 12.827 $\pm$ 4.858 & .902 $\pm$ .015 & .907 $\pm$ .014 & .901 $\pm$ .015 & 3.232 $\pm$ 2.454 \\
 & \algostoNN & \textemdash & .944 & \textemdash & .002 & \textemdash & .941 & \textemdash & .004 & \textemdash & .941 & \textemdash & .014 & \textemdash & .944 & \textemdash & .336 \\
\midrule
\multirow[c]{5}{*}{\rotatebox[origin=c]{90}{\small{Fashion}}} & \algoours & .970 $\pm$ .028 & .972 $\pm$ .025 & .970 $\pm$ .027 & .016 & .944 $\pm$ .038 & .949 $\pm$ .035 & .944 $\pm$ .037 & .046 & .910 $\pm$ .027 & .917 $\pm$ .026 & .910 $\pm$ .027 & .140 & .901 $\pm$ .026 & .909 $\pm$ .025 & .901 $\pm$ .026 & 1.255 \\
 & \algoblanchard & .970 $\pm$ .029 & .978 $\pm$ .019 & .970 $\pm$ .028 & 122.508 $\pm$ 16.085 & .942 $\pm$ .038 & .952 $\pm$ .032 & .943 $\pm$ .038 & 39.957 $\pm$ 8.610 & .910 $\pm$ .031 & .919 $\pm$ .029 & .910 $\pm$ .031 & 12.649 $\pm$ 4.846 & .899 $\pm$ .028 & .905 $\pm$ .027 & .899 $\pm$ .028 & 3.206 $\pm$ 2.566 \\
 & \algocatoni & .970 $\pm$ .028 & .983 $\pm$ .017 & .970 $\pm$ .027 & 122.364 $\pm$ 15.860 & .945 $\pm$ .038 & .954 $\pm$ .036 & .945 $\pm$ .037 & 38.555 $\pm$ 8.873 & .912 $\pm$ .032 & .919 $\pm$ .031 & .912 $\pm$ .032 & 12.167 $\pm$ 4.762 & .899 $\pm$ .027 & .905 $\pm$ .026 & .899 $\pm$ .027 & 3.122 $\pm$ 2.392 \\
 & \algorivasplata & .970 $\pm$ .028 & .977 $\pm$ .021 & .971 $\pm$ .027 & 123.328 $\pm$ 15.929 & .943 $\pm$ .038 & .950 $\pm$ .033 & .943 $\pm$ .038 & 39.300 $\pm$ 8.991 & .908 $\pm$ .031 & .916 $\pm$ .029 & .908 $\pm$ .031 & 12.627 $\pm$ 4.890 & .899 $\pm$ .028 & .905 $\pm$ .027 & .899 $\pm$ .028 & 3.591 $\pm$ 2.610 \\
 & \algostoNN & \textemdash & .990 & \textemdash & .008 & \textemdash & .975 & \textemdash & .023 & \textemdash & .950 & \textemdash & .070 & \textemdash & .944 & \textemdash & .627 \\
\midrule
\multirow[c]{5}{*}{\rotatebox[origin=c]{90}{\small{CIFAR-10}}} & \algoours & .899 $\pm$ .000 & .907 $\pm$ .000 & .899 $\pm$ .000 & 3.113 & .896 $\pm$ .002 & .914 $\pm$ .002 & .894 $\pm$ .002 & 107.797 & .826 $\pm$ .011 & .885 $\pm$ .009 & .825 $\pm$ .010 & 76.475 & .786 $\pm$ .019 & .851 $\pm$ .015 & .788 $\pm$ .018 & 714.351 \\
 & \algoblanchard & .899 $\pm$ .000 & .940 $\pm$ .001 & .898 $\pm$ .000 & 314.983 $\pm$ 26.377 & .888 $\pm$ .004 & .927 $\pm$ .002 & .885 $\pm$ .003 & 28.250 $\pm$ 25.255 & .823 $\pm$ .010 & .885 $\pm$ .008 & .822 $\pm$ .010 & 422.401 $\pm$ 29.323 & .798 $\pm$ .019 & .856 $\pm$ .015 & .799 $\pm$ .018 & 292.706 $\pm$ 25.318 \\
 & \algocatoni & .899 $\pm$ .000 & .941 $\pm$ .000 & .898 $\pm$ .000 & 285.415 $\pm$ 25.085 & .894 $\pm$ .002 & .930 $\pm$ .004 & .892 $\pm$ .002 & 169.713 $\pm$ 19.543 & .857 $\pm$ .010 & .915 $\pm$ .009 & .856 $\pm$ .010 & 273.554 $\pm$ 23.212 & .815 $\pm$ .019 & .864 $\pm$ .017 & .816 $\pm$ .018 & 209.069 $\pm$ 21.230 \\
 & \algorivasplata & .899 $\pm$ .001 & .930 $\pm$ .001 & .898 $\pm$ .000 & 362.070 $\pm$ 28.420 & .864 $\pm$ .004 & .933 $\pm$ .002 & .862 $\pm$ .004 & 1568.007 $\pm$ 55.492 & .748 $\pm$ .010 & .837 $\pm$ .007 & .750 $\pm$ .009 & 1219.178 $\pm$ 49.610 & .769 $\pm$ .018 & .828 $\pm$ .015 & .771 $\pm$ .017 & 526.068 $\pm$ 33.837 \\
 & \algostoNN & \textemdash & .942 & \textemdash & 1.557 & \textemdash & .945 & \textemdash & 53.898 & \textemdash & .914 & \textemdash & 38.237 & \textemdash & .884 & \textemdash & 357.175 \\
\midrule
 &  & \multicolumn{4}{c}{$\sigma^2=10^{-6}$} & \multicolumn{4}{c}{$\sigma^2=10^{-5}$} & \multicolumn{4}{c}{$\sigma^2=10^{-4}$} & \multicolumn{4}{c}{$\sigma^2=10^{-3}$} \\
\midrule
 & $\lr=10^{-4}$ & $\Risk_{\Tcal}(h)$ & Bnd & $\Risk_{\Scal}(h)$ & Div & $\Risk_{\Tcal}(h)$ & Bnd & $\Risk_{\Scal}(h)$ & Div & $\Risk_{\Tcal}(h)$ & Bnd & $\Risk_{\Scal}(h)$ & Div & $\Risk_{\Tcal}(h)$ & Bnd & $\Risk_{\Scal}(h)$ & Div \\
\midrule
\multirow[c]{5}{*}{\rotatebox[origin=c]{90}{\small{MNIST}}} & \algoours & .901 $\pm$ .002 & .909 $\pm$ .002 & .901 $\pm$ .002 & 3.767 & .896 $\pm$ .014 & .904 $\pm$ .013 & .896 $\pm$ .014 & .835 & .898 $\pm$ .016 & .905 $\pm$ .015 & .898 $\pm$ .016 & 1.062 & .901 $\pm$ .015 & .909 $\pm$ .014 & .901 $\pm$ .015 & 6.022 \\
 & \algoblanchard & .900 $\pm$ .003 & .990 $\pm$ .000 & .900 $\pm$ .003 & 12004.196 $\pm$ 152.632 & .894 $\pm$ .017 & .986 $\pm$ .006 & .894 $\pm$ .016 & 3837.785 $\pm$ 93.560 & .888 $\pm$ .021 & .957 $\pm$ .013 & .888 $\pm$ .020 & 1221.198 $\pm$ 49.920 & .898 $\pm$ .015 & .939 $\pm$ .012 & .897 $\pm$ .015 & 391.343 $\pm$ 28.182 \\
 & \algocatoni & .900 $\pm$ .003 & .997 $\pm$ .002 & .900 $\pm$ .003 & 5694.194 $\pm$ 102.906 & .889 $\pm$ .020 & .967 $\pm$ .012 & .889 $\pm$ .019 & 3331.617 $\pm$ 78.945 & .879 $\pm$ .025 & .941 $\pm$ .016 & .880 $\pm$ .025 & 1481.726 $\pm$ 53.973 & .888 $\pm$ .023 & .937 $\pm$ .015 & .888 $\pm$ .023 & 567.893 $\pm$ 33.441 \\
 & \algorivasplata & .900 $\pm$ .004 & .990 $\pm$ .000 & .900 $\pm$ .003 & 1199.818 $\pm$ 152.557 & .892 $\pm$ .017 & .970 $\pm$ .009 & .892 $\pm$ .016 & 3846.699 $\pm$ 84.643 & .886 $\pm$ .020 & .940 $\pm$ .015 & .886 $\pm$ .020 & 1224.463 $\pm$ 49.970 & .897 $\pm$ .018 & .928 $\pm$ .015 & .897 $\pm$ .018 & 393.757 $\pm$ 29.158 \\
 & \algostoNN & \textemdash & .944 & \textemdash & 1.884 & \textemdash & .940 & \textemdash & .417 & \textemdash & .941 & \textemdash & .531 & \textemdash & .944 & \textemdash & 3.011 \\
\midrule
\multirow[c]{5}{*}{\rotatebox[origin=c]{90}{\small{Fashion}}} & \algoours & .977 $\pm$ .024 & .979 $\pm$ .021 & .977 $\pm$ .023 & 3.926 & .947 $\pm$ .038 & .951 $\pm$ .035 & .947 $\pm$ .038 & 1.623 & .907 $\pm$ .030 & .914 $\pm$ .029 & .907 $\pm$ .030 & 2.947 & .900 $\pm$ .026 & .910 $\pm$ .025 & .900 $\pm$ .026 & 15.978 \\
 & \algoblanchard & .984 $\pm$ .015 & .990 $\pm$ .000 & .984 $\pm$ .015 & 12019.121 $\pm$ 166.251 & .912 $\pm$ .029 & .988 $\pm$ .004 & .911 $\pm$ .029 & 3846.861 $\pm$ 84.568 & .883 $\pm$ .029 & .953 $\pm$ .019 & .883 $\pm$ .029 & 1232.645 $\pm$ 5.285 & .403 $\pm$ .041 & .648 $\pm$ .038 & .399 $\pm$ .041 & 3853.231 $\pm$ 87.867 \\
 & \algocatoni & .983 $\pm$ .018 & 1.000 $\pm$ .000 & .983 $\pm$ .017 & 5654.642 $\pm$ 114.040 & .903 $\pm$ .021 & .985 $\pm$ .012 & .902 $\pm$ .021 & 4354.538 $\pm$ 94.427 & .751 $\pm$ .033 & .867 $\pm$ .023 & .750 $\pm$ .033 & 2702.652 $\pm$ 76.863 & .504 $\pm$ .041 & .673 $\pm$ .037 & .502 $\pm$ .041 & 3172.609 $\pm$ 78.698 \\
 & \algorivasplata & .983 $\pm$ .016 & .990 $\pm$ .000 & .983 $\pm$ .016 & 11976.720 $\pm$ 165.964 & .905 $\pm$ .023 & .975 $\pm$ .007 & .905 $\pm$ .023 & 3855.872 $\pm$ 84.676 & .855 $\pm$ .035 & .916 $\pm$ .027 & .855 $\pm$ .035 & 125.110 $\pm$ 51.837 & .365 $\pm$ .032 & .559 $\pm$ .032 & .359 $\pm$ .033 & 4823.725 $\pm$ 103.813 \\
 & \algostoNN & \textemdash & .990 & \textemdash & 1.963 & \textemdash & .977 & \textemdash & .812 & \textemdash & .948 & \textemdash & 1.473 & \textemdash & .944 & \textemdash & 7.989 \\
\midrule
\multirow[c]{5}{*}{\rotatebox[origin=c]{90}{\small{CIFAR-10}}} & \algoours & .899 $\pm$ .000 & .915 $\pm$ .000 & .899 $\pm$ .000 & 63.416 & .890 $\pm$ .003 & .932 $\pm$ .003 & .886 $\pm$ .003 & 68.353 & .786 $\pm$ .011 & .888 $\pm$ .008 & .787 $\pm$ .010 & 2072.610 & .769 $\pm$ .017 & .859 $\pm$ .013 & .770 $\pm$ .017 & 1406.824 \\
 & \algoblanchard & .869 $\pm$ .002 & .990 $\pm$ .000 & .866 $\pm$ .001 & 27237.938 $\pm$ 251.770 & .813 $\pm$ .004 & .990 $\pm$ .000 & .812 $\pm$ .003 & 12052.733 $\pm$ 159.732 & .697 $\pm$ .011 & .920 $\pm$ .005 & .700 $\pm$ .009 & 5137.799 $\pm$ 103.680 & .674 $\pm$ .020 & .861 $\pm$ .014 & .675 $\pm$ .020 & 2814.450 $\pm$ 76.004 \\
 & \algocatoni & .928 $\pm$ .001 & 1.000 $\pm$ .000 & .925 $\pm$ .001 & 2145276.795 $\pm$ 2095.160 & .821 $\pm$ .002 & 1.000 $\pm$ .000 & .821 $\pm$ .002 & 375019.277 $\pm$ 896.780 & .689 $\pm$ .011 & .870 $\pm$ .007 & .692 $\pm$ .010 & 5292.535 $\pm$ 106.380 & .629 $\pm$ .019 & .805 $\pm$ .015 & .628 $\pm$ .019 & 4159.131 $\pm$ 96.763 \\
 & \algorivasplata & .867 $\pm$ .002 & .990 $\pm$ .000 & .864 $\pm$ .001 & 35956.152 $\pm$ 268.304 & .812 $\pm$ .004 & .976 $\pm$ .001 & .811 $\pm$ .003 & 12135.134 $\pm$ 157.621 & .698 $\pm$ .010 & .874 $\pm$ .006 & .701 $\pm$ .009 & 5191.665 $\pm$ 102.712 & .677 $\pm$ .020 & .819 $\pm$ .015 & .678 $\pm$ .019 & 2839.514 $\pm$ 81.432 \\
 & \algostoNN & \textemdash & .947 & \textemdash & 31.708 & \textemdash & .954 & \textemdash & 34.176 & \textemdash & .908 & \textemdash & 1036.305 & \textemdash & .886 & \textemdash & 703.412 \\
\bottomrule
\end{tabular}
}
\label{chap:dis-pra:table:1_prior_0.0}
\end{table}
\end{landscape} 

\begin{landscape}
\begin{table}[t]
\caption{
\looseness=-1
Comparison of \algoours, \algorivasplata, \algoblanchard and \algocatoni based on the disintegrated bounds, and \algostoNN based on the randomized bounds learned with two learning rates $\lr{\ \in}\{10^{-4}, 10^{-6}\}$ and different variances $\sigma^2{\in}\{10^{-3}, 10^{-4}, 10^{-5}, 10^{-6}\}$.
We report the test risk ($\Risk_{\dT}(\h)$), the bound value (Bnd), the empirical risk ($\Risk_{\dS}(\h)$), and the divergence (Div) associated with each bound (the \textsc{Rényi} divergence for \algoours, the KL divergence for \algostoNN, and the disintegrated KL divergence for \algorivasplata, \algoblanchard and \algocatoni).
More precisely, we report the mean $\pm$ the standard deviation for $400$ neural networks sampled from $\AQ$ for \algoours, \algorivasplata, \algoblanchard, and \algocatoni.
We consider, in this table, that the split ratio is $0.1$.
}
\resizebox{0.63\paperheight}{!}{
\begin{tabular}{rr|clcl|clcl|clcl|clcl}
\toprule
 &  & \multicolumn{4}{c}{$\sigma^2=10^{-6}$} & \multicolumn{4}{c}{$\sigma^2=10^{-5}$} & \multicolumn{4}{c}{$\sigma^2=10^{-4}$} & \multicolumn{4}{c}{$\sigma^2=10^{-3}$} \\
\midrule
 & $\lr=10^{-6}$ & $\Risk_{\Tcal}(h)$ & Bnd & $\Risk_{\Scal}(h)$ & Div & $\Risk_{\Tcal}(h)$ & Bnd & $\Risk_{\Scal}(h)$ & Div & $\Risk_{\Tcal}(h)$ & Bnd & $\Risk_{\Scal}(h)$ & Div & $\Risk_{\Tcal}(h)$ & Bnd & $\Risk_{\Scal}(h)$ & Div \\
\midrule
\multirow[c]{5}{*}{\rotatebox[origin=c]{90}{\small{MNIST}}} & \algoours & .035 $\pm$ .000 & .044 $\pm$ .000 & .039 $\pm$ .000 & .622 & .024 $\pm$ .000 & .034 $\pm$ .000 & .029 $\pm$ .000 & 2.122 & .029 $\pm$ .002 & .040 $\pm$ .002 & .034 $\pm$ .002 & 12.754 & .034 $\pm$ .004 & .044 $\pm$ .004 & .038 $\pm$ .004 & 7.303 \\
 & \algoblanchard & .034 $\pm$ .000 & .058 $\pm$ .002 & .038 $\pm$ .000 & 99.876 $\pm$ 14.858 & .024 $\pm$ .000 & .038 $\pm$ .001 & .030 $\pm$ .000 & 21.775 $\pm$ 6.848 & .034 $\pm$ .002 & .043 $\pm$ .002 & .038 $\pm$ .002 & 3.949 $\pm$ 2.877 & .039 $\pm$ .005 & .047 $\pm$ .005 & .043 $\pm$ .005 & .590 $\pm$ 1.085 \\
 & \algocatoni & .035 $\pm$ .000 & .064 $\pm$ .001 & .039 $\pm$ .000 & 119.663 $\pm$ 15.854 & .024 $\pm$ .000 & .038 $\pm$ .001 & .030 $\pm$ .000 & 26.277 $\pm$ 7.490 & .033 $\pm$ .002 & .041 $\pm$ .002 & .037 $\pm$ .002 & 4.067 $\pm$ 2.882 & .038 $\pm$ .005 & .045 $\pm$ .005 & .042 $\pm$ .004 & .759 $\pm$ 1.217 \\
 & \algorivasplata & .034 $\pm$ .000 & .052 $\pm$ .001 & .038 $\pm$ .000 & 104.880 $\pm$ 15.268 & .024 $\pm$ .000 & .036 $\pm$ .001 & .029 $\pm$ .000 & 23.007 $\pm$ 7.187 & .033 $\pm$ .002 & .042 $\pm$ .002 & .037 $\pm$ .002 & 4.116 $\pm$ 2.845 & .038 $\pm$ .005 & .046 $\pm$ .004 & .042 $\pm$ .004 & .775 $\pm$ 1.231 \\
 & \algostoNN & \textemdash & .080 & \textemdash & .311 & \textemdash & .067 & \textemdash & 1.061 & \textemdash & .074 & \textemdash & 6.377 & \textemdash & .079 & \textemdash & 3.651 \\
\midrule
\multirow[c]{5}{*}{\rotatebox[origin=c]{90}{\small{Fashion}}} & \algoours & .166 $\pm$ .001 & .169 $\pm$ .000 & .159 $\pm$ .000 & .580 & .157 $\pm$ .001 & .160 $\pm$ .001 & .150 $\pm$ .001 & 2.128 & .160 $\pm$ .002 & .161 $\pm$ .003 & .151 $\pm$ .002 & 3.503 & .176 $\pm$ .006 & .179 $\pm$ .006 & .168 $\pm$ .005 & 1.268 \\
 & \algoblanchard & .165 $\pm$ .001 & .192 $\pm$ .002 & .159 $\pm$ .000 & 96.822 $\pm$ 14.116 & .157 $\pm$ .001 & .166 $\pm$ .002 & .150 $\pm$ .001 & 21.592 $\pm$ 6.681 & .163 $\pm$ .003 & .162 $\pm$ .003 & .153 $\pm$ .003 & 3.846 $\pm$ 2.660 & .178 $\pm$ .005 & .178 $\pm$ .005 & .170 $\pm$ .005 & .463 $\pm$ .954 \\
 & \algocatoni & .165 $\pm$ .001 & .190 $\pm$ .003 & .159 $\pm$ .000 & 119.927 $\pm$ 15.938 & .157 $\pm$ .001 & .163 $\pm$ .002 & .150 $\pm$ .001 & 26.363 $\pm$ 7.355 & .162 $\pm$ .003 & .161 $\pm$ .003 & .152 $\pm$ .003 & 4.152 $\pm$ 2.945 & .177 $\pm$ .006 & .178 $\pm$ .006 & .169 $\pm$ .006 & .548 $\pm$ 1.032 \\
 & \algorivasplata & .165 $\pm$ .001 & .183 $\pm$ .002 & .158 $\pm$ .000 & 101.954 $\pm$ 14.463 & .157 $\pm$ .001 & .163 $\pm$ .002 & .150 $\pm$ .001 & 23.098 $\pm$ 6.977 & .162 $\pm$ .003 & .161 $\pm$ .003 & .153 $\pm$ .003 & 3.852 $\pm$ 2.798 & .177 $\pm$ .006 & .177 $\pm$ .006 & .169 $\pm$ .006 & .516 $\pm$ .985 \\
 & \algostoNN & \textemdash & .227 & \textemdash & .290 & \textemdash & .216 & \textemdash & 1.064 & \textemdash & .218 & \textemdash & 1.751 & \textemdash & .237 & \textemdash & .634 \\
\midrule
\multirow[c]{5}{*}{\rotatebox[origin=c]{90}{\small{CIFAR-10}}} & \algoours & .479 $\pm$ .000 & .487 $\pm$ .000 & .472 $\pm$ .000 & .052 & .479 $\pm$ .000 & .493 $\pm$ .000 & .477 $\pm$ .000 & .065 & .458 $\pm$ .001 & .479 $\pm$ .000 & .463 $\pm$ .000 & .299 & .480 $\pm$ .002 & .495 $\pm$ .001 & .480 $\pm$ .001 & .793 \\
 & \algoblanchard & .479 $\pm$ .000 & .550 $\pm$ .003 & .472 $\pm$ .000 & 27.644 $\pm$ 22.868 & .479 $\pm$ .000 & .522 $\pm$ .003 & .477 $\pm$ .000 & 85.476 $\pm$ 12.781 & .458 $\pm$ .001 & .489 $\pm$ .003 & .463 $\pm$ .000 & 24.608 $\pm$ 7.136 & .481 $\pm$ .002 & .495 $\pm$ .002 & .480 $\pm$ .001 & 5.093 $\pm$ 3.299 \\
 & \algocatoni & .479 $\pm$ .000 & .546 $\pm$ .005 & .472 $\pm$ .000 & 269.855 $\pm$ 22.883 & .479 $\pm$ .000 & .511 $\pm$ .003 & .477 $\pm$ .000 & 85.113 $\pm$ 12.806 & .458 $\pm$ .001 & .483 $\pm$ .002 & .463 $\pm$ .000 & 25.453 $\pm$ 7.155 & .480 $\pm$ .002 & .495 $\pm$ .001 & .480 $\pm$ .001 & 5.468 $\pm$ 3.315 \\
 & \algorivasplata & .479 $\pm$ .000 & .528 $\pm$ .002 & .472 $\pm$ .000 & 27.588 $\pm$ 22.859 & .479 $\pm$ .000 & .511 $\pm$ .002 & .477 $\pm$ .000 & 85.745 $\pm$ 13.357 & .458 $\pm$ .001 & .484 $\pm$ .002 & .463 $\pm$ .001 & 25.051 $\pm$ 7.005 & .481 $\pm$ .002 & .494 $\pm$ .001 & .480 $\pm$ .001 & 5.155 $\pm$ 3.260 \\
 & \algostoNN & \textemdash & .558 & \textemdash & .026 & \textemdash & .564 & \textemdash & .032 & \textemdash & .550 & \textemdash & .150 & \textemdash & .566 & \textemdash & .397 \\
\midrule
 &  & \multicolumn{4}{c}{$\sigma^2=10^{-6}$} & \multicolumn{4}{c}{$\sigma^2=10^{-5}$} & \multicolumn{4}{c}{$\sigma^2=10^{-4}$} & \multicolumn{4}{c}{$\sigma^2=10^{-3}$} \\
\midrule
 & $\lr=10^{-4}$ & $\Risk_{\Tcal}(h)$ & Bnd & $\Risk_{\Scal}(h)$ & Div & $\Risk_{\Tcal}(h)$ & Bnd & $\Risk_{\Scal}(h)$ & Div & $\Risk_{\Tcal}(h)$ & Bnd & $\Risk_{\Scal}(h)$ & Div & $\Risk_{\Tcal}(h)$ & Bnd & $\Risk_{\Scal}(h)$ & Div \\
\midrule
\multirow[c]{5}{*}{\rotatebox[origin=c]{90}{\small{MNIST}}} & \algoours & .035 $\pm$ .000 & .048 $\pm$ .000 & .039 $\pm$ .000 & 35.348 & .024 $\pm$ .000 & .037 $\pm$ .001 & .029 $\pm$ .000 & 3.753 & .022 $\pm$ .001 & .042 $\pm$ .001 & .027 $\pm$ .001 & 153.773 & .025 $\pm$ .002 & .041 $\pm$ .002 & .029 $\pm$ .002 & 97.840 \\
 & \algoblanchard & .032 $\pm$ .000 & .442 $\pm$ .003 & .036 $\pm$ .000 & 1181.482 $\pm$ 14.449 & .022 $\pm$ .000 & .206 $\pm$ .003 & .027 $\pm$ .000 & 3851.110 $\pm$ 84.274 & .019 $\pm$ .001 & .102 $\pm$ .002 & .023 $\pm$ .001 & 1306.371 $\pm$ 51.396 & .024 $\pm$ .002 & .065 $\pm$ .003 & .027 $\pm$ .002 & 411.772 $\pm$ 29.458 \\
 & \algocatoni & .035 $\pm$ .000 & .362 $\pm$ .003 & .039 $\pm$ .000 & 11925.734 $\pm$ 145.511 & .024 $\pm$ .000 & .152 $\pm$ .002 & .029 $\pm$ .000 & 3841.248 $\pm$ 84.033 & .027 $\pm$ .002 & .084 $\pm$ .002 & .032 $\pm$ .001 & 1235.287 $\pm$ 49.454 & .027 $\pm$ .002 & .059 $\pm$ .003 & .030 $\pm$ .002 & 403.300 $\pm$ 28.587 \\
 & \algorivasplata & .030 $\pm$ .000 & .289 $\pm$ .002 & .034 $\pm$ .000 & 12022.576 $\pm$ 151.157 & .021 $\pm$ .000 & .134 $\pm$ .002 & .026 $\pm$ .000 & 3912.803 $\pm$ 85.146 & .018 $\pm$ .000 & .072 $\pm$ .001 & .022 $\pm$ .000 & 1348.169 $\pm$ 53.400 & .023 $\pm$ .002 & .051 $\pm$ .002 & .026 $\pm$ .001 & 424.971 $\pm$ 29.301 \\
 & \algostoNN & \textemdash & .084 & \textemdash & 17.674 & \textemdash & .069 & \textemdash & 15.376 & \textemdash & .072 & \textemdash & 76.887 & \textemdash & .072 & \textemdash & 48.920 \\
\midrule
\multirow[c]{5}{*}{\rotatebox[origin=c]{90}{\small{Fashion}}} & \algoours & .166 $\pm$ .001 & .172 $\pm$ .000 & .159 $\pm$ .000 & 13.084 & .157 $\pm$ .001 & .163 $\pm$ .001 & .150 $\pm$ .001 & 16.513 & .159 $\pm$ .002 & .164 $\pm$ .002 & .149 $\pm$ .002 & 2.344 & .176 $\pm$ .005 & .181 $\pm$ .005 & .168 $\pm$ .005 & 11.331 \\
 & \algoblanchard & .160 $\pm$ .001 & .588 $\pm$ .003 & .153 $\pm$ .000 & 1089.829 $\pm$ 137.125 & .150 $\pm$ .001 & .379 $\pm$ .003 & .141 $\pm$ .001 & 3744.491 $\pm$ 83.656 & .155 $\pm$ .002 & .271 $\pm$ .003 & .145 $\pm$ .002 & 1221.062 $\pm$ 49.548 & .173 $\pm$ .005 & .233 $\pm$ .006 & .165 $\pm$ .004 & 369.721 $\pm$ 27.211 \\
 & \algocatoni & .165 $\pm$ .001 & .500 $\pm$ .003 & .159 $\pm$ .000 & 11954.591 $\pm$ 141.463 & .156 $\pm$ .001 & .311 $\pm$ .002 & .148 $\pm$ .001 & 3826.848 $\pm$ 86.111 & .158 $\pm$ .002 & .248 $\pm$ .003 & .148 $\pm$ .002 & 1226.282 $\pm$ 5.332 & .174 $\pm$ .005 & .252 $\pm$ .006 & .166 $\pm$ .004 & 393.542 $\pm$ 27.890 \\
 & \algorivasplata & .158 $\pm$ .001 & .459 $\pm$ .002 & .151 $\pm$ .000 & 11541.128 $\pm$ 14.706 & .149 $\pm$ .001 & .302 $\pm$ .002 & .140 $\pm$ .001 & 3878.145 $\pm$ 85.782 & .154 $\pm$ .002 & .230 $\pm$ .002 & .144 $\pm$ .001 & 1244.035 $\pm$ 49.268 & .172 $\pm$ .005 & .212 $\pm$ .005 & .164 $\pm$ .004 & 378.990 $\pm$ 27.559 \\
 & \algostoNN & \textemdash & .229 & \textemdash & 6.542 & \textemdash & .219 & \textemdash & 8.257 & \textemdash & .219 & \textemdash & 1.172 & \textemdash & .239 & \textemdash & 5.666 \\
\midrule
\multirow[c]{5}{*}{\rotatebox[origin=c]{90}{\small{CIFAR-10}}} & \algoours & .479 $\pm$ .000 & .489 $\pm$ .000 & .472 $\pm$ .000 & 4.882 & .479 $\pm$ .000 & .496 $\pm$ .000 & .477 $\pm$ .000 & 9.273 & .458 $\pm$ .001 & .480 $\pm$ .000 & .463 $\pm$ .000 & 4.988 & .480 $\pm$ .002 & .497 $\pm$ .001 & .479 $\pm$ .001 & 8.681 \\
 & \algoblanchard & .479 $\pm$ .000 & .957 $\pm$ .001 & .471 $\pm$ .000 & 22201.935 $\pm$ 218.369 & .479 $\pm$ .000 & .854 $\pm$ .002 & .477 $\pm$ .000 & 8777.551 $\pm$ 125.716 & .457 $\pm$ .001 & .699 $\pm$ .003 & .461 $\pm$ .000 & 2758.075 $\pm$ 77.155 & .474 $\pm$ .001 & .613 $\pm$ .003 & .472 $\pm$ .001 & 903.948 $\pm$ 4.742 \\
 & \algocatoni & .479 $\pm$ .000 & .995 $\pm$ .000 & .471 $\pm$ .000 & 26347.736 $\pm$ 225.908 & .479 $\pm$ .000 & .771 $\pm$ .002 & .477 $\pm$ .000 & 8566.272 $\pm$ 124.834 & .455 $\pm$ .001 & .650 $\pm$ .002 & .459 $\pm$ .000 & 3117.566 $\pm$ 75.178 & .468 $\pm$ .001 & .621 $\pm$ .001 & .466 $\pm$ .001 & 1481.520 $\pm$ 52.533 \\
 & \algorivasplata & .479 $\pm$ .000 & .915 $\pm$ .001 & .471 $\pm$ .000 & 29489.241 $\pm$ 241.010 & .479 $\pm$ .000 & .765 $\pm$ .002 & .477 $\pm$ .000 & 867.264 $\pm$ 126.038 & .456 $\pm$ .001 & .633 $\pm$ .002 & .460 $\pm$ .000 & 2776.052 $\pm$ 72.901 & .472 $\pm$ .001 & .572 $\pm$ .002 & .470 $\pm$ .001 & 937.091 $\pm$ 42.116 \\
 & \algostoNN & \textemdash & .559 & \textemdash & 2.441 & \textemdash & .566 & \textemdash & 4.637 & \textemdash & .551 & \textemdash & 2.494 & \textemdash & .567 & \textemdash & 4.340 \\
\bottomrule
\end{tabular}
}
\label{chap:dis-pra:table:1_prior_0.1}
\end{table}
\end{landscape} 

\begin{landscape}
\begin{table}[t]
\caption{
\looseness=-1
Comparison of \algoours, \algorivasplata, \algoblanchard and \algocatoni based on the disintegrated bounds, and \algostoNN based on the randomized bounds learned with two learning rates $\lr{\ \in}\{10^{-4}, 10^{-6}\}$ and different variances $\sigma^2{\in}\{10^{-3}, 10^{-4}, 10^{-5}, 10^{-6}\}$.
We report the test risk ($\Risk_{\dT}(\h)$), the bound value (Bnd), the empirical risk ($\Risk_{\dS}(\h)$), and the divergence (Div) associated with each bound (the \textsc{Rényi} divergence for \algoours, the KL divergence for \algostoNN, and the disintegrated KL divergence for \algorivasplata, \algoblanchard and \algocatoni).
More precisely, we report the mean $\pm$ the standard deviation for $400$ neural networks sampled from $\AQ$ for \algoours, \algorivasplata, \algoblanchard, and \algocatoni.
We consider, in this table, that the split ratio is $0.2$.
}
\resizebox{0.63\paperheight}{!}{
\begin{tabular}{rr|clcl|clcl|clcl|clcl}
\toprule
 &  & \multicolumn{4}{c}{$\sigma^2=10^{-6}$} & \multicolumn{4}{c}{$\sigma^2=10^{-5}$} & \multicolumn{4}{c}{$\sigma^2=10^{-4}$} & \multicolumn{4}{c}{$\sigma^2=10^{-3}$} \\
\midrule
 & $\lr=10^{-6}$ & $\Risk_{\Tcal}(h)$ & Bnd & $\Risk_{\Scal}(h)$ & Div & $\Risk_{\Tcal}(h)$ & Bnd & $\Risk_{\Scal}(h)$ & Div & $\Risk_{\Tcal}(h)$ & Bnd & $\Risk_{\Scal}(h)$ & Div & $\Risk_{\Tcal}(h)$ & Bnd & $\Risk_{\Scal}(h)$ & Div \\
\midrule
\multirow[c]{5}{*}{\rotatebox[origin=c]{90}{\small{MNIST}}} & \algoours & .016 $\pm$ .000 & .023 $\pm$ .000 & .019 $\pm$ .000 & .336 & .015 $\pm$ .000 & .023 $\pm$ .000 & .019 $\pm$ .000 & .748 & .014 $\pm$ .001 & .020 $\pm$ .001 & .016 $\pm$ .000 & 2.096 & .019 $\pm$ .002 & .024 $\pm$ .002 & .020 $\pm$ .002 & 2.244 \\
 & \algoblanchard & .016 $\pm$ .000 & .034 $\pm$ .001 & .019 $\pm$ .000 & 97.590 $\pm$ 14.260 & .015 $\pm$ .000 & .026 $\pm$ .001 & .019 $\pm$ .000 & 21.153 $\pm$ 6.514 & .015 $\pm$ .001 & .020 $\pm$ .001 & .016 $\pm$ .001 & 3.362 $\pm$ 2.569 & .020 $\pm$ .002 & .024 $\pm$ .002 & .021 $\pm$ .002 & .371 $\pm$ .875 \\
 & \algocatoni & .016 $\pm$ .000 & .034 $\pm$ .001 & .019 $\pm$ .000 & 116.744 $\pm$ 15.447 & .015 $\pm$ .000 & .027 $\pm$ .002 & .019 $\pm$ .000 & 24.135 $\pm$ 7.075 & .015 $\pm$ .001 & .020 $\pm$ .001 & .016 $\pm$ .001 & 3.352 $\pm$ 2.667 & .020 $\pm$ .002 & .024 $\pm$ .002 & .021 $\pm$ .002 & .410 $\pm$ .890 \\
 & \algorivasplata & .016 $\pm$ .000 & .030 $\pm$ .001 & .019 $\pm$ .000 & 101.334 $\pm$ 14.728 & .015 $\pm$ .000 & .024 $\pm$ .001 & .019 $\pm$ .000 & 21.663 $\pm$ 6.603 & .015 $\pm$ .001 & .020 $\pm$ .001 & .016 $\pm$ .001 & 3.409 $\pm$ 2.666 & .020 $\pm$ .002 & .024 $\pm$ .002 & .021 $\pm$ .002 & .446 $\pm$ .927 \\
 & \algostoNN & \textemdash & .052 & \textemdash & .168 & \textemdash & .051 & \textemdash & .374 & \textemdash & .047 & \textemdash & 1.048 & \textemdash & .053 & \textemdash & 1.122 \\
\midrule
\multirow[c]{5}{*}{\rotatebox[origin=c]{90}{\small{Fashion}}} & \algoours & .165 $\pm$ .002 & .169 $\pm$ .001 & .157 $\pm$ .001 & 4.811 & .148 $\pm$ .003 & .155 $\pm$ .002 & .143 $\pm$ .002 & 1.856 & .145 $\pm$ .005 & .153 $\pm$ .006 & .139 $\pm$ .005 & 15.453 & .160 $\pm$ .005 & .166 $\pm$ .005 & .155 $\pm$ .005 & 1.633 \\
 & \algoblanchard & .163 $\pm$ .002 & .190 $\pm$ .003 & .155 $\pm$ .001 & 96.264 $\pm$ 14.472 & .152 $\pm$ .003 & .163 $\pm$ .003 & .147 $\pm$ .003 & 21.099 $\pm$ 6.507 & .155 $\pm$ .007 & .160 $\pm$ .007 & .151 $\pm$ .007 & 3.929 $\pm$ 2.841 & .163 $\pm$ .006 & .165 $\pm$ .006 & .158 $\pm$ .006 & .340 $\pm$ .885 \\
 & \algocatoni & .163 $\pm$ .002 & .190 $\pm$ .004 & .156 $\pm$ .001 & 121.542 $\pm$ 16.499 & .150 $\pm$ .002 & .158 $\pm$ .003 & .144 $\pm$ .002 & 27.241 $\pm$ 7.318 & .151 $\pm$ .006 & .155 $\pm$ .006 & .146 $\pm$ .006 & 5.120 $\pm$ 3.150 & .162 $\pm$ .005 & .165 $\pm$ .005 & .157 $\pm$ .005 & .444 $\pm$ .968 \\
 & \algorivasplata & .161 $\pm$ .001 & .180 $\pm$ .002 & .153 $\pm$ .001 & 106.403 $\pm$ 14.044 & .150 $\pm$ .002 & .158 $\pm$ .003 & .145 $\pm$ .003 & 23.134 $\pm$ 7.064 & .153 $\pm$ .006 & .157 $\pm$ .006 & .148 $\pm$ .007 & 4.439 $\pm$ 2.924 & .162 $\pm$ .006 & .165 $\pm$ .005 & .157 $\pm$ .005 & .417 $\pm$ .928 \\
 & \algostoNN & \textemdash & .226 & \textemdash & 2.405 & \textemdash & .210 & \textemdash & 5.428 & \textemdash & .207 & \textemdash & 7.727 & \textemdash & .223 & \textemdash & .816 \\
\midrule
\multirow[c]{5}{*}{\rotatebox[origin=c]{90}{\small{CIFAR-10}}} & \algoours & .390 $\pm$ .000 & .407 $\pm$ .000 & .391 $\pm$ .000 & .040 & .404 $\pm$ .000 & .414 $\pm$ .000 & .398 $\pm$ .000 & .070 & .396 $\pm$ .001 & .411 $\pm$ .000 & .395 $\pm$ .000 & .155 & .416 $\pm$ .002 & .432 $\pm$ .001 & .415 $\pm$ .001 & .970 \\
 & \algoblanchard & .390 $\pm$ .000 & .473 $\pm$ .004 & .391 $\pm$ .000 & 271.616 $\pm$ 23.555 & .404 $\pm$ .000 & .445 $\pm$ .003 & .398 $\pm$ .000 & 84.868 $\pm$ 13.050 & .396 $\pm$ .001 & .422 $\pm$ .003 & .395 $\pm$ .000 & 23.962 $\pm$ 7.208 & .416 $\pm$ .002 & .432 $\pm$ .002 & .416 $\pm$ .001 & 4.496 $\pm$ 3.018 \\
 & \algocatoni & .390 $\pm$ .000 & .473 $\pm$ .006 & .391 $\pm$ .000 & 27.502 $\pm$ 23.371 & .404 $\pm$ .000 & .434 $\pm$ .003 & .398 $\pm$ .000 & 84.848 $\pm$ 12.992 & .396 $\pm$ .001 & .415 $\pm$ .002 & .395 $\pm$ .000 & 24.505 $\pm$ 6.942 & .416 $\pm$ .002 & .431 $\pm$ .001 & .415 $\pm$ .001 & 4.859 $\pm$ 3.176 \\
 & \algorivasplata & .390 $\pm$ .000 & .450 $\pm$ .002 & .391 $\pm$ .000 & 271.700 $\pm$ 23.586 & .403 $\pm$ .000 & .433 $\pm$ .002 & .398 $\pm$ .000 & 85.027 $\pm$ 13.047 & .396 $\pm$ .001 & .416 $\pm$ .002 & .395 $\pm$ .000 & 23.955 $\pm$ 7.093 & .416 $\pm$ .002 & .431 $\pm$ .002 & .416 $\pm$ .001 & 4.610 $\pm$ 3.084 \\
 & \algostoNN & \textemdash & .477 & \textemdash & .020 & \textemdash & .485 & \textemdash & .035 & \textemdash & .482 & \textemdash & .077 & \textemdash & .503 & \textemdash & .485 \\
\midrule
 &  & \multicolumn{4}{c}{$\sigma^2=10^{-6}$} & \multicolumn{4}{c}{$\sigma^2=10^{-5}$} & \multicolumn{4}{c}{$\sigma^2=10^{-4}$} & \multicolumn{4}{c}{$\sigma^2=10^{-3}$} \\
\midrule
 & $\lr=10^{-4}$ & $\Risk_{\Tcal}(h)$ & Bnd & $\Risk_{\Scal}(h)$ & Div & $\Risk_{\Tcal}(h)$ & Bnd & $\Risk_{\Scal}(h)$ & Div & $\Risk_{\Tcal}(h)$ & Bnd & $\Risk_{\Scal}(h)$ & Div & $\Risk_{\Tcal}(h)$ & Bnd & $\Risk_{\Scal}(h)$ & Div \\
\midrule
\multirow[c]{5}{*}{\rotatebox[origin=c]{90}{\small{MNIST}}} & \algoours & .016 $\pm$ .000 & .025 $\pm$ .000 & .019 $\pm$ .000 & 14.490 & .015 $\pm$ .000 & .024 $\pm$ .000 & .019 $\pm$ .000 & 8.583 & .014 $\pm$ .000 & .021 $\pm$ .001 & .016 $\pm$ .000 & 13.055 & .016 $\pm$ .001 & .023 $\pm$ .001 & .017 $\pm$ .001 & 25.556 \\
 & \algoblanchard & .016 $\pm$ .000 & .430 $\pm$ .004 & .018 $\pm$ .000 & 11405.062 $\pm$ 153.554 & .014 $\pm$ .000 & .200 $\pm$ .003 & .018 $\pm$ .000 & 3799.912 $\pm$ 89.585 & .013 $\pm$ .000 & .086 $\pm$ .002 & .014 $\pm$ .000 & 1187.859 $\pm$ 48.700 & .015 $\pm$ .001 & .049 $\pm$ .002 & .016 $\pm$ .001 & 38.983 $\pm$ 27.857 \\
 & \algocatoni & .016 $\pm$ .000 & .355 $\pm$ .002 & .019 $\pm$ .000 & 11954.106 $\pm$ 15.709 & .015 $\pm$ .000 & .149 $\pm$ .003 & .019 $\pm$ .000 & 3828.342 $\pm$ 83.937 & .014 $\pm$ .001 & .064 $\pm$ .002 & .016 $\pm$ .001 & 1218.708 $\pm$ 48.514 & .017 $\pm$ .001 & .041 $\pm$ .002 & .018 $\pm$ .001 & 389.726 $\pm$ 29.076 \\
 & \algorivasplata & .015 $\pm$ .000 & .272 $\pm$ .002 & .018 $\pm$ .000 & 1173.953 $\pm$ 149.364 & .013 $\pm$ .000 & .122 $\pm$ .002 & .017 $\pm$ .000 & 3691.345 $\pm$ 82.512 & .012 $\pm$ .000 & .056 $\pm$ .001 & .013 $\pm$ .000 & 1206.615 $\pm$ 5.381 & .015 $\pm$ .001 & .037 $\pm$ .001 & .015 $\pm$ .001 & 391.881 $\pm$ 28.344 \\
 & \algostoNN & \textemdash & .053 & \textemdash & 7.245 & \textemdash & .052 & \textemdash & 4.292 & \textemdash & .048 & \textemdash & 6.528 & \textemdash & .051 & \textemdash & 12.778 \\
\midrule
\multirow[c]{5}{*}{\rotatebox[origin=c]{90}{\small{Fashion}}} & \algoours & .165 $\pm$ .002 & .172 $\pm$ .001 & .157 $\pm$ .001 & 23.705 & .141 $\pm$ .002 & .156 $\pm$ .002 & .137 $\pm$ .002 & 52.736 & .131 $\pm$ .003 & .147 $\pm$ .003 & .126 $\pm$ .003 & 7.515 & .156 $\pm$ .004 & .165 $\pm$ .004 & .151 $\pm$ .003 & 16.954 \\
 & \algoblanchard & .136 $\pm$ .001 & .598 $\pm$ .003 & .130 $\pm$ .001 & 11334.327 $\pm$ 145.083 & .125 $\pm$ .001 & .379 $\pm$ .003 & .121 $\pm$ .001 & 3998.068 $\pm$ 88.992 & .124 $\pm$ .001 & .247 $\pm$ .003 & .117 $\pm$ .001 & 126.184 $\pm$ 48.814 & .152 $\pm$ .003 & .216 $\pm$ .004 & .147 $\pm$ .003 & 364.531 $\pm$ 28.029 \\
 & \algocatoni & .162 $\pm$ .001 & .525 $\pm$ .004 & .154 $\pm$ .001 & 11965.668 $\pm$ 152.681 & .141 $\pm$ .002 & .309 $\pm$ .003 & .137 $\pm$ .002 & 384.802 $\pm$ 84.123 & .132 $\pm$ .003 & .224 $\pm$ .004 & .127 $\pm$ .002 & 1239.918 $\pm$ 49.594 & .155 $\pm$ .004 & .232 $\pm$ .005 & .150 $\pm$ .004 & 394.607 $\pm$ 28.146 \\
 & \algorivasplata & .131 $\pm$ .001 & .455 $\pm$ .002 & .127 $\pm$ .001 & 1193.209 $\pm$ 155.390 & .123 $\pm$ .001 & .290 $\pm$ .002 & .119 $\pm$ .001 & 4005.169 $\pm$ 89.793 & .123 $\pm$ .001 & .204 $\pm$ .002 & .116 $\pm$ .001 & 1294.726 $\pm$ 49.874 & .152 $\pm$ .004 & .195 $\pm$ .004 & .146 $\pm$ .003 & 378.905 $\pm$ 27.422 \\
 & \algostoNN & \textemdash & .228 & \textemdash & 11.853 & \textemdash & .209 & \textemdash & 26.368 & \textemdash & .198 & \textemdash & 35.258 & \textemdash & .221 & \textemdash & 8.477 \\
\midrule
\multirow[c]{5}{*}{\rotatebox[origin=c]{90}{\small{CIFAR-10}}} & \algoours & .390 $\pm$ .000 & .411 $\pm$ .000 & .391 $\pm$ .000 & 13.286 & .404 $\pm$ .000 & .415 $\pm$ .000 & .398 $\pm$ .000 & 3.305 & .396 $\pm$ .001 & .412 $\pm$ .000 & .395 $\pm$ .000 & 3.136 & .415 $\pm$ .001 & .433 $\pm$ .001 & .415 $\pm$ .001 & 6.064 \\
 & \algoblanchard & .389 $\pm$ .000 & .990 $\pm$ .000 & .391 $\pm$ .000 & 75424.764 $\pm$ 397.521 & .403 $\pm$ .000 & .820 $\pm$ .002 & .397 $\pm$ .000 & 8815.324 $\pm$ 126.764 & .395 $\pm$ .001 & .651 $\pm$ .003 & .394 $\pm$ .000 & 2738.066 $\pm$ 75.053 & .408 $\pm$ .001 & .557 $\pm$ .003 & .405 $\pm$ .001 & 918.500 $\pm$ 42.347 \\
 & \algocatoni & .390 $\pm$ .000 & .990 $\pm$ .000 & .391 $\pm$ .000 & 26434.787 $\pm$ 228.500 & .403 $\pm$ .000 & .726 $\pm$ .003 & .397 $\pm$ .000 & 8651.380 $\pm$ 126.473 & .394 $\pm$ .001 & .620 $\pm$ .002 & .393 $\pm$ .000 & 4178.302 $\pm$ 9.315 & .401 $\pm$ .001 & .556 $\pm$ .001 & .396 $\pm$ .001 & 1462.235 $\pm$ 55.526 \\
 & \algorivasplata & .389 $\pm$ .000 & .902 $\pm$ .001 & .391 $\pm$ .000 & 31497.669 $\pm$ 249.683 & .403 $\pm$ .000 & .715 $\pm$ .002 & .397 $\pm$ .000 & 8707.893 $\pm$ 133.239 & .394 $\pm$ .001 & .578 $\pm$ .003 & .393 $\pm$ .000 & 2741.257 $\pm$ 74.942 & .405 $\pm$ .001 & .512 $\pm$ .002 & .402 $\pm$ .001 & 967.818 $\pm$ 43.629 \\
 & \algostoNN & \textemdash & .480 & \textemdash & 6.643 & \textemdash & .486 & \textemdash & 1.653 & \textemdash & .483 & \textemdash & 1.568 & \textemdash & .503 & \textemdash & 3.032 \\
\bottomrule
\end{tabular}
}
\label{chap:dis-pra:table:1_prior_0.2}
\end{table}
\end{landscape} 

\begin{landscape}
\begin{table}[t]
\caption{
\looseness=-1
Comparison of \algoours, \algorivasplata, \algoblanchard and \algocatoni based on the disintegrated bounds, and \algostoNN based on the randomized bounds learned with two learning rates $\lr{\ \in}\{10^{-4}, 10^{-6}\}$ and different variances $\sigma^2{\in}\{10^{-3}, 10^{-4}, 10^{-5}, 10^{-6}\}$.
We report the test risk ($\Risk_{\dT}(\h)$), the bound value (Bnd), the empirical risk ($\Risk_{\dS}(\h)$), and the divergence (Div) associated with each bound (the \textsc{Rényi} divergence for \algoours, the KL divergence for \algostoNN, and the disintegrated KL divergence for \algorivasplata, \algoblanchard and \algocatoni).
More precisely, we report the mean $\pm$ the standard deviation for $400$ neural networks sampled from $\AQ$ for \algoours, \algorivasplata, \algoblanchard, and \algocatoni.
We consider, in this table, that the split ratio is $0.3$.
}
\resizebox{0.63\paperheight}{!}{
\begin{tabular}{rr|clcl|clcl|clcl|clcl}
\toprule
 &  & \multicolumn{4}{c}{$\sigma^2=10^{-6}$} & \multicolumn{4}{c}{$\sigma^2=10^{-5}$} & \multicolumn{4}{c}{$\sigma^2=10^{-4}$} & \multicolumn{4}{c}{$\sigma^2=10^{-3}$} \\
\midrule
 & $\lr=10^{-6}$ & $\Risk_{\Tcal}(h)$ & Bnd & $\Risk_{\Scal}(h)$ & Div & $\Risk_{\Tcal}(h)$ & Bnd & $\Risk_{\Scal}(h)$ & Div & $\Risk_{\Tcal}(h)$ & Bnd & $\Risk_{\Scal}(h)$ & Div & $\Risk_{\Tcal}(h)$ & Bnd & $\Risk_{\Scal}(h)$ & Div \\
\midrule
\multirow[c]{5}{*}{\rotatebox[origin=c]{90}{\small{MNIST}}} & \algoours & .012 $\pm$ .000 & .017 $\pm$ .000 & .013 $\pm$ .000 & .181 & .009 $\pm$ .000 & .015 $\pm$ .000 & .011 $\pm$ .000 & .155 & .012 $\pm$ .000 & .020 $\pm$ .000 & .016 $\pm$ .000 & 1.655 & .013 $\pm$ .001 & .019 $\pm$ .001 & .015 $\pm$ .001 & .615 \\
 & \algoblanchard & .012 $\pm$ .000 & .027 $\pm$ .001 & .013 $\pm$ .000 & 93.915 $\pm$ 14.109 & .012 $\pm$ .000 & .021 $\pm$ .001 & .014 $\pm$ .000 & 19.292 $\pm$ 6.037 & .012 $\pm$ .000 & .020 $\pm$ .001 & .016 $\pm$ .000 & 3.023 $\pm$ 2.430 & .014 $\pm$ .001 & .018 $\pm$ .001 & .015 $\pm$ .001 & .368 $\pm$ .831 \\
 & \algocatoni & .012 $\pm$ .000 & .025 $\pm$ .001 & .013 $\pm$ .000 & 113.574 $\pm$ 15.436 & .012 $\pm$ .000 & .023 $\pm$ .002 & .014 $\pm$ .000 & 22.347 $\pm$ 6.877 & .012 $\pm$ .000 & .020 $\pm$ .001 & .016 $\pm$ .000 & 2.918 $\pm$ 2.341 & .013 $\pm$ .001 & .018 $\pm$ .001 & .015 $\pm$ .001 & .336 $\pm$ .807 \\
 & \algorivasplata & .012 $\pm$ .000 & .023 $\pm$ .001 & .013 $\pm$ .000 & 96.392 $\pm$ 14.300 & .012 $\pm$ .000 & .020 $\pm$ .001 & .014 $\pm$ .000 & 19.905 $\pm$ 6.254 & .012 $\pm$ .000 & .020 $\pm$ .001 & .016 $\pm$ .000 & 2.931 $\pm$ 2.446 & .013 $\pm$ .001 & .018 $\pm$ .001 & .015 $\pm$ .001 & .355 $\pm$ .813 \\
 & \algostoNN & \textemdash & .042 & \textemdash & .091 & \textemdash & .039 & \textemdash & .077 & \textemdash & .047 & \textemdash & .827 & \textemdash & .045 & \textemdash & .308 \\
\midrule
\multirow[c]{5}{*}{\rotatebox[origin=c]{90}{\small{Fashion}}} & \algoours & .126 $\pm$ .000 & .134 $\pm$ .000 & .124 $\pm$ .000 & .328 & .126 $\pm$ .001 & .130 $\pm$ .001 & .119 $\pm$ .001 & 1.692 & .122 $\pm$ .002 & .126 $\pm$ .002 & .115 $\pm$ .002 & 4.617 & .139 $\pm$ .005 & .145 $\pm$ .005 & .133 $\pm$ .005 & 2.425 \\
 & \algoblanchard & .126 $\pm$ .000 & .157 $\pm$ .003 & .124 $\pm$ .000 & 88.034 $\pm$ 13.485 & .126 $\pm$ .001 & .136 $\pm$ .002 & .120 $\pm$ .001 & 18.852 $\pm$ 6.115 & .124 $\pm$ .002 & .127 $\pm$ .002 & .118 $\pm$ .002 & 3.014 $\pm$ 2.395 & .142 $\pm$ .006 & .144 $\pm$ .006 & .137 $\pm$ .006 & .370 $\pm$ .819 \\
 & \algocatoni & .126 $\pm$ .000 & .159 $\pm$ .004 & .124 $\pm$ .000 & 114.259 $\pm$ 15.300 & .126 $\pm$ .001 & .133 $\pm$ .002 & .120 $\pm$ .001 & 22.607 $\pm$ 6.871 & .124 $\pm$ .002 & .126 $\pm$ .002 & .118 $\pm$ .002 & 3.100 $\pm$ 2.513 & .141 $\pm$ .006 & .144 $\pm$ .006 & .136 $\pm$ .006 & .390 $\pm$ .898 \\
 & \algorivasplata & .126 $\pm$ .000 & .148 $\pm$ .002 & .124 $\pm$ .000 & 93.107 $\pm$ 13.630 & .126 $\pm$ .001 & .133 $\pm$ .002 & .120 $\pm$ .001 & 19.724 $\pm$ 6.320 & .124 $\pm$ .002 & .126 $\pm$ .002 & .118 $\pm$ .002 & 2.980 $\pm$ 2.451 & .142 $\pm$ .006 & .144 $\pm$ .006 & .136 $\pm$ .006 & .371 $\pm$ .869 \\
 & \algostoNN & \textemdash & .187 & \textemdash & .164 & \textemdash & .182 & \textemdash & .846 & \textemdash & .178 & \textemdash & 2.309 & \textemdash & .199 & \textemdash & 1.212 \\
\midrule
\multirow[c]{5}{*}{\rotatebox[origin=c]{90}{\small{CIFAR-10}}} & \algoours & .369 $\pm$ .000 & .375 $\pm$ .000 & .358 $\pm$ .000 & .028 & .351 $\pm$ .000 & .368 $\pm$ .000 & .352 $\pm$ .000 & .041 & .359 $\pm$ .001 & .377 $\pm$ .000 & .360 $\pm$ .000 & .183 & .419 $\pm$ .001 & .433 $\pm$ .001 & .416 $\pm$ .001 & .759 \\
 & \algoblanchard & .369 $\pm$ .000 & .446 $\pm$ .004 & .358 $\pm$ .000 & 269.789 $\pm$ 22.724 & .351 $\pm$ .000 & .401 $\pm$ .004 & .352 $\pm$ .000 & 84.113 $\pm$ 12.530 & .359 $\pm$ .001 & .388 $\pm$ .003 & .360 $\pm$ .000 & 22.878 $\pm$ 6.728 & .419 $\pm$ .001 & .432 $\pm$ .003 & .416 $\pm$ .001 & 4.089 $\pm$ 2.818 \\
 & \algocatoni & .369 $\pm$ .000 & .450 $\pm$ .007 & .358 $\pm$ .000 & 269.843 $\pm$ 24.225 & .351 $\pm$ .000 & .390 $\pm$ .004 & .352 $\pm$ .000 & 84.500 $\pm$ 12.608 & .359 $\pm$ .001 & .381 $\pm$ .002 & .360 $\pm$ .000 & 23.567 $\pm$ 7.181 & .419 $\pm$ .001 & .432 $\pm$ .001 & .416 $\pm$ .001 & 4.285 $\pm$ 2.942 \\
 & \algorivasplata & .369 $\pm$ .000 & .421 $\pm$ .003 & .358 $\pm$ .000 & 27.224 $\pm$ 24.187 & .351 $\pm$ .000 & .388 $\pm$ .002 & .352 $\pm$ .000 & 84.250 $\pm$ 13.274 & .359 $\pm$ .001 & .382 $\pm$ .002 & .360 $\pm$ .000 & 23.053 $\pm$ 6.724 & .419 $\pm$ .001 & .431 $\pm$ .002 & .416 $\pm$ .001 & 4.141 $\pm$ 2.985 \\
 & \algostoNN & \textemdash & .445 & \textemdash & .014 & \textemdash & .438 & \textemdash & .020 & \textemdash & .447 & \textemdash & .092 & \textemdash & .504 & \textemdash & .380 \\
\midrule
 &  & \multicolumn{4}{c}{$\sigma^2=10^{-6}$} & \multicolumn{4}{c}{$\sigma^2=10^{-5}$} & \multicolumn{4}{c}{$\sigma^2=10^{-4}$} & \multicolumn{4}{c}{$\sigma^2=10^{-3}$} \\
\midrule
 & $\lr=10^{-4}$ & $\Risk_{\Tcal}(h)$ & Bnd & $\Risk_{\Scal}(h)$ & Div & $\Risk_{\Tcal}(h)$ & Bnd & $\Risk_{\Scal}(h)$ & Div & $\Risk_{\Tcal}(h)$ & Bnd & $\Risk_{\Scal}(h)$ & Div & $\Risk_{\Tcal}(h)$ & Bnd & $\Risk_{\Scal}(h)$ & Div \\
\midrule
\multirow[c]{5}{*}{\rotatebox[origin=c]{90}{\small{MNIST}}} & \algoours & .012 $\pm$ .000 & .019 $\pm$ .000 & .013 $\pm$ .000 & 24.837 & .012 $\pm$ .000 & .020 $\pm$ .000 & .014 $\pm$ .000 & 12.358 & .012 $\pm$ .000 & .021 $\pm$ .000 & .015 $\pm$ .000 & 13.908 & .013 $\pm$ .001 & .019 $\pm$ .001 & .014 $\pm$ .001 & 16.179 \\
 & \algoblanchard & .012 $\pm$ .000 & .467 $\pm$ .004 & .013 $\pm$ .000 & 11819.223 $\pm$ 154.992 & .011 $\pm$ .000 & .211 $\pm$ .003 & .014 $\pm$ .000 & 3808.981 $\pm$ 86.014 & .010 $\pm$ .000 & .094 $\pm$ .003 & .014 $\pm$ .000 & 121.397 $\pm$ 51.944 & .012 $\pm$ .001 & .046 $\pm$ .002 & .013 $\pm$ .001 & 372.832 $\pm$ 26.602 \\
 & \algocatoni & .012 $\pm$ .000 & .339 $\pm$ .002 & .013 $\pm$ .000 & 1196.394 $\pm$ 15.704 & .012 $\pm$ .000 & .159 $\pm$ .003 & .014 $\pm$ .000 & 3838.459 $\pm$ 88.155 & .012 $\pm$ .000 & .070 $\pm$ .002 & .016 $\pm$ .000 & 1218.505 $\pm$ 51.783 & .013 $\pm$ .001 & .037 $\pm$ .001 & .014 $\pm$ .001 & 386.824 $\pm$ 28.233 \\
 & \algorivasplata & .012 $\pm$ .000 & .289 $\pm$ .003 & .013 $\pm$ .000 & 1191.037 $\pm$ 152.759 & .011 $\pm$ .000 & .128 $\pm$ .002 & .014 $\pm$ .000 & 3768.785 $\pm$ 9.947 & .010 $\pm$ .000 & .061 $\pm$ .001 & .013 $\pm$ .000 & 1231.638 $\pm$ 49.362 & .011 $\pm$ .001 & .033 $\pm$ .001 & .012 $\pm$ .000 & 382.225 $\pm$ 28.481 \\
 & \algostoNN & \textemdash & .044 & \textemdash & 12.418 & \textemdash & .046 & \textemdash & 6.179 & \textemdash & .047 & \textemdash & 6.954 & \textemdash & .045 & \textemdash & 8.089 \\
\midrule
\multirow[c]{5}{*}{\rotatebox[origin=c]{90}{\small{Fashion}}} & \algoours & .126 $\pm$ .000 & .137 $\pm$ .000 & .124 $\pm$ .000 & 12.401 & .125 $\pm$ .001 & .132 $\pm$ .001 & .119 $\pm$ .001 & 14.631 & .120 $\pm$ .002 & .128 $\pm$ .002 & .113 $\pm$ .001 & 26.499 & .133 $\pm$ .003 & .143 $\pm$ .003 & .127 $\pm$ .003 & 23.702 \\
 & \algoblanchard & .123 $\pm$ .000 & .602 $\pm$ .003 & .121 $\pm$ .000 & 10558.872 $\pm$ 139.107 & .119 $\pm$ .001 & .383 $\pm$ .004 & .112 $\pm$ .001 & 3893.091 $\pm$ 86.176 & .113 $\pm$ .001 & .239 $\pm$ .003 & .106 $\pm$ .001 & 1204.211 $\pm$ 5.815 & .132 $\pm$ .003 & .195 $\pm$ .004 & .125 $\pm$ .003 & 362.146 $\pm$ 27.801 \\
 & \algocatoni & .126 $\pm$ .000 & .531 $\pm$ .004 & .124 $\pm$ .000 & 11966.223 $\pm$ 148.195 & .125 $\pm$ .001 & .299 $\pm$ .003 & .118 $\pm$ .001 & 3829.806 $\pm$ 85.864 & .119 $\pm$ .002 & .209 $\pm$ .002 & .113 $\pm$ .001 & 1225.310 $\pm$ 48.090 & .134 $\pm$ .004 & .202 $\pm$ .005 & .127 $\pm$ .003 & 395.243 $\pm$ 29.182 \\
 & \algorivasplata & .123 $\pm$ .000 & .458 $\pm$ .003 & .120 $\pm$ .000 & 11209.156 $\pm$ 143.319 & .118 $\pm$ .001 & .287 $\pm$ .002 & .111 $\pm$ .001 & 3815.804 $\pm$ 85.091 & .112 $\pm$ .001 & .196 $\pm$ .002 & .105 $\pm$ .001 & 126.956 $\pm$ 49.255 & .130 $\pm$ .003 & .173 $\pm$ .004 & .124 $\pm$ .003 & 376.904 $\pm$ 27.549 \\
 & \algostoNN & \textemdash & .189 & \textemdash & 6.200 & \textemdash & .184 & \textemdash & 7.316 & \textemdash & .179 & \textemdash & 13.250 & \textemdash & .195 & \textemdash & 11.851 \\
\midrule
\multirow[c]{5}{*}{\rotatebox[origin=c]{90}{\small{CIFAR-10}}} & \algoours & .369 $\pm$ .000 & .379 $\pm$ .000 & .358 $\pm$ .000 & 11.657 & .351 $\pm$ .000 & .369 $\pm$ .000 & .352 $\pm$ .000 & 2.267 & .359 $\pm$ .001 & .378 $\pm$ .000 & .360 $\pm$ .000 & 2.616 & .418 $\pm$ .001 & .434 $\pm$ .001 & .415 $\pm$ .001 & 5.675 \\
 & \algoblanchard & .369 $\pm$ .000 & .990 $\pm$ .000 & .358 $\pm$ .000 & 40152.974 $\pm$ 291.721 & .351 $\pm$ .000 & .809 $\pm$ .003 & .351 $\pm$ .000 & 8753.816 $\pm$ 136.801 & .358 $\pm$ .001 & .635 $\pm$ .004 & .359 $\pm$ .000 & 2728.436 $\pm$ 73.835 & .412 $\pm$ .001 & .568 $\pm$ .004 & .407 $\pm$ .001 & 91.026 $\pm$ 44.096 \\
 & \algocatoni & .369 $\pm$ .000 & .986 $\pm$ .000 & .358 $\pm$ .000 & 24477.984 $\pm$ 223.367 & .351 $\pm$ .000 & .708 $\pm$ .003 & .351 $\pm$ .000 & 8463.452 $\pm$ 135.001 & .357 $\pm$ .001 & .578 $\pm$ .002 & .357 $\pm$ .000 & 3401.221 $\pm$ 84.878 & .405 $\pm$ .001 & .561 $\pm$ .002 & .399 $\pm$ .001 & 1354.100 $\pm$ 51.315 \\
 & \algorivasplata & .369 $\pm$ .000 & .868 $\pm$ .001 & .358 $\pm$ .000 & 24424.968 $\pm$ 223.601 & .351 $\pm$ .000 & .694 $\pm$ .002 & .351 $\pm$ .000 & 8665.339 $\pm$ 136.361 & .358 $\pm$ .001 & .555 $\pm$ .003 & .358 $\pm$ .000 & 274.651 $\pm$ 74.784 & .409 $\pm$ .001 & .521 $\pm$ .003 & .403 $\pm$ .001 & 955.211 $\pm$ 44.609 \\
 & \algostoNN & \textemdash & .448 & \textemdash & 5.829 & \textemdash & .439 & \textemdash & 1.134 & \textemdash & .448 & \textemdash & 1.308 & \textemdash & .504 & \textemdash & 2.838 \\
\bottomrule
\end{tabular}
}
\label{chap:dis-pra:table:1_prior_0.3}
\end{table}
\end{landscape} 

\begin{landscape}
\begin{table}[t]
\caption{
\looseness=-1
Comparison of \algoours, \algorivasplata, \algoblanchard and \algocatoni based on the disintegrated bounds, and \algostoNN based on the randomized bounds learned with two learning rates $\lr{\ \in}\{10^{-4}, 10^{-6}\}$ and different variances $\sigma^2{\in}\{10^{-3}, 10^{-4}, 10^{-5}, 10^{-6}\}$.
We report the test risk ($\Risk_{\dT}(\h)$), the bound value (Bnd), the empirical risk ($\Risk_{\dS}(\h)$), and the divergence (Div) associated with each bound (the \textsc{Rényi} divergence for \algoours, the KL divergence for \algostoNN, and the disintegrated KL divergence for \algorivasplata, \algoblanchard and \algocatoni).
More precisely, we report the mean $\pm$ the standard deviation for $400$ neural networks sampled from $\AQ$ for \algoours, \algorivasplata, \algoblanchard, and \algocatoni.
We consider, in this table, that the split ratio is $0.4$.
}
\resizebox{0.63\paperheight}{!}{
\begin{tabular}{rr|clcl|clcl|clcl|clcl}
\toprule
 &  & \multicolumn{4}{c}{$\sigma^2=10^{-6}$} & \multicolumn{4}{c}{$\sigma^2=10^{-5}$} & \multicolumn{4}{c}{$\sigma^2=10^{-4}$} & \multicolumn{4}{c}{$\sigma^2=10^{-3}$} \\
\midrule
 & $\lr=10^{-6}$ & $\Risk_{\Tcal}(h)$ & Bnd & $\Risk_{\Scal}(h)$ & Div & $\Risk_{\Tcal}(h)$ & Bnd & $\Risk_{\Scal}(h)$ & Div & $\Risk_{\Tcal}(h)$ & Bnd & $\Risk_{\Scal}(h)$ & Div & $\Risk_{\Tcal}(h)$ & Bnd & $\Risk_{\Scal}(h)$ & Div \\
\midrule
\multirow[c]{5}{*}{\rotatebox[origin=c]{90}{\small{MNIST}}} & \algoours & .010 $\pm$ .000 & .017 $\pm$ .000 & .013 $\pm$ .000 & .194 & .012 $\pm$ .000 & .018 $\pm$ .000 & .014 $\pm$ .000 & .138 & .009 $\pm$ .000 & .015 $\pm$ .000 & .011 $\pm$ .000 & .235 & .014 $\pm$ .001 & .020 $\pm$ .001 & .015 $\pm$ .001 & 1.111 \\
 & \algoblanchard & .010 $\pm$ .000 & .028 $\pm$ .001 & .013 $\pm$ .000 & 88.323 $\pm$ 13.740 & .012 $\pm$ .000 & .021 $\pm$ .001 & .014 $\pm$ .000 & 16.792 $\pm$ 5.702 & .009 $\pm$ .000 & .014 $\pm$ .001 & .011 $\pm$ .000 & 2.449 $\pm$ 2.313 & .014 $\pm$ .001 & .019 $\pm$ .001 & .016 $\pm$ .001 & .244 $\pm$ .765 \\
 & \algocatoni & .010 $\pm$ .000 & .026 $\pm$ .001 & .013 $\pm$ .000 & 109.202 $\pm$ 15.634 & .012 $\pm$ .000 & .023 $\pm$ .002 & .014 $\pm$ .000 & 19.918 $\pm$ 6.526 & .009 $\pm$ .000 & .015 $\pm$ .001 & .011 $\pm$ .000 & 2.486 $\pm$ 2.362 & .014 $\pm$ .001 & .019 $\pm$ .001 & .016 $\pm$ .001 & .298 $\pm$ .762 \\
 & \algorivasplata & .010 $\pm$ .000 & .024 $\pm$ .001 & .013 $\pm$ .000 & 91.872 $\pm$ 14.470 & .012 $\pm$ .000 & .019 $\pm$ .001 & .014 $\pm$ .000 & 17.002 $\pm$ 5.882 & .009 $\pm$ .000 & .014 $\pm$ .000 & .011 $\pm$ .000 & 2.529 $\pm$ 2.251 & .014 $\pm$ .001 & .019 $\pm$ .001 & .016 $\pm$ .001 & .308 $\pm$ .778 \\
 & \algostoNN & \textemdash & .043 & \textemdash & .097 & \textemdash & .044 & \textemdash & .069 & \textemdash & .039 & \textemdash & .117 & \textemdash & .047 & \textemdash & .555 \\
\midrule
\multirow[c]{5}{*}{\rotatebox[origin=c]{90}{\small{Fashion}}} & \algoours & .118 $\pm$ .001 & .123 $\pm$ .000 & .112 $\pm$ .000 & .269 & .113 $\pm$ .001 & .118 $\pm$ .001 & .107 $\pm$ .001 & .743 & .117 $\pm$ .002 & .121 $\pm$ .002 & .110 $\pm$ .002 & 2.600 & .131 $\pm$ .004 & .138 $\pm$ .004 & .126 $\pm$ .004 & 1.229 \\
 & \algoblanchard & .118 $\pm$ .001 & .145 $\pm$ .003 & .112 $\pm$ .000 & 82.403 $\pm$ 13.230 & .113 $\pm$ .001 & .123 $\pm$ .002 & .107 $\pm$ .001 & 16.836 $\pm$ 5.583 & .119 $\pm$ .002 & .121 $\pm$ .003 & .112 $\pm$ .003 & 2.641 $\pm$ 2.369 & .133 $\pm$ .004 & .136 $\pm$ .004 & .128 $\pm$ .004 & .297 $\pm$ .731 \\
 & \algocatoni & .118 $\pm$ .001 & .151 $\pm$ .004 & .112 $\pm$ .000 & 109.988 $\pm$ 15.347 & .113 $\pm$ .001 & .120 $\pm$ .002 & .107 $\pm$ .001 & 19.889 $\pm$ 6.689 & .118 $\pm$ .002 & .120 $\pm$ .003 & .112 $\pm$ .003 & 2.615 $\pm$ 2.234 & .132 $\pm$ .004 & .136 $\pm$ .004 & .128 $\pm$ .004 & .300 $\pm$ .811 \\
 & \algorivasplata & .118 $\pm$ .001 & .137 $\pm$ .002 & .112 $\pm$ .000 & 87.804 $\pm$ 13.640 & .113 $\pm$ .001 & .120 $\pm$ .002 & .107 $\pm$ .001 & 17.491 $\pm$ 6.144 & .118 $\pm$ .002 & .121 $\pm$ .003 & .112 $\pm$ .003 & 2.549 $\pm$ 2.175 & .133 $\pm$ .005 & .137 $\pm$ .004 & .128 $\pm$ .004 & .322 $\pm$ .794 \\
 & \algostoNN & \textemdash & .174 & \textemdash & .135 & \textemdash & .168 & \textemdash & .372 & \textemdash & .172 & \textemdash & 1.300 & \textemdash & .191 & \textemdash & .615 \\
\midrule
\multirow[c]{5}{*}{\rotatebox[origin=c]{90}{\small{CIFAR-10}}} & \algoours & .334 $\pm$ .000 & .346 $\pm$ .000 & .328 $\pm$ .000 & .025 & .322 $\pm$ .000 & .331 $\pm$ .000 & .313 $\pm$ .000 & .050 & .323 $\pm$ .001 & .334 $\pm$ .000 & .316 $\pm$ .000 & .160 & .333 $\pm$ .001 & .341 $\pm$ .001 & .323 $\pm$ .001 & .461 \\
 & \algoblanchard & .334 $\pm$ .000 & .421 $\pm$ .004 & .328 $\pm$ .000 & 269.875 $\pm$ 23.982 & .322 $\pm$ .000 & .364 $\pm$ .004 & .313 $\pm$ .000 & 83.082 $\pm$ 13.029 & .323 $\pm$ .001 & .345 $\pm$ .004 & .316 $\pm$ .000 & 21.614 $\pm$ 6.670 & .333 $\pm$ .001 & .340 $\pm$ .002 & .323 $\pm$ .001 & 3.630 $\pm$ 2.750 \\
 & \algocatoni & .334 $\pm$ .000 & .433 $\pm$ .008 & .328 $\pm$ .000 & 27.270 $\pm$ 24.201 & .322 $\pm$ .000 & .355 $\pm$ .005 & .313 $\pm$ .000 & 84.148 $\pm$ 13.578 & .323 $\pm$ .001 & .338 $\pm$ .002 & .316 $\pm$ .000 & 22.547 $\pm$ 6.801 & .333 $\pm$ .001 & .338 $\pm$ .001 & .323 $\pm$ .001 & 3.831 $\pm$ 2.801 \\
 & \algorivasplata & .334 $\pm$ .000 & .394 $\pm$ .003 & .328 $\pm$ .000 & 27.133 $\pm$ 24.109 & .322 $\pm$ .000 & .351 $\pm$ .003 & .313 $\pm$ .000 & 83.438 $\pm$ 13.033 & .323 $\pm$ .001 & .339 $\pm$ .002 & .316 $\pm$ .000 & 21.688 $\pm$ 6.718 & .333 $\pm$ .001 & .339 $\pm$ .002 & .323 $\pm$ .001 & 3.667 $\pm$ 2.757 \\
 & \algostoNN & \textemdash & .414 & \textemdash & .013 & \textemdash & .399 & \textemdash & .025 & \textemdash & .403 & \textemdash & .080 & \textemdash & .409 & \textemdash & .230 \\
\midrule
 &  & \multicolumn{4}{c}{$\sigma^2=10^{-6}$} & \multicolumn{4}{c}{$\sigma^2=10^{-5}$} & \multicolumn{4}{c}{$\sigma^2=10^{-4}$} & \multicolumn{4}{c}{$\sigma^2=10^{-3}$} \\
\midrule
 & $\lr=10^{-4}$ & $\Risk_{\Tcal}(h)$ & Bnd & $\Risk_{\Scal}(h)$ & Div & $\Risk_{\Tcal}(h)$ & Bnd & $\Risk_{\Scal}(h)$ & Div & $\Risk_{\Tcal}(h)$ & Bnd & $\Risk_{\Scal}(h)$ & Div & $\Risk_{\Tcal}(h)$ & Bnd & $\Risk_{\Scal}(h)$ & Div \\
\midrule
\multirow[c]{5}{*}{\rotatebox[origin=c]{90}{\small{MNIST}}} & \algoours & .010 $\pm$ .000 & .019 $\pm$ .000 & .013 $\pm$ .000 & 23.992 & .012 $\pm$ .000 & .019 $\pm$ .000 & .014 $\pm$ .000 & 7.767 & .009 $\pm$ .000 & .015 $\pm$ .000 & .011 $\pm$ .000 & 3.165 & .012 $\pm$ .001 & .019 $\pm$ .001 & .013 $\pm$ .001 & 18.413 \\
 & \algoblanchard & .010 $\pm$ .000 & .500 $\pm$ .004 & .013 $\pm$ .000 & 1123.328 $\pm$ 151.115 & .012 $\pm$ .000 & .236 $\pm$ .004 & .014 $\pm$ .000 & 381.840 $\pm$ 94.218 & .009 $\pm$ .000 & .096 $\pm$ .003 & .011 $\pm$ .000 & 1184.214 $\pm$ 47.208 & .011 $\pm$ .001 & .048 $\pm$ .002 & .012 $\pm$ .001 & 363.194 $\pm$ 26.547 \\
 & \algocatoni & .010 $\pm$ .000 & .369 $\pm$ .003 & .013 $\pm$ .000 & 1191.598 $\pm$ 154.180 & .012 $\pm$ .000 & .180 $\pm$ .003 & .014 $\pm$ .000 & 3826.581 $\pm$ 85.362 & .009 $\pm$ .000 & .070 $\pm$ .002 & .011 $\pm$ .000 & 1217.723 $\pm$ 49.984 & .012 $\pm$ .001 & .039 $\pm$ .002 & .014 $\pm$ .001 & 384.476 $\pm$ 29.126 \\
 & \algorivasplata & .010 $\pm$ .000 & .316 $\pm$ .003 & .013 $\pm$ .000 & 11557.703 $\pm$ 151.498 & .012 $\pm$ .000 & .142 $\pm$ .002 & .014 $\pm$ .000 & 3751.391 $\pm$ 84.542 & .009 $\pm$ .000 & .061 $\pm$ .002 & .011 $\pm$ .000 & 1172.156 $\pm$ 46.933 & .010 $\pm$ .001 & .035 $\pm$ .001 & .012 $\pm$ .001 & 373.003 $\pm$ 27.844 \\
 & \algostoNN & \textemdash & .045 & \textemdash & 11.996 & \textemdash & .045 & \textemdash & 3.884 & \textemdash & .040 & \textemdash & 1.583 & \textemdash & .045 & \textemdash & 9.207 \\
\midrule
\multirow[c]{5}{*}{\rotatebox[origin=c]{90}{\small{Fashion}}} & \algoours & .118 $\pm$ .000 & .127 $\pm$ .000 & .112 $\pm$ .000 & 17.987 & .113 $\pm$ .001 & .119 $\pm$ .001 & .107 $\pm$ .001 & 6.361 & .114 $\pm$ .002 & .123 $\pm$ .002 & .107 $\pm$ .002 & 22.582 & .125 $\pm$ .003 & .137 $\pm$ .003 & .122 $\pm$ .003 & 16.872 \\
 & \algoblanchard & .115 $\pm$ .001 & .659 $\pm$ .004 & .110 $\pm$ .000 & 11835.780 $\pm$ 161.816 & .110 $\pm$ .001 & .395 $\pm$ .004 & .104 $\pm$ .000 & 3828.562 $\pm$ 94.279 & .108 $\pm$ .001 & .244 $\pm$ .004 & .102 $\pm$ .001 & 1185.882 $\pm$ 5.575 & .123 $\pm$ .003 & .192 $\pm$ .004 & .119 $\pm$ .002 & 346.265 $\pm$ 27.827 \\
 & \algocatoni & .118 $\pm$ .001 & .566 $\pm$ .004 & .112 $\pm$ .000 & 11921.114 $\pm$ 153.739 & .113 $\pm$ .001 & .304 $\pm$ .003 & .107 $\pm$ .000 & 3822.647 $\pm$ 85.225 & .114 $\pm$ .002 & .208 $\pm$ .003 & .107 $\pm$ .002 & 1217.879 $\pm$ 52.353 & .125 $\pm$ .003 & .196 $\pm$ .004 & .121 $\pm$ .002 & 388.473 $\pm$ 29.475 \\
 & \algorivasplata & .114 $\pm$ .000 & .476 $\pm$ .003 & .109 $\pm$ .000 & 11206.239 $\pm$ 149.549 & .110 $\pm$ .001 & .292 $\pm$ .003 & .103 $\pm$ .000 & 3745.930 $\pm$ 84.367 & .106 $\pm$ .001 & .197 $\pm$ .003 & .101 $\pm$ .001 & 1229.005 $\pm$ 51.052 & .122 $\pm$ .003 & .170 $\pm$ .004 & .118 $\pm$ .003 & 361.652 $\pm$ 28.452 \\
 & \algostoNN & \textemdash & .177 & \textemdash & 8.994 & \textemdash & .169 & \textemdash & 3.180 & \textemdash & .172 & \textemdash & 11.291 & \textemdash & .189 & \textemdash & 8.436 \\
\midrule
\multirow[c]{5}{*}{\rotatebox[origin=c]{90}{\small{CIFAR-10}}} & \algoours & .334 $\pm$ .000 & .350 $\pm$ .000 & .328 $\pm$ .000 & 12.067 & .322 $\pm$ .000 & .332 $\pm$ .000 & .313 $\pm$ .000 & 4.172 & .323 $\pm$ .001 & .336 $\pm$ .000 & .316 $\pm$ .000 & 3.382 & .332 $\pm$ .001 & .343 $\pm$ .001 & .322 $\pm$ .001 & 6.855 \\
 & \algoblanchard & .334 $\pm$ .000 & .977 $\pm$ .001 & .328 $\pm$ .000 & 28565.558 $\pm$ 245.568 & .322 $\pm$ .000 & .803 $\pm$ .003 & .313 $\pm$ .000 & 8479.553 $\pm$ 126.804 & .321 $\pm$ .001 & .614 $\pm$ .004 & .315 $\pm$ .000 & 2727.786 $\pm$ 7.572 & .327 $\pm$ .001 & .487 $\pm$ .004 & .317 $\pm$ .001 & 887.578 $\pm$ 42.449 \\
 & \algocatoni & .334 $\pm$ .000 & .983 $\pm$ .000 & .328 $\pm$ .000 & 24136.528 $\pm$ 211.963 & .322 $\pm$ .000 & .694 $\pm$ .004 & .313 $\pm$ .000 & 7928.671 $\pm$ 122.159 & .320 $\pm$ .001 & .515 $\pm$ .002 & .314 $\pm$ .000 & 237.703 $\pm$ 65.952 & .323 $\pm$ .001 & .468 $\pm$ .002 & .312 $\pm$ .001 & 1157.073 $\pm$ 47.283 \\
 & \algorivasplata & .334 $\pm$ .000 & .922 $\pm$ .001 & .328 $\pm$ .000 & 33282.032 $\pm$ 246.654 & .322 $\pm$ .000 & .680 $\pm$ .003 & .312 $\pm$ .000 & 8493.458 $\pm$ 128.894 & .320 $\pm$ .001 & .527 $\pm$ .003 & .314 $\pm$ .000 & 2739.108 $\pm$ 7.556 & .325 $\pm$ .001 & .436 $\pm$ .003 & .314 $\pm$ .001 & 91.066 $\pm$ 43.389 \\
 & \algostoNN & \textemdash & .417 & \textemdash & 6.033 & \textemdash & .400 & \textemdash & 2.086 & \textemdash & .403 & \textemdash & 1.691 & \textemdash & .410 & \textemdash & 3.427 \\
\bottomrule
\end{tabular}
}
\label{chap:dis-pra:table:1_prior_0.4}
\end{table}
\end{landscape} 

\begin{landscape}
\begin{table}[t]
\caption{
\looseness=-1
Comparison of \algoours, \algorivasplata, \algoblanchard and \algocatoni based on the disintegrated bounds, and \algostoNN based on the randomized bounds learned with two learning rates $\lr{\ \in}\{10^{-4}, 10^{-6}\}$ and different variances $\sigma^2{\in}\{10^{-3}, 10^{-4}, 10^{-5}, 10^{-6}\}$.
We report the test risk ($\Risk_{\dT}(\h)$), the bound value (Bnd), the empirical risk ($\Risk_{\dS}(\h)$), and the divergence (Div) associated with each bound (the \textsc{Rényi} divergence for \algoours, the KL divergence for \algostoNN, and the disintegrated KL divergence for \algorivasplata, \algoblanchard and \algocatoni).
More precisely, we report the mean $\pm$ the standard deviation for $400$ neural networks sampled from $\AQ$ for \algoours, \algorivasplata, \algoblanchard, and \algocatoni.
We consider, in this table, that the split ratio is $0.5$.
}
\resizebox{0.63\paperheight}{!}{
\begin{tabular}{rr|clcl|clcl|clcl|clcl}
\toprule
 &  & \multicolumn{4}{c}{$\sigma^2=10^{-6}$} & \multicolumn{4}{c}{$\sigma^2=10^{-5}$} & \multicolumn{4}{c}{$\sigma^2=10^{-4}$} & \multicolumn{4}{c}{$\sigma^2=10^{-3}$} \\
\midrule
 & $\lr=10^{-6}$ & $\Risk_{\Tcal}(h)$ & Bnd & $\Risk_{\Scal}(h)$ & Div & $\Risk_{\Tcal}(h)$ & Bnd & $\Risk_{\Scal}(h)$ & Div & $\Risk_{\Tcal}(h)$ & Bnd & $\Risk_{\Scal}(h)$ & Div & $\Risk_{\Tcal}(h)$ & Bnd & $\Risk_{\Scal}(h)$ & Div \\
\midrule
\multirow[c]{5}{*}{\rotatebox[origin=c]{90}{\small{MNIST}}} & \algoours & .008 $\pm$ .000 & .015 $\pm$ .000 & .010 $\pm$ .000 & .084 & .006 $\pm$ .000 & .012 $\pm$ .000 & .009 $\pm$ .000 & .053 & .008 $\pm$ .000 & .014 $\pm$ .000 & .010 $\pm$ .000 & .179 & .014 $\pm$ .001 & .019 $\pm$ .001 & .014 $\pm$ .001 & .576 \\
 & \algoblanchard & .008 $\pm$ .000 & .025 $\pm$ .001 & .010 $\pm$ .000 & 81.167 $\pm$ 12.801 & .006 $\pm$ .000 & .014 $\pm$ .001 & .009 $\pm$ .000 & 15.518 $\pm$ 5.438 & .009 $\pm$ .000 & .014 $\pm$ .001 & .010 $\pm$ .000 & 2.140 $\pm$ 2.072 & .015 $\pm$ .001 & .018 $\pm$ .001 & .015 $\pm$ .001 & .284 $\pm$ .649 \\
 & \algocatoni & .008 $\pm$ .000 & .022 $\pm$ .001 & .010 $\pm$ .000 & 104.063 $\pm$ 14.662 & .006 $\pm$ .000 & .015 $\pm$ .000 & .009 $\pm$ .000 & 17.676 $\pm$ 5.963 & .008 $\pm$ .000 & .014 $\pm$ .001 & .010 $\pm$ .000 & 2.152 $\pm$ 2.085 & .015 $\pm$ .001 & .018 $\pm$ .001 & .015 $\pm$ .001 & .252 $\pm$ .680 \\
 & \algorivasplata & .008 $\pm$ .000 & .021 $\pm$ .001 & .010 $\pm$ .000 & 84.581 $\pm$ 13.035 & .006 $\pm$ .000 & .013 $\pm$ .001 & .009 $\pm$ .000 & 15.545 $\pm$ 5.594 & .008 $\pm$ .000 & .014 $\pm$ .000 & .010 $\pm$ .000 & 2.185 $\pm$ 1.992 & .015 $\pm$ .001 & .018 $\pm$ .001 & .015 $\pm$ .001 & .276 $\pm$ .693 \\
 & \algostoNN & \textemdash & .039 & \textemdash & .042 & \textemdash & .035 & \textemdash & .026 & \textemdash & .038 & \textemdash & .090 & \textemdash & .045 & \textemdash & .288 \\
\midrule
\multirow[c]{5}{*}{\rotatebox[origin=c]{90}{\small{Fashion}}} & \algoours & .106 $\pm$ .000 & .113 $\pm$ .000 & .101 $\pm$ .000 & .133 & .104 $\pm$ .001 & .110 $\pm$ .000 & .099 $\pm$ .000 & .327 & .108 $\pm$ .002 & .112 $\pm$ .001 & .101 $\pm$ .001 & .903 & .120 $\pm$ .004 & .127 $\pm$ .003 & .115 $\pm$ .003 & .868 \\
 & \algoblanchard & .106 $\pm$ .000 & .136 $\pm$ .003 & .101 $\pm$ .000 & 77.573 $\pm$ 12.564 & .104 $\pm$ .001 & .115 $\pm$ .003 & .099 $\pm$ .000 & 15.278 $\pm$ 5.599 & .109 $\pm$ .002 & .111 $\pm$ .002 & .102 $\pm$ .001 & 2.153 $\pm$ 2.081 & .122 $\pm$ .004 & .126 $\pm$ .004 & .117 $\pm$ .004 & .248 $\pm$ .715 \\
 & \algocatoni & .106 $\pm$ .000 & .145 $\pm$ .005 & .101 $\pm$ .000 & 104.356 $\pm$ 14.712 & .104 $\pm$ .001 & .112 $\pm$ .002 & .099 $\pm$ .000 & 17.566 $\pm$ 5.996 & .109 $\pm$ .002 & .110 $\pm$ .001 & .102 $\pm$ .001 & 2.217 $\pm$ 2.084 & .122 $\pm$ .004 & .125 $\pm$ .004 & .117 $\pm$ .004 & .262 $\pm$ .699 \\
 & \algorivasplata & .106 $\pm$ .000 & .127 $\pm$ .002 & .101 $\pm$ .000 & 82.150 $\pm$ 12.955 & .104 $\pm$ .001 & .112 $\pm$ .001 & .099 $\pm$ .000 & 15.509 $\pm$ 5.629 & .109 $\pm$ .002 & .111 $\pm$ .001 & .102 $\pm$ .001 & 2.178 $\pm$ 2.060 & .122 $\pm$ .004 & .126 $\pm$ .004 & .117 $\pm$ .004 & .264 $\pm$ .704 \\
 & \algostoNN & \textemdash & .162 & \textemdash & .066 & \textemdash & .159 & \textemdash & .164 & \textemdash & .162 & \textemdash & .451 & \textemdash & .179 & \textemdash & .434 \\
\midrule
\multirow[c]{5}{*}{\rotatebox[origin=c]{90}{\small{CIFAR-10}}} & \algoours & .312 $\pm$ .000 & .323 $\pm$ .000 & .304 $\pm$ .000 & .027 & .281 $\pm$ .000 & .304 $\pm$ .000 & .285 $\pm$ .000 & .035 & .298 $\pm$ .001 & .310 $\pm$ .000 & .291 $\pm$ .000 & .101 & .315 $\pm$ .001 & .329 $\pm$ .001 & .309 $\pm$ .001 & .368 \\
 & \algoblanchard & .312 $\pm$ .000 & .405 $\pm$ .004 & .304 $\pm$ .000 & 268.149 $\pm$ 22.835 & .281 $\pm$ .000 & .339 $\pm$ .004 & .285 $\pm$ .000 & 8.690 $\pm$ 12.628 & .298 $\pm$ .001 & .320 $\pm$ .004 & .291 $\pm$ .000 & 19.648 $\pm$ 6.249 & .315 $\pm$ .001 & .327 $\pm$ .003 & .310 $\pm$ .001 & 3.213 $\pm$ 2.590 \\
 & \algocatoni & .312 $\pm$ .000 & .428 $\pm$ .009 & .304 $\pm$ .000 & 269.415 $\pm$ 22.884 & .281 $\pm$ .000 & .333 $\pm$ .005 & .285 $\pm$ .000 & 83.414 $\pm$ 13.018 & .298 $\pm$ .001 & .314 $\pm$ .003 & .291 $\pm$ .000 & 2.711 $\pm$ 6.481 & .315 $\pm$ .001 & .326 $\pm$ .001 & .310 $\pm$ .001 & 3.273 $\pm$ 2.597 \\
 & \algorivasplata & .312 $\pm$ .000 & .375 $\pm$ .003 & .304 $\pm$ .000 & 268.589 $\pm$ 22.845 & .281 $\pm$ .000 & .325 $\pm$ .003 & .285 $\pm$ .000 & 81.532 $\pm$ 12.712 & .298 $\pm$ .001 & .315 $\pm$ .002 & .291 $\pm$ .000 & 19.813 $\pm$ 6.288 & .315 $\pm$ .001 & .327 $\pm$ .002 & .310 $\pm$ .001 & 3.233 $\pm$ 2.599 \\
 & \algostoNN & \textemdash & .391 & \textemdash & .013 & \textemdash & .370 & \textemdash & .017 & \textemdash & .377 & \textemdash & .050 & \textemdash & .397 & \textemdash & .184 \\
\midrule
 &  & \multicolumn{4}{c}{$\sigma^2=10^{-6}$} & \multicolumn{4}{c}{$\sigma^2=10^{-5}$} & \multicolumn{4}{c}{$\sigma^2=10^{-4}$} & \multicolumn{4}{c}{$\sigma^2=10^{-3}$} \\
\midrule
 & $\lr=10^{-4}$ & $\Risk_{\Tcal}(h)$ & Bnd & $\Risk_{\Scal}(h)$ & Div & $\Risk_{\Tcal}(h)$ & Bnd & $\Risk_{\Scal}(h)$ & Div & $\Risk_{\Tcal}(h)$ & Bnd & $\Risk_{\Scal}(h)$ & Div & $\Risk_{\Tcal}(h)$ & Bnd & $\Risk_{\Scal}(h)$ & Div \\
\midrule
\multirow[c]{5}{*}{\rotatebox[origin=c]{90}{\small{MNIST}}} & \algoours & .008 $\pm$ .000 & .017 $\pm$ .000 & .010 $\pm$ .000 & 29.993 & .006 $\pm$ .000 & .013 $\pm$ .000 & .009 $\pm$ .000 & 3.162 & .008 $\pm$ .000 & .015 $\pm$ .000 & .010 $\pm$ .000 & 1.418 & .013 $\pm$ .001 & .019 $\pm$ .001 & .013 $\pm$ .001 & 12.231 \\
 & \algoblanchard & .008 $\pm$ .000 & .574 $\pm$ .005 & .010 $\pm$ .000 & 11894.556 $\pm$ 155.958 & .006 $\pm$ .000 & .256 $\pm$ .004 & .009 $\pm$ .000 & 3826.515 $\pm$ 86.973 & .008 $\pm$ .000 & .108 $\pm$ .003 & .010 $\pm$ .000 & 1184.777 $\pm$ 48.158 & .010 $\pm$ .001 & .052 $\pm$ .002 & .011 $\pm$ .000 & 36.865 $\pm$ 28.054 \\
 & \algocatoni & .008 $\pm$ .000 & .396 $\pm$ .003 & .010 $\pm$ .000 & 11986.455 $\pm$ 15.722 & .006 $\pm$ .000 & .192 $\pm$ .002 & .009 $\pm$ .000 & 3824.971 $\pm$ 85.072 & .008 $\pm$ .000 & .079 $\pm$ .002 & .010 $\pm$ .000 & 1213.611 $\pm$ 48.751 & .013 $\pm$ .001 & .042 $\pm$ .002 & .014 $\pm$ .001 & 384.275 $\pm$ 28.556 \\
 & \algorivasplata & .008 $\pm$ .000 & .362 $\pm$ .003 & .010 $\pm$ .000 & 11905.971 $\pm$ 15.609 & .006 $\pm$ .000 & .148 $\pm$ .003 & .009 $\pm$ .000 & 377.259 $\pm$ 84.127 & .008 $\pm$ .000 & .067 $\pm$ .002 & .010 $\pm$ .000 & 118.841 $\pm$ 5.043 & .010 $\pm$ .001 & .036 $\pm$ .001 & .011 $\pm$ .000 & 369.675 $\pm$ 27.947 \\
 & \algostoNN & \textemdash & .041 & \textemdash & 14.996 & \textemdash & .035 & \textemdash & 1.581 & \textemdash & .039 & \textemdash & .709 & \textemdash & .045 & \textemdash & 6.116 \\
\midrule
\multirow[c]{5}{*}{\rotatebox[origin=c]{90}{\small{Fashion}}} & \algoours & .106 $\pm$ .000 & .114 $\pm$ .000 & .101 $\pm$ .000 & 6.310 & .103 $\pm$ .001 & .113 $\pm$ .000 & .099 $\pm$ .000 & 9.312 & .106 $\pm$ .002 & .115 $\pm$ .001 & .100 $\pm$ .001 & 14.924 & .115 $\pm$ .003 & .126 $\pm$ .003 & .110 $\pm$ .002 & 18.364 \\
 & \algoblanchard & .105 $\pm$ .000 & .674 $\pm$ .004 & .101 $\pm$ .000 & 10795.464 $\pm$ 143.426 & .102 $\pm$ .000 & .412 $\pm$ .004 & .098 $\pm$ .000 & 3685.940 $\pm$ 82.481 & .103 $\pm$ .001 & .253 $\pm$ .004 & .097 $\pm$ .001 & 1178.401 $\pm$ 48.359 & .113 $\pm$ .002 & .186 $\pm$ .004 & .108 $\pm$ .002 & 338.697 $\pm$ 27.104 \\
 & \algocatoni & .106 $\pm$ .000 & .623 $\pm$ .005 & .101 $\pm$ .000 & 11971.564 $\pm$ 15.589 & .104 $\pm$ .001 & .321 $\pm$ .004 & .099 $\pm$ .000 & 3825.370 $\pm$ 87.728 & .107 $\pm$ .002 & .208 $\pm$ .003 & .100 $\pm$ .001 & 1214.976 $\pm$ 48.846 & .116 $\pm$ .003 & .184 $\pm$ .004 & .111 $\pm$ .003 & 388.197 $\pm$ 27.580 \\
 & \algorivasplata & .105 $\pm$ .000 & .503 $\pm$ .003 & .100 $\pm$ .000 & 11139.304 $\pm$ 15.540 & .102 $\pm$ .000 & .307 $\pm$ .003 & .097 $\pm$ .000 & 381.075 $\pm$ 87.924 & .102 $\pm$ .001 & .201 $\pm$ .003 & .096 $\pm$ .001 & 1201.832 $\pm$ 48.877 & .112 $\pm$ .002 & .161 $\pm$ .003 & .107 $\pm$ .002 & 349.146 $\pm$ 27.482 \\
 & \algostoNN & \textemdash & .163 & \textemdash & 3.155 & \textemdash & .161 & \textemdash & 4.656 & \textemdash & .163 & \textemdash & 7.462 & \textemdash & .176 & \textemdash & 9.182 \\
\midrule
\multirow[c]{5}{*}{\rotatebox[origin=c]{90}{\small{CIFAR-10}}} & \algoours & .312 $\pm$ .000 & .328 $\pm$ .000 & .304 $\pm$ .000 & 12.006 & .281 $\pm$ .000 & .304 $\pm$ .000 & .285 $\pm$ .000 & 1.802 & .297 $\pm$ .001 & .311 $\pm$ .000 & .291 $\pm$ .000 & 2.056 & .314 $\pm$ .001 & .330 $\pm$ .001 & .309 $\pm$ .001 & 4.782 \\
 & \algoblanchard & .312 $\pm$ .000 & .990 $\pm$ .000 & .304 $\pm$ .000 & 48007.471 $\pm$ 31.730 & .280 $\pm$ .000 & .825 $\pm$ .003 & .284 $\pm$ .000 & 8824.774 $\pm$ 134.331 & .296 $\pm$ .001 & .617 $\pm$ .004 & .290 $\pm$ .000 & 2723.775 $\pm$ 66.832 & .309 $\pm$ .001 & .490 $\pm$ .004 & .303 $\pm$ .001 & 888.277 $\pm$ 41.530 \\
 & \algocatoni & .312 $\pm$ .000 & .980 $\pm$ .000 & .304 $\pm$ .000 & 21278.808 $\pm$ 207.839 & .280 $\pm$ .000 & .681 $\pm$ .004 & .284 $\pm$ .000 & 6951.932 $\pm$ 118.540 & .296 $\pm$ .001 & .496 $\pm$ .003 & .290 $\pm$ .000 & 2145.470 $\pm$ 6.045 & .305 $\pm$ .001 & .457 $\pm$ .002 & .299 $\pm$ .001 & 103.494 $\pm$ 47.021 \\
 & \algorivasplata & .312 $\pm$ .000 & .964 $\pm$ .001 & .304 $\pm$ .000 & 42834.626 $\pm$ 284.116 & .280 $\pm$ .000 & .690 $\pm$ .003 & .284 $\pm$ .000 & 8675.531 $\pm$ 136.658 & .296 $\pm$ .001 & .521 $\pm$ .003 & .290 $\pm$ .000 & 2718.415 $\pm$ 66.664 & .307 $\pm$ .001 & .434 $\pm$ .003 & .301 $\pm$ .001 & 921.068 $\pm$ 42.158 \\
 & \algostoNN & \textemdash & .394 & \textemdash & 6.003 & \textemdash & .371 & \textemdash & .901 & \textemdash & .378 & \textemdash & 1.028 & \textemdash & .397 & \textemdash & 2.391 \\
\bottomrule
\end{tabular}
}
\label{chap:dis-pra:table:1_prior_0.5}
\end{table}
\end{landscape} 

\begin{landscape}
\begin{table}[t]
\caption{
\looseness=-1
Comparison of \algoours, \algorivasplata, \algoblanchard and \algocatoni based on the disintegrated bounds, and \algostoNN based on the randomized bounds learned with two learning rates $\lr{\ \in}\{10^{-4}, 10^{-6}\}$ and different variances $\sigma^2{\in}\{10^{-3}, 10^{-4}, 10^{-5}, 10^{-6}\}$.
We report the test risk ($\Risk_{\dT}(\h)$), the bound value (Bnd), the empirical risk ($\Risk_{\dS}(\h)$), and the divergence (Div) associated with each bound (the \textsc{Rényi} divergence for \algoours, the KL divergence for \algostoNN, and the disintegrated KL divergence for \algorivasplata, \algoblanchard and \algocatoni).
More precisely, we report the mean $\pm$ the standard deviation for $400$ neural networks sampled from $\AQ$ for \algoours, \algorivasplata, \algoblanchard, and \algocatoni.
We consider, in this table, that the split ratio is $0.6$.
}
\resizebox{0.63\paperheight}{!}{
\begin{tabular}{rr|clcl|clcl|clcl|clcl}
\toprule
 &  & \multicolumn{4}{c}{$\sigma^2=10^{-6}$} & \multicolumn{4}{c}{$\sigma^2=10^{-5}$} & \multicolumn{4}{c}{$\sigma^2=10^{-4}$} & \multicolumn{4}{c}{$\sigma^2=10^{-3}$} \\
\midrule
 & $\lr=10^{-6}$ & $\Risk_{\Tcal}(h)$ & Bnd & $\Risk_{\Scal}(h)$ & Div & $\Risk_{\Tcal}(h)$ & Bnd & $\Risk_{\Scal}(h)$ & Div & $\Risk_{\Tcal}(h)$ & Bnd & $\Risk_{\Scal}(h)$ & Div & $\Risk_{\Tcal}(h)$ & Bnd & $\Risk_{\Scal}(h)$ & Div \\
\midrule
\multirow[c]{5}{*}{\rotatebox[origin=c]{90}{\small{MNIST}}} & \algoours & .008 $\pm$ .000 & .014 $\pm$ .000 & .010 $\pm$ .000 & .040 & .007 $\pm$ .000 & .014 $\pm$ .000 & .009 $\pm$ .000 & .068 & .008 $\pm$ .000 & .013 $\pm$ .000 & .009 $\pm$ .000 & .092 & .008 $\pm$ .000 & .014 $\pm$ .001 & .009 $\pm$ .000 & .128 \\
 & \algoblanchard & .008 $\pm$ .000 & .026 $\pm$ .002 & .010 $\pm$ .000 & 75.043 $\pm$ 11.586 & .007 $\pm$ .000 & .016 $\pm$ .001 & .009 $\pm$ .000 & 13.220 $\pm$ 4.956 & .008 $\pm$ .000 & .012 $\pm$ .001 & .009 $\pm$ .000 & 1.774 $\pm$ 1.772 & .008 $\pm$ .000 & .012 $\pm$ .001 & .009 $\pm$ .000 & .190 $\pm$ .594 \\
 & \algocatoni & .008 $\pm$ .000 & .022 $\pm$ .001 & .010 $\pm$ .000 & 96.561 $\pm$ 13.980 & .007 $\pm$ .000 & .016 $\pm$ .000 & .009 $\pm$ .000 & 15.107 $\pm$ 5.370 & .008 $\pm$ .000 & .013 $\pm$ .001 & .009 $\pm$ .000 & 1.835 $\pm$ 1.837 & .008 $\pm$ .000 & .013 $\pm$ .000 & .009 $\pm$ .000 & .219 $\pm$ .619 \\
 & \algorivasplata & .008 $\pm$ .000 & .021 $\pm$ .001 & .010 $\pm$ .000 & 76.898 $\pm$ 12.301 & .007 $\pm$ .000 & .014 $\pm$ .001 & .009 $\pm$ .000 & 13.370 $\pm$ 4.931 & .008 $\pm$ .000 & .013 $\pm$ .000 & .009 $\pm$ .000 & 1.695 $\pm$ 1.741 & .008 $\pm$ .000 & .013 $\pm$ .001 & .009 $\pm$ .000 & .183 $\pm$ .580 \\
 & \algostoNN & \textemdash & .038 & \textemdash & .020 & \textemdash & .037 & \textemdash & .034 & \textemdash & .037 & \textemdash & .046 & \textemdash & .037 & \textemdash & .064 \\
\midrule
\multirow[c]{5}{*}{\rotatebox[origin=c]{90}{\small{Fashion}}} & \algoours & .109 $\pm$ .000 & .115 $\pm$ .000 & .102 $\pm$ .000 & .128 & .114 $\pm$ .001 & .117 $\pm$ .001 & .104 $\pm$ .001 & .436 & .101 $\pm$ .001 & .108 $\pm$ .001 & .096 $\pm$ .001 & .452 & .110 $\pm$ .003 & .116 $\pm$ .003 & .103 $\pm$ .003 & .438 \\
 & \algoblanchard & .109 $\pm$ .000 & .139 $\pm$ .003 & .102 $\pm$ .000 & 7.878 $\pm$ 11.599 & .114 $\pm$ .001 & .121 $\pm$ .003 & .104 $\pm$ .001 & 13.041 $\pm$ 5.012 & .102 $\pm$ .001 & .106 $\pm$ .002 & .096 $\pm$ .001 & 1.840 $\pm$ 1.864 & .111 $\pm$ .003 & .113 $\pm$ .003 & .104 $\pm$ .002 & .184 $\pm$ .600 \\
 & \algocatoni & .109 $\pm$ .000 & .152 $\pm$ .006 & .102 $\pm$ .000 & 96.732 $\pm$ 13.464 & .114 $\pm$ .001 & .119 $\pm$ .002 & .104 $\pm$ .001 & 15.103 $\pm$ 5.363 & .102 $\pm$ .001 & .105 $\pm$ .001 & .096 $\pm$ .001 & 1.825 $\pm$ 1.886 & .111 $\pm$ .003 & .112 $\pm$ .003 & .104 $\pm$ .003 & .224 $\pm$ .610 \\
 & \algorivasplata & .109 $\pm$ .000 & .129 $\pm$ .002 & .102 $\pm$ .000 & 75.029 $\pm$ 11.918 & .114 $\pm$ .001 & .118 $\pm$ .002 & .104 $\pm$ .001 & 13.495 $\pm$ 5.112 & .102 $\pm$ .001 & .106 $\pm$ .001 & .096 $\pm$ .001 & 1.798 $\pm$ 1.859 & .111 $\pm$ .003 & .114 $\pm$ .003 & .104 $\pm$ .002 & .219 $\pm$ .610 \\
 & \algostoNN & \textemdash & .164 & \textemdash & .064 & \textemdash & .167 & \textemdash & .218 & \textemdash & .157 & \textemdash & .226 & \textemdash & .165 & \textemdash & .219 \\
\midrule
\multirow[c]{5}{*}{\rotatebox[origin=c]{90}{\small{CIFAR-10}}} & \algoours & .277 $\pm$ .000 & .297 $\pm$ .000 & .276 $\pm$ .000 & .021 & .288 $\pm$ .000 & .307 $\pm$ .000 & .286 $\pm$ .000 & .027 & .273 $\pm$ .001 & .284 $\pm$ .000 & .263 $\pm$ .000 & .079 & .281 $\pm$ .001 & .302 $\pm$ .001 & .281 $\pm$ .001 & .227 \\
 & \algoblanchard & .277 $\pm$ .000 & .386 $\pm$ .005 & .276 $\pm$ .000 & 262.952 $\pm$ 24.385 & .288 $\pm$ .000 & .346 $\pm$ .005 & .286 $\pm$ .000 & 76.609 $\pm$ 12.923 & .273 $\pm$ .001 & .293 $\pm$ .004 & .263 $\pm$ .000 & 17.724 $\pm$ 6.241 & .281 $\pm$ .001 & .299 $\pm$ .002 & .281 $\pm$ .001 & 2.580 $\pm$ 2.299 \\
 & \algocatoni & .277 $\pm$ .000 & .398 $\pm$ .001 & .276 $\pm$ .000 & 268.083 $\pm$ 24.567 & .288 $\pm$ .000 & .343 $\pm$ .007 & .286 $\pm$ .000 & 82.887 $\pm$ 13.493 & .273 $\pm$ .001 & .287 $\pm$ .003 & .263 $\pm$ .000 & 18.978 $\pm$ 6.437 & .281 $\pm$ .001 & .297 $\pm$ .001 & .281 $\pm$ .001 & 2.661 $\pm$ 2.317 \\
 & \algorivasplata & .277 $\pm$ .000 & .354 $\pm$ .004 & .276 $\pm$ .000 & 263.581 $\pm$ 24.435 & .288 $\pm$ .000 & .330 $\pm$ .003 & .286 $\pm$ .000 & 77.488 $\pm$ 12.464 & .273 $\pm$ .001 & .288 $\pm$ .002 & .263 $\pm$ .000 & 17.704 $\pm$ 5.927 & .281 $\pm$ .001 & .299 $\pm$ .002 & .281 $\pm$ .001 & 2.619 $\pm$ 2.297 \\
 & \algostoNN & \textemdash & .363 & \textemdash & .010 & \textemdash & .374 & \textemdash & .014 & \textemdash & .349 & \textemdash & .040 & \textemdash & .368 & \textemdash & .113 \\
\midrule
 &  & \multicolumn{4}{c}{$\sigma^2=10^{-6}$} & \multicolumn{4}{c}{$\sigma^2=10^{-5}$} & \multicolumn{4}{c}{$\sigma^2=10^{-4}$} & \multicolumn{4}{c}{$\sigma^2=10^{-3}$} \\
\midrule
 & $\lr=10^{-4}$ & $\Risk_{\Tcal}(h)$ & Bnd & $\Risk_{\Scal}(h)$ & Div & $\Risk_{\Tcal}(h)$ & Bnd & $\Risk_{\Scal}(h)$ & Div & $\Risk_{\Tcal}(h)$ & Bnd & $\Risk_{\Scal}(h)$ & Div & $\Risk_{\Tcal}(h)$ & Bnd & $\Risk_{\Scal}(h)$ & Div \\
\midrule
\multirow[c]{5}{*}{\rotatebox[origin=c]{90}{\small{MNIST}}} & \algoours & .008 $\pm$ .000 & .016 $\pm$ .000 & .010 $\pm$ .000 & 9.520 & .007 $\pm$ .000 & .014 $\pm$ .000 & .009 $\pm$ .000 & 3.594 & .008 $\pm$ .000 & .014 $\pm$ .000 & .009 $\pm$ .000 & 1.877 & .008 $\pm$ .000 & .014 $\pm$ .001 & .009 $\pm$ .000 & 6.589 \\
 & \algoblanchard & .008 $\pm$ .000 & .657 $\pm$ .005 & .010 $\pm$ .000 & 1209.158 $\pm$ 157.539 & .007 $\pm$ .000 & .304 $\pm$ .005 & .009 $\pm$ .000 & 3795.285 $\pm$ 88.141 & .008 $\pm$ .000 & .124 $\pm$ .004 & .009 $\pm$ .000 & 1183.704 $\pm$ 5.113 & .007 $\pm$ .000 & .052 $\pm$ .002 & .009 $\pm$ .000 & 347.860 $\pm$ 25.275 \\
 & \algocatoni & .008 $\pm$ .000 & .452 $\pm$ .004 & .010 $\pm$ .000 & 12032.708 $\pm$ 157.184 & .007 $\pm$ .000 & .225 $\pm$ .003 & .009 $\pm$ .000 & 3834.246 $\pm$ 89.809 & .008 $\pm$ .000 & .093 $\pm$ .003 & .009 $\pm$ .000 & 1225.575 $\pm$ 51.027 & .007 $\pm$ .000 & .039 $\pm$ .002 & .008 $\pm$ .000 & 39.374 $\pm$ 26.987 \\
 & \algorivasplata & .008 $\pm$ .000 & .423 $\pm$ .004 & .010 $\pm$ .000 & 11943.688 $\pm$ 156.365 & .007 $\pm$ .000 & .179 $\pm$ .003 & .009 $\pm$ .000 & 3787.407 $\pm$ 87.968 & .008 $\pm$ .000 & .075 $\pm$ .002 & .009 $\pm$ .000 & 1173.457 $\pm$ 49.846 & .007 $\pm$ .000 & .035 $\pm$ .002 & .008 $\pm$ .000 & 348.717 $\pm$ 26.495 \\
 & \algostoNN & \textemdash & .039 & \textemdash & 4.760 & \textemdash & .038 & \textemdash & 1.797 & \textemdash & .037 & \textemdash & .938 & \textemdash & .038 & \textemdash & 3.294 \\
\midrule
\multirow[c]{5}{*}{\rotatebox[origin=c]{90}{\small{Fashion}}} & \algoours & .109 $\pm$ .000 & .119 $\pm$ .000 & .102 $\pm$ .000 & 16.776 & .114 $\pm$ .001 & .119 $\pm$ .001 & .104 $\pm$ .001 & 7.869 & .101 $\pm$ .001 & .111 $\pm$ .001 & .095 $\pm$ .001 & 14.224 & .109 $\pm$ .002 & .116 $\pm$ .002 & .101 $\pm$ .002 & 9.187 \\
 & \algoblanchard & .108 $\pm$ .000 & .743 $\pm$ .004 & .101 $\pm$ .000 & 11048.501 $\pm$ 146.969 & .112 $\pm$ .001 & .468 $\pm$ .005 & .101 $\pm$ .001 & 3798.865 $\pm$ 87.270 & .099 $\pm$ .001 & .268 $\pm$ .005 & .093 $\pm$ .001 & 1144.740 $\pm$ 49.199 & .106 $\pm$ .002 & .183 $\pm$ .004 & .099 $\pm$ .002 & 328.466 $\pm$ 24.435 \\
 & \algocatoni & .109 $\pm$ .000 & .712 $\pm$ .005 & .102 $\pm$ .000 & 1191.096 $\pm$ 15.212 & .114 $\pm$ .001 & .367 $\pm$ .005 & .104 $\pm$ .001 & 3831.104 $\pm$ 88.371 & .101 $\pm$ .001 & .216 $\pm$ .003 & .095 $\pm$ .001 & 1221.392 $\pm$ 5.970 & .108 $\pm$ .002 & .175 $\pm$ .003 & .101 $\pm$ .002 & 386.528 $\pm$ 26.498 \\
 & \algorivasplata & .108 $\pm$ .000 & .557 $\pm$ .003 & .101 $\pm$ .000 & 11148.085 $\pm$ 145.818 & .111 $\pm$ .001 & .340 $\pm$ .003 & .100 $\pm$ .001 & 3757.976 $\pm$ 83.965 & .098 $\pm$ .001 & .209 $\pm$ .003 & .092 $\pm$ .001 & 1176.081 $\pm$ 49.829 & .106 $\pm$ .002 & .156 $\pm$ .003 & .098 $\pm$ .002 & 34.716 $\pm$ 24.874 \\
 & \algostoNN & \textemdash & .168 & \textemdash & 8.388 & \textemdash & .168 & \textemdash & 3.935 & \textemdash & .159 & \textemdash & 7.112 & \textemdash & .165 & \textemdash & 4.594 \\
\midrule
\multirow[c]{5}{*}{\rotatebox[origin=c]{90}{\small{CIFAR-10}}} & \algoours & .277 $\pm$ .000 & .301 $\pm$ .000 & .276 $\pm$ .000 & 8.466 & .288 $\pm$ .000 & .308 $\pm$ .000 & .286 $\pm$ .000 & 2.415 & .273 $\pm$ .001 & .285 $\pm$ .000 & .263 $\pm$ .000 & 2.256 & .280 $\pm$ .001 & .303 $\pm$ .001 & .280 $\pm$ .001 & 2.747 \\
 & \algoblanchard & .277 $\pm$ .000 & .990 $\pm$ .000 & .276 $\pm$ .000 & 58878.209 $\pm$ 356.845 & .288 $\pm$ .000 & .868 $\pm$ .003 & .286 $\pm$ .000 & 8858.838 $\pm$ 134.545 & .272 $\pm$ .001 & .625 $\pm$ .005 & .262 $\pm$ .000 & 2709.659 $\pm$ 76.197 & .278 $\pm$ .001 & .480 $\pm$ .005 & .277 $\pm$ .001 & 86.940 $\pm$ 43.864 \\
 & \algocatoni & .277 $\pm$ .000 & .974 $\pm$ .000 & .276 $\pm$ .000 & 17581.286 $\pm$ 185.476 & .288 $\pm$ .000 & .662 $\pm$ .005 & .286 $\pm$ .000 & 5118.582 $\pm$ 105.636 & .272 $\pm$ .001 & .456 $\pm$ .003 & .262 $\pm$ .000 & 1548.107 $\pm$ 58.565 & .277 $\pm$ .001 & .426 $\pm$ .002 & .274 $\pm$ .001 & 783.103 $\pm$ 41.593 \\
 & \algorivasplata & .277 $\pm$ .000 & .990 $\pm$ .000 & .276 $\pm$ .000 & 82459.214 $\pm$ 398.763 & .288 $\pm$ .000 & .733 $\pm$ .003 & .286 $\pm$ .000 & 8674.850 $\pm$ 13.468 & .272 $\pm$ .001 & .518 $\pm$ .004 & .262 $\pm$ .000 & 2709.173 $\pm$ 77.205 & .277 $\pm$ .001 & .418 $\pm$ .004 & .275 $\pm$ .001 & 874.307 $\pm$ 44.089 \\
 & \algostoNN & \textemdash & .366 & \textemdash & 4.233 & \textemdash & .374 & \textemdash & 1.207 & \textemdash & .350 & \textemdash & 1.128 & \textemdash & .369 & \textemdash & 1.374 \\
\bottomrule
\end{tabular}
}
\label{chap:dis-pra:table:1_prior_0.6}
\end{table}
\end{landscape} 

\begin{landscape}
\begin{table}[t]
\caption{
\looseness=-1
Comparison of \algoours, \algorivasplata, \algoblanchard and \algocatoni based on the disintegrated bounds, and \algostoNN based on the randomized bounds learned with two learning rates $\lr{\ \in}\{10^{-4}, 10^{-6}\}$ and different variances $\sigma^2{\in}\{10^{-3}, 10^{-4}, 10^{-5}, 10^{-6}\}$.
We report the test risk ($\Risk_{\dT}(\h)$), the bound value (Bnd), the empirical risk ($\Risk_{\dS}(\h)$), and the divergence (Div) associated with each bound (the \textsc{Rényi} divergence for \algoours, the KL divergence for \algostoNN, and the disintegrated KL divergence for \algorivasplata, \algoblanchard and \algocatoni).
More precisely, we report the mean $\pm$ the standard deviation for $400$ neural networks sampled from $\AQ$ for \algoours, \algorivasplata, \algoblanchard, and \algocatoni.
We consider, in this table, that the split ratio is $0.7$.
}
\resizebox{0.63\paperheight}{!}{
\begin{tabular}{rr|clcl|clcl|clcl|clcl}
\toprule
 &  & \multicolumn{4}{c}{$\sigma^2=10^{-6}$} & \multicolumn{4}{c}{$\sigma^2=10^{-5}$} & \multicolumn{4}{c}{$\sigma^2=10^{-4}$} & \multicolumn{4}{c}{$\sigma^2=10^{-3}$} \\
\midrule
 & $\lr=10^{-6}$ & $\Risk_{\Tcal}(h)$ & Bnd & $\Risk_{\Scal}(h)$ & Div & $\Risk_{\Tcal}(h)$ & Bnd & $\Risk_{\Scal}(h)$ & Div & $\Risk_{\Tcal}(h)$ & Bnd & $\Risk_{\Scal}(h)$ & Div & $\Risk_{\Tcal}(h)$ & Bnd & $\Risk_{\Scal}(h)$ & Div \\
\midrule
\multirow[c]{5}{*}{\rotatebox[origin=c]{90}{\small{MNIST}}} & \algoours & .011 $\pm$ .000 & .019 $\pm$ .000 & .013 $\pm$ .000 & .047 & .010 $\pm$ .000 & .018 $\pm$ .000 & .012 $\pm$ .000 & .125 & .010 $\pm$ .000 & .017 $\pm$ .000 & .012 $\pm$ .000 & .116 & .010 $\pm$ .001 & .018 $\pm$ .001 & .012 $\pm$ .001 & .132 \\
 & \algoblanchard & .011 $\pm$ .000 & .032 $\pm$ .002 & .013 $\pm$ .000 & 65.017 $\pm$ 11.099 & .010 $\pm$ .000 & .019 $\pm$ .001 & .012 $\pm$ .000 & 1.819 $\pm$ 4.995 & .010 $\pm$ .000 & .016 $\pm$ .001 & .012 $\pm$ .000 & 1.551 $\pm$ 1.635 & .010 $\pm$ .001 & .016 $\pm$ .001 & .012 $\pm$ .001 & .115 $\pm$ .560 \\
 & \algocatoni & .011 $\pm$ .000 & .028 $\pm$ .001 & .013 $\pm$ .000 & 84.529 $\pm$ 13.023 & .010 $\pm$ .000 & .021 $\pm$ .000 & .012 $\pm$ .000 & 11.910 $\pm$ 5.053 & .010 $\pm$ .000 & .017 $\pm$ .001 & .012 $\pm$ .000 & 1.228 $\pm$ 1.637 & .010 $\pm$ .001 & .017 $\pm$ .001 & .012 $\pm$ .001 & .173 $\pm$ .512 \\
 & \algorivasplata & .011 $\pm$ .000 & .026 $\pm$ .001 & .013 $\pm$ .000 & 68.055 $\pm$ 11.606 & .010 $\pm$ .000 & .018 $\pm$ .001 & .012 $\pm$ .000 & 1.637 $\pm$ 4.962 & .010 $\pm$ .000 & .016 $\pm$ .000 & .012 $\pm$ .000 & 1.408 $\pm$ 1.639 & .010 $\pm$ .001 & .016 $\pm$ .001 & .012 $\pm$ .001 & .160 $\pm$ .529 \\
 & \algostoNN & \textemdash & .044 & \textemdash & .023 & \textemdash & .043 & \textemdash & .062 & \textemdash & .042 & \textemdash & .058 & \textemdash & .043 & \textemdash & .066 \\
\midrule
\multirow[c]{5}{*}{\rotatebox[origin=c]{90}{\small{Fashion}}} & \algoours & .099 $\pm$ .000 & .112 $\pm$ .000 & .098 $\pm$ .000 & .067 & .107 $\pm$ .001 & .115 $\pm$ .001 & .100 $\pm$ .001 & .542 & .098 $\pm$ .002 & .107 $\pm$ .001 & .093 $\pm$ .001 & .353 & .108 $\pm$ .003 & .117 $\pm$ .002 & .102 $\pm$ .002 & .312 \\
 & \algoblanchard & .099 $\pm$ .000 & .138 $\pm$ .004 & .098 $\pm$ .000 & 61.733 $\pm$ 1.862 & .107 $\pm$ .001 & .119 $\pm$ .003 & .101 $\pm$ .001 & 1.651 $\pm$ 4.230 & .099 $\pm$ .001 & .104 $\pm$ .002 & .094 $\pm$ .001 & 1.342 $\pm$ 1.664 & .108 $\pm$ .003 & .113 $\pm$ .003 & .103 $\pm$ .002 & .143 $\pm$ .534 \\
 & \algocatoni & .099 $\pm$ .000 & .155 $\pm$ .007 & .098 $\pm$ .000 & 83.929 $\pm$ 12.212 & .107 $\pm$ .001 & .116 $\pm$ .003 & .101 $\pm$ .001 & 11.543 $\pm$ 4.870 & .099 $\pm$ .002 & .103 $\pm$ .002 & .094 $\pm$ .001 & 1.437 $\pm$ 1.594 & .108 $\pm$ .003 & .112 $\pm$ .003 & .103 $\pm$ .002 & .153 $\pm$ .545 \\
 & \algorivasplata & .099 $\pm$ .000 & .128 $\pm$ .002 & .098 $\pm$ .000 & 65.737 $\pm$ 11.733 & .107 $\pm$ .001 & .116 $\pm$ .002 & .101 $\pm$ .001 & 1.958 $\pm$ 4.794 & .099 $\pm$ .002 & .105 $\pm$ .002 & .094 $\pm$ .001 & 1.491 $\pm$ 1.618 & .108 $\pm$ .003 & .114 $\pm$ .003 & .103 $\pm$ .002 & .155 $\pm$ .546 \\
 & \algostoNN & \textemdash & .161 & \textemdash & .034 & \textemdash & .164 & \textemdash & .271 & \textemdash & .155 & \textemdash & .177 & \textemdash & .166 & \textemdash & .156 \\
\midrule
\multirow[c]{5}{*}{\rotatebox[origin=c]{90}{\small{CIFAR-10}}} & \algoours & .277 $\pm$ .000 & .296 $\pm$ .000 & .272 $\pm$ .000 & .016 & .266 $\pm$ .000 & .281 $\pm$ .000 & .257 $\pm$ .000 & .022 & .253 $\pm$ .001 & .272 $\pm$ .000 & .248 $\pm$ .000 & .069 & .236 $\pm$ .001 & .258 $\pm$ .001 & .235 $\pm$ .001 & .118 \\
 & \algoblanchard & .277 $\pm$ .000 & .399 $\pm$ .006 & .272 $\pm$ .000 & 257.371 $\pm$ 23.327 & .266 $\pm$ .000 & .322 $\pm$ .005 & .257 $\pm$ .000 & 7.190 $\pm$ 11.685 & .253 $\pm$ .001 & .281 $\pm$ .005 & .248 $\pm$ .000 & 15.214 $\pm$ 5.838 & .236 $\pm$ .001 & .255 $\pm$ .002 & .235 $\pm$ .001 & 2.223 $\pm$ 2.016 \\
 & \algocatoni & .277 $\pm$ .000 & .399 $\pm$ .002 & .272 $\pm$ .000 & 269.048 $\pm$ 23.489 & .266 $\pm$ .000 & .328 $\pm$ .009 & .257 $\pm$ .000 & 81.217 $\pm$ 13.476 & .253 $\pm$ .001 & .275 $\pm$ .004 & .248 $\pm$ .000 & 16.576 $\pm$ 6.087 & .236 $\pm$ .001 & .253 $\pm$ .002 & .235 $\pm$ .001 & 2.248 $\pm$ 2.082 \\
 & \algorivasplata & .277 $\pm$ .000 & .362 $\pm$ .004 & .272 $\pm$ .000 & 258.993 $\pm$ 23.725 & .266 $\pm$ .000 & .305 $\pm$ .004 & .257 $\pm$ .000 & 72.737 $\pm$ 12.750 & .253 $\pm$ .001 & .275 $\pm$ .003 & .248 $\pm$ .000 & 15.342 $\pm$ 5.948 & .236 $\pm$ .001 & .255 $\pm$ .002 & .235 $\pm$ .001 & 2.220 $\pm$ 2.180 \\
 & \algostoNN & \textemdash & .362 & \textemdash & .008 & \textemdash & .345 & \textemdash & .011 & \textemdash & .336 & \textemdash & .034 & \textemdash & .322 & \textemdash & .059 \\
\midrule
 &  & \multicolumn{4}{c}{$\sigma^2=10^{-6}$} & \multicolumn{4}{c}{$\sigma^2=10^{-5}$} & \multicolumn{4}{c}{$\sigma^2=10^{-4}$} & \multicolumn{4}{c}{$\sigma^2=10^{-3}$} \\
\midrule
 & $\lr=10^{-4}$ & $\Risk_{\Tcal}(h)$ & Bnd & $\Risk_{\Scal}(h)$ & Div & $\Risk_{\Tcal}(h)$ & Bnd & $\Risk_{\Scal}(h)$ & Div & $\Risk_{\Tcal}(h)$ & Bnd & $\Risk_{\Scal}(h)$ & Div & $\Risk_{\Tcal}(h)$ & Bnd & $\Risk_{\Scal}(h)$ & Div \\
\midrule
\multirow[c]{5}{*}{\rotatebox[origin=c]{90}{\small{MNIST}}} & \algoours & .011 $\pm$ .000 & .025 $\pm$ .000 & .013 $\pm$ .000 & 45.094 & .010 $\pm$ .000 & .019 $\pm$ .000 & .012 $\pm$ .000 & 7.479 & .010 $\pm$ .000 & .018 $\pm$ .000 & .012 $\pm$ .000 & 5.269 & .010 $\pm$ .000 & .018 $\pm$ .001 & .011 $\pm$ .001 & 6.510 \\
 & \algoblanchard & .011 $\pm$ .000 & .737 $\pm$ .004 & .013 $\pm$ .000 & 11285.050 $\pm$ 147.363 & .010 $\pm$ .000 & .381 $\pm$ .006 & .012 $\pm$ .000 & 3785.071 $\pm$ 85.889 & .010 $\pm$ .000 & .160 $\pm$ .005 & .011 $\pm$ .000 & 1181.043 $\pm$ 46.219 & .009 $\pm$ .000 & .067 $\pm$ .003 & .011 $\pm$ .001 & 34.267 $\pm$ 26.244 \\
 & \algocatoni & .011 $\pm$ .000 & .547 $\pm$ .004 & .013 $\pm$ .000 & 11965.668 $\pm$ 153.481 & .010 $\pm$ .000 & .283 $\pm$ .004 & .012 $\pm$ .000 & 3811.642 $\pm$ 88.111 & .010 $\pm$ .000 & .120 $\pm$ .004 & .011 $\pm$ .000 & 1212.373 $\pm$ 48.835 & .009 $\pm$ .000 & .050 $\pm$ .002 & .010 $\pm$ .000 & 383.387 $\pm$ 27.059 \\
 & \algorivasplata & .011 $\pm$ .000 & .509 $\pm$ .004 & .013 $\pm$ .000 & 11555.623 $\pm$ 15.287 & .010 $\pm$ .000 & .226 $\pm$ .004 & .012 $\pm$ .000 & 3695.054 $\pm$ 9.289 & .009 $\pm$ .000 & .096 $\pm$ .003 & .011 $\pm$ .000 & 1171.892 $\pm$ 47.812 & .009 $\pm$ .000 & .044 $\pm$ .002 & .010 $\pm$ .000 & 343.025 $\pm$ 25.804 \\
 & \algostoNN & \textemdash & .050 & \textemdash & 22.547 & \textemdash & .044 & \textemdash & 3.740 & \textemdash & .043 & \textemdash & 2.634 & \textemdash & .043 & \textemdash & 3.255 \\
\midrule
\multirow[c]{5}{*}{\rotatebox[origin=c]{90}{\small{Fashion}}} & \algoours & .099 $\pm$ .000 & .116 $\pm$ .000 & .098 $\pm$ .000 & 11.922 & .107 $\pm$ .001 & .117 $\pm$ .001 & .101 $\pm$ .001 & 6.556 & .097 $\pm$ .001 & .109 $\pm$ .001 & .092 $\pm$ .001 & 9.235 & .105 $\pm$ .002 & .118 $\pm$ .002 & .100 $\pm$ .002 & 1.362 \\
 & \algoblanchard & .098 $\pm$ .000 & .795 $\pm$ .004 & .098 $\pm$ .000 & 10179.790 $\pm$ 138.889 & .101 $\pm$ .001 & .524 $\pm$ .006 & .096 $\pm$ .001 & 3752.748 $\pm$ 9.952 & .095 $\pm$ .001 & .291 $\pm$ .005 & .090 $\pm$ .001 & 1091.018 $\pm$ 47.577 & .104 $\pm$ .002 & .195 $\pm$ .005 & .098 $\pm$ .002 & 309.857 $\pm$ 24.422 \\
 & \algocatoni & .099 $\pm$ .000 & .808 $\pm$ .002 & .098 $\pm$ .000 & 11999.071 $\pm$ 158.418 & .107 $\pm$ .001 & .425 $\pm$ .006 & .100 $\pm$ .001 & 3817.800 $\pm$ 91.674 & .098 $\pm$ .001 & .235 $\pm$ .004 & .093 $\pm$ .001 & 1216.042 $\pm$ 5.641 & .106 $\pm$ .002 & .182 $\pm$ .004 & .101 $\pm$ .002 & 376.493 $\pm$ 27.018 \\
 & \algorivasplata & .098 $\pm$ .000 & .619 $\pm$ .004 & .097 $\pm$ .000 & 10768.160 $\pm$ 146.634 & .099 $\pm$ .001 & .369 $\pm$ .004 & .094 $\pm$ .001 & 3565.270 $\pm$ 88.164 & .094 $\pm$ .001 & .224 $\pm$ .004 & .089 $\pm$ .001 & 1137.876 $\pm$ 48.421 & .103 $\pm$ .002 & .164 $\pm$ .003 & .097 $\pm$ .002 & 318.512 $\pm$ 24.741 \\
 & \algostoNN & \textemdash & .164 & \textemdash & 5.961 & \textemdash & .166 & \textemdash & 3.278 & \textemdash & .156 & \textemdash & 4.618 & \textemdash & .166 & \textemdash & 5.181 \\
\midrule
\multirow[c]{5}{*}{\rotatebox[origin=c]{90}{\small{CIFAR-10}}} & \algoours & .277 $\pm$ .000 & .303 $\pm$ .000 & .272 $\pm$ .000 & 12.803 & .266 $\pm$ .000 & .282 $\pm$ .000 & .257 $\pm$ .000 & 2.312 & .253 $\pm$ .001 & .272 $\pm$ .000 & .248 $\pm$ .000 & 1.641 & .236 $\pm$ .001 & .259 $\pm$ .001 & .235 $\pm$ .001 & 1.929 \\
 & \algoblanchard & .277 $\pm$ .000 & .990 $\pm$ .000 & .272 $\pm$ .000 & 2577.092 $\pm$ 236.075 & .266 $\pm$ .000 & .901 $\pm$ .003 & .257 $\pm$ .000 & 8788.732 $\pm$ 134.680 & .253 $\pm$ .001 & .662 $\pm$ .005 & .247 $\pm$ .000 & 2683.054 $\pm$ 73.139 & .235 $\pm$ .001 & .464 $\pm$ .006 & .233 $\pm$ .001 & 85.586 $\pm$ 41.917 \\
 & \algocatoni & .277 $\pm$ .000 & 1.000 $\pm$ .000 & .272 $\pm$ .000 & 177807.417 $\pm$ 546.892 & .266 $\pm$ .000 & .601 $\pm$ .005 & .257 $\pm$ .000 & 331.757 $\pm$ 83.561 & .253 $\pm$ .001 & .416 $\pm$ .003 & .247 $\pm$ .000 & 85.973 $\pm$ 4.961 & .234 $\pm$ .001 & .369 $\pm$ .003 & .233 $\pm$ .001 & 485.863 $\pm$ 31.335 \\
 & \algorivasplata & .277 $\pm$ .000 & .990 $\pm$ .000 & .272 $\pm$ .000 & 48522.489 $\pm$ 309.735 & .266 $\pm$ .000 & .762 $\pm$ .003 & .257 $\pm$ .000 & 850.968 $\pm$ 131.507 & .252 $\pm$ .001 & .542 $\pm$ .004 & .247 $\pm$ .000 & 2696.074 $\pm$ 73.062 & .234 $\pm$ .001 & .393 $\pm$ .004 & .232 $\pm$ .001 & 858.936 $\pm$ 41.972 \\
 & \algostoNN & \textemdash & .366 & \textemdash & 6.401 & \textemdash & .346 & \textemdash & 1.156 & \textemdash & .336 & \textemdash & .821 & \textemdash & .322 & \textemdash & .965 \\
\bottomrule
\end{tabular}
}
\label{chap:dis-pra:table:1_prior_0.7}
\end{table}
\end{landscape} 

\begin{landscape}
\begin{table}[t]
\caption{
\looseness=-1
Comparison of \algoours, \algorivasplata, \algoblanchard and \algocatoni based on the disintegrated bounds, and \algostoNN based on the randomized bounds learned with two learning rates $\lr{\ \in}\{10^{-4}, 10^{-6}\}$ and different variances $\sigma^2{\in}\{10^{-3}, 10^{-4}, 10^{-5}, 10^{-6}\}$.
We report the test risk ($\Risk_{\dT}(\h)$), the bound value (Bnd), the empirical risk ($\Risk_{\dS}(\h)$), and the divergence (Div) associated with each bound (the \textsc{Rényi} divergence for \algoours, the KL divergence for \algostoNN, and the disintegrated KL divergence for \algorivasplata, \algoblanchard and \algocatoni).
More precisely, we report the mean $\pm$ the standard deviation for $400$ neural networks sampled from $\AQ$ for \algoours, \algorivasplata, \algoblanchard, and \algocatoni.
We consider, in this table, that the split ratio is $0.8$.
}
\resizebox{0.63\paperheight}{!}{
\begin{tabular}{rr|clcl|clcl|clcl|clcl}
\toprule
 &  & \multicolumn{4}{c}{$\sigma^2=10^{-6}$} & \multicolumn{4}{c}{$\sigma^2=10^{-5}$} & \multicolumn{4}{c}{$\sigma^2=10^{-4}$} & \multicolumn{4}{c}{$\sigma^2=10^{-3}$} \\
\midrule
 & $\lr=10^{-6}$ & $\Risk_{\Tcal}(h)$ & Bnd & $\Risk_{\Scal}(h)$ & Div & $\Risk_{\Tcal}(h)$ & Bnd & $\Risk_{\Scal}(h)$ & Div & $\Risk_{\Tcal}(h)$ & Bnd & $\Risk_{\Scal}(h)$ & Div & $\Risk_{\Tcal}(h)$ & Bnd & $\Risk_{\Scal}(h)$ & Div \\
\midrule
\multirow[c]{5}{*}{\rotatebox[origin=c]{90}{\small{MNIST}}} & \algoours & .011 $\pm$ .000 & .020 $\pm$ .000 & .013 $\pm$ .000 & .064 & .008 $\pm$ .000 & .017 $\pm$ .000 & .010 $\pm$ .000 & .050 & .011 $\pm$ .000 & .018 $\pm$ .000 & .011 $\pm$ .000 & .112 & .010 $\pm$ .001 & .016 $\pm$ .001 & .009 $\pm$ .001 & .073 \\
 & \algoblanchard & .011 $\pm$ .000 & .034 $\pm$ .003 & .013 $\pm$ .000 & 49.248 $\pm$ 1.541 & .008 $\pm$ .000 & .018 $\pm$ .001 & .010 $\pm$ .000 & 8.031 $\pm$ 3.654 & .011 $\pm$ .000 & .016 $\pm$ .001 & .011 $\pm$ .000 & .810 $\pm$ 1.248 & .010 $\pm$ .001 & .014 $\pm$ .001 & .010 $\pm$ .001 & .102 $\pm$ .448 \\
 & \algocatoni & .011 $\pm$ .000 & .030 $\pm$ .002 & .013 $\pm$ .000 & 66.244 $\pm$ 11.961 & .008 $\pm$ .000 & .018 $\pm$ .001 & .010 $\pm$ .000 & 8.685 $\pm$ 3.987 & .011 $\pm$ .000 & .019 $\pm$ .001 & .011 $\pm$ .000 & 1.011 $\pm$ 1.283 & .010 $\pm$ .001 & .016 $\pm$ .001 & .010 $\pm$ .001 & .131 $\pm$ .422 \\
 & \algorivasplata & .011 $\pm$ .000 & .028 $\pm$ .002 & .013 $\pm$ .000 & 5.344 $\pm$ 1.600 & .008 $\pm$ .000 & .017 $\pm$ .001 & .010 $\pm$ .000 & 7.757 $\pm$ 4.187 & .011 $\pm$ .000 & .017 $\pm$ .001 & .011 $\pm$ .000 & .861 $\pm$ 1.361 & .010 $\pm$ .001 & .014 $\pm$ .001 & .010 $\pm$ .001 & .090 $\pm$ .460 \\
 & \algostoNN & \textemdash & .046 & \textemdash & .032 & \textemdash & .041 & \textemdash & .025 & \textemdash & .043 & \textemdash & .056 & \textemdash & .040 & \textemdash & .037 \\
\midrule
\multirow[c]{5}{*}{\rotatebox[origin=c]{90}{\small{Fashion}}} & \algoours & .103 $\pm$ .000 & .117 $\pm$ .000 & .099 $\pm$ .000 & .068 & .098 $\pm$ .001 & .114 $\pm$ .001 & .096 $\pm$ .001 & .178 & .104 $\pm$ .001 & .117 $\pm$ .002 & .099 $\pm$ .002 & .587 & .107 $\pm$ .004 & .119 $\pm$ .004 & .101 $\pm$ .003 & .328 \\
 & \algoblanchard & .103 $\pm$ .000 & .144 $\pm$ .004 & .099 $\pm$ .000 & 5.069 $\pm$ 9.537 & .098 $\pm$ .001 & .116 $\pm$ .003 & .096 $\pm$ .001 & 8.105 $\pm$ 3.874 & .104 $\pm$ .001 & .113 $\pm$ .002 & .100 $\pm$ .002 & .990 $\pm$ 1.435 & .109 $\pm$ .004 & .114 $\pm$ .004 & .102 $\pm$ .004 & .102 $\pm$ .444 \\
 & \algocatoni & .103 $\pm$ .000 & .168 $\pm$ .009 & .099 $\pm$ .000 & 66.761 $\pm$ 1.939 & .098 $\pm$ .001 & .115 $\pm$ .004 & .096 $\pm$ .001 & 8.698 $\pm$ 3.974 & .104 $\pm$ .001 & .112 $\pm$ .002 & .100 $\pm$ .002 & .934 $\pm$ 1.413 & .109 $\pm$ .004 & .113 $\pm$ .004 & .102 $\pm$ .004 & .100 $\pm$ .457 \\
 & \algorivasplata & .103 $\pm$ .000 & .132 $\pm$ .003 & .099 $\pm$ .000 & 52.096 $\pm$ 1.745 & .098 $\pm$ .001 & .113 $\pm$ .002 & .096 $\pm$ .001 & 7.820 $\pm$ 4.154 & .104 $\pm$ .001 & .114 $\pm$ .002 & .100 $\pm$ .002 & .939 $\pm$ 1.417 & .108 $\pm$ .004 & .115 $\pm$ .004 & .102 $\pm$ .004 & .100 $\pm$ .464 \\
 & \algostoNN & \textemdash & .165 & \textemdash & .034 & \textemdash & .162 & \textemdash & .089 & \textemdash & .166 & \textemdash & .294 & \textemdash & .168 & \textemdash & .164 \\
\midrule
\multirow[c]{5}{*}{\rotatebox[origin=c]{90}{\small{CIFAR-10}}} & \algoours & .249 $\pm$ .000 & .265 $\pm$ .000 & .237 $\pm$ .000 & .014 & .247 $\pm$ .000 & .271 $\pm$ .000 & .243 $\pm$ .000 & .018 & .259 $\pm$ .001 & .281 $\pm$ .001 & .252 $\pm$ .001 & .055 & .249 $\pm$ .001 & .274 $\pm$ .001 & .245 $\pm$ .001 & .072 \\
 & \algoblanchard & .249 $\pm$ .000 & .384 $\pm$ .007 & .237 $\pm$ .000 & 24.108 $\pm$ 22.114 & .247 $\pm$ .000 & .316 $\pm$ .006 & .243 $\pm$ .000 & 59.096 $\pm$ 1.459 & .259 $\pm$ .001 & .289 $\pm$ .006 & .252 $\pm$ .001 & 11.804 $\pm$ 5.001 & .249 $\pm$ .001 & .269 $\pm$ .003 & .245 $\pm$ .001 & 1.578 $\pm$ 1.705 \\
 & \algocatoni & .249 $\pm$ .000 & .368 $\pm$ .003 & .237 $\pm$ .000 & 27.284 $\pm$ 23.618 & .247 $\pm$ .000 & .337 $\pm$ .012 & .243 $\pm$ .000 & 73.833 $\pm$ 11.692 & .259 $\pm$ .001 & .285 $\pm$ .005 & .252 $\pm$ .001 & 12.808 $\pm$ 5.089 & .249 $\pm$ .001 & .266 $\pm$ .002 & .245 $\pm$ .001 & 1.635 $\pm$ 1.773 \\
 & \algorivasplata & .249 $\pm$ .000 & .341 $\pm$ .005 & .237 $\pm$ .000 & 244.258 $\pm$ 22.339 & .247 $\pm$ .000 & .298 $\pm$ .004 & .243 $\pm$ .000 & 61.907 $\pm$ 11.135 & .259 $\pm$ .001 & .283 $\pm$ .003 & .252 $\pm$ .001 & 11.818 $\pm$ 4.923 & .249 $\pm$ .001 & .269 $\pm$ .002 & .245 $\pm$ .001 & 1.629 $\pm$ 1.763 \\
 & \algostoNN & \textemdash & .328 & \textemdash & .007 & \textemdash & .334 & \textemdash & .009 & \textemdash & .344 & \textemdash & .028 & \textemdash & .337 & \textemdash & .036 \\
\midrule
 &  & \multicolumn{4}{c}{$\sigma^2=10^{-6}$} & \multicolumn{4}{c}{$\sigma^2=10^{-5}$} & \multicolumn{4}{c}{$\sigma^2=10^{-4}$} & \multicolumn{4}{c}{$\sigma^2=10^{-3}$} \\
\midrule
 & $\lr=10^{-4}$ & $\Risk_{\Tcal}(h)$ & Bnd & $\Risk_{\Scal}(h)$ & Div & $\Risk_{\Tcal}(h)$ & Bnd & $\Risk_{\Scal}(h)$ & Div & $\Risk_{\Tcal}(h)$ & Bnd & $\Risk_{\Scal}(h)$ & Div & $\Risk_{\Tcal}(h)$ & Bnd & $\Risk_{\Scal}(h)$ & Div \\
\midrule
\multirow[c]{5}{*}{\rotatebox[origin=c]{90}{\small{MNIST}}} & \algoours & .011 $\pm$ .000 & .030 $\pm$ .000 & .013 $\pm$ .000 & 53.875 & .008 $\pm$ .000 & .018 $\pm$ .000 & .010 $\pm$ .000 & 4.369 & .011 $\pm$ .000 & .019 $\pm$ .000 & .011 $\pm$ .000 & 5.063 & .009 $\pm$ .001 & .016 $\pm$ .001 & .009 $\pm$ .001 & 4.854 \\
 & \algoblanchard & .011 $\pm$ .000 & .828 $\pm$ .004 & .013 $\pm$ .000 & 10014.066 $\pm$ 14.769 & .008 $\pm$ .000 & .491 $\pm$ .007 & .010 $\pm$ .000 & 3707.758 $\pm$ 86.461 & .011 $\pm$ .000 & .211 $\pm$ .007 & .011 $\pm$ .000 & 1151.660 $\pm$ 47.238 & .009 $\pm$ .001 & .076 $\pm$ .005 & .009 $\pm$ .001 & 303.402 $\pm$ 25.072 \\
 & \algocatoni & .011 $\pm$ .000 & .684 $\pm$ .004 & .013 $\pm$ .000 & 12238.359 $\pm$ 158.595 & .008 $\pm$ .000 & .343 $\pm$ .005 & .010 $\pm$ .000 & 3834.114 $\pm$ 88.516 & .011 $\pm$ .000 & .168 $\pm$ .006 & .011 $\pm$ .000 & 121.777 $\pm$ 48.460 & .009 $\pm$ .001 & .060 $\pm$ .003 & .008 $\pm$ .001 & 356.740 $\pm$ 25.649 \\
 & \algorivasplata & .011 $\pm$ .000 & .662 $\pm$ .005 & .013 $\pm$ .000 & 1207.265 $\pm$ 161.842 & .008 $\pm$ .000 & .305 $\pm$ .005 & .010 $\pm$ .000 & 3785.930 $\pm$ 87.976 & .011 $\pm$ .000 & .125 $\pm$ .004 & .010 $\pm$ .000 & 1141.437 $\pm$ 46.910 & .009 $\pm$ .001 & .048 $\pm$ .002 & .008 $\pm$ .000 & 305.573 $\pm$ 23.629 \\
 & \algostoNN & \textemdash & .055 & \textemdash & 26.937 & \textemdash & .042 & \textemdash & 2.185 & \textemdash & .044 & \textemdash & 2.532 & \textemdash & .040 & \textemdash & 2.427 \\
\midrule
\multirow[c]{5}{*}{\rotatebox[origin=c]{90}{\small{Fashion}}} & \algoours & .102 $\pm$ .000 & .121 $\pm$ .000 & .099 $\pm$ .000 & 1.120 & .098 $\pm$ .001 & .115 $\pm$ .001 & .096 $\pm$ .001 & 3.956 & .102 $\pm$ .001 & .118 $\pm$ .002 & .098 $\pm$ .001 & 7.830 & .103 $\pm$ .003 & .118 $\pm$ .003 & .097 $\pm$ .002 & 8.797 \\
 & \algoblanchard & .101 $\pm$ .000 & .990 $\pm$ .000 & .098 $\pm$ .000 & 27936.970 $\pm$ 235.840 & .096 $\pm$ .001 & .585 $\pm$ .007 & .094 $\pm$ .001 & 321.105 $\pm$ 81.006 & .098 $\pm$ .001 & .348 $\pm$ .007 & .094 $\pm$ .001 & 1045.641 $\pm$ 44.087 & .101 $\pm$ .002 & .208 $\pm$ .006 & .095 $\pm$ .002 & 273.641 $\pm$ 24.046 \\
 & \algocatoni & .103 $\pm$ .000 & .865 $\pm$ .002 & .099 $\pm$ .000 & 12143.837 $\pm$ 161.857 & .098 $\pm$ .001 & .536 $\pm$ .008 & .096 $\pm$ .001 & 3802.871 $\pm$ 87.750 & .103 $\pm$ .001 & .286 $\pm$ .006 & .098 $\pm$ .001 & 1202.907 $\pm$ 47.928 & .105 $\pm$ .003 & .191 $\pm$ .005 & .098 $\pm$ .003 & 354.246 $\pm$ 25.507 \\
 & \algorivasplata & .102 $\pm$ .000 & .746 $\pm$ .004 & .098 $\pm$ .000 & 11305.448 $\pm$ 149.693 & .096 $\pm$ .001 & .438 $\pm$ .005 & .093 $\pm$ .001 & 3458.977 $\pm$ 83.715 & .097 $\pm$ .001 & .264 $\pm$ .004 & .093 $\pm$ .001 & 1101.567 $\pm$ 44.816 & .099 $\pm$ .002 & .172 $\pm$ .004 & .094 $\pm$ .002 & 285.588 $\pm$ 24.451 \\
 & \algostoNN & \textemdash & .168 & \textemdash & 5.060 & \textemdash & .163 & \textemdash & 1.978 & \textemdash & .166 & \textemdash & 3.915 & \textemdash & .166 & \textemdash & 4.399 \\
\midrule
\multirow[c]{5}{*}{\rotatebox[origin=c]{90}{\small{CIFAR-10}}} & \algoours & .249 $\pm$ .000 & .274 $\pm$ .000 & .237 $\pm$ .000 & 14.083 & .247 $\pm$ .000 & .273 $\pm$ .000 & .243 $\pm$ .000 & 1.770 & .259 $\pm$ .001 & .282 $\pm$ .001 & .252 $\pm$ .001 & 1.098 & .248 $\pm$ .001 & .275 $\pm$ .001 & .245 $\pm$ .001 & 1.461 \\
 & \algoblanchard & .249 $\pm$ .000 & .990 $\pm$ .000 & .237 $\pm$ .000 & 26575.507 $\pm$ 218.278 & .247 $\pm$ .000 & .925 $\pm$ .002 & .243 $\pm$ .000 & 7135.143 $\pm$ 117.030 & .259 $\pm$ .001 & .739 $\pm$ .006 & .251 $\pm$ .001 & 2581.211 $\pm$ 74.799 & .247 $\pm$ .001 & .526 $\pm$ .007 & .243 $\pm$ .001 & 831.790 $\pm$ 4.592 \\
 & \algocatoni & .249 $\pm$ .000 & 1.000 $\pm$ .000 & .237 $\pm$ .000 & 154168.585 $\pm$ 539.590 & .247 $\pm$ .000 & .677 $\pm$ .008 & .243 $\pm$ .000 & 3148.174 $\pm$ 83.069 & .259 $\pm$ .001 & .549 $\pm$ .006 & .252 $\pm$ .001 & 1735.530 $\pm$ 57.888 & .248 $\pm$ .001 & .425 $\pm$ .005 & .244 $\pm$ .001 & 675.780 $\pm$ 38.306 \\
 & \algorivasplata & .249 $\pm$ .000 & .990 $\pm$ .000 & .237 $\pm$ .000 & 35062.089 $\pm$ 246.257 & .247 $\pm$ .000 & .824 $\pm$ .003 & .243 $\pm$ .000 & 8092.236 $\pm$ 125.162 & .259 $\pm$ .001 & .610 $\pm$ .005 & .251 $\pm$ .001 & 2652.857 $\pm$ 75.369 & .247 $\pm$ .001 & .441 $\pm$ .005 & .242 $\pm$ .001 & 84.056 $\pm$ 4.952 \\
 & \algostoNN & \textemdash & .334 & \textemdash & 7.041 & \textemdash & .335 & \textemdash & .885 & \textemdash & .345 & \textemdash & .549 & \textemdash & .337 & \textemdash & .731 \\
\bottomrule
\end{tabular}
}
\label{chap:dis-pra:table:1_prior_0.8}
\end{table}
\end{landscape} 

\begin{landscape}
\begin{table}[t]
\caption{
\looseness=-1
Comparison of \algoours, \algorivasplata, \algoblanchard and \algocatoni based on the disintegrated bounds, and \algostoNN based on the randomized bounds learned with two learning rates $\lr{\ \in}\{10^{-4}, 10^{-6}\}$ and different variances $\sigma^2{\in}\{10^{-3}, 10^{-4}, 10^{-5}, 10^{-6}\}$.
We report the test risk ($\Risk_{\dT}(\h)$), the bound value (Bnd), the empirical risk ($\Risk_{\dS}(\h)$), and the divergence (Div) associated with each bound (the \textsc{Rényi} divergence for \algoours, the KL divergence for \algostoNN, and the disintegrated KL divergence for \algorivasplata, \algoblanchard and \algocatoni).
More precisely, we report the mean $\pm$ the standard deviation for $400$ neural networks sampled from $\AQ$ for \algoours, \algorivasplata, \algoblanchard, and \algocatoni.
We consider, in this table, that the split ratio is $0.9$.
}
\resizebox{0.63\paperheight}{!}{
\begin{tabular}{rr|clcl|clcl|clcl|clcl}
\toprule
 &  & \multicolumn{4}{c}{$\sigma^2=10^{-6}$} & \multicolumn{4}{c}{$\sigma^2=10^{-5}$} & \multicolumn{4}{c}{$\sigma^2=10^{-4}$} & \multicolumn{4}{c}{$\sigma^2=10^{-3}$} \\
\midrule
 & $\lr=10^{-6}$ & $\Risk_{\Tcal}(h)$ & Bnd & $\Risk_{\Scal}(h)$ & Div & $\Risk_{\Tcal}(h)$ & Bnd & $\Risk_{\Scal}(h)$ & Div & $\Risk_{\Tcal}(h)$ & Bnd & $\Risk_{\Scal}(h)$ & Div & $\Risk_{\Tcal}(h)$ & Bnd & $\Risk_{\Scal}(h)$ & Div \\
\midrule
\multirow[c]{5}{*}{\rotatebox[origin=c]{90}{\small{MNIST}}} & \algoours & .008 $\pm$ .000 & .018 $\pm$ .000 & .008 $\pm$ .000 & .029 & .011 $\pm$ .000 & .020 $\pm$ .000 & .010 $\pm$ .000 & .052 & .009 $\pm$ .000 & .018 $\pm$ .001 & .009 $\pm$ .000 & .059 & .008 $\pm$ .000 & .019 $\pm$ .001 & .009 $\pm$ .001 & .023 \\
 & \algoblanchard & .008 $\pm$ .000 & .033 $\pm$ .004 & .009 $\pm$ .000 & 35.446 $\pm$ 8.610 & .011 $\pm$ .000 & .020 $\pm$ .002 & .010 $\pm$ .000 & 4.933 $\pm$ 2.958 & .009 $\pm$ .000 & .015 $\pm$ .001 & .009 $\pm$ .000 & .490 $\pm$ .960 & .008 $\pm$ .001 & .016 $\pm$ .001 & .009 $\pm$ .001 & .059 $\pm$ .299 \\
 & \algocatoni & .008 $\pm$ .000 & .026 $\pm$ .002 & .009 $\pm$ .000 & 41.267 $\pm$ 9.234 & .011 $\pm$ .000 & .019 $\pm$ .001 & .010 $\pm$ .000 & 4.564 $\pm$ 3.263 & .009 $\pm$ .000 & .016 $\pm$ .001 & .009 $\pm$ .000 & .581 $\pm$ .989 & .008 $\pm$ .001 & .017 $\pm$ .001 & .009 $\pm$ .001 & .078 $\pm$ .320 \\
 & \algorivasplata & .008 $\pm$ .000 & .025 $\pm$ .002 & .009 $\pm$ .000 & 35.856 $\pm$ 8.648 & .011 $\pm$ .000 & .019 $\pm$ .001 & .010 $\pm$ .000 & 4.620 $\pm$ 2.983 & .009 $\pm$ .000 & .015 $\pm$ .001 & .009 $\pm$ .000 & .448 $\pm$ 1.045 & .008 $\pm$ .000 & .016 $\pm$ .001 & .009 $\pm$ .001 & .041 $\pm$ .330 \\
 & \algostoNN & \textemdash & .041 & \textemdash & .014 & \textemdash & .045 & \textemdash & .026 & \textemdash & .042 & \textemdash & .030 & \textemdash & .043 & \textemdash & .012 \\
\midrule
\multirow[c]{5}{*}{\rotatebox[origin=c]{90}{\small{Fashion}}} & \algoours & .094 $\pm$ .000 & .113 $\pm$ .000 & .089 $\pm$ .000 & .029 & .091 $\pm$ .001 & .119 $\pm$ .001 & .095 $\pm$ .001 & .107 & .092 $\pm$ .002 & .113 $\pm$ .001 & .089 $\pm$ .001 & .097 & .103 $\pm$ .003 & .124 $\pm$ .003 & .099 $\pm$ .003 & .045 \\
 & \algoblanchard & .094 $\pm$ .000 & .140 $\pm$ .006 & .089 $\pm$ .000 & 32.563 $\pm$ 8.007 & .091 $\pm$ .001 & .119 $\pm$ .004 & .095 $\pm$ .001 & 4.567 $\pm$ 2.912 & .092 $\pm$ .002 & .106 $\pm$ .002 & .089 $\pm$ .001 & .468 $\pm$ 1.101 & .104 $\pm$ .003 & .116 $\pm$ .003 & .099 $\pm$ .003 & .063 $\pm$ .300 \\
 & \algocatoni & .094 $\pm$ .000 & .146 $\pm$ .002 & .089 $\pm$ .000 & 4.355 $\pm$ 9.121 & .091 $\pm$ .001 & .120 $\pm$ .005 & .095 $\pm$ .001 & 4.895 $\pm$ 3.064 & .092 $\pm$ .002 & .106 $\pm$ .002 & .089 $\pm$ .001 & .473 $\pm$ 1.052 & .103 $\pm$ .003 & .117 $\pm$ .003 & .099 $\pm$ .003 & .079 $\pm$ .319 \\
 & \algorivasplata & .094 $\pm$ .000 & .127 $\pm$ .004 & .089 $\pm$ .000 & 33.175 $\pm$ 8.710 & .091 $\pm$ .001 & .117 $\pm$ .002 & .095 $\pm$ .001 & 4.774 $\pm$ 3.003 & .092 $\pm$ .002 & .107 $\pm$ .002 & .089 $\pm$ .001 & .479 $\pm$ .924 & .103 $\pm$ .003 & .118 $\pm$ .003 & .099 $\pm$ .002 & .045 $\pm$ .330 \\
 & \algostoNN & \textemdash & .159 & \textemdash & .015 & \textemdash & .166 & \textemdash & .053 & \textemdash & .159 & \textemdash & .048 & \textemdash & .172 & \textemdash & .023 \\
\midrule
\multirow[c]{5}{*}{\rotatebox[origin=c]{90}{\small{CIFAR-10}}} & \algoours & .231 $\pm$ .000 & .268 $\pm$ .000 & .228 $\pm$ .000 & .011 & .235 $\pm$ .000 & .267 $\pm$ .000 & .227 $\pm$ .000 & .009 & .218 $\pm$ .001 & .253 $\pm$ .001 & .214 $\pm$ .001 & .024 & .231 $\pm$ .001 & .264 $\pm$ .002 & .224 $\pm$ .002 & .036 \\
 & \algoblanchard & .231 $\pm$ .000 & .418 $\pm$ .010 & .228 $\pm$ .000 & 193.922 $\pm$ 19.216 & .235 $\pm$ .000 & .312 $\pm$ .009 & .227 $\pm$ .000 & 39.705 $\pm$ 8.929 & .218 $\pm$ .000 & .256 $\pm$ .007 & .214 $\pm$ .001 & 6.919 $\pm$ 3.722 & .231 $\pm$ .001 & .255 $\pm$ .003 & .224 $\pm$ .002 & .878 $\pm$ 1.248 \\
 & \algocatoni & .231 $\pm$ .000 & .388 $\pm$ .005 & .228 $\pm$ .000 & 255.538 $\pm$ 22.306 & .235 $\pm$ .000 & .337 $\pm$ .003 & .227 $\pm$ .000 & 53.736 $\pm$ 1.302 & .218 $\pm$ .000 & .257 $\pm$ .007 & .214 $\pm$ .001 & 7.060 $\pm$ 3.626 & .231 $\pm$ .001 & .255 $\pm$ .003 & .224 $\pm$ .002 & .857 $\pm$ 1.264 \\
 & \algorivasplata & .231 $\pm$ .000 & .364 $\pm$ .007 & .228 $\pm$ .000 & 202.026 $\pm$ 19.688 & .235 $\pm$ .000 & .293 $\pm$ .006 & .227 $\pm$ .000 & 42.458 $\pm$ 9.250 & .218 $\pm$ .001 & .251 $\pm$ .004 & .214 $\pm$ .001 & 6.780 $\pm$ 3.575 & .231 $\pm$ .001 & .256 $\pm$ .002 & .224 $\pm$ .002 & .854 $\pm$ 1.275 \\
 & \algostoNN & \textemdash & .328 & \textemdash & .005 & \textemdash & .327 & \textemdash & .005 & \textemdash & .312 & \textemdash & .012 & \textemdash & .324 & \textemdash & .018 \\
\midrule
 &  & \multicolumn{4}{c}{$\sigma^2=10^{-6}$} & \multicolumn{4}{c}{$\sigma^2=10^{-5}$} & \multicolumn{4}{c}{$\sigma^2=10^{-4}$} & \multicolumn{4}{c}{$\sigma^2=10^{-3}$} \\
\midrule
 & $\lr=10^{-4}$ & $\Risk_{\Tcal}(h)$ & Bnd & $\Risk_{\Scal}(h)$ & Div & $\Risk_{\Tcal}(h)$ & Bnd & $\Risk_{\Scal}(h)$ & Div & $\Risk_{\Tcal}(h)$ & Bnd & $\Risk_{\Scal}(h)$ & Div & $\Risk_{\Tcal}(h)$ & Bnd & $\Risk_{\Scal}(h)$ & Div \\
\midrule
\multirow[c]{5}{*}{\rotatebox[origin=c]{90}{\small{MNIST}}} & \algoours & .008 $\pm$ .000 & .018 $\pm$ .000 & .009 $\pm$ .000 & 2.107 & .011 $\pm$ .000 & .021 $\pm$ .000 & .010 $\pm$ .000 & 1.329 & .008 $\pm$ .000 & .019 $\pm$ .001 & .008 $\pm$ .000 & 3.598 & .008 $\pm$ .001 & .020 $\pm$ .001 & .009 $\pm$ .001 & 4.216 \\
 & \algoblanchard & .008 $\pm$ .000 & .982 $\pm$ .001 & .008 $\pm$ .000 & 11722.999 $\pm$ 157.452 & .011 $\pm$ .000 & .706 $\pm$ .008 & .010 $\pm$ .000 & 3475.807 $\pm$ 77.708 & .008 $\pm$ .000 & .331 $\pm$ .011 & .008 $\pm$ .000 & 1076.767 $\pm$ 46.059 & .008 $\pm$ .000 & .108 $\pm$ .008 & .009 $\pm$ .001 & 242.819 $\pm$ 23.775 \\
 & \algocatoni & .008 $\pm$ .000 & 1.000 $\pm$ .000 & .008 $\pm$ .000 & 60838.120 $\pm$ 346.289 & .011 $\pm$ .000 & .515 $\pm$ .007 & .010 $\pm$ .000 & 3728.586 $\pm$ 86.803 & .008 $\pm$ .000 & .243 $\pm$ .007 & .008 $\pm$ .000 & 1166.491 $\pm$ 48.086 & .008 $\pm$ .000 & .087 $\pm$ .006 & .009 $\pm$ .001 & 277.823 $\pm$ 25.431 \\
 & \algorivasplata & .008 $\pm$ .000 & .879 $\pm$ .003 & .008 $\pm$ .000 & 12257.175 $\pm$ 152.738 & .010 $\pm$ .000 & .481 $\pm$ .007 & .010 $\pm$ .000 & 3602.529 $\pm$ 78.717 & .008 $\pm$ .000 & .201 $\pm$ .007 & .008 $\pm$ .000 & 1126.882 $\pm$ 47.430 & .008 $\pm$ .001 & .067 $\pm$ .004 & .009 $\pm$ .001 & 242.366 $\pm$ 22.337 \\
 & \algostoNN & \textemdash & .042 & \textemdash & 1.053 & \textemdash & .045 & \textemdash & .664 & \textemdash & .042 & \textemdash & 1.799 & \textemdash & .043 & \textemdash & 2.108 \\
\midrule
\multirow[c]{5}{*}{\rotatebox[origin=c]{90}{\small{Fashion}}} & \algoours & .094 $\pm$ .000 & .115 $\pm$ .000 & .089 $\pm$ .000 & 2.501 & .091 $\pm$ .001 & .121 $\pm$ .001 & .095 $\pm$ .001 & 2.925 & .092 $\pm$ .002 & .114 $\pm$ .001 & .088 $\pm$ .001 & 3.069 & .102 $\pm$ .002 & .125 $\pm$ .003 & .098 $\pm$ .002 & 3.159 \\
 & \algoblanchard & .094 $\pm$ .000 & .990 $\pm$ .000 & .089 $\pm$ .000 & 19455.864 $\pm$ 19.460 & .089 $\pm$ .001 & .792 $\pm$ .007 & .093 $\pm$ .001 & 3402.546 $\pm$ 86.590 & .090 $\pm$ .001 & .461 $\pm$ .010 & .087 $\pm$ .001 & 1002.861 $\pm$ 44.393 & .102 $\pm$ .002 & .244 $\pm$ .009 & .098 $\pm$ .002 & 206.177 $\pm$ 2.051 \\
 & \algocatoni & .094 $\pm$ .000 & 1.000 $\pm$ .000 & .089 $\pm$ .000 & 60888.029 $\pm$ 346.501 & .091 $\pm$ .001 & .813 $\pm$ .012 & .095 $\pm$ .001 & 3756.375 $\pm$ 9.419 & .092 $\pm$ .002 & .390 $\pm$ .010 & .089 $\pm$ .001 & 1161.884 $\pm$ 52.073 & .103 $\pm$ .003 & .215 $\pm$ .007 & .099 $\pm$ .002 & 277.284 $\pm$ 25.479 \\
 & \algorivasplata & .094 $\pm$ .000 & .990 $\pm$ .000 & .089 $\pm$ .000 & 27137.315 $\pm$ 227.934 & .088 $\pm$ .001 & .597 $\pm$ .007 & .093 $\pm$ .001 & 3371.321 $\pm$ 86.352 & .090 $\pm$ .001 & .331 $\pm$ .007 & .086 $\pm$ .001 & 1003.481 $\pm$ 48.362 & .101 $\pm$ .002 & .195 $\pm$ .006 & .097 $\pm$ .002 & 207.442 $\pm$ 21.896 \\
 & \algostoNN & \textemdash & .160 & \textemdash & 1.250 & \textemdash & .167 & \textemdash & 1.463 & \textemdash & .160 & \textemdash & 1.535 & \textemdash & .172 & \textemdash & 1.579 \\
\midrule
\multirow[c]{5}{*}{\rotatebox[origin=c]{90}{\small{CIFAR-10}}} & \algoours & .231 $\pm$ .000 & .279 $\pm$ .000 & .228 $\pm$ .000 & 12.925 & .235 $\pm$ .000 & .268 $\pm$ .000 & .227 $\pm$ .000 & 1.371 & .218 $\pm$ .001 & .254 $\pm$ .001 & .214 $\pm$ .001 & .715 & .231 $\pm$ .001 & .264 $\pm$ .002 & .224 $\pm$ .002 & 1.019 \\
 & \algoblanchard & .231 $\pm$ .000 & .990 $\pm$ .000 & .228 $\pm$ .000 & 26032.808 $\pm$ 222.475 & .235 $\pm$ .000 & .986 $\pm$ .001 & .227 $\pm$ .000 & 6875.633 $\pm$ 112.137 & .217 $\pm$ .000 & .831 $\pm$ .006 & .214 $\pm$ .001 & 2292.053 $\pm$ 68.347 & .230 $\pm$ .001 & .606 $\pm$ .010 & .222 $\pm$ .001 & 76.644 $\pm$ 39.246 \\
 & \algocatoni & .231 $\pm$ .000 & 1.000 $\pm$ .000 & .228 $\pm$ .000 & 17684.651 $\pm$ 576.711 & .235 $\pm$ .000 & .980 $\pm$ .000 & .227 $\pm$ .000 & 8265.727 $\pm$ 123.941 & .218 $\pm$ .000 & .834 $\pm$ .011 & .214 $\pm$ .001 & 2664.069 $\pm$ 73.915 & .231 $\pm$ .001 & .517 $\pm$ .009 & .224 $\pm$ .002 & 85.593 $\pm$ 41.022 \\
 & \algorivasplata & .231 $\pm$ .000 & .988 $\pm$ .001 & .228 $\pm$ .000 & 14284.846 $\pm$ 169.166 & .235 $\pm$ .000 & .919 $\pm$ .002 & .227 $\pm$ .000 & 7121.350 $\pm$ 114.645 & .217 $\pm$ .000 & .699 $\pm$ .006 & .213 $\pm$ .001 & 2502.412 $\pm$ 68.728 & .229 $\pm$ .001 & .494 $\pm$ .007 & .221 $\pm$ .001 & 776.237 $\pm$ 39.540 \\
 & \algostoNN & \textemdash & .335 & \textemdash & 6.462 & \textemdash & .328 & \textemdash & .685 & \textemdash & .313 & \textemdash & .358 & \textemdash & .324 & \textemdash & .510 \\
\bottomrule
\end{tabular}
}
\label{chap:dis-pra:table:1_prior_0.9}
\end{table}
\end{landscape} 

\begin{landscape} \begin{table}[t]
\caption{
Comparison of the bound values before performing Step {\bf 2)} of our Training Method for \algoours, \algorivasplata, \algoblanchard and \algocatoni. 
More precisely, for each split and each variance $\sigma^2{\in}\{10^{-3}, 10^{-4}, 10^{-5}, 10^{-6}\}$, we report the mean $\pm$ the standard deviation (for $400$ neural networks sampled from $\P$) of the test risk ($\Risk_{\dT}(\h)$), the empirical risk ($\Risk_{\dS}(\h)$), and the value of the bounds of \Cref{chap:dis-pra:corollary:nn,chap:dis-pra:corollary:nn-rbc}.
We consider in this table that the dataset is MNIST.
}
\begin{subtable}[h]{0.50\textwidth}
    \centering
    \resizebox{0.35\paperheight}{!}{
    \begin{tabular}{cccccccc}
\toprule
 & Split & $\Risk_{\Tcal}(h)$ & $\Risk_{\Scal}(h)$ & \cref{chap:dis-pra:corollary:nn} & \cref{chap:dis-pra:eq:nn-rivasplata} & \cref{chap:dis-pra:eq:nn-blanchard} & \cref{chap:dis-pra:eq:nn-catoni} \\
\midrule
\multirow[c]{10}{*}{\rotatebox[origin=c]{90}{\small{$\sigma^2=10^{-6}$}}} & .0 & .901 $\pm$ .002 & .901 $\pm$ .002 & .908 $\pm$ .002 & .906 $\pm$ .002 & .905 $\pm$ .002 & .906 $\pm$ .002 \\
 & .1 & .035 $\pm$ .000 & .039 $\pm$ .000 & .045 $\pm$ .000 & .043 $\pm$ .000 & .043 $\pm$ .000 & .042 $\pm$ .000 \\
 & .2 & .016 $\pm$ .000 & .019 $\pm$ .000 & .023 $\pm$ .000 & .022 $\pm$ .000 & .022 $\pm$ .000 & .022 $\pm$ .000 \\
 & .3 & .012 $\pm$ .000 & .013 $\pm$ .000 & .017 $\pm$ .000 & .016 $\pm$ .000 & .015 $\pm$ .000 & .015 $\pm$ .000 \\
 & .4 & .010 $\pm$ .000 & .013 $\pm$ .000 & .017 $\pm$ .000 & .016 $\pm$ .000 & .016 $\pm$ .000 & .016 $\pm$ .000 \\
 & .5 & .008 $\pm$ .000 & .010 $\pm$ .000 & .015 $\pm$ .000 & .013 $\pm$ .000 & .013 $\pm$ .000 & .014 $\pm$ .000 \\
 & .6 & .008 $\pm$ .000 & .010 $\pm$ .000 & .014 $\pm$ .000 & .013 $\pm$ .000 & .013 $\pm$ .000 & .014 $\pm$ .000 \\
 & .7 & .011 $\pm$ .000 & .013 $\pm$ .000 & .019 $\pm$ .000 & .017 $\pm$ .000 & .017 $\pm$ .000 & .018 $\pm$ .000 \\
 & .8 & .011 $\pm$ .000 & .013 $\pm$ .000 & .020 $\pm$ .000 & .018 $\pm$ .000 & .018 $\pm$ .000 & .020 $\pm$ .000 \\
 & .9 & .008 $\pm$ .000 & .009 $\pm$ .000 & .018 $\pm$ .000 & .015 $\pm$ .000 & .014 $\pm$ .000 & .015 $\pm$ .000 \\
\midrule
\multirow[c]{10}{*}{\rotatebox[origin=c]{90}{\small{$\sigma^2=10^{-5}$}}} & .0 & .897 $\pm$ .013 & .897 $\pm$ .012 & .904 $\pm$ .012 & .902 $\pm$ .012 & .902 $\pm$ .012 & .903 $\pm$ .012 \\
 & .1 & .024 $\pm$ .000 & .030 $\pm$ .001 & .035 $\pm$ .001 & .034 $\pm$ .001 & .033 $\pm$ .001 & .033 $\pm$ .001 \\
 & .2 & .015 $\pm$ .000 & .019 $\pm$ .000 & .023 $\pm$ .000 & .022 $\pm$ .000 & .021 $\pm$ .000 & .021 $\pm$ .000 \\
 & .3 & .009 $\pm$ .000 & .011 $\pm$ .000 & .015 $\pm$ .000 & .014 $\pm$ .000 & .013 $\pm$ .000 & .013 $\pm$ .000 \\
 & .4 & .012 $\pm$ .000 & .014 $\pm$ .000 & .018 $\pm$ .000 & .017 $\pm$ .000 & .017 $\pm$ .000 & .017 $\pm$ .000 \\
 & .5 & .006 $\pm$ .000 & .009 $\pm$ .000 & .012 $\pm$ .000 & .011 $\pm$ .000 & .011 $\pm$ .000 & .012 $\pm$ .000 \\
 & .6 & .007 $\pm$ .000 & .009 $\pm$ .000 & .014 $\pm$ .000 & .013 $\pm$ .000 & .012 $\pm$ .000 & .013 $\pm$ .000 \\
 & .7 & .010 $\pm$ .000 & .012 $\pm$ .000 & .018 $\pm$ .000 & .016 $\pm$ .000 & .016 $\pm$ .000 & .017 $\pm$ .000 \\
 & .8 & .008 $\pm$ .000 & .010 $\pm$ .000 & .017 $\pm$ .000 & .015 $\pm$ .000 & .014 $\pm$ .000 & .017 $\pm$ .000 \\
 & .9 & .011 $\pm$ .000 & .010 $\pm$ .000 & .020 $\pm$ .000 & .017 $\pm$ .000 & .017 $\pm$ .000 & .018 $\pm$ .000 \\
\bottomrule
\end{tabular}
}
\end{subtable}
\begin{subtable}[h]{1.1\textwidth}
    \centering
    \resizebox{0.35\paperheight}{!}{
    \begin{tabular}{cccccccc}
\toprule
 & Split & $\Risk_{\Tcal}(h)$ & $\Risk_{\Scal}(h)$ & \cref{chap:dis-pra:corollary:nn} & \cref{chap:dis-pra:eq:nn-rivasplata} & \cref{chap:dis-pra:eq:nn-blanchard} & \cref{chap:dis-pra:eq:nn-catoni} \\
\midrule
\multirow[c]{10}{*}{\rotatebox[origin=c]{90}{\small{$\sigma^2=10^{-4}$}}} & .0 & .898 $\pm$ .017 & .898 $\pm$ .017 & .905 $\pm$ .016 & .903 $\pm$ .016 & .902 $\pm$ .016 & .903 $\pm$ .016 \\
 & .1 & .035 $\pm$ .003 & .039 $\pm$ .002 & .045 $\pm$ .002 & .044 $\pm$ .002 & .043 $\pm$ .002 & .043 $\pm$ .002 \\
 & .2 & .015 $\pm$ .001 & .016 $\pm$ .001 & .020 $\pm$ .001 & .019 $\pm$ .001 & .019 $\pm$ .001 & .019 $\pm$ .001 \\
 & .3 & .012 $\pm$ .000 & .016 $\pm$ .000 & .020 $\pm$ .001 & .019 $\pm$ .001 & .019 $\pm$ .001 & .019 $\pm$ .001 \\
 & .4 & .009 $\pm$ .000 & .011 $\pm$ .000 & .015 $\pm$ .000 & .014 $\pm$ .000 & .014 $\pm$ .000 & .014 $\pm$ .000 \\
 & .5 & .008 $\pm$ .000 & .010 $\pm$ .000 & .015 $\pm$ .000 & .013 $\pm$ .000 & .013 $\pm$ .000 & .014 $\pm$ .000 \\
 & .6 & .008 $\pm$ .000 & .009 $\pm$ .000 & .013 $\pm$ .000 & .012 $\pm$ .000 & .012 $\pm$ .000 & .013 $\pm$ .000 \\
 & .7 & .010 $\pm$ .000 & .012 $\pm$ .000 & .017 $\pm$ .000 & .016 $\pm$ .000 & .015 $\pm$ .000 & .016 $\pm$ .000 \\
 & .8 & .011 $\pm$ .000 & .011 $\pm$ .000 & .018 $\pm$ .000 & .016 $\pm$ .000 & .016 $\pm$ .000 & .018 $\pm$ .000 \\
 & .9 & .009 $\pm$ .000 & .009 $\pm$ .000 & .018 $\pm$ .001 & .015 $\pm$ .001 & .015 $\pm$ .001 & .016 $\pm$ .001 \\
\midrule
\multirow[c]{10}{*}{\rotatebox[origin=c]{90}{\small{$\sigma^2=10^{-3}$}}} & .0 & .903 $\pm$ .014 & .902 $\pm$ .014 & .909 $\pm$ .013 & .907 $\pm$ .013 & .907 $\pm$ .013 & .907 $\pm$ .013 \\
 & .1 & .041 $\pm$ .005 & .045 $\pm$ .005 & .050 $\pm$ .005 & .049 $\pm$ .005 & .048 $\pm$ .005 & .048 $\pm$ .005 \\
 & .2 & .020 $\pm$ .002 & .022 $\pm$ .002 & .026 $\pm$ .002 & .025 $\pm$ .002 & .025 $\pm$ .002 & .024 $\pm$ .002 \\
 & .3 & .014 $\pm$ .001 & .015 $\pm$ .001 & .019 $\pm$ .001 & .018 $\pm$ .001 & .018 $\pm$ .001 & .018 $\pm$ .001 \\
 & .4 & .015 $\pm$ .001 & .016 $\pm$ .001 & .021 $\pm$ .001 & .020 $\pm$ .001 & .019 $\pm$ .001 & .019 $\pm$ .001 \\
 & .5 & .015 $\pm$ .001 & .015 $\pm$ .001 & .020 $\pm$ .001 & .019 $\pm$ .001 & .018 $\pm$ .001 & .018 $\pm$ .001 \\
 & .6 & .008 $\pm$ .000 & .010 $\pm$ .000 & .014 $\pm$ .001 & .013 $\pm$ .001 & .012 $\pm$ .000 & .013 $\pm$ .000 \\
 & .7 & .010 $\pm$ .001 & .012 $\pm$ .001 & .018 $\pm$ .001 & .016 $\pm$ .001 & .016 $\pm$ .001 & .017 $\pm$ .001 \\
 & .8 & .010 $\pm$ .001 & .010 $\pm$ .001 & .016 $\pm$ .001 & .014 $\pm$ .001 & .014 $\pm$ .001 & .016 $\pm$ .001 \\
 & .9 & .008 $\pm$ .000 & .009 $\pm$ .001 & .019 $\pm$ .001 & .016 $\pm$ .001 & .015 $\pm$ .001 & .017 $\pm$ .001 \\
\bottomrule
\end{tabular}
    }
\end{subtable}
\label{chap:dis-pra:table:2_data_mnist}
\end{table}\end{landscape} 


\begin{landscape}
\begin{table}[t]
\caption{
Comparison of the bound values before performing Step {\bf 2)} of our Training Method for \algoours, \algorivasplata, \algoblanchard and \algocatoni. 
More precisely, for each split and each variance $\sigma^2{\in}\{10^{-3}, 10^{-4}, 10^{-5}, 10^{-6}\}$, we report the mean $\pm$ the standard deviation (for $400$ neural networks sampled from $\P$) of the test risk ($\Risk_{\dT}(\h)$), the empirical risk ($\Risk_{\dS}(\h)$), and the value of the bounds of \Cref{chap:dis-pra:corollary:nn,chap:dis-pra:corollary:nn-rbc}.
We consider in this table that the dataset is Fashion-MNIST.
}
\begin{subtable}[h]{0.50\textwidth}
    \centering
    \resizebox{0.35\paperheight}{!}{
    \begin{tabular}{cccccccc}
\toprule
 & Split & $\Risk_{\Tcal}(h)$ & $\Risk_{\Scal}(h)$ & \cref{chap:dis-pra:corollary:nn} & \cref{chap:dis-pra:eq:nn-rivasplata} & \cref{chap:dis-pra:eq:nn-blanchard} & \cref{chap:dis-pra:eq:nn-catoni} \\
\midrule
\multirow[c]{10}{*}{\rotatebox[origin=c]{90}{\small{$\sigma^2=10^{-6}$}}} & .0 & .970 $\pm$ .028 & .970 $\pm$ .027 & .972 $\pm$ .025 & .971 $\pm$ .025 & .971 $\pm$ .026 & .972 $\pm$ .026 \\
 & .1 & .166 $\pm$ .001 & .159 $\pm$ .000 & .169 $\pm$ .000 & .167 $\pm$ .000 & .166 $\pm$ .000 & .167 $\pm$ .000 \\
 & .2 & .168 $\pm$ .002 & .160 $\pm$ .001 & .170 $\pm$ .001 & .168 $\pm$ .001 & .167 $\pm$ .001 & .168 $\pm$ .001 \\
 & .3 & .126 $\pm$ .000 & .124 $\pm$ .000 & .134 $\pm$ .000 & .132 $\pm$ .000 & .131 $\pm$ .000 & .131 $\pm$ .000 \\
 & .4 & .118 $\pm$ .001 & .112 $\pm$ .000 & .123 $\pm$ .000 & .120 $\pm$ .000 & .119 $\pm$ .000 & .119 $\pm$ .000 \\
 & .5 & .106 $\pm$ .000 & .101 $\pm$ .000 & .113 $\pm$ .000 & .110 $\pm$ .000 & .109 $\pm$ .000 & .109 $\pm$ .000 \\
 & .6 & .109 $\pm$ .000 & .102 $\pm$ .000 & .115 $\pm$ .000 & .112 $\pm$ .000 & .110 $\pm$ .000 & .110 $\pm$ .000 \\
 & .7 & .099 $\pm$ .000 & .098 $\pm$ .000 & .112 $\pm$ .000 & .109 $\pm$ .000 & .108 $\pm$ .000 & .107 $\pm$ .000 \\
 & .8 & .103 $\pm$ .000 & .099 $\pm$ .000 & .117 $\pm$ .000 & .112 $\pm$ .000 & .111 $\pm$ .000 & .110 $\pm$ .000 \\
 & .9 & .094 $\pm$ .000 & .089 $\pm$ .000 & .113 $\pm$ .000 & .107 $\pm$ .000 & .105 $\pm$ .000 & .106 $\pm$ .000 \\
\midrule
\multirow[c]{10}{*}{\rotatebox[origin=c]{90}{\small{$\sigma^2=10^{-5}$}}} & .0 & .945 $\pm$ .038 & .945 $\pm$ .037 & .949 $\pm$ .035 & .948 $\pm$ .035 & .948 $\pm$ .036 & .948 $\pm$ .036 \\
 & .1 & .158 $\pm$ .001 & .151 $\pm$ .001 & .161 $\pm$ .001 & .159 $\pm$ .001 & .158 $\pm$ .001 & .159 $\pm$ .001 \\
 & .2 & .157 $\pm$ .003 & .151 $\pm$ .003 & .162 $\pm$ .003 & .159 $\pm$ .003 & .158 $\pm$ .003 & .159 $\pm$ .003 \\
 & .3 & .126 $\pm$ .001 & .121 $\pm$ .001 & .131 $\pm$ .001 & .128 $\pm$ .001 & .127 $\pm$ .001 & .128 $\pm$ .001 \\
 & .4 & .114 $\pm$ .001 & .107 $\pm$ .001 & .118 $\pm$ .001 & .115 $\pm$ .001 & .114 $\pm$ .001 & .114 $\pm$ .001 \\
 & .5 & .104 $\pm$ .001 & .099 $\pm$ .000 & .110 $\pm$ .000 & .108 $\pm$ .000 & .107 $\pm$ .000 & .106 $\pm$ .000 \\
 & .6 & .115 $\pm$ .001 & .104 $\pm$ .001 & .117 $\pm$ .001 & .114 $\pm$ .001 & .113 $\pm$ .001 & .112 $\pm$ .001 \\
 & .7 & .107 $\pm$ .001 & .101 $\pm$ .001 & .115 $\pm$ .001 & .111 $\pm$ .001 & .110 $\pm$ .001 & .109 $\pm$ .001 \\
 & .8 & .098 $\pm$ .001 & .096 $\pm$ .001 & .114 $\pm$ .001 & .109 $\pm$ .001 & .108 $\pm$ .001 & .107 $\pm$ .001 \\
 & .9 & .091 $\pm$ .001 & .095 $\pm$ .001 & .119 $\pm$ .001 & .113 $\pm$ .001 & .111 $\pm$ .001 & .112 $\pm$ .001 \\
\bottomrule
\end{tabular}
}
\end{subtable}
\begin{subtable}[h]{1.1\textwidth}
    \centering
    \resizebox{0.35\paperheight}{!}{
    \begin{tabular}{cccccccc}
\toprule
 & Split & $\Risk_{\Tcal}(h)$ & $\Risk_{\Scal}(h)$ & \cref{chap:dis-pra:corollary:nn} & \cref{chap:dis-pra:eq:nn-rivasplata} & \cref{chap:dis-pra:eq:nn-blanchard} & \cref{chap:dis-pra:eq:nn-catoni} \\
\midrule
\multirow[c]{10}{*}{\rotatebox[origin=c]{90}{\small{$\sigma^2=10^{-4}$}}} & .0 & .912 $\pm$ .027 & .912 $\pm$ .027 & .918 $\pm$ .026 & .916 $\pm$ .027 & .916 $\pm$ .027 & .916 $\pm$ .026 \\
 & .1 & .164 $\pm$ .003 & .154 $\pm$ .003 & .164 $\pm$ .003 & .162 $\pm$ .003 & .161 $\pm$ .003 & .162 $\pm$ .004 \\
 & .2 & .164 $\pm$ .009 & .160 $\pm$ .009 & .170 $\pm$ .010 & .168 $\pm$ .010 & .167 $\pm$ .010 & .168 $\pm$ .010 \\
 & .3 & .125 $\pm$ .002 & .119 $\pm$ .002 & .129 $\pm$ .002 & .126 $\pm$ .002 & .126 $\pm$ .002 & .126 $\pm$ .002 \\
 & .4 & .119 $\pm$ .003 & .113 $\pm$ .003 & .124 $\pm$ .003 & .121 $\pm$ .003 & .120 $\pm$ .003 & .120 $\pm$ .003 \\
 & .5 & .109 $\pm$ .002 & .102 $\pm$ .001 & .113 $\pm$ .001 & .110 $\pm$ .001 & .109 $\pm$ .001 & .109 $\pm$ .001 \\
 & .6 & .102 $\pm$ .001 & .096 $\pm$ .001 & .109 $\pm$ .001 & .105 $\pm$ .001 & .105 $\pm$ .001 & .104 $\pm$ .001 \\
 & .7 & .099 $\pm$ .002 & .094 $\pm$ .001 & .108 $\pm$ .001 & .104 $\pm$ .001 & .103 $\pm$ .001 & .102 $\pm$ .001 \\
 & .8 & .104 $\pm$ .001 & .100 $\pm$ .002 & .118 $\pm$ .002 & .113 $\pm$ .002 & .112 $\pm$ .002 & .111 $\pm$ .002 \\
 & .9 & .092 $\pm$ .002 & .089 $\pm$ .001 & .113 $\pm$ .001 & .107 $\pm$ .001 & .105 $\pm$ .001 & .106 $\pm$ .001 \\
\midrule
\multirow[c]{10}{*}{\rotatebox[origin=c]{90}{\small{$\sigma^2=10^{-3}$}}} & .0 & .899 $\pm$ .026 & .899 $\pm$ .027 & .906 $\pm$ .026 & .904 $\pm$ .026 & .904 $\pm$ .026 & .905 $\pm$ .025 \\
 & .1 & .178 $\pm$ .006 & .170 $\pm$ .006 & .181 $\pm$ .006 & .178 $\pm$ .006 & .177 $\pm$ .006 & .179 $\pm$ .006 \\
 & .2 & .164 $\pm$ .006 & .159 $\pm$ .006 & .169 $\pm$ .006 & .167 $\pm$ .006 & .166 $\pm$ .006 & .167 $\pm$ .006 \\
 & .3 & .143 $\pm$ .007 & .138 $\pm$ .007 & .148 $\pm$ .007 & .146 $\pm$ .007 & .145 $\pm$ .007 & .145 $\pm$ .007 \\
 & .4 & .133 $\pm$ .005 & .129 $\pm$ .005 & .140 $\pm$ .005 & .137 $\pm$ .005 & .137 $\pm$ .005 & .137 $\pm$ .005 \\
 & .5 & .122 $\pm$ .004 & .117 $\pm$ .004 & .129 $\pm$ .004 & .126 $\pm$ .004 & .125 $\pm$ .004 & .125 $\pm$ .004 \\
 & .6 & .111 $\pm$ .003 & .104 $\pm$ .003 & .117 $\pm$ .003 & .114 $\pm$ .003 & .113 $\pm$ .003 & .112 $\pm$ .003 \\
 & .7 & .109 $\pm$ .003 & .103 $\pm$ .003 & .118 $\pm$ .003 & .114 $\pm$ .003 & .113 $\pm$ .003 & .112 $\pm$ .003 \\
 & .8 & .108 $\pm$ .004 & .102 $\pm$ .004 & .120 $\pm$ .004 & .115 $\pm$ .004 & .114 $\pm$ .004 & .113 $\pm$ .004 \\
 & .9 & .103 $\pm$ .003 & .099 $\pm$ .002 & .124 $\pm$ .003 & .118 $\pm$ .003 & .116 $\pm$ .003 & .116 $\pm$ .003 \\
\bottomrule
\end{tabular}
    }
\end{subtable}
\label{chap:dis-pra:table:2_data_fashion}
\end{table}
\end{landscape} 


\begin{landscape}
\begin{table}[t]
\caption{
Comparison of the bound values before performing Step {\bf 2)} of our Training Method for \algoours, \algorivasplata, \algoblanchard and \algocatoni. 
More precisely, for each split and each variance $\sigma^2{\in}\{10^{-3}, 10^{-4}, 10^{-5}, 10^{-6}\}$, we report the mean $\pm$ the standard deviation (for $400$ neural networks sampled from $\P$) of the test risk ($\Risk_{\dT}(\h)$), the empirical risk ($\Risk_{\dS}(\h)$), and the value of the bounds of \Cref{chap:dis-pra:corollary:nn,chap:dis-pra:corollary:nn-rbc}.
We consider in this table that the dataset is CIFAR-10.
}
\begin{subtable}[h]{0.50\textwidth}
    \centering
    \resizebox{0.35\paperheight}{!}{
    \begin{tabular}{cccccccc}
\toprule
 & Split & $\Risk_{\Tcal}(h)$ & $\Risk_{\Scal}(h)$ & \cref{chap:dis-pra:corollary:nn} & \cref{chap:dis-pra:eq:nn-rivasplata} & \cref{chap:dis-pra:eq:nn-blanchard} & \cref{chap:dis-pra:eq:nn-catoni} \\
\midrule
\multirow[c]{10}{*}{\rotatebox[origin=c]{90}{\small{$\sigma^2=10^{-6}$}}} & .0 & .899 $\pm$ .000 & .899 $\pm$ .000 & .906 $\pm$ .000 & .904 $\pm$ .000 & .903 $\pm$ .000 & .904 $\pm$ .000 \\
 & .1 & .476 $\pm$ .000 & .470 $\pm$ .000 & .486 $\pm$ .000 & .482 $\pm$ .000 & .481 $\pm$ .000 & .485 $\pm$ .000 \\
 & .2 & .390 $\pm$ .000 & .389 $\pm$ .000 & .406 $\pm$ .000 & .402 $\pm$ .000 & .401 $\pm$ .000 & .404 $\pm$ .000 \\
 & .3 & .370 $\pm$ .000 & .358 $\pm$ .000 & .374 $\pm$ .000 & .371 $\pm$ .000 & .370 $\pm$ .000 & .372 $\pm$ .000 \\
 & .4 & .334 $\pm$ .000 & .328 $\pm$ .000 & .346 $\pm$ .000 & .342 $\pm$ .000 & .341 $\pm$ .000 & .342 $\pm$ .000 \\
 & .5 & .307 $\pm$ .000 & .302 $\pm$ .000 & .321 $\pm$ .000 & .317 $\pm$ .000 & .316 $\pm$ .000 & .317 $\pm$ .000 \\
 & .6 & .274 $\pm$ .000 & .276 $\pm$ .000 & .297 $\pm$ .000 & .293 $\pm$ .000 & .291 $\pm$ .000 & .291 $\pm$ .000 \\
 & .7 & .275 $\pm$ .000 & .272 $\pm$ .000 & .296 $\pm$ .000 & .290 $\pm$ .000 & .289 $\pm$ .000 & .288 $\pm$ .000 \\
 & .8 & .249 $\pm$ .000 & .237 $\pm$ .000 & .265 $\pm$ .000 & .259 $\pm$ .000 & .257 $\pm$ .000 & .256 $\pm$ .000 \\
 & .9 & .227 $\pm$ .000 & .230 $\pm$ .000 & .269 $\pm$ .000 & .260 $\pm$ .000 & .258 $\pm$ .000 & .258 $\pm$ .000 \\
\midrule
\multirow[c]{10}{*}{\rotatebox[origin=c]{90}{\small{$\sigma^2=10^{-5}$}}} & .0 & .899 $\pm$ .001 & .899 $\pm$ .000 & .906 $\pm$ .000 & .904 $\pm$ .000 & .904 $\pm$ .000 & .904 $\pm$ .000 \\
 & .1 & .476 $\pm$ .000 & .478 $\pm$ .000 & .494 $\pm$ .000 & .490 $\pm$ .000 & .489 $\pm$ .000 & .493 $\pm$ .000 \\
 & .2 & .403 $\pm$ .000 & .398 $\pm$ .000 & .414 $\pm$ .000 & .410 $\pm$ .000 & .409 $\pm$ .000 & .412 $\pm$ .000 \\
 & .3 & .349 $\pm$ .000 & .350 $\pm$ .000 & .367 $\pm$ .000 & .363 $\pm$ .000 & .362 $\pm$ .000 & .364 $\pm$ .000 \\
 & .4 & .322 $\pm$ .000 & .313 $\pm$ .000 & .330 $\pm$ .000 & .327 $\pm$ .000 & .326 $\pm$ .000 & .327 $\pm$ .000 \\
 & .5 & .281 $\pm$ .000 & .283 $\pm$ .000 & .302 $\pm$ .000 & .298 $\pm$ .000 & .297 $\pm$ .000 & .297 $\pm$ .000 \\
 & .6 & .290 $\pm$ .000 & .286 $\pm$ .000 & .307 $\pm$ .000 & .303 $\pm$ .000 & .301 $\pm$ .000 & .301 $\pm$ .000 \\
 & .7 & .266 $\pm$ .000 & .257 $\pm$ .000 & .281 $\pm$ .000 & .276 $\pm$ .000 & .274 $\pm$ .000 & .274 $\pm$ .000 \\
 & .8 & .247 $\pm$ .000 & .243 $\pm$ .000 & .271 $\pm$ .000 & .265 $\pm$ .000 & .263 $\pm$ .000 & .262 $\pm$ .000 \\
 & .9 & .236 $\pm$ .000 & .227 $\pm$ .000 & .266 $\pm$ .000 & .257 $\pm$ .000 & .255 $\pm$ .000 & .255 $\pm$ .000 \\
\bottomrule
\end{tabular}
}
\end{subtable}
\begin{subtable}[h]{1.1\textwidth}
    \centering
    \resizebox{0.35\paperheight}{!}{
    \begin{tabular}{cccccccc}
\toprule
 & Split & $\Risk_{\Tcal}(h)$ & $\Risk_{\Scal}(h)$ & \cref{chap:dis-pra:corollary:nn} & \cref{chap:dis-pra:eq:nn-rivasplata} & \cref{chap:dis-pra:eq:nn-blanchard} & \cref{chap:dis-pra:eq:nn-catoni} \\
\midrule
\multirow[c]{10}{*}{\rotatebox[origin=c]{90}{\small{$\sigma^2=10^{-4}$}}} & .0 & .900 $\pm$ .004 & .900 $\pm$ .003 & .907 $\pm$ .003 & .905 $\pm$ .003 & .905 $\pm$ .003 & .905 $\pm$ .003 \\
 & .1 & .458 $\pm$ .001 & .464 $\pm$ .001 & .479 $\pm$ .001 & .476 $\pm$ .001 & .475 $\pm$ .001 & .478 $\pm$ .001 \\
 & .2 & .395 $\pm$ .001 & .396 $\pm$ .000 & .412 $\pm$ .000 & .409 $\pm$ .000 & .408 $\pm$ .000 & .411 $\pm$ .000 \\
 & .3 & .361 $\pm$ .001 & .361 $\pm$ .000 & .378 $\pm$ .000 & .375 $\pm$ .000 & .373 $\pm$ .000 & .376 $\pm$ .000 \\
 & .4 & .323 $\pm$ .001 & .316 $\pm$ .000 & .334 $\pm$ .000 & .330 $\pm$ .000 & .329 $\pm$ .000 & .331 $\pm$ .000 \\
 & .5 & .296 $\pm$ .001 & .291 $\pm$ .000 & .310 $\pm$ .000 & .306 $\pm$ .000 & .304 $\pm$ .000 & .305 $\pm$ .000 \\
 & .6 & .271 $\pm$ .001 & .263 $\pm$ .000 & .284 $\pm$ .000 & .279 $\pm$ .000 & .278 $\pm$ .000 & .278 $\pm$ .000 \\
 & .7 & .253 $\pm$ .001 & .246 $\pm$ .000 & .270 $\pm$ .000 & .265 $\pm$ .000 & .263 $\pm$ .000 & .262 $\pm$ .000 \\
 & .8 & .259 $\pm$ .001 & .252 $\pm$ .001 & .281 $\pm$ .001 & .275 $\pm$ .001 & .273 $\pm$ .001 & .272 $\pm$ .001 \\
 & .9 & .217 $\pm$ .000 & .216 $\pm$ .001 & .255 $\pm$ .001 & .246 $\pm$ .001 & .243 $\pm$ .001 & .244 $\pm$ .001 \\
\midrule
\multirow[c]{10}{*}{\rotatebox[origin=c]{90}{\small{$\sigma^2=10^{-3}$}}} & .0 & .905 $\pm$ .012 & .904 $\pm$ .012 & .911 $\pm$ .011 & .909 $\pm$ .011 & .909 $\pm$ .011 & .909 $\pm$ .011 \\
 & .1 & .479 $\pm$ .002 & .480 $\pm$ .001 & .496 $\pm$ .001 & .493 $\pm$ .001 & .491 $\pm$ .001 & .495 $\pm$ .001 \\
 & .2 & .415 $\pm$ .002 & .415 $\pm$ .001 & .432 $\pm$ .001 & .428 $\pm$ .001 & .427 $\pm$ .001 & .430 $\pm$ .001 \\
 & .3 & .417 $\pm$ .001 & .416 $\pm$ .001 & .434 $\pm$ .001 & .430 $\pm$ .001 & .429 $\pm$ .001 & .431 $\pm$ .001 \\
 & .4 & .333 $\pm$ .001 & .323 $\pm$ .001 & .341 $\pm$ .001 & .337 $\pm$ .001 & .336 $\pm$ .001 & .338 $\pm$ .001 \\
 & .5 & .316 $\pm$ .001 & .311 $\pm$ .001 & .331 $\pm$ .001 & .327 $\pm$ .001 & .325 $\pm$ .001 & .326 $\pm$ .001 \\
 & .6 & .280 $\pm$ .001 & .281 $\pm$ .001 & .302 $\pm$ .001 & .298 $\pm$ .001 & .296 $\pm$ .001 & .296 $\pm$ .001 \\
 & .7 & .239 $\pm$ .001 & .234 $\pm$ .001 & .257 $\pm$ .001 & .252 $\pm$ .001 & .250 $\pm$ .001 & .250 $\pm$ .001 \\
 & .8 & .249 $\pm$ .001 & .245 $\pm$ .001 & .274 $\pm$ .001 & .268 $\pm$ .001 & .266 $\pm$ .001 & .264 $\pm$ .001 \\
 & .9 & .233 $\pm$ .001 & .232 $\pm$ .002 & .272 $\pm$ .002 & .263 $\pm$ .002 & .260 $\pm$ .002 & .260 $\pm$ .002 \\
\bottomrule
\end{tabular}
    }
\end{subtable}
\label{chap:dis-pra:table:2_data_cifar10}
\end{table}
\end{landscape}
\end{noaddcontents}