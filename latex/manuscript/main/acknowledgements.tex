\section*{Remerciements}

Ces quelques lignes viennent d'un matin de juillet qui parachève l'écriture de ce manuscrit, produit d'un voyage rarement hésitant mais toujours incertain. Le soleil se lève, abreuvant de ses rayons un bosquet de noms toujours croissant et impatient d'arriver au plus vite sur le papier pour faire fleurir sur vos lèvres, lecteurs et lectrices, le coin d'un sourire. 
Une métaphore en guise de commencement: cette thèse est une lance finement ouvragée, je n'en suis que le modeste fer, sobre et discret qui s'attêle à la tâche, là où vous tous formez cette hampe magnifique qui donne prestance, extension et force.\\
Commençons par la prestance, toute scientifique, de ce manuscrit qui n'aurait pu être sans Benjamin, qui m'a accompagné pendant quatre ans et qui m'a appris les divers aspect du métier et m'a fait évoluer d'étudiant enthousiaste à docteur en herbe. Un merci particulier pour m'avoir laissé bricoler mon cadre de vie insolite à Paris qui, pour mes proches, m'a maintenu loin du labo. Tout autre  Un grand merci également à Umut Simsekli, Paul Viallard, Pierre Jobic, Omar Rivasplata ainsi que John Shawe-Taylor pour m'avoir fait découvrir, sous divers aspects, la richesse de la collaboration qui offre un sens bien plus humain aux choses de la science. Point d'orgue sur cette liste déjà fournie, un grand merci aux membres du jury, Stéphane Chrétien qui a eu la gentillesse de s'intéresser à mes travaux il y a déja quelques mois, Claire Boyer et Gérard Biau que j'ai connu en temps qu'enseignants à Jussieu aux prémisses de ma thèse et dont la présence dans ce comité clôt magnifiquement cette boucle. Enfin, un grand merci à Frédéric Chazal et Pascal Germain d'être rapporteur et d'avoir consacré leur précieux temps à cette production scientifique qui je l'espère, aura suscité un certain interêt (et avec un peu de chance, un interêt certain).\\
Continuons par l'extension, toute spirituelle, qui m'a été prodiguée lors de cette histoire par mes amies et amis qui m'enrichissent autant qu'ils me font rire chaque jour durant, sans la merveilleuse mosaïque de leurs passions et reflexions, nul doute que ma thèse serait restée une coquille vide et sans passion. Je songe à ce magicien d'Anatole, à Anaïs qui chante à Pierre qui apparaît une deuxième fois dans ces remerciements, à Antoine qui aime trop Grenoble pour que je le voie souvent, à Farf qui a troqué la pizza cachanaise pour l'italienne? Je songe à mes amis qui dansent, tous membres la Guillotine (nom à redéfinir au demeurant): Keko et Chanus qui secrètement adorent me voir parler du grand capital, Kaou, Selene, Vince, Antoine, Juliette, Abou, Calvin, Bertrand. Merci également à Jadou et Phil qui se déhanchent comme jamais quand je joue, et dont je sais la présence, même si je ne vous vois que peu ces temps-ci, merci à Miles pour ta vision de la guitare qui a résonné à la mienne et l'a faconnée. A tous, merci pour votre musique. 
