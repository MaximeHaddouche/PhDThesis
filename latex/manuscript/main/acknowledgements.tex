\section*{Remerciements}

Ces quelques lignes viennent d'un matin de juillet qui parachève l'écriture de ce manuscrit, produit d'un voyage rarement hésitant mais toujours incertain. Alors que le soleil se lève, ses rayons font croître en moi un vivace bosquet de noms qui fleurira très vite sur le papier. 
Mais par où commencer, une métaphore peut être? Je vois cette thèse comme une lance finement ouvragée dont je ne suis que le fer, sobre et discret qui s'attele à la tâche, là où vous tous formez cette hampe magnifique qui donne prestance, extension et force.\\
Commençons par la prestance, toute scientifique, de ce manuscrit qui n'aurait pu être sans Benjamin, qui m'a accompagné pendant quatre ans, qui m'a appris les divers aspects du métier et m'a fait évoluer d'étudiant enthousiaste à docteur en herbe. Un merci particulier pour m'avoir laissé bricoler mon cadre de vie insolite à Paris qui, pour rester avec mes proches, m'a maintenu loin du labo. Un grand merci également à Umut Simsekli, Paul Viallard, Pierre Jobic, Omar Rivasplata ainsi que John Shawe-Taylor pour m'avoir fait découvrir, sous divers aspects, la richesse de la collaboration qui offre un sens bien plus humain à la chose scientifique. Point d'orgue sur cette liste déjà fournie, un grand merci aux membres du jury, Stéphane Chrétien qui a eu la gentillesse de s'intéresser à mes travaux il y a déja quelques mois, Claire Boyer et Gérard Biau que j'ai connu en temps qu'enseignants aux prémisses de ma thèse et dont la présence dans ce comité clôt magnifiquement cette boucle. Enfin, un grand merci à Frédéric Chazal et Pascal Germain d'être rapporteur et d'avoir consacré leur précieux temps à cette production scientifique qui je l'espère, aura suscité un certain interêt (et avec un peu de chance, un interêt certain).\\
Continuons par l'extension, toute spirituelle, qui m'a été prodiguée lors de cette histoire par mes amies et amis qui m'enrichissent autant qu'ils me font rire chaque jour. Sans la merveilleuse mosaïque de leurs passions et reflexions, nul doute que ma thèse serait terne et sans saveur. Je songe à ce magicien d'Anatole, cette chanteuse d'Anaïs, à Pierre qui apparaît une deuxième fois dans ces remerciements, à Antoine qui aime trop Grenoble pour que je le voie souvent, à Farf qui a troqué la pizza cachanaise pour l'italienne. Je songe à mes amis qui dansent, membres la Guillotine (nom à redéfinir au demeurant): Keko et Chanus qui secrètement adorent me voir parler du grand capital, Kaou, Sélene, Vince, Antoine, Angelo, Maxime, Juliette, Abou, Calvin, Bertrand. Merci également à Jadou et Phil dont je sais la présence, même si je ne vous vois que peu ces temps-ci, merci à Miles pour avoir partagé ta vision de la guitare, merci à Hugo et Pia qui seront les prochains artistes pop-electro de cette génération. A tous, merci pour votre musique. 
Merci également à tous ceux qui m'ont enchanté de temps à autre par leur présence au fil des ans: Armand, Samy, Valentin qui est toujours là pour une session tennis tout comme Thibault, mes anciens colocs Quentin et Aaron qui ne sont jamais bien loin. Merci aussi à Liam et Sophie qui sont à ce jour les dernières belles rencontres que j'ai faites.
Je songe également aux copains de prépa que j'ai toujours le plaisir de croiser de temps à autres, Merwan, Pierre (troisième apparition dans ces remerciements), Clément, Mat' Blanke que j'ai eu la chance de redécouvrir pendant cette thèse ainsi que Song. Merci aussi à Martin (Verdammt), au Greg d'être là de la soirée au déjeuner (ou bien l'inverse).\\  
Que de noms déjà mentionnés qui masquent des analyses fines, de l'art vivace, des discussions précieuses. Et tant encore à venir: merci au Hossen qui a bien des fois changé mon rapport aux choses, à Tom et ses analyses d'une justesse rare, Matthias, le réalisateur le plus prometteur de Paris, Gabriel qui développera sans nul doute un regard frais et caustique sur l'astrologie romaine, Ulysse qui philosophe avec sa bienveillance rare, et Ariane qui porte sur le monde son regard solaire.\\
Finalement, la force, c'est ce que m'ont donné les plus courageux d'entre vous qui me supportent de façon bien trop régulière. Vous m'avez sans cesse sorti du silence froid de ma chambre où les maths ont foisonné trois années durant. Je pense à ma très chère Mathilde qui a toujours eu du temps malgré des études de médecine de plus en plus prenantes, à Yvan et Simon et leur bienveillance rare, à Antoine, Jacques et François-Xavier qui incarnent cette alliance apaisée d'un regard perçant aux choses et d'une écoute sincère. Je pense à Louis qui a rendu possible mon projet musical et qui est là dans moults facettes de ma vie. Je pense au concile de Vincennes: Hada, Monjuju, Yannis, Yaacov, Alex, Brian qui rendent tout léger, sauf leur ventre, depuis l'adolescence voire même avant. Finalement, je pense à ma famille qui est là depuis le début, à mon père, à ma mère qui m'ont élevé, dans tout les sens du terme, vingt-huit années durant, à mon grand frère JB qui prend le temps de m'instruire sur le tennis et d'improviser des bons plans, à Kevin et Clémence qui m'ont toujours accueilli au coin de leur salon pour un jeu de plateau, et enfin à mon Nain et ma Naine à qui je transmets ce que je peux. À vous tous, merci.

\newpage
\textit{À Anaïs, qui rend commun ce qui est rare.}