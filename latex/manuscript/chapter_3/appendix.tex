\chapter{Appendix of Chapter~\ref{chap:mv-robustness}}
\label{ap:mv-robustness}

\minitoc

\begin{noaddcontents}
\section{Proof of \Cref{chap:mv-robustness:proposition:risks}}
\label{ap:mv-robustness:sec:proof-risks}

\proprisks*
\begin{proof} First, we prove $\RiskM_{\Dpert^{1}}(\MVQ){=}\Risk_{\Dpert}(\MVQ)$. We have
\begin{align*}
    \RiskM_{\Dpert^{1}}(\MVQ) &= 1- \PP_{((\x,\y), \Epert)\sim\Dpert^{1}} \LP \forall \epsilon\in\Epert, \MVQ(\x+\epsilon) = \y \RP\\
    &= 1- \PP_{((\x,\y), \Epert)\sim\Dpert^{1}} \LP \forall \epsilon\in\{\epsilon_1\}, \MVQ(\x+\epsilon) = \y \RP\\
    &= 1- \PP_{((\x,\y), \Epert)\sim\Dpert^{1}} \LP \MVQ(\x+\epsilon_1) = \y \RP = \Risk_{\Dpert}(\MVQ).
\end{align*}
\looseness=-1
Then, we prove the inequality $\RiskM_{\Dpert^{n'}}(\MVQ) \le \RiskM_{\Dpert^{n}}(\MVQ)$ from the fact that the indicator function $\indic\LB\cdot\RB$ is upper-bounded by $1$. Indeed, from \Cref{chap:mv-robustness:def:average_max_risk} we have 
\allowdisplaybreaks
\begin{align*}
    1-\RiskM_{\Dpert^{n}}(\MVQ) &= \EE_{(\x,\y)\sim\D} \ \EE_{\Epert\sim\DX^{n}} \indic\LB\forall\epsilon\in\Epert, \MVQ(\x+\epsilon) = \y\RB\\
    &= \EE_{(\x,\y)\sim\D}\LB \prod_{i=1}^{{n}}\ \EE_{\epsilon_i\sim\DX}\indic\LB\MVQ(\x+\epsilon_i) = \y\RB\RB\\
    &\le \EE_{(\x,\y)\sim\D}\LB \prod_{i=1}^{{n'}}\ \EE_{\epsilon_i\sim\DX}\indic\LB\MVQ(\x+\epsilon_i) = \y\RB\RB\\
    &= \EE_{(\x,\y)\sim\D}\ \EE_{\Epert'\sim\DX^{n'}} \indic\LB\forall\epsilon\in\Epert', \MVQ(\x+\epsilon) = \y\RB\\
    &= 1-\RiskM_{\Dpert^{n'}}(\MVQ).
\end{align*}
Lastly, to prove the right-most inequality, we have to use the fact that the expectation over the set $\Xpert$ is bounded by the maximum over the set $\Xpert$.  
We have
\begin{align*}
    \RiskM_{\Dpert^{{n}}}(\MVQ)
    &= \EE_{(\x,\y)\sim\D} \EE_{\epsilon_1\sim\DX} {\ldots}\EE_{\ \epsilon_{n}\sim\DX}\indic\LB\exists\epsilon{\in}\{\epsilon_1,\ldots, \epsilon_{n}\}, \MVQ(\x+\epsilon)\ne \y\RB\\
    &\le \EE_{(\x,\y)\sim\D} \max_{\epsilon_1\in\Xpert}\ldots\max_{\epsilon_{n}\in\Xpert}\indic \LB\exists\epsilon\in\{\epsilon_1,{\ldots,}\epsilon_{n}\}, \MVQ(\x+\epsilon)\ne \y\RB\\
    &= \EE_{(\x,\y)\sim\D} \max_{\epsilon_1\in\Xpert}{\ldots} \max_{\epsilon_{{n}-1}\in\Xpert} \indic \LB\exists\epsilon\in\{\epsilon_1,\ldots, \epsilon^*\}, \MVQ(\x+\epsilon)\ne \y\RB\\
    &= \EE_{(\x,\y)\sim\D} \indic \LB \MVQ(\x+\epsilon^*)\ne \y\RB\\
    &= \EE_{(\x,\y)\sim\D} \max_{\epsilon\in\Xpert}\indic \LB \MVQ(\x+\epsilon)\ne \y\RB\ =\  \ \RiskA_{\D}(\MVQ).
\end{align*}
Merging the three equations proves the claim.
\end{proof}

\section{Proof of \Cref{chap:mv-robustness:proposition:avg-worst-risk}}
\label{ap:mv-robustness:sec:proof-avg-worst-risk}

In this section, we provide the proof of \Cref{chap:mv-robustness:proposition:avg-worst-risk} that relies on \Cref{chap:mv-robustness:lemma:Delta,chap:mv-robustness:lemma:Pi} which are also described and proved.
\Cref{chap:mv-robustness:lemma:Delta} shows that $\Risk_{\Dpert}(\MVQ)$ equals $\Risk_{\Gamma}(\MVQ)$.

\begin{lemma}
For any distribution $\Dpert$ on $(\X{\times}\Y){\times}\Xpert$ and its associated distribution $\Gamma$, for any posterior $\Q$ on $\H$, we have
\label{chap:mv-robustness:lemma:Delta}
\begin{align*}
    \Risk_{\Dpert}(\MVQ) = \Pr_{(\x+\epsilon, \y)\sim\Gamma}\LB \MVQ(\x+\epsilon){\ne}y\RB = \Risk_{\Gamma}(\MVQ).
\end{align*}
\end{lemma}
\begin{proof}
Starting from the averaged risk $\Risk_{\Dpert}(\MVQ)=\EE_{((\x,\y){,}\epsilon)\sim\Dpert}\indic\LB\MVQ(\x+\epsilon){\ne}y\RB$, we have
\begin{align*}
    \Risk_{\Dpert}(\MVQ) &= \EE_{(\x'{+}\epsilon', \y')\sim \Gamma}\!\tfrac{1}{\Gamma(\x'{+}\epsilon'{,}\y')}\!\LB\Pr_{((\x,\y),\epsilon)\sim\Dpert}\!\LB\MVQ(\x+\epsilon){\ne}y, \x'{+}\epsilon'{=}\x+\epsilon, \y'{=}y\RB\RB\\
    &= \EE_{(\x'{+}\epsilon', \y')\sim \Gamma}\!\tfrac{1}{\Gamma(\x'{+}\epsilon'{,}\y')}\!\LB
    \EE_{((\x,\y),\epsilon)\sim\Dpert}\!\indic\!\LB\MVQ(\x+\epsilon){\ne}y\RB \indic\!\LB \x'{+}\epsilon'{=}\x+\epsilon, \y'{=}y\RB\RB.\\
\end{align*}
\looseness=-1
In other words, the double expectation only rearranges the terms of the original expectation: given an example $(\x'{+}\epsilon'{,}\y')$, we gather probabilities such that $\MVQ(\x+\epsilon){\ne}y$ with $(\x+\epsilon{,}\y) {=} (\x'{+}\epsilon'{,}\y')$ in the inner expectation, while integrating over all couple $(\x'{+}\epsilon', \y')\in\X{\times}\Y$ in the outer expectation.
Then, from the fact that when $\x'{+}\epsilon'{=}\x+\epsilon$ and $\y'{=}y$, $\indic\!\LB\MVQ(\x+\epsilon){\ne}y\RB=\indic\!\LB\MVQ(\x'{+}\epsilon'){\ne} \y'\RB$, we have 
\begin{align*}
    \Risk_{\Dpert}(\MVQ) &= \EE_{(\x'{+}\epsilon', \y')\sim \Gamma}\!\tfrac{1}{\Gamma(\x'{+}\epsilon'{, }\y')}\!\LB\EE_{((\x,\y),\epsilon)\sim\Dpert}\!\indic\!\LB\MVQ(\x'{+}\epsilon'){\ne}\y'\RB\!\indic\!\LB \x'{+}\epsilon'{=}\x+\epsilon, \y'{=}y\RB\RB\\
    &= \EE_{(\x'{+}\epsilon', \y')\sim\Gamma}\!\tfrac{1}{\Gamma(\x'{+}\epsilon'{, }\y')}\!\LB\indic\!\LB\MVQ(\x'{+}\epsilon'){\ne}\y'\RB\EE_{((\x,\y),\epsilon)\sim\Dpert}\!\!\indic\!\LB \x'{+}\epsilon'{=}\x+\epsilon, \y'{=}y\RB\RB.
\end{align*}
Finally, by definition of $\Gamma(\x'{+}\epsilon'{,}\y')$, we can deduce that
\begin{align*}
    \Risk_{\Dpert}(\MVQ) &=\EE_{(\x'{+}\epsilon', \y')\sim \Gamma}\!\tfrac{1}{\Gamma(\x'{+}\epsilon'{, }\y')}\!\LB\indic\!\LB\MVQ(\x'{+}\epsilon'){\ne}\y'\RB\Gamma(\x'{+}\epsilon'{,}\y') \RB\\
    &= \EE_{(\x'{+}\epsilon', \y')\sim \Gamma}\indic\!\LB\MVQ(\x'{+}\epsilon'){\ne}\y'\RB = \Risk_{\Gamma}(\MVQ).
\end{align*}\end{proof}

Similarly, \Cref{chap:mv-robustness:lemma:Pi} shows that $\RiskA_{\D}(\MVQ)$ is equivalent to $\Risk_{\gamma}(\MVQ)$.
\begin{lemma} 
For any distribution $\D$ on $\X\times\Y$ and its associated distribution $\gamma$, for any posterior $\Q$ on $\H$, we have
\label{chap:mv-robustness:lemma:Pi}
\begin{align*}
    \RiskA_{\D}(\MVQ) = \Pr_{(\x+\epsilon{,}\y){\sim}\gamma}\!\LB \MVQ(\x+\epsilon){\ne}y\RB = \Risk_{\gamma}(\MVQ).
\end{align*}
\end{lemma}
\begin{proof} 
The proof is similar to the one of \Cref{chap:mv-robustness:lemma:Delta}.
Indeed, starting from the definition of $\RiskA_{\D}(\MVQ) = \EE_{(\x,\y)\sim\D}\indic\!\LB\MVQ(\x+\epsilon^*(\x{,}\y)) \neq \y\RB$, we have
\begin{align*}
    &\RiskA_{\D}(\MVQ)\\
    &= \EE_{(\x'{+}\epsilon', \y')\sim \gamma}\tfrac{1}{\gamma(\x'{+}\epsilon', \y')}\!\LB\EE_{(\x,\y)\sim\D}\indic\LB\MVQ(\x{+}\epsilon^*(\x{,}\y))\ne \y\RB\!\indic\!\LB \x'{+}\epsilon'{=}\x{+}\epsilon^*(\x{,}\y), \y'{=}y\RB\RB\\
    &= \EE_{(\x'+\epsilon', \y')\sim\gamma}\tfrac{1}{\gamma(\x'{+}\epsilon', \y')}\!\LB\EE_{(\x{,}\y)\sim\D}\indic\LB\MVQ(\x'{+}\epsilon')\ne \y'\RB\!\indic\!\LB \x'{+}\epsilon'{=}\x{+}\epsilon^*(\x{,}\y), \y'{=}y\RB\RB.
\end{align*}
Finally, by definition of $\gamma(\x'{+}\epsilon', \y')$, we can deduce that
\begin{align*}
    \RiskA_{\D}(\MVQ) &\!=\!\!\EE_{(\x'+\epsilon', \y')\sim \gamma}\tfrac{1}{\gamma(\x'+\epsilon', \y')}\LB\indic\LB\MVQ(\x'{+}\epsilon')\ne \y'\RB\gamma(\x'{+}\epsilon'{,}\y')\RB\\
    &= \EE_{(\x'+\epsilon', \y')\sim\gamma}\!\!\indic\LB\MVQ(\x'{+}\epsilon'){\ne}\y'\RB\! = \Risk_{\gamma}(\MVQ).
\end{align*}
\end{proof}

We can now prove \Cref{chap:mv-robustness:proposition:avg-worst-risk}.

\propavgworstrisk*
\begin{proof}
    From \Cref{chap:mv-robustness:lemma:Delta,chap:mv-robustness:lemma:Pi}, we have 
    \begin{align*}
        \Risk_{\Dpert}(\MVQ) = \Risk_{\Gamma}(\MVQ), \text{\quad and\quad} \RiskA_{\D}(\MVQ)=\Risk_{\gamma}(\MVQ).
    \end{align*}
    Then, we apply Lemma~4 of \citet{OhnishiHonorio2021}, we have
    \begin{align*}
    \Risk_{\gamma}(\MVQ) \le \TV(\gamma\|\Gamma) + \Risk_{\Gamma}(\MVQ)\quad 
    \iff \quad \RiskA_{\D}(\MVQ) \le \TV(\gamma\|\Gamma) + \Risk_{\Dpert}(\MVQ).
    \end{align*}
\end{proof}

\section{Proof of \Cref{chap:mv-robustness:theorem:2-adver-gibbs}}
\label{ap:mv-robustness:sec:proof-2-adver-gibbs}

\theoremadvergibbs*
\begin{proof}
By the definition of the majority vote and from \textsc{Markov}'s inequality (\Cref{ap:tools:theorem:first-markov}), we have 
\begin{align*}
    \frac12 \Risk_{\Dpert}(\MVQ)
    &= \frac12\, \PP_{( (\x,\y),\epsilon)\sim\Dpert} \LP \y \EE_{\h\sim\Q} \h(\x+\epsilon) \le 0\RP\\
    &= \frac12 \, \PP_{( (\x,\y),\epsilon)\sim\Dpert}\LP 1 - \y \EE_{\h\sim\Q} \h(\x+\epsilon) \ge 1\RP\\
    &\leq  \EE_{( (\x,\y),\epsilon)\sim\Dpert} \frac12 \Big[ 1 - \y \EE_{\h\sim\Q} \h(\x+\epsilon) \Big]\\
    &= \RiskS_{\Dpert}(\Q).
\end{align*}
    Similarly we have
\begin{align*}
    \frac12\RiskM_{\Dpert^{n\!}}(\MVQ) &=\frac12\,\PP_{( (\x,\y),\Epert)\sim\Dpert^{n\!}} \LP \exists \epsilon \in\Epert, \y \EE_{\h\sim\Q} \h(\x+\epsilon)\le 0 \RP\\
    &=\frac12\,\PP_{( (\x,\y),\Epert)\sim\Dpert^{n\!}} \LP \min_{\epsilon\in\Epert} \Big( \y \EE_{\h\sim\Q} \h(\x+\epsilon)\Big) \le 0 \RP\\
    &= \frac12 \, \PP_{( (\x,\y),\epsilon)\sim\Dpert}\LP 1 - \min_{\epsilon\in\Epert} \Big( \y \EE_{\h\sim\Q} \h(\x+\epsilon)\Big) \ge 1\RP\\
    &\leq \EE_{( (\x,\y),\epsilon)\sim\Dpert} \frac12 \Big[ 1 - \min_{\epsilon\in\Epert}\Big( \y \EE_{\h\sim\Q} \h(\x+\epsilon)\Big) \Big]\\
    &= \RiskMS_{\Dpert^{n}}(\Q).
\end{align*}
\end{proof}

\section{Proof of \Cref{chap:mv-robustness:theorem:chromatic}}
\label{ap:mv-robustness:sec:proof-chromatic}

\theoremchromatic*
\begin{proof}
Let $G{=}(V, E)$ be the graph representing the dependencies between the random variables where 
{\it (i)} the set of vertices is $V {=} \Spert$, {\it (ii)} the set of edges $E$ is defined such that \mbox{$(((\x, \y), \epsilon), ((\x', \y'), \epsilon')) \notin E \Leftrightarrow x \ne \x'$}.
Then, applying Theorem~8~of~\citet{RalaivolaSzafranskiStempfel2010} with our notations gives
\begin{align*}
    \kl(\RiskS_{\Spert}(\Q)\|\RiskS_{\Dpert}(\Q)) \le \frac{\chi({G})}{mn} \Bigg[\KL(\Q\|\P) + \ln \frac{mn+\chi({G})}{\delta\chi({G})}
     \Bigg],
\end{align*}
where $\chi(G)$ is the fractional chromatic number of $G$.
From a property of \citet{ScheinermanUllman2011}, we have
\begin{align*}
    c(G) \le \chi(G) \le \Delta(G)+1,
\end{align*}
where $c(G)$ is the order of the largest clique in $G$ and $\Delta(G)$ is the maximum degree of a vertex in $G$.
By construction of $G$, $c(G){=}n$ and $\Delta(G){=}n{-}1$. 
Thus, $\chi(G){=}n$ and rearranging the terms proves  \Cref{chap:mv-robustness:eq:seeger-chromatic}.
Finally, by applying \textsc{Pinsker}'s inequality (\ie, $|a{-}b|{\le} \sqrt{\tfrac{1}{2}\kl(a\|b)}$), we obtain \Cref{chap:mv-robustness:eq:mcallester-chromatic}.
\end{proof}

\section{Proof of \Cref{chap:mv-robustness:theorem:bound-average-max}}
\label{ap:mv-robustness:sec:proof-bound-average-max}

\theoremboundaveragemax*
\begin{proof}
\newcommand{\Loss}[1]{L_{#1}}
Let $\Loss{\h{,}(\x{,}\y){,}\epsilon}{=}\frac{1}{2}\big[1{-}\y\h(\x+\epsilon)\big]$ for the sake of readability.
Given $\h\in\H$, the losses $\max_{\epsilon\in\Epert_1}\!\Loss{\h{,} (\x_1{,}\y_1){,} \epsilon}$, $\ldots$ $\max_{\epsilon\in\Epert_1}\!\Loss{\h{,} (\x_m{,}\y_m){,} \epsilon}$ are \iid.
Hence, we can apply Theorem 20 of \citet{GermainLacasseLavioletteMarchandRoy2015} and \textsc{Pinsker}'s inequality, \ie,\ the inequality $|q{-}p|\! \le\! \sqrt{\!\frac{1}{2}\kl(q\|p)}$ (\Cref{ap:pac-bayes:theorem:pinsker}) to obtain
\begin{align*}
    \EE_{\h\sim\Q}\EE_{(\x,\y),\Epert) \sim\Dpert^{n\!}} \max_{\epsilon\in\Epert}\ \Loss{\h{,}(\x{,}\y){,}\epsilon}
  \le \EE_{\h\sim\Q}\frac{1}{\m}\sum_{i=1}^{\m}\max_{\epsilon\in\Epert_i}\Loss{\h{,}(\x_i{,}\y_i){,}\epsilon} +
  \sqrt{\frac{\KL(\Q\|\P)+\ln\tfrac{2\sqrt{\m}}{\delta}}{2\m}}.
\end{align*}
Then, we lower-bound the left-hand side of the inequality with $\RiskMS_{\Dpert^{n}}(\Q)$, we have
\begin{align*}
   \RiskMS_{\Dpert^{n}}(\Q) \le \EE_{\h\sim\Q} \EE_{( (\x,\y),\Epert)\sim\Dpert^{n\!}}\ \max_{\epsilon\in\Epert}\Loss{\h, (\x,\y),\epsilon}.
\end{align*}
Finally, from the definition of $\theta_i^h$, and from Lemma~4 of \citet{OhnishiHonorio2021}, we have
\begin{align*}
    \EE_{\h\sim\Q}\frac{1}{\m}\sum_{i=1}^{\m}\max_{\epsilon\in\Epert_i}\Loss{\h{,}(\x_i{,}\y_i){,}\epsilon}
    &=  \EE_{\h\sim\Q}\frac{1}{\m}\sum_{i=1}^{\m}\EE_{\epsilon\sim\theta_i^{\h}}\Loss{\h{,}(\x_i{,}\y_i){,}\epsilon}\\
    &\le \EE_{\h\sim\Q}\frac{1}{\m}\sum_{i=1}^{\m}\TV(\theta_i^\h\|\Theta_i) + \EE_{\h\sim\Q}\frac{1}{\m}\sum_{i=1}^{\m}\EE_{\epsilon\sim\Theta_i}\Loss{\h{,}(\x_i{,}\y_i){,}\epsilon}\\
    &= \EE_{\h\sim\Q}\frac{1}{\m}\sum_{i=1}^{\m}\TV(\theta_i^\h\|\Theta_i) + \frac{1}{\m}\sum_{i=1}^{\m}\EE_{\epsilon\sim\Theta_i}\EE_{\h\sim\Q}\Loss{\h{,}(\x_i{,}\y_i){,}\epsilon}\\
    &\le \EE_{\h\sim\Q}\frac{1}{\m}\sum_{i=1}^{\m}\TV(\theta_i^\h\|\Theta_i) + \RiskMS_{\Spert}(\Q).
\end{align*}
\end{proof}

\section{Proof of \Cref{chap:mv-robustness:corollary:seeger-chromatic-T,chap:mv-robustness:corollary:max-mcallester-T}}
\label{ap:mv-robustness:sec:proof-bound}

We start to prove \Cref{chap:mv-robustness:corollary:seeger-chromatic-T}.

\corollaryseegerchromaticT*
\begin{proof}

Let $\Dpert_1,\dots, \Dpert_\iter$ be $\iter$ distributions defined as $\Dpert_1= \D(\x,\y)\DX^1(\epsilon),$ $\dots,$ $\Dpert_\iter=\D(\x,\y)\DX^\iter(\epsilon)$ on $(\X{\times}\Y){\times}\Xpert$ where each distribution $\DX^\t$ depends on the example $(\x,\y)$ and possibly on the fixed prior $\Pt$.
Then, for all distributions $\Dpert_\t$, we can derive a bound on the risk $\RiskS_{\Dpert_\t}(\Q)$ which holds with probability at least $1-\frac{\delta}{\iter}$, we have
\begin{align*}
    \Pr_{\Spert_t\sim(\Dpert_t^n)^\m}\!\!&\LB\forall\Q,\  \kl(\RiskS_{\Spert_\t}(\Q)\|\RiskS_{\Dpert_\t}(\Q)) \!\le\! \frac{1}{\m}\!\LB\KL(\Q\|\Pt){+}\ln\frac{\iter(\m{+}1)}{\delta} \RB \RB {\ge} 1{-}\tfrac{\delta}{\iter}.
\end{align*}
Then, from a union bound argument, we have 
\begin{align*}
    \Pr_{\Spert_1\sim(\Dpert_1^n)^\m,\dots, \Spert_T\sim(\Dpert_T^n)^\m}&\Bigg[\forall\Q,\  \kl(\RiskS_{\Spert_1}(\Q)\|\RiskS_{\Dpert_1}(\Q)) \!\le\! \frac{1}{\m}\!\LB\KL(\Q\|\P_t){+}\ln\frac{\iter(\m{+}1)}{\delta} \RB,\\
    &\hspace{-2cm}\dots, \text{ and } \kl(\RiskS_{\Spert_T}(\Q)\|\RiskS_{\Dpert_T}(\Q)) \!\le\! \frac{1}{\m}\!\LB\KL(\Q\|\P_T){+}\ln\frac{\iter(\m{+}1)}{\delta} \RB
    \Bigg]{\ge}1{-}\delta.  
\end{align*}
Hence, we have 
\begin{align*}
    \kl\LP\RiskS_{\Spert}(\Q) \| \RiskS_{\Dpert}(\Q)\RP \le \frac{1}{\m}\!\LB\KL(\Q\|\P){+}\ln\frac{\iter(\m{+}1)}{\delta}\RB,
\end{align*}
where $\DX$ can be dependent on the selected prior $\P$.
From \Cref{chap:pac-bayes:def:invert-kl}, we can obtain the claimed result.
\end{proof}

We can prove \Cref{chap:mv-robustness:corollary:max-mcallester-T} similarly to \Cref{chap:mv-robustness:corollary:seeger-chromatic-T}.

\corollarymaxmcallesterT*
\begin{proof}
From a union bound argument, we obtain the claimed result.
\end{proof}

\section{About the (Differentiable) Decision Trees}
\label{ap:mv-robustness:section:tree}
In this section, we introduce the differentiable decision trees, \ie, the voters of our majority vote.
Note that we adapt the model of \citet{KontschiederFiterauCriminisiBulo2016} in order to fit with our framework: a voter must output a real between $-1$ and $+1$.
An example of such a tree is represented in \Cref{ap:mv-robustness:fig:tree}.

\begin{figure}[h!]
\begin{center}
\includestandalone[width=0.5\textwidth]{chapter_3/figures/tree}
\end{center}
\caption{Representation of a (differentiable) decision tree of depth $l=2$; The root is the node $0$ and the leafs are $4$; $5$; $6$ and $7$. The probability $p_i(\x)$ (respectively $1{-}p_i(\x)$) to go left (respectively right) at the node $i$ is represented by $p_i$ (we omitted the dependence on $\x$ for simplicity). Similarly, the predicted label (a ``score'' between $-1$ and $+1$) at the leaf $i$ is represented by $s_i$.}
\label{ap:mv-robustness:fig:tree}
\end{figure}

This differentiable decision tree is stochastic by nature: at each node $i$ of the tree, we continue recursively to the left sub-tree with a probability of $p_i(\x)$ and to the right sub-tree with a probability of $1{-}p_i(\x)$; When we attain a leaf $j$, the tree predicts the label $s_j$.
Precisely, the probability $p_i(\x)$ is constructed by {\it (i)} selecting randomly 50\% of the input features $\x$ and applying a random mask $M_i\in\R^d$ on $\x$ (where the $k$-th entry of the mask is $1$ if the $k$-th feature is selected and $0$ otherwise), by {\it (ii)} multiplying this quantity by a learned weight vector $v_i\in\R^d$, and by {\it (iii)} applying a sigmoid function to output a probability. 
Indeed, we have
\begin{align*}
p_i(\x) = \sigma\Big(\langle v_i, M_i{\odot}x\rangle\Big),
\end{align*}
where $\sigma(a)=\LB1+e^{-a}\RB^{-1}$ is the sigmoid function; $\langle a, b \rangle$ is the dot product between the vector $a$ and $b$ and $a \odot b$ is the elementwise product between the vector $a$ and $b$.
Moreover, $s_i$ is obtained by learning a parameter $u_i\in\R$ and applying a $\tanh$ function, \ie, we have 
\begin{align*}
     s_i = \tanh\!\Big( u_i\Big).
\end{align*}
Finally, instead of having a stochastic voter, $h$ will output the expected label predicted by the tree (see \citet{KontschiederFiterauCriminisiBulo2016} for more details). 
It can be computed by $\h(\x) = f(\x, 0, 0)$ with
\begin{align*}
f(\x, i, l') =\left\{\begin{array}{cc}
    s_i  &  \text{if } l'=l\\
    p_i(\x)f(\x, 2i{+}1, l'{+}1)+ (1-p_i(\x))f(\x, 2i{+}2, l'{+}1) &  \text{otherwise}
\end{array}\right..
\end{align*}

\section{Additional Experimental Results}
\label{ap:mv-robustness:sec:additional-results}
In this section, we present the detailed results for the 6 tasks (3 on MNIST and 3 on Fashion MNIST) on which we perform experiments that show the test risks and the bounds for the different scenarios of (Defense, Attack).
We train all the models using the same parameters as described in \Cref{chap:mv-robustness:sec:expe-desc}.
\Cref{ap:mv-robustness:tab:mnist-l2-details} and \Cref{ap:mv-robustness:tab:fashion-l2-details} complement \Cref{chap:mv-robustness:tab:mnist-1-7-details} to present the results for all the tasks when using the $\ell_2$-norm with $b=1$ (the maximum noise allowed by the norm).
Then, we run again the same experiment but we use the $\ell_\infty$-norm with $b=0.1$ and exhibit the results in \Cref{ap:mv-robustness:tab:mnist-infty-details} and \Cref{ap:mv-robustness:tab:fashion-infty-details}.
For the experiments on the 5 other tasks using the $\ell_2$-norm, we have a similar behavior than MNIST:1vs7.
Indeed, using the attacks \PGDU~and \IFGSMU~as defense mechanism allows to obtain better risks and also tighter bounds compared to the bounds obtained with a defense based on \U (which is a naive defense).
For the experiments on the 6 tasks using the $\ell_\infty$-norm, the trend is the same as with the $\ell_2$-norm, \ie, the appropriate defense leads to better risks and bounds.

We also run experiments that do not rely on the PAC-Bayesian framework.
In other words, we train the models following only Step 1 of our adversarial training procedure (\ie, \Cref{chap:mv-robustness:algo-step}) using classical attacks (\PGD~or \IFGSM): we refer to this experiment as a baseline.
In our cases, it means learning a majority vote $\MVPp$ that follows a distribution $\Pp$.
As a reminder, the studied scenarios for the baseline are all the pairs $(\text{Defense}, \text{Attack})$ belonging to the set $\{\text{---},\text{\U}, \text{\PGD}, \text{\IFGSM} \}{\times}\{\text{---}, \text{\PGD}, \text{\IFGSM}\}$.
We report the results in \Cref{ap:mv-robustness:tab:l2-baseline} and \Cref{ap:mv-robustness:tab:infty-baseline}.
With this experiment, we are now able to compare our defense based on \PGDU or \IFGSMU and a classical defense based on \PGD and \IFGSM.
Hence, considering the test risks $\RiskA_{\dT}(\MVQ)$ (columns ``Attack without {\sc u}'' of Tables~\ref{chap:mv-robustness:tab:mnist-1-7-details} to~\ref{ap:mv-robustness:tab:fashion-infty-details}) and $\RiskA_{\dT}(\MVPp)$ (in Tables~\ref{ap:mv-robustness:tab:l2-baseline} and~\ref{ap:mv-robustness:tab:infty-baseline}) , we observe similar results between the baseline and our framework.

\begin{table*}
\caption{Test risks and bounds for 2 tasks of MNIST with $n{=}100$ perturbations for all pairs (Defense,Attack) with the two voters' set $\H$ and $\Hsigned$. 
The results in \textbf{bold} correspond to the best values between results for $\H$ and $\Hsigned$.
To quantify the gap between our risks and the classical definition we put in \textit{italic} the risk of our models against the classical attacks: we replace \PGDU and \IFGSMU~by \PGD or \IFGSM~(\ie, we did {\it not} sample from the uniform distribution). 
Since \cref{chap:mv-robustness:eq:max-mcallester} upperbounds \cref{chap:mv-robustness:eq:max-mcallester-0} thanks to the TV term, we compute the two bound values of \Cref{chap:mv-robustness:theorem:bound-average-max}.
}
\begin{subtable}{1.0\linewidth}
\resizebox{\textwidth}{!}{
\newcommand{\sep}{ & }
\begin{tabular}{ll||c|c||c|c|c|c||c|c||c|c|c|c|c|c}
\toprule
\multicolumn{2}{c||}{$\ell_2$-norm} &  \multicolumn{6}{c||}{\cref{chap:mv-robustness:algo-step}\ {\rm with \cref{chap:mv-robustness:eq:seeger-chromatic}}} & \multicolumn{8}{c}{\cref{chap:mv-robustness:algo-step}\ {\rm with \cref{chap:mv-robustness:eq:max-mcallester}}}\\
\multicolumn{2}{c||}{$b=1$} & \multicolumn{2}{c}{\scriptsize\rm Attack  without {\sc u}} &   \multicolumn{4}{c||}{ } & \multicolumn{2}{c}{\scriptsize\rm Attack  without {\sc u}} &\multicolumn{6}{c}{ }  \\
& & \multicolumn{2}{c||}{$\RiskA_{\dT}(\MVQ)$} &  \multicolumn{2}{c|}{$\Risk_{\Tpert}(\MVQ)$} & \multicolumn{2}{c||}{\cref{chap:mv-robustness:theorem:chromatic}} & \multicolumn{2}{c||}{$\RiskA_{\dT}(\MVQ)$} & \multicolumn{2}{c|}{$\RiskM_{\Tpert}(\MVQ)$} & \multicolumn{2}{c}{\cref{chap:mv-robustness:theorem:bound-average-max} - \cref{chap:mv-robustness:eq:max-mcallester}} & \multicolumn{2}{c}{\cref{chap:mv-robustness:theorem:bound-average-max} - \cref{chap:mv-robustness:eq:max-mcallester-0}}\\[1mm]
Defense & Attack & $\Hsigned$ & $\H$ & $\Hsigned$ & $\H$ & $\Hsigned$ & $\H$ & $\Hsigned$ & $\H$ & $\Hsigned$ & $\H$ & $\Hsigned$ & $\H$ & $\Hsigned$ & $\H$\\
\midrule
--- & ---         & \it .015 \sep \it.015               & .015     \sep .015     & \bf 0.060 \sep .067     & \textit{.015}    \sep \textit{.015}     & .015     \sep .015     & \bf 0.129  \sep  0.135 & \bf 0.129 \sep .135 \\
--- & \PGDU       & \it .632 \sep \bf\textit{.628}      & \bf .520 \sep .526     & 1.059    \sep \bf .847 & \textit{.672}    \sep \bf\textit{.641}  & \bf .683 \sep .684     & \bf 1.718 \sep 2.405 & 1.392 \sep \bf .962 \\
--- & \IFGSMU     & \it .447 \sep \bf\textit{.443}      & \bf .157 \sep .166     & \bf 0.387 \sep .572     & \textit{.461}    \sep \bf\textit{.451}  & \bf .337 \sep .345     & \bf 1.137 \sep 2.090 &  0.776 \sep \bf .669 \\
\U  & ---         & \it .024 \sep \it.024               & .024     \sep .024     & \bf 0.073 \sep .083     & \textit{.024}    \sep \textit{.024}     & .024     \sep .024     & \bf 0.140  \sep  0.148 & \bf 0.140 \sep .148 \\
\U  & \PGDU       & \it .646 \sep \bf\textit{.619}      & \bf .486 \sep .500     & 1.016    \sep \bf .809 & \textit{.649}    \sep \bf \textit{.626} & \bf .648 \sep .650     & \bf 1.646 \sep 2.417 & 1.338 \sep \bf .915 \\
\U  & \IFGSMU     & \it .442 \sep \it.442               & \bf .128 \sep .139     & \bf 0.316 \sep .528     & \textit{.442}    \sep \textit{.442}     & \bf .281 \sep .293     & \bf 0.907  \sep 2.118 &  0.633 \sep \bf .617 \\
\hdashline
\PGDU & ---       & \bf\textit{.024} \sep \textit{.025} & \bf .024 \sep .025     & \bf 0.094 \sep .101     & \bf\textit{.024} \sep \textit{.025}     & \bf .024 \sep .025     & \bf 0.158  \sep  0.163 & \bf 0.158 \sep .163 \\
\PGDU & \PGDU     & \textit{.148} \sep \bf\textit{.135} & .111     \sep \bf .103 & 0.360     \sep \bf .355 & \textit{.146}    \sep \bf\textit{.136}  & .129     \sep \bf .120 & \bf 0.442  \sep 2.062 &  0.414 \sep \bf .403 \\
\PGDU & \IFGSMU   & \textit{.104} \sep \bf\textit{.103}  & .072     \sep .072     & 0.277     \sep .277     & \textit{.102}    \sep \textit{.102}     & .090     \sep \bf .084 & \bf 0.358  \sep 1.954 &  0.335 \sep \bf .328 \\
\IFGSMU & ---     & \textit{.027} \sep \bf\textit{.025} & .027     \sep \bf .025 & \bf 0.080 \sep .091     & \textit{.027}    \sep \bf \textit{.025} & .027     \sep \bf .025 & \bf 0.146  \sep  0.154 & \bf 0.146 \sep .154 \\
\IFGSMU & \PGDU   & \textit{.188} \sep \bf\textit{.178} & \bf .111 \sep .119     & \bf 0.383 \sep .405     & \textit{.190}    \sep \bf \textit{.178} & \bf .126 \sep .134     & \bf 0.501  \sep 2.063 &  0.454 \sep .454 \\
\IFGSMU & \IFGSMU & \textit{.126} \sep \bf\textit{.115} & .076     \sep \bf .070 & \bf 0.248 \sep .290     & \textit{.127}    \sep \bf \textit{.115} & .091     \sep \bf .085 & \bf 0.371  \sep 1.918 & \bf 0.329 \sep .342 \\
\bottomrule
\end{tabular}
}
\caption{MNIST 4vs9}
\end{subtable}

\begin{subtable}{1.0\linewidth}
\resizebox{\textwidth}{!}{
\newcommand{\sep}{ & }
\begin{tabular}{ll||c|c||c|c|c|c||c|c||c|c|c|c|c|c}
\toprule
\multicolumn{2}{c||}{$\ell_2$-norm} &  \multicolumn{6}{c||}{\cref{chap:mv-robustness:algo-step}\ {\rm with \cref{chap:mv-robustness:eq:seeger-chromatic}}} & \multicolumn{8}{c}{\cref{chap:mv-robustness:algo-step}\ {\rm with \cref{chap:mv-robustness:eq:max-mcallester}}}\\
\multicolumn{2}{c||}{$b=1$} & \multicolumn{2}{c}{\scriptsize\rm Attack  without {\sc u}} &   \multicolumn{4}{c||}{ } & \multicolumn{2}{c}{\scriptsize\rm Attack  without {\sc u}} &\multicolumn{6}{c}{ }  \\
& & \multicolumn{2}{c||}{$\RiskA_{\dT}(\MVQ)$} &  \multicolumn{2}{c|}{$\Risk_{\Tpert}(\MVQ)$} & \multicolumn{2}{c||}{\cref{chap:mv-robustness:theorem:chromatic}} & \multicolumn{2}{c||}{$\RiskA_{\dT}(\MVQ)$} & \multicolumn{2}{c|}{$\RiskM_{\Tpert}(\MVQ)$} & \multicolumn{2}{c}{\cref{chap:mv-robustness:theorem:bound-average-max} - \cref{chap:mv-robustness:eq:max-mcallester}} & \multicolumn{2}{c}{\cref{chap:mv-robustness:theorem:bound-average-max} - \cref{chap:mv-robustness:eq:max-mcallester-0}}\\[1mm]
Defense & Attack & $\Hsigned$ & $\H$ & $\Hsigned$ & $\H$ & $\Hsigned$ & $\H$ & $\Hsigned$ & $\H$ & $\Hsigned$ & $\H$ & $\Hsigned$ & $\H$ & $\Hsigned$ & $\H$\\
\midrule
--- & ---         & \textit{.015}    \sep \textit{.015}     &     .015 \sep     .015 & \bf .043 \sep     .045 &    \textit{.015} \sep     \textit{.015} &     .015 \sep     .015 & \bf .117 \sep      0.118 & \bf .117 \sep .118  \\
--- & \PGDU       & \textit{.279}    \sep \bf \textit{.271} & \bf .232 \sep     .234 &     .600 \sep \bf .453 &    \textit{.284} \sep \bf \textit{.274} &     .284 \sep     .284 & \bf .829 \sep     1.929 & .724 \sep \bf .524  \\
--- & \IFGSMU     & \textit{.143}    \sep \bf \textit{.137} & \bf .089 \sep     .090 & \bf .204 \sep     .227 &    \textit{.144} \sep \bf \textit{.139} & \bf .125 \sep     .127 & \bf .422 \sep     1.662 & .337 \sep \bf .293  \\
\U  & ---         & \textit{.017}    \sep \textit{.017}     &     .017 \sep     .017 & \bf .054 \sep     .055 & \textit{.017} \sep \textit{.017} &     .017 \sep     .017 & \bf .124 \sep      0.125 & \bf .124 \sep .125  \\
\U  & \PGDU       & \textit{.219}    \sep \bf \textit{.201} & \bf .172 \sep     .177 &     .433 \sep \bf .350 &    \textit{.219} \sep \bf \textit{.209} & \bf .217 \sep     .218 & \bf .671 \sep     1.810 & .565 \sep \bf .419  \\
\U  & \IFGSMU     & \textit{.122} \sep \textit{.122}     & \bf .052 \sep     .055 & \bf .119 \sep     .181 & \bf\textit{.122} \sep     \textit{.123} & \bf .077 \sep     .082 & \bf .307 \sep     1.554 & \bf .242 \sep .248  \\
\hdashline
\PGDU & ---       & \bf\textit{.013} \sep \textit{.015}     & \bf .013 \sep     .015 &     .061 \sep     .061 & \bf\textit{.013} \sep     \textit{.015} & \bf .013 \sep     .015 & \bf .131 \sep      0.130 & .131 \sep \bf .130  \\
\PGDU & \PGDU     & \textit{.057}    \sep \textit{.057}     &     .045 \sep \bf .041 & \bf .157 \sep     .160 &    \textit{.057} \sep \textit{.057} &     .055 \sep \bf .045 & \bf .227 \sep     1.536 & .218 \sep .218  \\
\PGDU & \IFGSMU   & \textit{.043} \sep \textit{.043}     & \bf .027 \sep     .031 & \bf .114 \sep     .119 & \bf\textit{.042} \sep     \textit{.043} &     .037 \sep \bf .035 & \bf .187 \sep     1.433 & \bf .179 \sep .181  \\
\IFGSMU & ---     & \textit{.014}    \sep \bf\textit{.012}  &     .014 \sep \bf .012 &     .057 \sep     .057 &    \textit{.014} \sep \bf \textit{.013} &     .014 \sep \bf .013 &     .128 \sep \bf  0.127 & .128 \sep \bf .127  \\
\IFGSMU & \PGDU   & \textit{.077}    \sep \bf\textit{.072}  &     .054 \sep \bf .043 & \bf .170 \sep     .174 &    \textit{.076} \sep \bf \textit{.075} &     .055 \sep \bf .052 & \bf .252 \sep     1.510 & \bf .233 \sep .236  \\
\IFGSMU & \IFGSMU & \textit{.055} \sep \bf\textit{.048}     &     .034 \sep \bf .030 & \bf .105 \sep     .121 &    \textit{.052} \sep \bf \textit{.051} &     .039 \sep \bf .032 & \bf .191 \sep     1.379 & \bf .177 \sep .185  \\
\bottomrule
\end{tabular}
}
\caption{MNIST 5vs6}
\end{subtable}
\label{ap:mv-robustness:tab:mnist-l2-details}
\end{table*}
\begin{table*}
\caption{Test risks and bounds for 3 tasks Fashion MNIST with $n{=}100$ perturbations for all pairs (Defense,Attack) with the two voters' set $\H$ and $\Hsigned$. 
The results in \textbf{bold} correspond to the best values between results for $\H$ and $\Hsigned$.
To quantify the gap between our risks and the classical definition we put in \textit{italic} the risk of our models against the classical attacks: we replace \PGDU and \IFGSMU~by \PGD or \IFGSM~(\ie, we did {\it not} sample from the uniform distribution). 
Since \cref{chap:mv-robustness:eq:max-mcallester} upperbounds \cref{chap:mv-robustness:eq:max-mcallester-0} thanks to the TV term, we compute the two bound values of \Cref{chap:mv-robustness:theorem:bound-average-max}.
}
\begin{subtable}{0.97\linewidth}
\resizebox{\textwidth}{!}{
\newcommand{\sep}{ & }
\begin{tabular}{ll||c|c||c|c|c|c||c|c||c|c|c|c|c|c}
\toprule
\multicolumn{2}{c||}{$\ell_2$-norm} &  \multicolumn{6}{c||}{\cref{chap:mv-robustness:algo-step}\ {\rm with \cref{chap:mv-robustness:eq:seeger-chromatic}}} & \multicolumn{8}{c}{\cref{chap:mv-robustness:algo-step}\ {\rm with \cref{chap:mv-robustness:eq:max-mcallester}}}\\
\multicolumn{2}{c||}{$b=1$} & \multicolumn{2}{c}{\scriptsize\rm Attack  without {\sc u}} &   \multicolumn{4}{c||}{ } & \multicolumn{2}{c}{\scriptsize\rm Attack  without {\sc u}} &\multicolumn{6}{c}{ }  \\
& & \multicolumn{2}{c||}{$\RiskA_{\dT}(\MVQ)$} &  \multicolumn{2}{c|}{$\Risk_{\Tpert}(\MVQ)$} & \multicolumn{2}{c||}{\cref{chap:mv-robustness:theorem:chromatic}} & \multicolumn{2}{c||}{$\RiskA_{\dT}(\MVQ)$} & \multicolumn{2}{c|}{$\RiskM_{\Tpert}(\MVQ)$} & \multicolumn{2}{c}{\cref{chap:mv-robustness:theorem:bound-average-max} - \cref{chap:mv-robustness:eq:max-mcallester}} & \multicolumn{2}{c}{\cref{chap:mv-robustness:theorem:bound-average-max} - \cref{chap:mv-robustness:eq:max-mcallester-0}}\\[1mm]
Defense & Attack & $\Hsigned$ & $\H$ & $\Hsigned$ & $\H$ & $\Hsigned$ & $\H$ & $\Hsigned$ & $\H$ & $\Hsigned$ & $\H$ & $\Hsigned$ & $\H$ & $\Hsigned$ & $\H$\\
\midrule
--- & ---         &    \textit{.021} \sep \bf\textit{.020} &     .021 \sep \bf .020 & \bf  0.060 \sep      0.070 & \textit{.019} \sep \textit{.019} & \bf .019 \sep \bf .019 & \bf  0.130 \sep  0.139 & \bf 0.130 \sep  0.139  \\
--- & \PGDU       &    \textit{.695} \sep \bf\textit{.650} & \bf .494 \sep     .568 & \bf 1.042 \sep     1.090 &    \bf\textit{.677} \sep \textit{.686} & \bf .588 \sep     .674 & \bf 1.326 \sep 2.307 & 1.152 \sep \bf 1.082  \\
--- & \IFGSMU     &    \textit{.451} \sep \textit{.451} & \bf .269 \sep     .328 & \bf  0.585 \sep      0.731 & \bf\textit{.405} \sep    \textit{.438} & \bf .295 \sep     .381 & \bf  0.878 \sep 1.971 & \bf 0.730 \sep  0.746  \\
\U  & ---         &    \textit{.071} \sep    \textit{.071} &     .071 \sep     .071 & \bf  0.185 \sep      0.191 &    \textit{.071} \sep    \textit{.071} &     .071 \sep     .071 & \bf  0.236 \sep  0.241 & \bf 0.236 \sep  0.241  \\
\U  & \PGDU       & \bf\textit{.423} \sep    \textit{.477} & \bf .418 \sep     .425 &      0.957 \sep \bf  0.755 & \textit{.486} \sep    \textit{.486} &     .513 \sep     .513 & \bf 1.372 \sep 2.173 & 1.151 \sep \bf 0.869  \\
\U  & \IFGSMU     & \bf\textit{.326} \sep \textit{.331} &     .105 \sep     .105 & \bf  0.273 \sep      0.422 &    \textit{.333} \sep \bf\textit{.331} &     .144 \sep \bf .142 & \bf  0.496 \sep 1.642 & \bf 0.397 \sep  0.504  \\
\hdashline
\PGDU & ---       &    \textit{.034} \sep \bf\textit{.032} &     .034 \sep \bf .032 & \bf  0.094 \sep      0.114 &    \textit{.034} \sep \bf\textit{.032} &     .034 \sep \bf .032 & \bf  0.158 \sep  0.174 & \bf 0.158 \sep  0.174  \\
\PGDU & \PGDU     & \bf\textit{.103} \sep    \textit{.115} & \bf .086 \sep     .091 & \bf  0.227 \sep      0.289 & \bf\textit{.102} \sep    \textit{.115} & \bf .096 \sep     .101 & \bf  0.299 \sep 1.985 & \bf 0.283 \sep  0.338  \\
\PGDU & \IFGSMU   & \bf\textit{.092} \sep    \textit{.099} & \bf .073 \sep     .076 & \bf  0.195 \sep      0.248 & \bf\textit{.092} \sep    \textit{.099} &     .082 \sep     .082 & \bf  0.266 \sep 1.914 & \bf 0.253 \sep  0.299  \\
\IFGSMU & ---     & \bf\textit{.028} \sep    \textit{.030} & \bf .028 \sep     .030 & \bf  0.091 \sep      0.105 & \bf\textit{.027} \sep    \textit{.030} & \bf .027 \sep     .030 & \bf  0.155 \sep  0.166 & \bf 0.155 \sep  0.166  \\
\IFGSMU & \PGDU   &    \textit{.115} \sep \bf\textit{.114} &     .085 \sep     .085 & \bf  0.254 \sep      0.287 & \bf\textit{.112} \sep    \textit{.114} & \bf .096 \sep     .101 & \bf  0.331 \sep 2.026 & \bf 0.313 \sep  0.337  \\
\IFGSMU & \IFGSMU & \bf\textit{.095} \sep    \textit{.097} & \bf .067 \sep     .068 & \bf  0.206 \sep      0.232 & \bf\textit{.093} \sep    \textit{.097} & \bf .080 \sep     .081 & \bf  0.282 \sep 1.927 & \bf 0.266 \sep  0.285  \\
\bottomrule
\end{tabular}
}
\caption{Fashion MNIST Sandall vs Ankle Boot}
\end{subtable}

\begin{subtable}{0.97\linewidth}
\resizebox{\textwidth}{!}{
\newcommand{\sep}{ & }
\begin{tabular}{ll||c|c||c|c|c|c||c|c||c|c|c|c|c|c}
\toprule
\multicolumn{2}{c||}{$\ell_2$-norm} &  \multicolumn{6}{c||}{\cref{chap:mv-robustness:algo-step}\ {\rm with \cref{chap:mv-robustness:eq:seeger-chromatic}}} & \multicolumn{8}{c}{\cref{chap:mv-robustness:algo-step}\ {\rm with \cref{chap:mv-robustness:eq:max-mcallester}}}\\
\multicolumn{2}{c||}{$b=1$} & \multicolumn{2}{c}{\scriptsize\rm Attack  without {\sc u}} &   \multicolumn{4}{c||}{ } & \multicolumn{2}{c}{\scriptsize\rm Attack  without {\sc u}} &\multicolumn{6}{c}{ }  \\
& & \multicolumn{2}{c||}{$\RiskA_{\dT}(\MVQ)$} &  \multicolumn{2}{c|}{$\Risk_{\Tpert}(\MVQ)$} & \multicolumn{2}{c||}{\cref{chap:mv-robustness:theorem:chromatic}} & \multicolumn{2}{c||}{$\RiskA_{\dT}(\MVQ)$} & \multicolumn{2}{c|}{$\RiskM_{\Tpert}(\MVQ)$} & \multicolumn{2}{c}{\cref{chap:mv-robustness:theorem:bound-average-max} - \cref{chap:mv-robustness:eq:max-mcallester}} & \multicolumn{2}{c}{\cref{chap:mv-robustness:theorem:bound-average-max} - \cref{chap:mv-robustness:eq:max-mcallester-0}}\\[1mm]
Defense & Attack & $\Hsigned$ & $\H$ & $\Hsigned$ & $\H$ & $\Hsigned$ & $\H$ & $\Hsigned$ & $\H$ & $\Hsigned$ & $\H$ & $\Hsigned$ & $\H$ & $\Hsigned$ & $\H$\\
\midrule
--- & ---         &    \textit{.038} \sep \bf\textit{.037} &     .038 \sep \bf .037 & \bf .088 \sep     .091 &    \textit{.038} \sep \bf\textit{.037} &     .038 \sep \bf .037 & \bf .153 \sep      0.155 & \bf  .153 \sep .155  \\
--- & \PGDU       &    \textit{.292} \sep \bf\textit{.248} &     .233 \sep \bf .112 &     .452 \sep \bf .363 &    \textit{.289} \sep \bf\textit{.272} &     .287 \sep \bf .246 & \bf .578 \sep     1.314 & .525 \sep \bf .479  \\
--- & \IFGSMU     &    \textit{.194} \sep \bf\textit{.154} &     .132 \sep \bf .075 &     .300 \sep \bf .262 &    \textit{.193} \sep \bf\textit{.181} &     .176 \sep \bf .148 & \bf .423 \sep     1.103 & .376 \sep \bf .359  \\
\U  & ---         &    \textit{.039} \sep    \textit{.039} & \bf .039 \sep \bf .039 & \bf .091 \sep     .093 &    \textit{.041} \sep \bf\textit{.039} &     .041 \sep \bf .039 & \bf .155 \sep      0.157 & \bf .155 \sep .157  \\
\U  & \PGDU       &    \textit{.240} \sep \bf\textit{.220} & \bf .099 \sep     .117 &     .346 \sep \bf .332 &    \textit{.250} \sep \bf\textit{.231} &     .250 \sep \bf .245 & \bf .553 \sep     1.228 & .490 \sep \bf .443  \\
\U  & \IFGSMU     &    \textit{.177} \sep \bf\textit{.171} & \bf .070 \sep     .078 & \bf .228 \sep     .247 &    \textit{.197} \sep \bf\textit{.185} &     .186 \sep \bf .164 & \bf .445 \sep     1.046 & .371 \sep \bf .346  \\
\hdashline
\PGDU & ---       &    \textit{.045} \sep \bf\textit{.044} &     .045 \sep \bf .044 &     .108 \sep \bf .105 &    \textit{.046} \sep \bf\textit{.045} &     .046 \sep \bf .045 &     .172 \sep \bf  0.167 & .172 \sep \bf .167  \\
\PGDU & \PGDU     &    \textit{.108} \sep \bf\textit{.100} & \bf .077 \sep     .082 & \bf .203 \sep     .211 &    \textit{.104} \sep \bf\textit{.100} & \bf .081 \sep     .087 & \bf .279 \sep     1.118 & .269 \sep \bf .264  \\
\PGDU & \IFGSMU   &    \textit{.094} \sep \bf\textit{.086} &     .071 \sep \bf .069 & \bf .184 \sep     .186 &    \textit{.090} \sep \bf\textit{.086} &     .076 \sep \bf .073 & \bf .257 \sep     1.015 & .248 \sep \bf .241  \\
\IFGSMU & ---     & \bf\textit{.041} \sep    \textit{.043} & \bf .041 \sep     .043 & \bf .094 \sep     .101 & \bf\textit{.039} \sep    \textit{.042} & \bf .039 \sep     .042 & \bf .158 \sep      0.163 & \bf .158 \sep .163  \\
\IFGSMU & \PGDU   & \bf\textit{.106} \sep    \textit{.114} & \bf .078 \sep     .092 & \bf .220 \sep     .226 & \bf\textit{.109} \sep    \textit{.113} & \bf .084 \sep     .095 & \bf .293 \sep     1.052 & .279 \sep \bf .275  \\
\IFGSMU & \IFGSMU & \bf\textit{.082} \sep    \textit{.087} & \bf .065 \sep     .072 & \bf .171 \sep     .176 & \bf\textit{.082} \sep    \textit{.089} & \bf .068 \sep     .078 & \bf .247 \sep      0.927 & .234 \sep \bf .232  \\
\bottomrule
\end{tabular}
}
\caption{Fashion MNIST Top vs Pullover}
\end{subtable}
\end{table*}

\begin{table*}
\begin{subtable}{0.97\linewidth}
\resizebox{\textwidth}{!}{
\newcommand{\sep}{ & }
\begin{tabular}{ll||c|c||c|c|c|c||c|c||c|c|c|c|c|c}
\toprule
\multicolumn{2}{c||}{$\ell_2$-norm} &  \multicolumn{6}{c||}{\cref{chap:mv-robustness:algo-step}\ {\rm with \cref{chap:mv-robustness:eq:seeger-chromatic}}} & \multicolumn{8}{c}{\cref{chap:mv-robustness:algo-step}\ {\rm with \cref{chap:mv-robustness:eq:max-mcallester}}}\\
\multicolumn{2}{c||}{$b=1$} & \multicolumn{2}{c}{\scriptsize\rm Attack  without {\sc u}} &   \multicolumn{4}{c||}{ } & \multicolumn{2}{c}{\scriptsize\rm Attack  without {\sc u}} &\multicolumn{6}{c}{ }  \\
& & \multicolumn{2}{c||}{$\RiskA_{\dT}(\MVQ)$} &  \multicolumn{2}{c|}{$\Risk_{\Tpert}(\MVQ)$} & \multicolumn{2}{c||}{\cref{chap:mv-robustness:theorem:chromatic}} & \multicolumn{2}{c||}{$\RiskA_{\dT}(\MVQ)$} & \multicolumn{2}{c|}{$\RiskM_{\Tpert}(\MVQ)$} & \multicolumn{2}{c}{\cref{chap:mv-robustness:theorem:bound-average-max} - \cref{chap:mv-robustness:eq:max-mcallester}} & \multicolumn{2}{c}{\cref{chap:mv-robustness:theorem:bound-average-max} - \cref{chap:mv-robustness:eq:max-mcallester-0}}\\[1mm]
Defense & Attack & $\Hsigned$ & $\H$ & $\Hsigned$ & $\H$ & $\Hsigned$ & $\H$ & $\Hsigned$ & $\H$ & $\Hsigned$ & $\H$ & $\Hsigned$ & $\H$ & $\Hsigned$ & $\H$\\
\midrule
--- & ---         &    \textit{.122} \sep    \textit{.122} &     .122 \sep     .122 & \bf  0.276 \sep      0.286 &    \textit{.122} \sep    \textit{.122} &     .122 \sep     .122 & \bf  0.318 \sep      0.328 & \bf 0.318 \sep   0.328  \\
--- & \PGDU       &    \textit{.744} \sep \bf\textit{.738} & \bf .674 \sep     .689 &     1.386 \sep \bf 1.066 &    \textit{.745} \sep \bf\textit{.740} & \bf .767 \sep     .768 & \bf 1.773 \sep     2.386 & 1.576 \sep  \bf 1.180  \\
--- & \IFGSMU     &    \textit{.652} \sep \bf\textit{.646} & \bf .454 \sep     .474 &      0.947 \sep \bf  0.887 &    \textit{.659} \sep \bf\textit{.648} & \bf .618 \sep     .632 & \bf 1.597 \sep     2.214 & 1.276 \sep  \bf 0.992  \\
\U  & ---         &    \textit{.204} \sep    \textit{.204} &     .204 \sep     .204 &     0.444  \sep      0.444 &    \textit{.204} \sep    \textit{.204} &     .204 \sep     .204 & \bf  0.475 \sep      0.476 & \bf 0.475 \sep   0.476  \\
\U  & \PGDU       &    \textit{.750} \sep \bf\textit{.714} &     .682 \sep \bf .671 &     1.350 \sep \bf 1.069 &    \textit{.750} \sep \bf\textit{.719} &     .752 \sep \bf .749 & \bf 1.732 \sep     2.063 & 1.524 \sep  \bf 1.189  \\
\U  & \IFGSMU     &    \textit{.605} \sep \bf\textit{.575} & \bf .423 \sep     .431 &      0.871 \sep \bf  0.866 &    \textit{.605} \sep \bf\textit{.578} &     .530 \sep \bf .526 & \bf 1.304 \sep     1.860 & 1.091 \sep  \bf 0.956 \\
\hdashline
\PGDU & ---       &    \textit{.168} \sep \bf\textit{.165} &     .168 \sep \bf .165 & \bf  0.423 \sep      0.428 &    \textit{.167} \sep \bf\textit{.165} &     .167 \sep \bf .165 &      0.463 \sep \bf  0.461 &  0.463 \sep  \bf 0.460  \\
\PGDU & \PGDU     & \bf\textit{.389} \sep    \textit{.402} & \bf .306 \sep     .369 &      0.768 \sep \bf  0.719 & \bf\textit{.390} \sep    \textit{.402} & \bf .319 \sep     .403 & \bf  0.847 \sep     2.354 &  0.810 \sep  \bf 0.755  \\
\PGDU & \IFGSMU   & \bf\textit{.361} \sep    \textit{.368} & \bf .298 \sep     .324 &      0.693 \sep \bf  0.672 & \bf\textit{.362} \sep    \textit{.368} & \bf .320 \sep     .361 & \bf  0.799 \sep     2.258 &  0.754 \sep  \bf 0.707  \\
\IFGSMU & ---     & \bf\textit{.150} \sep    \textit{.163} & \bf .150 \sep     .163 & \bf  0.424 \sep      0.428 & \bf\textit{.149} \sep    \textit{.163} & \bf .149 \sep     .163 & \bf  0.458 \sep      0.461 & \bf 0.458 \sep   0.461  \\
\IFGSMU & \PGDU   & \bf\textit{.391} \sep    \textit{.428} &     .347 \sep \bf .292 &      0.778 \sep \bf  0.757 & \bf\textit{.390} \sep    \textit{.426} &     .371 \sep \bf .298 & \bf  0.856 \sep     2.327 &  0.820 \sep  \bf 0.791  \\
\IFGSMU & \IFGSMU & \bf\textit{.356} \sep    \textit{.382} &     .291 \sep \bf .273 & \bf  0.685 \sep      0.689 & \bf\textit{.354} \sep    \textit{.382} &     .331 \sep \bf .278 & \bf  0.772 \sep     2.218 &  0.734 \sep  \bf 0.723  \\
\bottomrule
\end{tabular}
}
\caption{Fashion MNIST Coat vs Shirt}
\end{subtable}
\label{ap:mv-robustness:tab:fashion-l2-details}
\end{table*}
\begin{table*}
\caption{Test risks and bounds for 3 tasks of MNIST with $n{=}100$ perturbations for all pairs (Defense,Attack) with the two voters' set $\H$ and $\Hsigned$. 
The results in \textbf{bold} correspond to the best values between results for $\H$ and $\Hsigned$.
To quantify the gap between our risks and the classical definition we put in \textit{italic} the risk of our models against the classical attacks: we replace \PGDU and \IFGSMU~by \PGD or \IFGSM~(\ie, we did {\it not} sample from the uniform distribution). 
Since \cref{chap:mv-robustness:eq:max-mcallester} upperbounds \cref{chap:mv-robustness:eq:max-mcallester-0} thanks to the TV term, we compute the two bound values of \Cref{chap:mv-robustness:theorem:bound-average-max}.
}
\begin{subtable}{0.98\linewidth}
\resizebox{\textwidth}{!}{
\newcommand{\sep}{ & }
\begin{tabular}{ll||c|c||c|c|c|c||c|c||c|c|c|c|c|c}
\toprule
\multicolumn{2}{c||}{$\ell_\infty$-norm} &  \multicolumn{6}{c||}{\cref{chap:mv-robustness:algo-step}\ {\rm with \cref{chap:mv-robustness:eq:seeger-chromatic}}} & \multicolumn{8}{c}{\cref{chap:mv-robustness:algo-step}\ {\rm with \cref{chap:mv-robustness:eq:max-mcallester}}}\\
\multicolumn{2}{c||}{$b=0.1$} & \multicolumn{2}{c}{\scriptsize\rm Attack  without {\sc u}} &   \multicolumn{4}{c||}{ } & \multicolumn{2}{c}{\scriptsize\rm Attack  without {\sc u}} &\multicolumn{6}{c}{ }  \\
& & \multicolumn{2}{c||}{$\RiskA_{\dT}(\MVQ)$} &  \multicolumn{2}{c|}{$\Risk_{\Tpert}(\MVQ)$} & \multicolumn{2}{c||}{\cref{chap:mv-robustness:theorem:chromatic}} & \multicolumn{2}{c||}{$\RiskA_{\dT}(\MVQ)$} & \multicolumn{2}{c|}{$\RiskM_{\Tpert}(\MVQ)$} & \multicolumn{2}{c}{\cref{chap:mv-robustness:theorem:bound-average-max} - \cref{chap:mv-robustness:eq:max-mcallester}} & \multicolumn{2}{c}{\cref{chap:mv-robustness:theorem:bound-average-max} - \cref{chap:mv-robustness:eq:max-mcallester-0}}\\[1mm]
Defense & Attack & $\Hsigned$ & $\H$ & $\Hsigned$ & $\H$ & $\Hsigned$ & $\H$ & $\Hsigned$ & $\H$ & $\Hsigned$ & $\H$ & $\Hsigned$ & $\H$ & $\Hsigned$ & $\H$\\
\midrule
--- & ---         & \textit{.005}     \sep \textit{.005}     & .005     \sep .005 & \bf .017 \sep .019     & \textit{.005}     \sep \textit{.005}     & .005     \sep .005     & \bf 0.099  \sep  0.100 & \bf .099 \sep .100 \\
--- & \PGDU       & \textit{.454}     \sep \textit{.454} & \bf .375 \sep .384 & .770     \sep \bf .638 & \textit{.492}     \sep \bf \textit{.484} & .480     \sep \bf .476 & \bf 1.127 \sep 2.031 & .946 \sep \bf .716 \\
--- & \IFGSMU     & \textit{.428}     \sep \bf \textit{.423} & \bf .350 \sep .361 & .727     \sep \bf .610 & \textit{.474}     \sep \bf \textit{.465} & .448     \sep \bf .443 & \bf 1.061 \sep 2.008 & .886 \sep \bf .686 \\
\U  & ---         & \textit{.004}     \sep \textit{.004}     & .004     \sep .004 & \bf .018 \sep .019     & \textit{.004}     \sep \textit{.004}     & .004     \sep .004     &  \bf 0.099 \sep  0.100 & \bf .099 \sep .100 \\
\U  & \PGDU       & \bf \textit{.487} \sep \textit{.491}     & \bf .369 \sep .392 & .779     \sep \bf .667 & \textit{.512} \sep \bf\textit{.507}     & \bf .484 \sep .487     & \bf 1.179 \sep 2.083 & .972 \sep \bf .739 \\
\U  & \IFGSMU     & \bf\textit{.436}     \sep \textit{.442} & \bf .325 \sep .337 & .664     \sep \bf .598 & \textit{.466}     \sep \bf \textit{.459} & .417     \sep .417     & \bf 1.023 \sep 1.959 & .841 \sep \bf .671 \\
\hdashline
\PGDU & ---       & \textit{.006}     \sep \textit{.006}     & .006     \sep .006 & .024     \sep .024     & \bf \textit{.005} \sep \textit{.006}     & \bf .005 \sep .006     & 0.103      \sep 0.103 & .103 \sep .103 \\
\PGDU & \PGDU     & \bf \textit{.018} \sep \textit{.020}     & \bf .013 \sep .016 & \bf .046 \sep .050     & \bf \textit{.018} \sep \textit{.020}     & \bf .015 \sep .020     &  \bf 0.127 \sep 1.461 & \bf .122 \sep .123 \\
\PGDU & \IFGSMU   & \bf \textit{.020} \sep \textit{.021}     & \bf .012 \sep .016 & \bf .048 \sep .054     & \bf \textit{.019} \sep \textit{.021}     & \bf .015 \sep .020     &  \bf 0.130 \sep 1.455 & \bf .125 \sep .127 \\
\IFGSMU & ---     & \bf \textit{.006} \sep \textit{.007}     & \bf .006 \sep .007 & \bf .023 \sep .024     & \bf\textit{.006}     \sep \textit{.007} & \bf .006 \sep .007     &  \bf 0.102 \sep  0.103  & \bf .102 \sep .103 \\
\IFGSMU & \PGDU   & \bf \textit{.018} \sep \textit{.019}     & .016     \sep .016 & \bf .046 \sep .051     & \bf \textit{.018} \sep \textit{.019}     & \bf .018 \sep .019     &  \bf 0.126 \sep 1.489  & \bf .122 \sep .124 \\
\IFGSMU & \IFGSMU & \textit{.020}     \sep \textit{.020}     & \bf .015 \sep .016 & \bf .050 \sep .055     & \textit{.020} \sep \textit{.020} & .020     \sep \bf .019 & \bf 0.131  \sep 1.481  & \bf .126 \sep .127 \\
\bottomrule
\end{tabular}
}
\caption{MNIST 1 vs 7}
\end{subtable}

\begin{subtable}{0.98\linewidth}
\resizebox{\textwidth}{!}{
\newcommand{\sep}{ & }
\begin{tabular}{ll||c|c||c|c|c|c||c|c||c|c|c|c|c|c}
\toprule
\multicolumn{2}{c||}{$\ell_\infty$-norm} &  \multicolumn{6}{c||}{\cref{chap:mv-robustness:algo-step}\ {\rm with \cref{chap:mv-robustness:eq:seeger-chromatic}}} & \multicolumn{8}{c}{\cref{chap:mv-robustness:algo-step}\ {\rm with \cref{chap:mv-robustness:eq:max-mcallester}}}\\
\multicolumn{2}{c||}{$b=0.1$} & \multicolumn{2}{c}{\scriptsize\rm Attack  without {\sc u}} &   \multicolumn{4}{c||}{ } & \multicolumn{2}{c}{\scriptsize\rm Attack  without {\sc u}} &\multicolumn{6}{c}{ }  \\
& & \multicolumn{2}{c||}{$\RiskA_{\dT}(\MVQ)$} &  \multicolumn{2}{c|}{$\Risk_{\Tpert}(\MVQ)$} & \multicolumn{2}{c||}{\cref{chap:mv-robustness:theorem:chromatic}} & \multicolumn{2}{c||}{$\RiskA_{\dT}(\MVQ)$} & \multicolumn{2}{c|}{$\RiskM_{\Tpert}(\MVQ)$} & \multicolumn{2}{c}{\cref{chap:mv-robustness:theorem:bound-average-max} - \cref{chap:mv-robustness:eq:max-mcallester}} & \multicolumn{2}{c}{\cref{chap:mv-robustness:theorem:bound-average-max} - \cref{chap:mv-robustness:eq:max-mcallester-0}}\\[1mm]
Defense & Attack & $\Hsigned$ & $\H$ & $\Hsigned$ & $\H$ & $\Hsigned$ & $\H$ & $\Hsigned$ & $\H$ & $\Hsigned$ & $\H$ & $\Hsigned$ & $\H$ & $\Hsigned$ & $\H$\\
\midrule
--- & ---         & \textit{.015}     \sep \textit{.015}     & .015     \sep .015     & \bf 0.060  \sep  0.067     & \textit{.015}     \sep \textit{.015}     & .015     \sep .015      & \bf 0.129  \sep  0.135 &  \bf 0.129 \sep  0.135 \\
--- & \PGDU       & \bf\textit{.929}     \sep \textit{.930} & \bf .651 \sep .662     & 1.367     \sep \bf 1.125 & \bf\textit{.920}     \sep \textit{.925} & \bf .874 \sep .880      & \bf 2.213 \sep 2.661 &  1.792 \sep \bf 1.266 \\
--- & \IFGSMU     & \textit{.935} \sep \textit{.935}     & \bf .601 \sep .609     & 1.243     \sep \bf 1.088 & \bf\textit{.926} \sep \textit{.928}     & \bf .800 \sep .806      & \bf 2.047 \sep 2.615 &  1.649 \sep \bf 1.224 \\
\U  & ---         & \textit{.017}     \sep \textit{.017}     & .017     \sep .017     &  \bf 0.062 \sep  0.072     & \textit{.017}     \sep \textit{.017}     & .017     \sep .017      & \bf 0.131  \sep  0.139 &  \bf 0.131 \sep  0.139 \\
\U  & \PGDU       & \textit{.895}     \sep \textit{.895}     & \bf .615 \sep .623     & 1.302     \sep \bf 1.078 & \bf\textit{.884}     \sep \textit{.888} & \bf .815 \sep .818      & \bf 2.035 \sep 2.722 &  1.670 \sep \bf 1.208 \\
\U  & \IFGSMU     & \textit{.898} \sep \textit{.898}     & \bf .516 \sep .528     & 1.112     \sep \bf 1.027 & \bf\textit{.884} \sep \textit{.890}     & \bf .697 \sep .706      & \bf 1.875 \sep 2.658 &  1.497 \sep \bf 1.153 \\
\hdashline
\PGDU & ---       & \textit{.039}    \sep \bf\textit{.037} & .039     \sep \bf .037 & \bf 0.093  \sep  0.094     & \textit{.039}     \sep \bf\textit{.037} & .039     \sep \bf .037  & \bf 0.156 \sep  0.157 &  \bf 0.156 \sep  0.157 \\
\PGDU & \PGDU     & \bf\textit{.108} \sep \textit{.109}     & .090     \sep .090     & \bf 0.200  \sep  0.209     & \bf \textit{.108} \sep \textit{.109}   & \bf .110 \sep .112      & \bf 0.337 \sep 1.874 &   0.290 \sep \bf 0.271 \\
\PGDU & \IFGSMU   & \bf\textit{.121} \sep \textit{.124}     & \bf .101 \sep .103     & \bf 0.229  \sep  0.235     & \bf \textit{.121} \sep \textit{.124}   & .126     \sep \bf .125  & \bf 0.378 \sep 1.890 &   0.326 \sep \bf 0.297 \\
\IFGSMU & ---     & \textit{.046}    \sep \bf\textit{.044} & .046     \sep \bf .044 & \bf 0.102  \sep  0.119     & \textit{.046}     \sep \bf\textit{.044} & .046     \sep \bf  .044 & \bf 0.164 \sep  0.178  & \bf 0.164 \sep  0.178 \\
\IFGSMU & \PGDU   & \textit{.105}    \sep \bf\textit{.093} & .091     \sep \bf .078 & \bf 0.203  \sep  0.214     & \textit{.105}     \sep \bf\textit{.093} & .108     \sep \bf .089  & \bf 0.321 \sep 1.810  &   0.286 \sep \bf 0.269 \\
\IFGSMU & \IFGSMU & \textit{.119}    \sep \bf\textit{.095} & .102     \sep \bf .080 & \bf 0.220  \sep  0.229     & \textit{.119}     \sep \bf\textit{.095} & .122     \sep \bf .090  & \bf 0.357 \sep 1.821  &   0.309 \sep \bf 0.283 \\
\bottomrule
\end{tabular}
}
\caption{MNIST 4 vs 9}
\end{subtable}
\end{table*}

\begin{table*}
\begin{subtable}{0.98\linewidth}
\resizebox{\textwidth}{!}{
\newcommand{\sep}{ & }
\begin{tabular}{ll||c|c||c|c|c|c||c|c||c|c|c|c|c|c}
\toprule
\multicolumn{2}{c||}{$\ell_\infty$-norm} &  \multicolumn{6}{c||}{\cref{chap:mv-robustness:algo-step}\ {\rm with \cref{chap:mv-robustness:eq:seeger-chromatic}}} & \multicolumn{8}{c}{\cref{chap:mv-robustness:algo-step}\ {\rm with \cref{chap:mv-robustness:eq:max-mcallester}}}\\
\multicolumn{2}{c||}{$b=0.1$} & \multicolumn{2}{c}{\scriptsize\rm Attack  without {\sc u}} &   \multicolumn{4}{c||}{ } & \multicolumn{2}{c}{\scriptsize\rm Attack  without {\sc u}} &\multicolumn{6}{c}{ }  \\
& & \multicolumn{2}{c||}{$\RiskA_{\dT}(\MVQ)$} &  \multicolumn{2}{c|}{$\Risk_{\Tpert}(\MVQ)$} & \multicolumn{2}{c||}{\cref{chap:mv-robustness:theorem:chromatic}} & \multicolumn{2}{c||}{$\RiskA_{\dT}(\MVQ)$} & \multicolumn{2}{c|}{$\RiskM_{\Tpert}(\MVQ)$} & \multicolumn{2}{c}{\cref{chap:mv-robustness:theorem:bound-average-max} - \cref{chap:mv-robustness:eq:max-mcallester}} & \multicolumn{2}{c}{\cref{chap:mv-robustness:theorem:bound-average-max} - \cref{chap:mv-robustness:eq:max-mcallester-0}}\\[1mm]
Defense & Attack & $\Hsigned$ & $\H$ & $\Hsigned$ & $\H$ & $\Hsigned$ & $\H$ & $\Hsigned$ & $\H$ & $\Hsigned$ & $\H$ & $\Hsigned$ & $\H$ & $\Hsigned$ & $\H$\\
\midrule
--- & ---         & \textit{.015}     \sep \textit{.015}     & .015 \sep .015 & \bf .043 \sep .045     & \textit{.015}     \sep \textit{.015} & .015 \sep .015 &  \bf 0.117 \sep  0.118 &  \bf 0.117  \sep .118 \\
--- & \PGDU       & \textit{.500}     \sep \bf \textit{.499} & \bf .387 \sep .390 & .923 \sep \bf .744 & \textit{.502}     \sep \bf \textit{.500} & \bf .474 \sep .475 & \bf 1.361 \sep 2.275 &  1.146  \sep \bf.830 \\
--- & \IFGSMU     & \textit{.519} \sep \bf\textit{.505}     & \bf .395 \sep .398 & .915 \sep \bf .762 & \bf \textit{.514} \sep \textit{.516} & .481 \sep .481 & \bf 1.335 \sep 2.283 &  1.129  \sep \bf .847 \\
\U  & ---         & \textit{.015}     \sep \textit{.015}     & .015 \sep .015 & \bf .052 \sep .053     & \textit{.015}     \sep \textit{.015} & .015 \sep .015 & \bf 0.123 \sep  0.124 &  \bf 0.123  \sep .124 \\
\U  & \PGDU       & \bf \textit{.529} \sep \textit{.544}     & \bf .388 \sep .393 & .925 \sep \bf .761 & \bf \textit{.517} \sep \textit{.532} & \bf .481 \sep .482 & \bf 1.342 \sep 2.349 &  1.137  \sep \bf .848 \\
\U  & \IFGSMU     & \bf \textit{.536} \sep \textit{.544}     & \bf .372 \sep .379 & .881 \sep \bf .774 & \bf \textit{.523} \sep \textit{.544} & \bf .451 \sep .456 & \bf 1.268 \sep 2.348 &  1.077 \sep \bf .857\\
\hdashline
\PGDU & ---       & \textit{.015}     \sep \bf \textit{.014} & .015 \sep \bf .014 & \bf .060 \sep .064 & \textit{.015}     \sep \bf \textit{.014} & .015 \sep \bf .014 & \bf 0.130 \sep  0.133 &  \bf 0.130 \sep .133 \\
\PGDU & \PGDU     & \bf\textit{.055}     \sep \textit{.058} & \bf .037 \sep .039 & \bf .131 \sep .143 & \bf \textit{.056} \sep \textit{.057}     & .046 \sep .046 &  \bf 0.219 \sep 1.619 &  \bf 0.202  \sep .204 \\
\PGDU & \IFGSMU   & \bf \textit{.061} \sep \textit{.065}     & \bf .040 \sep .043 & \bf .146 \sep .154 & \bf\textit{.059}     \sep \textit{.062} & .050 \sep \bf .046 & \bf 0.232 \sep 1.626 &   0.216  \sep \bf .214 \\
\IFGSMU & ---     & \textit{.019}     \sep \bf \textit{.014} & .019 \sep \bf .014 & .069 \sep \bf .064 & \textit{.018}     \sep \bf \textit{.014} & .018 \sep \bf .014 &  0.136 \sep \bf 0.132  &   0.136  \sep \bf .132 \\
\IFGSMU & \PGDU   & \textit{.061} \sep \textit{.061} & \bf .040 \sep .050 & .143 \sep \bf .142 & \textit{.061}     \sep \textit{.061} & \bf .045 \sep .061 & \bf 0.218 \sep 1.694  &   0.208  \sep \bf .205 \\
\IFGSMU & \IFGSMU & \bf \textit{.066} \sep \textit{.069}     & \bf .044 \sep .054 & .154 \sep \bf .152 & \bf \textit{.065} \sep \textit{.069} & \bf .048 \sep .068 & \bf 0.228 \sep 1.708  &   0.216  \sep \bf .214\\
\bottomrule
\end{tabular}
}
\caption{MNIST 5 vs 6}
\end{subtable}
\label{ap:mv-robustness:tab:mnist-infty-details}
\end{table*}
\begin{table*}
\caption{Test risks and bounds for 3 tasks of Fashion MNIST with $n{=}100$ perturbations for all pairs (Defense,Attack) with the two voters' set $\H$ and $\Hsigned$. 
The results in \textbf{bold} correspond to the best values between results for $\H$ and $\Hsigned$.
To quantify the gap between our risks and the classical definition we put in \textit{italic} the risk of our models against the classical attacks: we replace \PGDU and \IFGSMU~by \PGD or \IFGSM~(\ie, we did {\it not} sample from the uniform distribution). 
Since \cref{chap:mv-robustness:eq:max-mcallester} upperbounds \cref{chap:mv-robustness:eq:max-mcallester-0} thanks to the TV term, we compute the two bound values of \Cref{chap:mv-robustness:theorem:bound-average-max}.
}
\begin{subtable}{0.99\linewidth}
\resizebox{\textwidth}{!}{
\newcommand{\sep}{ & }
\begin{tabular}{ll||c|c||c|c|c|c||c|c||c|c|c|c|c|c}
\toprule
\multicolumn{2}{c||}{$\ell_\infty$-norm} &  \multicolumn{6}{c||}{\cref{chap:mv-robustness:algo-step}\ {\rm with \cref{chap:mv-robustness:eq:seeger-chromatic}}} & \multicolumn{8}{c}{\cref{chap:mv-robustness:algo-step}\ {\rm with \cref{chap:mv-robustness:eq:max-mcallester}}}\\
\multicolumn{2}{c||}{$b=0.1$} & \multicolumn{2}{c}{\scriptsize\rm Attack  without {\sc u}} &   \multicolumn{4}{c||}{ } & \multicolumn{2}{c}{\scriptsize\rm Attack  without {\sc u}} &\multicolumn{6}{c}{ }  \\
& & \multicolumn{2}{c||}{$\RiskA_{\dT}(\MVQ)$} &  \multicolumn{2}{c|}{$\Risk_{\Tpert}(\MVQ)$} & \multicolumn{2}{c||}{\cref{chap:mv-robustness:theorem:chromatic}} & \multicolumn{2}{c||}{$\RiskA_{\dT}(\MVQ)$} & \multicolumn{2}{c|}{$\RiskM_{\Tpert}(\MVQ)$} & \multicolumn{2}{c}{\cref{chap:mv-robustness:theorem:bound-average-max} - \cref{chap:mv-robustness:eq:max-mcallester}} & \multicolumn{2}{c}{\cref{chap:mv-robustness:theorem:bound-average-max} - \cref{chap:mv-robustness:eq:max-mcallester-0}}\\[1mm]
Defense & Attack & $\Hsigned$ & $\H$ & $\Hsigned$ & $\H$ & $\Hsigned$ & $\H$ & $\Hsigned$ & $\H$ & $\Hsigned$ & $\H$ & $\Hsigned$ & $\H$ & $\Hsigned$ & $\H$\\
\midrule
--- & ---         & \textit{.021} \sep \bf \textit{.020} & .021 \sep \bf .020 & \bf 0.060 \sep 0.070 & \textit{.019} \sep \textit{.019} & .019 \sep .019 & \bf 0.130 \sep  0.139 &  \bf 0.130 \sep  0.139 \\
--- & \PGDU       & \textit{.951} \sep \bf \textit{.944} & \bf .606 \sep .719 & \bf 1.275 \sep 1.333 & \textit{.935} \sep \bf\textit{.920} & \bf .762 \sep .864 & \bf 1.617 \sep 2.503 &  1.421 \sep \bf 1.317 \\
--- & \IFGSMU     & \textit{.957} \sep \bf \textit{.947} & \bf .588 \sep .718 & \bf 1.231 \sep 1.336 & \textit{.950} \sep \textit{.950} & \bf .734 \sep .851 & \bf 1.587 \sep 2.495 &  1.395 \sep \bf 1.316 \\
\U  & ---         & \bf \textit{.076} \sep \textit{.077} & \bf .076 \sep .077 &  \bf 0.178 \sep 0.184 & \bf \textit{.076} \sep \textit{.077} & \bf .076 \sep .077 &  \bf 0.230 \sep  0.235 &  \bf 0.230 \sep  0.235 \\
\U  & \PGDU       & \textit{.964} \sep \bf \textit{.961} & \bf .714 \sep .719 & 1.496 \sep \bf 1.265 & \textit{.966} \sep \bf \textit{.963} & \bf .853 \sep .859 & \bf 2.098 \sep 2.417 &  1.785 \sep \bf 1.416 \\
\U  & \IFGSMU     & \textit{.978} \sep \bf \textit{.976} & \bf .627 \sep .632 & 1.306 \sep \bf 1.259 & \textit{.979} \sep \textit{.979} & \bf .758 \sep .762 & \bf 1.914 \sep 2.422 &  1.597\sep \bf 1.396 \\
\hdashline
\PGDU & ---       & \textit{.041} \sep \bf \textit{.040} & .041 \sep \bf .040 &  0.114 \sep  \bf 0.111 & \textit{.041} \sep \bf \textit{.040} & .041 \sep \bf .040 &  0.173 \sep  \bf 0.171 &   0.173 \sep \bf 0.171 \\
\PGDU & \PGDU     & \textit{.098} \sep \bf \textit{.097} & .089 \sep \bf .086 & \bf  0.207 \sep  0.210 & \textit{.099} \sep \bf \textit{.097} & .101 \sep \bf .100 &  \bf 0.306 \sep 1.826 &   0.281 \sep \bf 0.267 \\
\PGDU & \IFGSMU   & \textit{.113} \sep \bf \textit{.112} & .105 \sep \bf .101 &  \bf 0.244 \sep  0.246 & \textit{.115} \sep \bf \textit{.112} & .120 \sep \bf .113 &  \bf 0.353 \sep 1.853 &   0.321 \sep \bf 0.302 \\
\IFGSMU & ---     & \bf \textit{.045} \sep \textit{.047} & \bf .045 \sep .047 &  \bf 0.131 \sep  0.137 & \bf \textit{.045} \sep \textit{.047} & \bf .045 \sep .047 &  \bf 0.188 \sep  0.194  &  \bf 0.188 \sep  0.194 \\
\IFGSMU & \PGDU   & \bf \textit{.100} \sep \textit{.102} & .089 \sep \bf .085 &  \bf 0.203 \sep  0.232 & \bf \textit{.100} \sep \textit{.102} & \bf .102 \sep \bf .102 &  \bf 0.298 \sep 1.645  & \bf 0.274 \sep  0.287 \\
\IFGSMU & \IFGSMU & \bf \textit{.112} \sep \textit{.116} & .099 \sep \bf .096 &  \bf 0.232 \sep  0.260 & \bf \textit{.112} \sep \textit{.116} & .114 \sep \bf .112 &  \bf 0.328 \sep 1.687  &  \bf 0.301 \sep  0.313 \\
\bottomrule
\end{tabular}
}
\caption{Fashion MNIST Sandall vs Ankle Boot}
\end{subtable}

\begin{subtable}{0.99\linewidth}
\resizebox{\textwidth}{!}{
\newcommand{\sep}{ & }
\begin{tabular}{ll||c|c||c|c|c|c||c|c||c|c|c|c|c|c}
\toprule
\multicolumn{2}{c||}{$\ell_\infty$-norm} &  \multicolumn{6}{c||}{\cref{chap:mv-robustness:algo-step}\ {\rm with \cref{chap:mv-robustness:eq:seeger-chromatic}}} & \multicolumn{8}{c}{\cref{chap:mv-robustness:algo-step}\ {\rm with \cref{chap:mv-robustness:eq:max-mcallester}}}\\
\multicolumn{2}{c||}{$b=0.1$} & \multicolumn{2}{c}{\scriptsize\rm Attack  without {\sc u}} &   \multicolumn{4}{c||}{ } & \multicolumn{2}{c}{\scriptsize\rm Attack  without {\sc u}} &\multicolumn{6}{c}{ }  \\
& & \multicolumn{2}{c||}{$\RiskA_{\dT}(\MVQ)$} &  \multicolumn{2}{c|}{$\Risk_{\Tpert}(\MVQ)$} & \multicolumn{2}{c||}{\cref{chap:mv-robustness:theorem:chromatic}} & \multicolumn{2}{c||}{$\RiskA_{\dT}(\MVQ)$} & \multicolumn{2}{c|}{$\RiskM_{\Tpert}(\MVQ)$} & \multicolumn{2}{c}{\cref{chap:mv-robustness:theorem:bound-average-max} - \cref{chap:mv-robustness:eq:max-mcallester}} & \multicolumn{2}{c}{\cref{chap:mv-robustness:theorem:bound-average-max} - \cref{chap:mv-robustness:eq:max-mcallester-0}}\\[1mm]
Defense & Attack & $\Hsigned$ & $\H$ & $\Hsigned$ & $\H$ & $\Hsigned$ & $\H$ & $\Hsigned$ & $\H$ & $\Hsigned$ & $\H$ & $\Hsigned$ & $\H$ & $\Hsigned$ & $\H$\\
\midrule
--- & ---         & \textit{.038} \sep \bf \textit{.037} & .038 \sep \bf .037 & \bf .088 \sep .091 & \textit{.038} \sep \bf \textit{.037} & .038 \sep \bf .037 &  \bf 0.153 \sep  0.155 &  \bf 0.153 \sep .155 \\
--- & \PGDU       & \textit{.596} \sep \bf \textit{.515} & .477 \sep \bf .218 & .844 \sep \bf .662 & \textit{.590} \sep \bf \textit{.576} & .570 \sep \bf .502 & \bf 1.049 \sep 1.924 &   0.948 \sep \bf .857 \\
--- & \IFGSMU     & \textit{.723} \sep \bf \textit{.623} & .573 \sep \bf .257 & .971 \sep \bf .751 & \textit{.716} \sep \bf \textit{.695} & .678 \sep \bf .598 & \bf 1.189 \sep 2.031 &  1.080 \sep \bf .980 \\
\U  & ---         & \textit{.032} \sep \textit{.032} & .032 \sep .032 & \bf .083 \sep .085 & \bf \textit{.032} \sep \textit{.033} & \bf .032 \sep .033 &  \bf 0.149 \sep  0.151 &  \bf 0.149 \sep .151 \\
\U  & \PGDU       & \bf \textit{.438} \sep \textit{.439} & .356 \sep \bf .245 & .813 \sep \bf .563 & \textit{.435} \sep \textit{.435} & .423 \sep \bf .312 & \bf 1.082 \sep 1.867 &   0.959 \sep \bf .688 \\
\U  & \IFGSMU     & \bf \textit{.546} \sep \textit{.547} & .453 \sep \bf .325 & .974 \sep \bf .690 & \bf \textit{.544} \sep \textit{.547} & .530 \sep \bf .409 & \bf 1.266 \sep 2.009 &  1.128 \sep \bf .823 \\
\hdashline
\PGDU & ---       & \bf \textit{.048} \sep \textit{.053} & \bf .048 \sep .053 & \bf .115 \sep .130 & \bf \textit{.048} \sep \textit{.053} & \bf .048 \sep .053 & \bf 0.177 \sep  0.188 &  \bf 0.177 \sep .188 \\
\PGDU & \PGDU     & \bf \textit{.102} \sep \textit{.116} & \bf .089 \sep .099 & \bf .205 \sep .223 & \bf \textit{.102} \sep \textit{.116} & \bf .096 \sep .115 & \bf 0.282 \sep 1.323 &  \bf 0.266 \sep .278 \\
\PGDU & \IFGSMU   & \bf \textit{.120} \sep \textit{.135} & \bf .102 \sep .115 & \bf .237 \sep .255 & \bf \textit{.120} \sep \textit{.135} & \bf .109 \sep .133 &  \bf 0.318 \sep 1.380 & \bf 0.299 \sep .309 \\
\IFGSMU & ---     & \textit{.051} \sep \bf \textit{.045} & .051 \sep \bf .045 & .120 \sep \bf .115 & \textit{.051} \sep \bf \textit{.045} & .051 \sep \bf .045 &  0.179 \sep  \bf 0.175  &   0.179 \sep \bf .175 \\
\IFGSMU & \PGDU   & \textit{.106} \sep \bf \textit{.094} & .091 \sep \bf .085 & .211 \sep \bf .193 & \textit{.106} \sep \bf \textit{.094} & .102 \sep \bf .097 & \bf  0.292 \sep 1.488  &   0.273 \sep \bf .252 \\
\IFGSMU & \IFGSMU & \textit{.120} \sep \bf \textit{.111} & \bf .101 \sep .102 & .239 \sep \bf .218 & \textit{.119} \sep \bf \textit{.111} & .113 \sep .113 &  \bf 0.322 \sep 1.546  &   0.299 \sep \bf .277 \\
\bottomrule
\end{tabular}
}
\caption{Fashion MNIST Top vs Pullover}
\end{subtable}
\label{ap:mv-robustness:tab:fashion-infty-details}
\end{table*}

\begin{table*}
\begin{subtable}{0.99\linewidth}
\resizebox{\textwidth}{!}{
\newcommand{\sep}{ & }
\begin{tabular}{ll||c|c||c|c|c|c||c|c||c|c|c|c|c|c}
\toprule
\multicolumn{2}{c||}{$\ell_\infty$-norm} &  \multicolumn{6}{c||}{\cref{chap:mv-robustness:algo-step}\ {\rm with \cref{chap:mv-robustness:eq:seeger-chromatic}}} & \multicolumn{8}{c}{\cref{chap:mv-robustness:algo-step}\ {\rm with \cref{chap:mv-robustness:eq:max-mcallester}}}\\
\multicolumn{2}{c||}{$b=0.1$} & \multicolumn{2}{c}{\scriptsize\rm Attack  without {\sc u}} &   \multicolumn{4}{c||}{ } & \multicolumn{2}{c}{\scriptsize\rm Attack  without {\sc u}} &\multicolumn{6}{c}{ }  \\
& & \multicolumn{2}{c||}{$\RiskA_{\dT}(\MVQ)$} &  \multicolumn{2}{c|}{$\Risk_{\Tpert}(\MVQ)$} & \multicolumn{2}{c||}{\cref{chap:mv-robustness:theorem:chromatic}} & \multicolumn{2}{c||}{$\RiskA_{\dT}(\MVQ)$} & \multicolumn{2}{c|}{$\RiskM_{\Tpert}(\MVQ)$} & \multicolumn{2}{c}{\cref{chap:mv-robustness:theorem:bound-average-max} - \cref{chap:mv-robustness:eq:max-mcallester}} & \multicolumn{2}{c}{\cref{chap:mv-robustness:theorem:bound-average-max} - \cref{chap:mv-robustness:eq:max-mcallester-0}}\\[1mm]
Defense & Attack & $\Hsigned$ & $\H$ & $\Hsigned$ & $\H$ & $\Hsigned$ & $\H$ & $\Hsigned$ & $\H$ & $\Hsigned$ & $\H$ & $\Hsigned$ & $\H$ & $\Hsigned$ & $\H$\\
\midrule
--- & ---         & \textit{.122} \sep \textit{.122} & .122 \sep .122 &  \bf 0.276 \sep  0.286 & \textit{.122} \sep \textit{.122} & .122 \sep .122 &  \bf 0.318 \sep  0.328 &  \bf 0.318 \sep  0.328 \\
--- & \PGDU       & \bf \textit{.884} \sep \textit{.887} & \bf .781 \sep .795 & 1.579 \sep \bf 1.268 & \bf \textit{.882} \sep \textit{.886} & \bf .864 \sep .872 & \bf 2.020 \sep 2.640 &  1.803 \sep \bf 1.390 \\
--- & \IFGSMU     & \bf \textit{.901} \sep \textit{.902} & \bf .756 \sep .774 & 1.558 \sep \bf 1.272 & \bf\textit{.901} \sep \textit{.902} & \bf .865 \sep .876 & \bf 2.032 \sep 2.651 &  1.795 \sep \bf 1.393 \\
\U  & ---         & \textit{.166} \sep \textit{.166} & .166 \sep .166 & \bf 0.352 \sep 0.357 & .166 \sep \textit{.166} & \textit{.166} \sep .166 &  \bf 0.389 \sep  0.394 &  \bf 0.389 \sep  0.394 \\
\U  & \PGDU       & \bf \textit{.911} \sep \textit{.914} & \bf .796 \sep .798 & 1.402 \sep \bf 1.326 & \bf\textit{.913} \sep \textit{.914} & .896 \sep \bf .888 & \bf 1.934 \sep 2.325 &  1.713 \sep \bf 1.447 \\
\U  & \IFGSMU     & \bf \textit{.935} \sep \textit{.937} & \bf .787 \sep .798 & 1.392 \sep \bf 1.350 & \bf\textit{.934} \sep \textit{.936} & .887 \sep \bf .882 & \bf 1.905 \sep 2.378 &  1.693 \sep \bf 1.469 \\
\hdashline
\PGDU & ---       & \textit{.163} \sep \bf \textit{.162} & .163 \sep \bf .162 &  \bf 0.386 \sep  0.395 & \textit{.163} \sep \bf  \textit{.162} & .163 \sep \bf .162 &  \bf 0.419 \sep  0.430 &  \bf 0.419 \sep  0.430 \\
\PGDU & \PGDU     & \bf \textit{.394} \sep \textit{.396} & .359 \sep \bf .329 &  0.764 \sep \bf 0.673 & \bf\textit{.394} \sep \textit{.396} & .403 \sep \bf .394 & \bf 0.954 \sep 2.321 &   0.865 \sep \bf 0.726 \\
\PGDU & \IFGSMU   & \bf\textit{.475} \sep \textit{.480} & .442 \sep \bf .410 &  0.910 \sep \bf 0.769 & \bf \textit{.477} \sep \textit{.480} & .487 \sep \bf .472 & \bf 1.121 \sep 2.411 &  1.020 \sep \bf 0.826 \\
\IFGSMU & ---     & \bf \textit{.167} \sep \textit{.168} & \bf .167 \sep .168 &  0.411 \sep \bf  0.395 & \bf \textit{.167} \sep \textit{.168} & \bf .167 \sep .168 & 0.445 \sep  \bf 0.429  &   0.445 \sep \bf 0.429 \\
\IFGSMU & \PGDU   & \textit{.396} \sep \bf \textit{.373} & .359 \sep \bf .293 &  0.772 \sep \bf 0.641 & \textit{.396} \sep \bf \textit{.373} & .405 \sep \bf .328 & \bf 0.970 \sep 2.368  &   0.877 \sep \bf 0.692 \\
\IFGSMU & \IFGSMU & \textit{.465} \sep \bf \textit{.428} & .424 \sep \bf .334 &  0.891 \sep  \bf 0.705 & \textit{.465} \sep \bf \textit{.429} & .470 \sep \bf .372 & \bf 1.090 \sep 2.425  &   0.995 \sep \bf 0.758 \\
\bottomrule
\end{tabular}
}
\caption{Fashion MNIST Coat vs Shirt}
\end{subtable}
\end{table*}
\begin{table}
\caption{Test risks for 6 tasks of MNIST and Fashion MNIST datasets for all pairs (Defense,Attack) with the two voters' set $\H$ and $\Hsigned$ using $\ell_2$-norm.
The results of these tables are computed considering defenses of the literature, \ie, adversarial training using \PGD~or \IFGSM. We also add an adversarial training using \U~for the completeness of comparison between this baseline defense and our algorithm.
The results in \textbf{bold} correspond to the best values between results for $\H$ and $\Hsigned$.
}
\begin{subtable}{0.32\linewidth}
\newcommand{\sep}{ & }
\resizebox{0.9\linewidth}{!}{
\begin{tabular}{ll||cc}
\toprule
\multicolumn{2}{c}{$\ell_2$-norm, $b=1$} & \multicolumn{2}{c}{$\RiskA_{\dT}(\MVPp)$}\\[1mm]
Defense & Attack & $\Hsigned$ & $\H$ \\
\midrule
--- & ---       &  .005 \sep  .005 \\
--- & \PGD      & \bf .326 \sep .327 \\
--- & \IFGSM    & .122 \sep \bf .121 \\
\U  & ---       &  .005 \sep  .005 \\
\U  & \PGD      & .191 \sep \bf .190 \\
\U  & \IFGSM    & \bf .071 \sep .072 \\
\hdashline
\PGD & ---      &  .007 \sep  .007 \\
\PGD & \PGD     & .027 \sep \bf .026 \\
\PGD & \IFGSM   & .022 \sep \bf .021 \\
\IFGSM & ---    & \bf .005 \sep .006 \\
\IFGSM & \PGD   & .041 \sep \bf .035 \\
\IFGSM & \IFGSM &  .021 \sep  .021 \\
\bottomrule
\end{tabular}
}
\caption{MNIST 1 vs 7}
\end{subtable}
\begin{subtable}{0.32\linewidth}
\newcommand{\sep}{ & }
\resizebox{0.9\linewidth}{!}{
\begin{tabular}{ll||cc}
\toprule
\multicolumn{2}{c}{$\ell_2$-norm, $b=1$} & \multicolumn{2}{c}{$\RiskA_{\dT}(\MVPp)$}\\[1mm]
Defense & Attack & $\Hsigned$ & $\H$ \\
\midrule
--- & ---        &  .015 \sep  .015 \\
--- & \PGD       & .692 \sep .692 \\
--- & \IFGSM     & .464 \sep \bf .462 \\
\U  & ---        & .024 \sep .024 \\
\U  & \PGD       & .653 \sep .653 \\
\U  & \IFGSM     & .441 \sep \bf .438 \\
\hdashline
\PGD & ---      & \bf .024 \sep .027 \\
\PGD & \PGD     & \bf .136 \sep .138 \\
\PGD & \IFGSM   & \bf .097 \sep .102 \\
\IFGSM & ---    & \bf .022 \sep .027 \\
\IFGSM & \PGD   & \bf .166 \sep .186 \\
\IFGSM & \IFGSM & \bf .113 \sep .124 \\
\bottomrule
\end{tabular}
}
\caption{MNIST 4 vs 9}
\end{subtable}
\begin{subtable}{0.32\linewidth}
\newcommand{\sep}{ & }
\resizebox{0.9\linewidth}{!}{
\begin{tabular}{ll||cc}
\toprule
\multicolumn{2}{c}{$\ell_2$-norm, $b=1$} & \multicolumn{2}{c}{$\RiskA_{\dT}(\MVPp)$}\\[1mm]
Defense & Attack & $\Hsigned$ & $\H$ \\
\midrule
--- & ---       &  .015 \sep  .015 \\
--- & \PGD      &  .283 \sep  .283 \\
--- & \IFGSM    &  .144 \sep  .144 \\
\U  & ---       &  .017 \sep  .017 \\
\U  & \PGD      & .220 \sep \bf .219 \\
\U  & \IFGSM    &  .122 \sep  .122 \\
\hdashline
\PGD & ---      & .014 \sep \bf .013 \\
\PGD & \PGD     & .056 \sep \bf .055 \\
\PGD & \IFGSM   & .045 \sep \bf .041 \\
\IFGSM & ---    & \bf .013 \sep .014 \\
\IFGSM & \PGD   & .077 \sep \bf .070 \\
\IFGSM & \IFGSM & .053 \sep \bf .047 \\
\bottomrule
\end{tabular}
}
\caption{MNIST 5 vs 6}
\end{subtable}
\begin{subtable}{0.32\linewidth}
\newcommand{\sep}{ & }
\resizebox{0.9\linewidth}{!}{
\begin{tabular}{ll||cc}
\toprule
\multicolumn{2}{c}{$\ell_2$-norm, $b=1$} & \multicolumn{2}{c}{$\RiskA_{\dT}(\MVPp)$}\\[1mm]
Defense & Attack & $\Hsigned$ & $\H$ \\
\midrule
--- & ---       &  .019 \sep  .019 \\
--- & \PGD      & .709 \sep \bf .708 \\
--- & \IFGSM    & .426 \sep \bf .414 \\
\U  & ---       & \bf .071 \sep .072 \\
\U  & \PGD      & .531 \sep .531 \\
\U  & \IFGSM    & .331 \sep \bf .329 \\
\hdashline
\PGD & ---      & \bf .034 \sep .036 \\
\PGD & \PGD     & .107 \sep \bf .103 \\
\PGD & \IFGSM   & .091 \sep \bf .087 \\
\IFGSM & ---    & .031 \sep \bf .029 \\
\IFGSM & \PGD   & .125 \sep \bf .108 \\
\IFGSM & \IFGSM & .104 \sep \bf .090 \\
\bottomrule
\end{tabular}
}
\caption{\centering Fashion MNIST\linebreak Sandall vs Ankle Boot}
\end{subtable}
\begin{subtable}{0.32\linewidth}
\newcommand{\sep}{ & }
\resizebox{0.9\linewidth}{!}{
\begin{tabular}{ll||cc}
\toprule
\multicolumn{2}{c}{$\ell_2$-norm, $b=1$} & \multicolumn{2}{c}{$\RiskA_{\dT}(\MVPp)$}\\[1mm]
Defense & Attack & $\Hsigned$ & $\H$ \\
\midrule
--- & ---        & .038 \sep .038 \\
--- & \PGD       & .286 \sep \bf .285 \\
--- & \IFGSM     & .188 \sep \bf .186 \\
\U  & ---        & .041 \sep \bf .039 \\
\U  & \PGD       & .249 \sep \bf .248 \\
\U  & \IFGSM     & .197 \sep \bf .192 \\
\hdashline
\PGD & ---      & \bf .043 \sep .045 \\
\PGD & \PGD     & \bf .102 \sep .117 \\
\PGD & \IFGSM   & \bf .090 \sep .094 \\
\IFGSM & ---    & \bf .038 \sep .040 \\
\IFGSM & \PGD   & .120 \sep \bf .106 \\
\IFGSM & \IFGSM & .092 \sep \bf .080 \\
\bottomrule
\end{tabular}
}
\caption{\centering Fashion MNIST\linebreak Top vs Pullover}
\end{subtable}
\begin{subtable}{0.32\linewidth}
\newcommand{\sep}{ & }
\resizebox{0.9\linewidth}{!}{
\begin{tabular}{ll||cc}
\toprule
\multicolumn{2}{c}{$\ell_2$-norm, $b=1$} & \multicolumn{2}{c}{$\RiskA_{\dT}(\MVPp)$}\\[1mm]
Defense & Attack & $\Hsigned$ & $\H$ \\
\midrule
--- & ---       & .122 \sep .122 \\
--- & \PGD      & .768 \sep \bf .767 \\
--- & \IFGSM    & .683 \sep \bf .680 \\
\U  & ---       & .204 \sep .204 \\
\U  & \PGD      & \bf .753 \sep .754 \\
\U  & \IFGSM    & .607 \sep \bf .606 \\
\hdashline
\PGD & ---      & .182 \sep \bf .178 \\
\PGD & \PGD     & .453 \sep \bf .412 \\
\PGD & \IFGSM   & .408 \sep \bf .379 \\
\IFGSM & ---    & .148 \sep \bf .146 \\
\IFGSM & \PGD   & \bf .405 \sep .411 \\
\IFGSM & \IFGSM & .369 \sep \bf .364 \\
\bottomrule
\end{tabular}
}
\caption{\centering Fashion MNIST\linebreak Coat vs Shirt}
\end{subtable}
\label{ap:mv-robustness:tab:l2-baseline}
\end{table}
\begin{table}[ht]
\caption{Test risks for 6 tasks of MNIST and Fashion MNIST datasets for all pairs (Defense,Attack) with the two voters' set $\H$ and $\Hsigned$ using $\ell_\infty$-norm.
The results of these tables are computed considering defenses of the literature, \ie, adversarial training using \PGD~or \IFGSM. We also add an adversarial training using \U~for the completeness of comparison between this baseline defense and our algorithm.
The results in \textbf{bold} correspond to the best values between results for $\H$ and $\Hsigned$.
}

\begin{subtable}{0.32\linewidth}
\newcommand{\sep}{ & }
\resizebox{0.9\linewidth}{!}{
\begin{tabular}{ll||cc}
\toprule
\multicolumn{2}{c}{$\ell_\infty$-norm, $b=0.1$} & \multicolumn{2}{c}{$\RiskA_{\dT}(\MVPp)$}\\[1mm]
Defense & Attack & $\Hsigned$ & $\H$ \\
\midrule
--- & ---       &  .005 \sep  .005 \\
--- & \PGD      & .499 \sep \bf .498 \\
--- & \IFGSM    & \bf .479 \sep .480 \\
\U  & ---       &  .004 \sep  .004 \\
\U  & \PGD      & .516 \sep \bf .515 \\
\U  & \IFGSM    &  .467 \sep  .467 \\
\hdashline
\PGD & ---      & \bf .006 \sep .007 \\
\PGD & \PGD     &  .019 \sep  .019 \\
\PGD & \IFGSM   &  .021 \sep  .021 \\
\IFGSM & ---    &  .007 \sep  .007 \\
\IFGSM & \PGD   & \bf .017 \sep .018 \\
\IFGSM & \IFGSM & \bf .019 \sep .020 \\
\bottomrule
\end{tabular}
}
\caption{MNIST 1 vs 7}
\end{subtable}
\begin{subtable}{0.32\linewidth}
\newcommand{\sep}{ & }
\resizebox{0.9\linewidth}{!}{
\begin{tabular}{ll||cc}
\toprule
\multicolumn{2}{c}{$\ell_\infty$-norm, $b=0.1$} & \multicolumn{2}{c}{$\RiskA_{\dT}(\MVPp)$}\\[1mm]
Defense & Attack & $\Hsigned$ & $\H$ \\
\midrule
--- & ---        &  .015 \sep  .015  \\
--- & \PGD       &  .921 \sep  .921  \\
--- & \IFGSM     &  .923 \sep  .923  \\
\U  & ---        &  .017 \sep  .017  \\
\U  & \PGD       & .877 \sep \bf .876  \\
\U  & \IFGSM     & .877 \sep .877  \\
\hdashline
\PGD & ---      & .041 \sep \bf .040  \\
\PGD & \PGD     & \bf .108 \sep .109  \\
\PGD & \IFGSM   & \bf .122 \sep .123  \\
\IFGSM & ---    & .057 \sep \bf .044  \\
\IFGSM & \PGD   & .109 \sep \bf .101  \\
\IFGSM & \IFGSM & .119 \sep \bf .108  \\
\bottomrule
\end{tabular}
}
\caption{MNIST 4 vs 9}
\end{subtable}
\begin{subtable}{0.32\linewidth}
\newcommand{\sep}{ & }
\resizebox{0.9\linewidth}{!}{
\begin{tabular}{ll||cc}
\toprule
\multicolumn{2}{c}{$\ell_\infty$-norm, $b=0.1$} & \multicolumn{2}{c}{$\RiskA_{\dT}(\MVPp)$}\\[1mm]
Defense & Attack & $\Hsigned$ & $\H$ \\
\midrule
--- & ---       & .015 \sep .015 \\
--- & \PGD      &  .498 \sep  .498 \\
--- & \IFGSM    & .511 \sep \bf .510 \\
\U  & ---       &  .015 \sep  .015 \\
\U  & \PGD      & .512 \sep \bf .511 \\
\U  & \IFGSM    &  .511 \sep  .511 \\
\hdashline
\PGD & ---      &  .014 \sep  .014 \\
\PGD & \PGD     & .065 \sep \bf .058 \\
\PGD & \IFGSM   & .068 \sep \bf .065 \\
\IFGSM & ---    & .018 \sep \bf .017 \\
\IFGSM & \PGD   & \bf .061 \sep .063 \\
\IFGSM & \IFGSM & \bf .069 \sep .071 \\
\bottomrule
\end{tabular}
}
\caption{MNIST 5 vs 6}
\end{subtable}
\begin{subtable}{0.32\linewidth}
\newcommand{\sep}{ & }
\resizebox{0.9\linewidth}{!}{
\begin{tabular}{ll||cc}
\toprule
\multicolumn{2}{c}{$\ell_\infty$-norm, $b=0.1$} & \multicolumn{2}{c}{$\RiskA_{\dT}(\MVPp)$}\\[1mm]
Defense & Attack & $\Hsigned$ & $\H$ \\
\midrule
--- & ---       & .019 \sep .019 \\
--- & \PGD      & .938 \sep .938 \\
--- & \IFGSM    & \bf .948 \sep .949 \\
\U  & ---       & \bf .076 \sep .077 \\
\U  & \PGD      & .970 \sep \bf .969 \\
\U  & \IFGSM    & .981 \sep .981 \\
\hdashline
\PGD & ---      & .041 \sep \bf .040 \\
\PGD & \PGD     & .098 \sep \bf .097 \\
\PGD & \IFGSM   & .115 \sep \bf .111 \\
\IFGSM & ---    & .112 \sep \bf .047 \\
\IFGSM & \PGD   & \bf .045 \sep .100 \\
\IFGSM & \IFGSM & \bf .101 \sep .114 \\
\bottomrule
\end{tabular}
}
\caption{\centering Fashion MNIST\linebreak Sandall vs Ankell Boot}
\end{subtable}
\begin{subtable}{0.32\linewidth}
\newcommand{\sep}{ & }
\resizebox{0.9\linewidth}{!}{
\begin{tabular}{ll||cc}
\toprule
\multicolumn{2}{c}{$\ell_\infty$-norm, $b=0.1$} & \multicolumn{2}{c}{$\RiskA_{\dT}(\MVPp)$}\\[1mm]
Defense & Attack & $\Hsigned$ & $\H$ \\
\midrule
--- & ---        & .038 \sep .038 \\
--- & \PGD       & \bf .574 \sep .577 \\
--- & \IFGSM     & .700 \sep \bf .696 \\
\U  & ---        & \bf .032 \sep .033 \\
\U  & \PGD       & \bf .428 \sep .435 \\
\U  & \IFGSM     & \bf .540 \sep .550 \\
\hdashline
\PGD & ---      &  \bf .047\sep .049 \\
\PGD & \PGD     &  .101\sep \bf .097 \\
\PGD & \IFGSM   &  .118\sep \bf .112 \\
\IFGSM & ---    &  .049\sep \bf .048 \\
\IFGSM & \PGD   &  .100\sep \bf .090 \\
\IFGSM & \IFGSM &  .112\sep \bf .108 \\
\bottomrule
\end{tabular}
}
\caption{\centering Fashion MNIST\linebreak Top vs Pullover}
\end{subtable}
\begin{subtable}{0.32\linewidth}
\newcommand{\sep}{ & }
\resizebox{0.9\linewidth}{!}{
\begin{tabular}{ll||cc}
\toprule
\multicolumn{2}{c}{$\ell_\infty$-norm, $b=0.1$} & \multicolumn{2}{c}{$\RiskA_{\dT}(\MVPp)$}\\[1mm]
Defense & Attack & $\Hsigned$ & $\H$ \\
\midrule
--- & ---       & .122 \sep .122 \\
--- & \PGD      & .879 \sep .879 \\
--- & \IFGSM    & .898 \sep .898 \\
\U  & ---       & .166 \sep .166 \\
\U  & \PGD      & .913 \sep \bf .911 \\
\U  & \IFGSM    & .934 \sep \bf .933 \\
\hdashline
\PGD & ---      & \bf .164 \sep .167 \\
\PGD & \PGD     & .398 \sep \bf .395 \\
\PGD & \IFGSM   & \bf .479 \sep .481 \\
\IFGSM & ---    & \bf .163 \sep .169 \\
\IFGSM & \PGD   & \bf .356 \sep .391 \\
\IFGSM & \IFGSM & \bf .422 \sep .461 \\
\bottomrule
\end{tabular}
}
\caption{\centering Fashion MNIST\linebreak Coat vs Shirt}
\end{subtable}
\label{ap:mv-robustness:tab:infty-baseline}
\end{table}

\end{noaddcontents}