\documentclass{thesis}
\pagestyle{plain}
\aliaspagestyle{part}{empty}
\usepackage{dsfont}
\newcommand{\wbf}{\mathbf{w}}
\newcommand{\Wbf}{\mathbf{W}}
\newcommand{\xbf}{\ensuremath{\mathbf{x}}}
\newcommand{\zbf}{\ensuremath{\mathbf{z}}}
\newcommand{\zerobf}{\mathbf{0}}
\newcommand{\h}{h}
\newcommand{\dist}{\mathrm{dist}}

\newcommand{\Acal}{\ensuremath{\mathcal{A}}}
\newcommand{\Bcal}{\ensuremath{\mathcal{B}}}
\newcommand{\Ccal}{\ensuremath{\mathcal{C}}}
\newcommand{\Dcal}{\ensuremath{\mathcal{D}}}
\newcommand{\Fcal}{\ensuremath{\mathcal{F}}}
\newcommand{\Hcal}{\ensuremath{\mathcal{H}}}
\newcommand{\Mcal}{\ensuremath{\mathcal{M}}}
\newcommand{\Ncal}{\ensuremath{\mathcal{N}}}
\newcommand{\Pcal}{\ensuremath{\mathcal{P}}}
\newcommand{\Scal}{\ensuremath{\mathcal{S}}}
\newcommand{\Tcal}{\ensuremath{\mathcal{T}}}
\newcommand{\Xcal}{\ensuremath{\mathcal{X}}}
\newcommand{\Ycal}{\ensuremath{\mathcal{Y}}}
\newcommand{\Zcal}{\ensuremath{\mathcal{Z}}}

\newcommand{\Ebb}{\ensuremath{\mathbb{E}}}
\newcommand{\Pbb}{\ensuremath{\mathbb{P}}}
\newcommand{\Rbb}{\ensuremath{\mathbb{R}}}
\newcommand{\Nbb}{\ensuremath{\mathbb{N}}}

\newcommand{\Rfrak}{\ensuremath{\mathfrak{R}}}

\newcommand{\RA}{\right\rangle}
\newcommand{\LA}{\left\langle}
\newcommand{\LB}{\left[}
\newcommand{\RB}{\right]}
\newcommand{\LC}{\left\{}
\newcommand{\LM}{\left\|}
\newcommand{\RM}{\right\|}
%\newcommand{\RC}{\right\}}
%\newcommand{\RN}{\right\vert}
\newcommand{\LN}{\left\vert}
\newcommand{\LP}{\left(}
\newcommand{\RP}{\right)}

\newcommand{\wrt}{{\it w.r.t.}\xspace}
\newcommand{\eg}{{\it e.g.}\xspace}
\newcommand{\ie}{{\it i.e.}\xspace}
\newcommand{\iid}{{\it i.i.d.}\xspace}

\newcommand{\defeq}{:=}

\DeclareMathOperator*{\EE}{\Ebb}
\DeclareMathOperator*{\PP}{\Pbb}
\DeclareMathOperator*{\argmin}{\mathrm{argmin}}
\DeclareMathOperator*{\vect}{\mathrm{vec}}
\DeclareMathOperator*{\leaky}{\mathrm{Leaky}}
\DeclareMathOperator*{\proj}{\mathrm{Proj}}

\newcommand{\Irm}{\mathrm{I}}
\newcommand{\KL}{\mathrm{KL}}
\newcommand{\KLr}{\overline{\KL}}
\newcommand{\Hell}{H^2}
\newcommand{\TV}{TV}
\newcommand{\kl}{\mathrm{kl}}
\newcommand{\W}{\mathrm{W}}
\newcommand{\Lip}{\mathrm{Lip}}

\newcommand{\D}{\Dcal}
\newcommand{\Dm}{\Dcal_{m}}
\renewcommand{\H}{\Hcal}
\newcommand{\Hb}{\overline{\Hcal}}
\newcommand{\loss}{\ell}
\renewcommand{\P}{\mathrm{P}}
\newcommand{\Q}{\mathrm{Q}}
\newcommand{\R}{\Rbb}
\newcommand{\N}{\Nbb}
\renewcommand{\S}{\Scal}
\newcommand{\Sm}{\S_m}
\newcommand{\Risk}{\text{R}}
\newcommand{\Riskhat}{\hat{\Risk}}
\newcommand{\X}{\Xcal}
\newcommand{\x}{\xbf}
\newcommand{\y}{y}
\newcommand{\Y}{\Ycal}
\newcommand{\Z}{\Zcal}
\newcommand{\z}{\zbf}
\newcommand{\varepsilonbf}{\boldsymbol{\varepsilon}}
\newcommand{\rad}{\mathdbcal{E}}
\newcommand{\DS}{\D_\S}
\newcommand{\yeast}{{\sc Yeast}\xspace}
\newcommand{\phishing}{{\sc Phishing}\xspace}
\newcommand{\mushrooms}{{\sc Mushrooms}\xspace}
\newcommand{\mnist}{{\sc MNIST}\xspace}
\newcommand{\fashion}{{\sc FashionMNIST}\xspace}
\newcommand{\indic}{\mathds{1}}

\newcommand{\OPBTest}{\normalfont\textsc{OPBTest} }
\newcommand{\OPBTrain}{\normalfont\textsc{OPBTrain} }

\newcommand{\Vhat}{\hat{V}}
\newcommand{\Bhat}{\hat{B}}
\DeclareMathOperator*{\ReLU}{\mathrm{ReLU}}

\newcommand{\Cfrak}{\ensuremath{\mathfrak{C}}}

\newcommand{\Poinc}{\texttt{Poinc}}
\newcommand{\Lsob}{\texttt{L-Sob}}
\newcommand{\Ent}{\mathrm{Ent}}
\newcommand{\Var}{\mathrm{Var}}
\DeclareMathOperator*{\Err}{\mathrm{Err}}


\let\oldQ\Q
\let\oldP\P
\renewcommand{\Q}{\orange{\oldQ}}
\renewcommand{\P}{\green{\oldP}}

\usepackage[noend]{algpseudocode}



\begin{document}

%  ----------------------------------------------------------------------------------------------- %
\pagestyle{plain}

% ----------------------------------------------------------------------------------------------- %

\includepdf{main/begin.pdf}

% ----------------------------------------------------------------------------------------------- %

\section*{Remerciements}

Ces quelques lignes viennent d'un matin de juillet qui parachève l'écriture de ce manuscrit, produit d'un voyage rarement hésitant mais toujours incertain.  
Remercier certes, mais par où commencer, une métaphore peut être? Je vois cette thèse comme une lance finement ouvragée dont je ne suis que le fer, sobre et discret qui s'attelle à la tâche, là où vous tous formez cette hampe magnifique qui donne prestance, extension et force.\\
Commençons par la prestance, toute scientifique, de ce manuscrit qui n'aurait pu être sans Benjamin, qui m'a accompagné pendant quatre ans, qui m'a appris les divers aspects du métier et m'a fait évoluer d'étudiant enthousiaste à chercheur en herbe. Un merci particulier pour m'avoir laissé bricoler mon cadre de vie insolite à Paris qui, pour rester avec mes proches, m'a amené loin du labo. Un grand merci également à Umut Simsekli, Paul Viallard, Pierre Jobic, Omar Rivasplata ainsi que John Shawe-Taylor pour m'avoir fait découvrir, sous divers aspects, la richesse de la collaboration qui offre un sens bien plus humain à la chose scientifique. Point d'orgue sur cette liste déjà fournie, un grand merci aux membres du jury, Pascal Germain qui a eu la gentillesse de participer à ce jury, Claire Boyer et Gérard Biau que j'ai connu en tant qu'enseignants aux prémisses de ma thèse et dont la présence dans ce comité clôt magnifiquement cette boucle. Enfin, un grand merci à Frédéric Chazal et Stéphane Chrétien d'avoir été rapporteurs et d'avoir consacré leur précieux temps à cette production scientifique qui je l'espère, aura suscité un certain interêt (et avec un peu de chance, un interêt certain).\\
Continuons par l'extension, toute spirituelle, qui m'a été prodiguée lors de cette histoire par mes amies et amis qui m'enrichissent autant qu'ils me font rire chaque jour. Sans la merveilleuse mosaïque de leurs passions et reflexions, nul doute que ma thèse serait terne et sans saveur. Je songe à ce magicien d'Anatole, cette chanteuse d'Anaïs, à Pierre qui apparaît une deuxième fois dans ces remerciements, à Antoine qui aime trop Grenoble pour que je le voie souvent, à Farf qui a troqué la pizza cachanaise pour l'italienne. Je songe à mes amis qui dansent, membres la Guillotine (nom à redéfinir au demeurant): Keko et Chanus qui secrètement adorent me voir parler du grand capital, Kaou, Sélene, Vince, Antoine, Angelo, Maxime, Juliette, Abou, Calvin, Bertrand. Merci également à Jadou et Phil dont je sais la présence, merci à Miles pour avoir partagé ta vision de la guitare, merci à Hugo et Pia qui seront les prochains artistes pop-electro de cette génération. A tous, merci pour votre musique. 
Merci également à tous ceux qui m'ont enchanté de temps à autre par leur présence au fil des ans: Armand, Samy, Valentin qui est toujours là pour une session tennis tout comme Thibault, mes anciens colocs Quentin et Aaron qui ne sont jamais bien loin. Merci aussi à Liam et Sophie qui sont à ce jour les dernières belles rencontres que j'ai faites.
Je songe également aux copains de prépa que j'ai toujours le plaisir de croiser de temps à autres, Merwan, Pierre (troisième apparition dans ces remerciements), Clément, Mat' Blanke que j'ai eu la chance de redécouvrir pendant cette thèse ainsi que Song. Merci aussi à Martin (Verdammt), au Greg d'être là de la soirée au déjeuner (ou bien l'inverse).\\  
Que de noms déjà mentionnés qui masquent des analyses fines, de l'art vivace, des discussions précieuses. Et tant encore à venir: merci au Hossen qui a bien des fois changé mon rapport aux choses, à Tom et ses analyses d'une justesse rare, Mathias, le réalisateur le plus prometteur de Paris, Gabriel qui développera sans nul doute un regard frais et caustique sur l'astrologie romaine, Ulysse qui philosophe avec sa bienveillance rare, et Ariane qui porte sur le monde son regard solaire.\\
Finalement, la force, c'est ce que m'ont donné les plus courageux d'entre vous qui me supportent de façon bien trop régulière. Vous m'avez sans cesse sorti du silence froid de ma chambre où les maths ont foisonné trois années durant. Je pense à ma très chère Mathilde qui a toujours eu du temps malgré des études de médecine de plus en plus prenantes, à Yvan et Simon et leur bienveillance rare, à Antoine, Jacques et François-Xavier qui incarnent cette alliance apaisée d'un regard perçant aux choses et d'une écoute sincère. Je pense à Louis qui a rendu possible mon projet musical et qui est là dans moults facettes de ma vie. Je pense au concile de Vincennes: Hada, Monjuju, Yannis, Yaacov, Alex, Brian qui rendent tout léger, sauf leur ventre, depuis l'adolescence. Finalement, je pense à ma famille qui est là depuis le début, à mon père, à ma mère qui m'ont élevé dans tout les sens du terme, à mon grand frère JB qui prend le temps de m'instruire au tennis et d'improviser des bons plans, à Kevin et Clémence qui m'ont toujours accueilli au coin de leur salon pour un jeu de plateau, et enfin à mon Nain et ma Naine à qui je transmets ce que je peux. À vous tous, merci.

\newpage
\textit{À Anaïs, qui rend commun ce qui est rare.}
\begin{vplace}[0.7]

\setlength{\epigraphwidth}{0.72\textwidth}

\epigraph{Quote 1}{Author}

\vspace{1cm}

\epigraph{Quote 2}{\textsc{Author}}


\end{vplace}

% ----------------------------------------------------------------------------------------------- %

\dominitoc

\tableofcontents
\newpage
\listoffigures
\newpage
\listofalgorithms
\newpage
\listoftheorems

\newcommand{\wbf}{\mathbf{w}}
\newcommand{\Wbf}{\mathbf{W}}
\newcommand{\xbf}{\ensuremath{\mathbf{x}}}
\newcommand{\zbf}{\ensuremath{\mathbf{z}}}
\newcommand{\zerobf}{\mathbf{0}}
\newcommand{\h}{h}
\newcommand{\dist}{\mathrm{dist}}

\newcommand{\Acal}{\ensuremath{\mathcal{A}}}
\newcommand{\Bcal}{\ensuremath{\mathcal{B}}}
\newcommand{\Ccal}{\ensuremath{\mathcal{C}}}
\newcommand{\Dcal}{\ensuremath{\mathcal{D}}}
\newcommand{\Fcal}{\ensuremath{\mathcal{F}}}
\newcommand{\Hcal}{\ensuremath{\mathcal{H}}}
\newcommand{\Mcal}{\ensuremath{\mathcal{M}}}
\newcommand{\Ncal}{\ensuremath{\mathcal{N}}}
\newcommand{\Pcal}{\ensuremath{\mathcal{P}}}
\newcommand{\Scal}{\ensuremath{\mathcal{S}}}
\newcommand{\Tcal}{\ensuremath{\mathcal{T}}}
\newcommand{\Xcal}{\ensuremath{\mathcal{X}}}
\newcommand{\Ycal}{\ensuremath{\mathcal{Y}}}
\newcommand{\Zcal}{\ensuremath{\mathcal{Z}}}

\newcommand{\Ebb}{\ensuremath{\mathbb{E}}}
\newcommand{\Pbb}{\ensuremath{\mathbb{P}}}
\newcommand{\Rbb}{\ensuremath{\mathbb{R}}}
\newcommand{\Nbb}{\ensuremath{\mathbb{N}}}

\newcommand{\Rfrak}{\ensuremath{\mathfrak{R}}}

\newcommand{\RA}{\right\rangle}
\newcommand{\LA}{\left\langle}
\newcommand{\LB}{\left[}
\newcommand{\RB}{\right]}
\newcommand{\LC}{\left\{}
\newcommand{\LM}{\left\|}
\newcommand{\RM}{\right\|}
%\newcommand{\RC}{\right\}}
%\newcommand{\RN}{\right\vert}
\newcommand{\LN}{\left\vert}
\newcommand{\LP}{\left(}
\newcommand{\RP}{\right)}

\newcommand{\wrt}{{\it w.r.t.}\xspace}
\newcommand{\eg}{{\it e.g.}\xspace}
\newcommand{\ie}{{\it i.e.}\xspace}
\newcommand{\iid}{{\it i.i.d.}\xspace}

\newcommand{\defeq}{:=}

\DeclareMathOperator*{\EE}{\Ebb}
\DeclareMathOperator*{\PP}{\Pbb}
\DeclareMathOperator*{\argmin}{\mathrm{argmin}}
\DeclareMathOperator*{\vect}{\mathrm{vec}}
\DeclareMathOperator*{\leaky}{\mathrm{Leaky}}
\DeclareMathOperator*{\proj}{\mathrm{Proj}}

\newcommand{\Irm}{\mathrm{I}}
\newcommand{\KL}{\mathrm{KL}}
\newcommand{\KLr}{\overline{\KL}}
\newcommand{\Hell}{H^2}
\newcommand{\TV}{TV}
\newcommand{\kl}{\mathrm{kl}}
\newcommand{\W}{\mathrm{W}}
\newcommand{\Lip}{\mathrm{Lip}}

\newcommand{\D}{\Dcal}
\newcommand{\Dm}{\Dcal_{m}}
\renewcommand{\H}{\Hcal}
\newcommand{\Hb}{\overline{\Hcal}}
\newcommand{\loss}{\ell}
\renewcommand{\P}{\mathrm{P}}
\newcommand{\Q}{\mathrm{Q}}
\newcommand{\R}{\Rbb}
\newcommand{\N}{\Nbb}
\renewcommand{\S}{\Scal}
\newcommand{\Sm}{\S_m}
\newcommand{\Risk}{\text{R}}
\newcommand{\Riskhat}{\hat{\Risk}}
\newcommand{\X}{\Xcal}
\newcommand{\x}{\xbf}
\newcommand{\y}{y}
\newcommand{\Y}{\Ycal}
\newcommand{\Z}{\Zcal}
\newcommand{\z}{\zbf}
\newcommand{\varepsilonbf}{\boldsymbol{\varepsilon}}
\newcommand{\rad}{\mathdbcal{E}}
\newcommand{\DS}{\D_\S}
\newcommand{\yeast}{{\sc Yeast}\xspace}
\newcommand{\phishing}{{\sc Phishing}\xspace}
\newcommand{\mushrooms}{{\sc Mushrooms}\xspace}
\newcommand{\mnist}{{\sc MNIST}\xspace}
\newcommand{\fashion}{{\sc FashionMNIST}\xspace}
\newcommand{\indic}{\mathds{1}}

\newcommand{\OPBTest}{\normalfont\textsc{OPBTest} }
\newcommand{\OPBTrain}{\normalfont\textsc{OPBTrain} }

\newcommand{\Vhat}{\hat{V}}
\newcommand{\Bhat}{\hat{B}}
\DeclareMathOperator*{\ReLU}{\mathrm{ReLU}}

\newcommand{\Cfrak}{\ensuremath{\mathfrak{C}}}

\newcommand{\Poinc}{\texttt{Poinc}}
\newcommand{\Lsob}{\texttt{L-Sob}}
\newcommand{\Ent}{\mathrm{Ent}}
\newcommand{\Var}{\mathrm{Var}}
\DeclareMathOperator*{\Err}{\mathrm{Err}}


\let\oldQ\Q
\let\oldP\P
\renewcommand{\Q}{\orange{\oldQ}}
\renewcommand{\P}{\green{\oldP}}


% ----------------------------------------------------------------------------------------------- %

\let\proof\newproof
\let\endproof\endnewproof
 
% ----------------------------------------------------------------------------------------------- %
\counterwithout{figure}{section}
\chapter*{Preamble: What is generalisation?}
\addcontentsline{toc}{chapter}{Preamble: What is generalisation?}\mtcaddchapter


\addtextlist{lof}{Preamble}

This manuscript tackles the notion of \emph{generalisation} a notion built upon the general notion of \emph{learning}. For a brief moment, let's take the luxury of forgetting about machines and concentrate on learning at its most human. First and foremost, a human being (or learner) is structured around experiences, either lived or passed on by others. 

The learner then benefit from these experiences in various ways, for instance, by considering a mediated experience to be true (fire burns) and acting on this assumption, whereas reiteration or denial of this same information may be symptoms of zero truth value.  These scenarios can just as easily appear for a lived experience (the question of hallucinations). This first dichotomy in information processing is intrinsically linked to a clearly stated question: does fire burn? Can I trust my senses or have I hallucinated? In these cases, learning has taken place by subjecting the experience to its truth value in relation to a simple question (in this case with two outcomes).  This vision can easily be extended to a multiple (and finite) tree of possibilities. Indeed, we can extend the burning question as follows: what is the intensity of the burn as a function of the temperature of the fire? We can then establish a multitude of answers representing various degrees of burn. 

However, many questions cannot be reduced to a finite number of possibilities. For example, what is fire? To answer this question, it is nevertheless possible to exploit multiple facets of experience (wood fire, twig, rock) to propose that fire is the chemical reaction of oxygen in the air with a combustible material, with a supply of energy serving as the trigger. 

Then, a legitimate question is: why has mankind understood the nature of fire? This fundamental understanding emerged from practical considerations: how can we stop being cold? Can we eat meat other than raw to reduce the risk of illness? It then takes multiple interactions with the environment to generate experiences and then learn from them to gradually respond to a complex need (how to make a fire to keep yourself warm?).

Thus, through this preliminary analysis, we have found several premises of understanding human learning.

\begin{itemize}
    \item How is learning formalised structurally?
    The learner must base the experience on simple questions to acquire primary certainties. These latter acquired, it is possible to reach complex questions by interweaving more and more elementary considerations.
    \item Where does the need to learn come from?
    From a practical point of view, the emergence of these complex questions often arises from a relationship between the being and its environment, making it possible to develop contextual objectives. The learner then gradually becomes capable of responding to complex needs through a succession of simple actions.
\end{itemize}
 



\begin{comment}
Pour un bref instant, prenons le luxe d'oublier les machines pour se concentrer sur l'apprentissage en ce qu'il a de plus humain. Un humain, en premier lieu, va se structurer autour d'expériences, vécues ou transmises par autrui. L'être apprenant va alors tirer bénéfice de ces vécus via diverses modalités, par exemple, en considérant une expérience médiée comme vraie (le feu brûle) et agir en fonction de ce postulat alors que la réitération ou la négation de cette même information peuvent être des symptômes d'une valeur de vérité nulle.  Ces scénarios peuvent tout aussi bien apparaître pour une expérience vécue (hallucinations). Cette première dichotomie quant au traitement de l'information est intrinsèquement liée à une question clairement énoncée : est-ce que le feu brûle? Puis-je me fier à mes sens ou ai-je halluciné ? Dans ces cas de figure, l'apprentissage a eu lieu à travers l'assujettissement de l'expérience à sa valeur de vérité par rapport à une question simple (ici à deux issues).  Cette vision peut facilement s'étendre à une arborescence multiple (et finie) de possibles. En effet, on peut étendre la question de la brûlure comme suit: quelle est l'intensité de la brûlure en fonction de la température du feu? On peut dès lors établir une multitude de réponses représentant divers degrés de brûlure.

Néanmoins, de nombreuses questions ne peuvent se réduire à un nombre fini de possibilités. Par exemple, qu'est-ce que le feu? Pour répondre à cette question, il est néanmoins possible d'exploiter de multiples facettes d'expériences (feu de bois, brindille, roche) pour proposer le feu comme étant la réaction chimique de l'oxygène de l'air avec un matériau combustible, un apport d'énergie servant de déclencheur. 

Il est alors légitime de se demander pourquoi l'apprenant a eu besoin de comprendre la vraie nature du feu. Cette  compréhension fondamentale des choses émerge de considérations pratiques : comment ne plus avoir froid? Peut-on manger de la viande autrement que crue pour diminuer les risques de maladie? Il faut alors de multiples interactions avec l'environnement pour générer des expériences et ensuite apprendre d'elles pour répondre graduellement à un besoin complexe (comment faire un feu pour se réchauffer?).

Ainsi, par cette analyse préliminaire, nous avons trouvé plusieurs prémices de compréhension de l'apprentissage chez l'homme.

1/ Comment l'apprentissage se formalise-t-il structurellement ?
L'apprenant doit abâtardir l'expérience à des questions simples pour acquérir des certitudes primaires. Ces dernières acquises, il est possible d'atteindre des questions complexes en imbriquant de plus en plus de considérations élémentaires. 

2/ D'où provient le besoin d'apprendre ?
D'un point de vue pratique, l'émergence de ces questions complexes dérive bien souvent d'un rapport de l'être à son environnement, permettant d'élaborer des objectifs contextuels. L'apprenant devient alors graduellement capable de répondre à des besoins complexes par une succession d'actions simples.

TODO: parler des facettes humaines de généralisation: abstraire pour atteindre des principes généraux: oui mais ne pas perdre de vue les cas particuliers. --> Liens interpolations et extrapolation

Parallèle: dvpt de la compréhension humaine consciente: espace de dimension finie dont la dimension ne cesse de croitre par l'apprentissage: similaire aux deep nets
A contrario: mécaniques de la compréhension humaine, consciente ou inconsciente -> espace de dimension infinie

DONC: on pourra de mieux en mieux comprendre et singer la compréhension humaine via des espaces de dimension finie car c tout ce que notre compréhension consciente permet, néanmoins cette compréhension aura toujours cette limite dénombrable, qui sera peut etre plus tard juste invisible (exemple actuel: chatGPT: 1kg de plomb aussi lourd que deux kilos de plumes)
\end{comment}




JURY:


Rapporteurs: Alquier (sur)/Chopin (moins)

Membres: Seldin (rapporteur mais peut etre pas fou pour le rapport)/ Pascal Germain (rapporteur) Gérard (si Pierre pas dispo), John Shawe-Taylor (examinateur), Emilie Morvant (présidente) + le Maitre + Arnak Dalalyan (trop proche?), Alessandro Rudi
 


CHALLENGE HERE: being very rigorous on the lit review.
\newpage
\counterwithin{figure}{chapter}

\newrefcontext[sorting=ydnt]
\chapter*{List of Publications}\addcontentsline{toc}{chapter}{List of Publications}
\begin{refsection}[publications.bib]
\nocite{*}
\printbibliography[heading={subbibliography}, keyword={conference}, title={Conference article}]
\printbibliography[heading={subbibliography}, keyword={journal}, title={Journal article}]
\printbibliography[heading={subbibliography}, keyword={report}, title={Research Report}]
\end{refsection}

\adjustmtc[4]

% ----------------------------------------------------------------------------------------------- %
\pagestyle{headings}

\part{Background}
\label{part:background}

\chapter[A survey on PAC-Bayes Generalisation Bounds]{A survey on PAC-Bayes Generalisation Bounds}
\label{chap:intro-stat-learning}

\minitoc

\addchapterlof
\addchapterloa
\addchapterloe

\begin{abstract}
    This chapter provides a brief introduction of statistical learning theory and various modern kind of generalisation bounds (go from uniform convergence bounds to algorithmic stability, search for other kinds of non PAC-Bayes bounds). Need to recall the historical and modern shapes of generalisation bounds in ML. 
   
\end{abstract}

\newpage

\section{A brief survey of generalisation bounds}

\section{PAC-Bayes learning}
%!TEX root = main.tex
\chapter{PAC-Bayes with Weak Statistical Assumptions: Generalisation Bounds for Martingales and Heavy-Tailed losses}
\label{chap: pb-ht}

\addchapterlof
\addchapterloa
\addchapterloe

\vspace{-1.5cm}
\begin{center}
\textbf{This chapter is based on the following paper}\\[0.1cm]
\end{center}
\printpublication{haddouche2023pac}

\minitoc


\begin{abstract}
\Cref{chap: pb-ht} provide PAC-Bayes bounds holding with weak statistical assumptions (finite variance, which is what we consider as heavy-tailed in this manuscript), this is promising to encompass various learning situations involving optimisation algorithms such as heavy-tailed SGD \citep{gurbuzbalaban2020heavy} where assumptions such as bounded or subgaussian losses do not hold. Furthermore those results go beyond \iid assumption on $\S$ and hold for all datasets $(\Sm)_{m\geq 1}$ simultaneously. Such a flexible setting is in line with various optimisation frameworks, where new data can be available after the beginning of the learning process and be incorporated on-the-fly to the ongoing training, regardless of their potential correlation with previous data. Then, the theoretical results proposed in this chapter are a promising step toward practical settings where data may exhibit heavy-tailed behaviours and the loss function to be unbounded.
\end{abstract}

\section{Introduction}

In \Cref{chap:intro-pac-bayes}, McAllester's and Catoni's bound \citep{mcallester2003pac,catoni2007pac} have been presented as key theoretical results with practical repercussions through their associated learning algorithm. However, the bounded or subgaussian assumption on the loss makes those results limited to tackle many real-life situations.
Indeed, from an optimisation perspective, as stated in Section 1.4 of \Cref{chap:intro-pac-bayes}, generalisation bounds should hold with weak statistical assumptions to make PAC-Bayes compatible with heavy-tailed data. Several works already proposed routes to overcome the boundedness constraint:  \citet[][Chapter 5]{catoni2004statistical} already proposed PAC-Bayes bounds for classification tasks and regressions ones with quadratic loss under a subexponential assumption. This technique has later been exploited in \citet{alquier2013sparse} for the single-index model, and by \citet{guedj2013pac} for nonparametric sparse additive regression, both under the assumption that the noise is subexponential. However all these works are dealing with light-tailed losses.\footnote{A loss $\ell$ is light(heavy)-tailed if for all $h$, $\ell(h,.)$ is light(heavy)-tailed}
\citet{alquier2018simpler,holland2019pac, kuzborskij2019efron, haddouche2021pac} proposed extensions beyond light-tailed losses.
This chapter stands in the continuation of this spirit while developing and exploiting a novel technical toolbox.
To better highlight the novelty of our approach, we first present the two classical building blocks of PAC-Bayes.

\subsection{Understanding PAC-Bayes: a celebrated route of proof}
In the following subsection, we exploit again, for the sake of pedagagogy, the general pattern of proof for PAC-Bayes bounds described in \Cref{eq: catoni} to prove Catoni's bound. 
\subsubsection{Two essential building blocks for a preliminary bound}

For the rest of this section, similarly to \Cref{chap:intro-pac-bayes}, we assume access to a non-negative loss function $\ell(h,z)$ taking as argument a predictor $h\in\mathcal{H}$  and data $z\in\mathcal{Z}$ (think of $z$ as a pair input-output $(x,y)$ for supervised learning problems, or as a single datum $x$ in unsupervised learning). We also assume access to a $m$-sized sample $\Sm= (z_1,...,z_m)\in\mathcal{Z}^m$. $\Sm$ is then used to learn a posterior distribution $\Q$ on $\mathcal{H}$, from a prior $\P$.

PAC-Bayesian proofs are built upon two cornerstones. The first one is the change of measure inequality, recalled in \Cref{l: change_meas}.
This property is applied to a certain function $f_m: \mathcal{Z}^m \times \mathcal{H}\rightarrow \mathbb{R}$ of the data and a candidate predictor: for all posteriors $\Q$,
\begin{align}
\label{eq: change_meas_pacb-chap-2}
\mathbb{E}_{h\sim \Q}[f_m(\Sm,h)] \leq \KL(\Q,\P) + \log\left( \mathbb{E}_{h\sim \P}[\exp(f_m(\Sm,h))]  \right).
\end{align}
To deal with the random variable  $X(\Sm):=\mathbb{E}_{h\sim \P}[\exp(f_m(\Sm,h))] $, our second building block is Markov's inequality $\left(\mathbb{P}(X>a) \leq \frac{\mathbb{E}[X]}{a}\right)$ which we apply for a fixed $\delta\in (0,1)$ on $X(\Sm)$ with $a= \mathbb{E}_{\Sm}[X(\Sm)]/\delta$.
Taking the complementary event gives that for any $m$, with probability at least $1-\delta$ over the sample $\Sm$, $X(\Sm)\leq \mathbb{E}_{\Sm}[X(\Sm)]/\delta$, thus:
%\begin{lemma}
%\label{l: markov}
%For any non-negative integrable random variable $X$, the following holds for any $a>0$:

%\[ \mathbb{P}(X>a) \leq \frac{\mathbb{E}[X]}{a}.  \]
%\end{lemma}


\begin{equation}
  \label{eq: prelim_pb_bound-chap-2}
  \mathbb{E}_{h\sim \Q}[f_m(\Sm,h)] \leq \operatorname{KL}(\Q,\P) + \log(1/\delta) + \log\left( \mathbb{E}_{h\sim \P}\mathbb{E}_{\Sm}[\exp(f_m(\Sm,h))]  \right).
\end{equation}

\subsubsection{From preliminary to complete bounds}


From the preliminary result of \Cref{eq: prelim_pb_bound-chap-2}, there exists several ways to obtain PAC-Bayesian generalisation bounds, all being tied to specific choices of $f$ and the assumptions on the dataset $\Sm$. However, they all rely on the control of an exponential moment implied by Markov's inequality: this is a strong constraint which has been at the heart of the classical assumption appearing in PAC-Bayes learning.
For instance, McAllester's bound \eqref{eq: mcallester} and Catoni's bound \eqref{eq: catoni}, exploits in particular, a data-free prior, an \iid assumption on $\Sm$ and a light-tailed loss. Most of the existing results stand with those assumptions (see \emph{e.g.}, \citealp{catoni2007pac,germain2009pac,guedj2013pac,tolstikhin2013pac,guedj2018pac,mhammedi2019pac,wu2022split}).
Indeed, in many of these works, either a boundedness or a subgaussian assumption on the loss is used.  \citet{catoni2004statistical} extended PAC-Bayes learning to the subexponential case. Many works tried to mitigate at least one of the following three assumptions.

  \begin{itemize}
    \item \textbf{Data-free priors.} With an alternative set of techniques, \citet{catoni2007pac} obtained bounds with localised (\emph{i.e.}, data-dependent) priors. More recently, \citet{lever2010distribution,parradohernandez2012pac,lever2013tighter,oneto2016pac,dziugaite2017computing,mhammedi2019pac} also obtained PAC-Bayes bound with data-dependent priors.
    \item \textbf{The \iid assumption on $\Sm$.} The work of \citet{fard2010pac} established links between reinforcement learning and PAC-Bayes theory. This naturally led to the study of PAC-Bayesian bound for martingales instead of iid data \citep{seldin2011pac,seldin2012bandit,seldin2012pac}.
    \item \textbf{Light-tailed loss.} PAC-Bayes bounds for heavy-tailed losses (\emph{i.e.}, without subgaussian or subexponential assumptions) have been studied. \citet{audibert2011robust} provide PAC-Bayes bounds for least square estimators with heavy-tailed random variables. Their results was suboptimal with respect to the intrinsic dimension and was followed by further works from  \citet{catoni2016pac,catoni2016pac}.
    More recently, this question has been adressed in the works of \citet{alquier2018simpler, holland2019pac,kuzborskij2019efron,haddouche2021pac}, extending PAC-Bayes to heavy-tailed losses under additional technical assumptions.
  \end{itemize}





Several questions then legitimately arise.
\paragraph{Can we avoid these three assumptions simultaneously?}
The answer is yes: for instance the work of \citet{rivasplata2020pac} proposed a preliminary PAC-Bayes bound holding with none of the three assumptions listed above. Building on their theorem, \citet{haddouche2022online} only exploited a bounded loss assumption to derive a PAC-Bayesian framework for online learning, requiring no assumption on data and allowing data (history in their context)-dependent priors.

\paragraph{Can we obtain PAC-Bayes bounds without the change of measure inequality?} Yes, for instance \citet{alquier2018simpler} proposed PAC-Bayes bounds involving $f$-divergences and exploiting Holder's inequality instead of Lemma \ref{l: change_meas}. More recently, \citet{ohnishi2021novel,picard2022change} developed a broader discussion about generalising the change of measure inequality for a wide range of $f$-divergences. We note also that \citet{germain2009pac} proposed a version of the classical route of proof stated above avoiding the use of the change of measure inequality.
This comes at the cost of additional technical assumptions (see \citealp[][Theorem 1]{haddouche2021pac} for a statement of the theorem in a proper measure-theoretic framework).

\paragraph{Can we avoid Markov's inequality?}
We mentioned above that several works avoided the change of measure inequality to obtain PAC-Bayesian bounds, but can we do the same with Markov's inequality? This is of interest as avoiding Markov could avoid assumptions such as subgaussiannity to provide PAC-Bayes bound.
The answer is yes but this is a rare breed. To the best of our knowledge, only two papers are explicitly not using Markov's inequality: \citet{kakade2008complexity} obtained a PAC-Bayes bound using results on Rademacher complexity based on the McDiarmid concentration inequality, and \citet{kuzborskij2019efron} exploited a concentration inequality from \citet{delapena2009self}, up to a technical assumption to obtain results for unbounded losses. Both of those works do not require a bound on an exponential moment to hold.



\subsection{Originality of our approach}

Avoiding Markov's inequality appears challenging in PAC-Bayes but leads to fruitful results as those in \citet{kuzborskij2019efron}.

In this work, we exploit a generalisation of Markov's inequality for supermartingales: Ville's inequality (as noticed by \citealt{doob1939jean}). This result has, to our knowledge, never been used in PAC-Bayes before.

\begin{lemma}[Ville's maximal inequality for supermartingales]
\label{l: ville_ineq}
Let $\left(\mathcal{F}_{t}\right)$ be a filtration adapted to $\left(Z_{t}\right)$, a non-negative super-martingale with $Z_{0}=1$ almost surely, \ie $(Z_t)_{t\geq 1}$ is a discrete process such that for any $t\in\Nbb$, $\mathbb{E}\left[Z_{t} \mid \mathcal{F}_{t-1}\right] \leq Z_{t-1}$ a.s., $t \geq 1$, then, for any $0<\delta<1$, it holds
$$
\mathbb{P}\left(\exists T \geq 1: Z_{T}>\delta^{-1}\right) \leq \delta.
$$
\end{lemma}

\begin{proof}
  We apply the optional stopping theorem \citep[][Thm 4.8.4]{durrett2019probability} with Markov's inequality defining the stopping time $i=\inf \{t>1$ : $\left.Z_{t}>\delta^{-1}\right\}$ so that
  $$
  \mathbb{P}\left(\exists t \geq 1: Z_{t}>\delta^{-1}\right)=\mathbb{P}\left(Z_{i}>\delta^{-1}\right) \leq \mathbb{E}\left[Z_{i}\right] \delta \leq \mathbb{E}\left[Z_{0}\right] \delta \leq \delta.
  $$
\end{proof}

A major interest of Ville's result is that it holds for a countable sequence of random variables simultaneously. This point is new in PAC-Bayes and will allow us to obtain bounds holding for a countable (not necessarily finite) dataset $\S$.

\paragraph{On which supermartingale do we apply Ville's bound ?}

To fully exploit Lemma \ref{l: ville_ineq}, we now take a countable dataset $\S= (\z_i)_{i\geq 1}\in\mathcal{Z}^{\Nbb}$.
Recall that, because we use the change of measure inequality, we have to deal with the following exponential random variable appearing in \cref{eq: change_meas_pacb-chap-2} for any $m\geq 1$:
\[ Z_m:= \mathbb{E}_{h\sim \P}[\exp(f_m(\S,h))].   \]
Our goal is to choose a sequence of functions $f_m:\Zcal^\Nbb \times \Hcal \rightarrow \Rbb$ such that $(Z_m)_{m\geq 1}$ is a supermartingale. A way to do so comes from \citet{bercu2008exponential}.
\begin{lemma}[Towards the design of a supermartingale]
\label{l: bercu_touati}
Let $\left(M_{m}\right)$ be a locally square-integrable martingale with respect to the filtration $(\mathcal{F}_m)$. For all $\eta \in \mathbb{R}$ and $m \geq 0$, one has:
$$\mathbb{E}\left[\exp \left(\eta \Delta M_{m}-\frac{\eta^{2}}{2}\left(\Delta[M]_{m}+\Delta\langle M\rangle_{m}\right)\right) \mid \mathcal{F}_{m-1}\right] \leq 1,
$$
where $\Delta M_{m}=M_{m}-M_{m-1}, \Delta[M]_{m}=\Delta M_{m}^{2}$ and $\Delta\langle M\rangle_{m}=\mathbb{E}\left[\Delta M_{m}^{2} \mid \mathcal{F}_{m-1}\right]$.

We define $
V_{m}(\eta)=\exp \left(\eta M_{m}-\frac{\eta^{2}}{2}\left([M]_{m}+\langle M\rangle_{m}\right)\right) .
$
Then, for all $\eta \in \mathbb{R},\left(V_{m}(\eta)\right)$ is a positive supermartingale with $\mathbb{E}\left[V_{m}(\eta)\right] \leq 1$ where $[M]_{m}(h):=\sum_{i=1}^m \Delta[M]_{m}(h),
\langle M\rangle_{m}(h):=\sum_{i=1}^m\Delta\langle M\rangle_{m}(h).$
\end{lemma}
In the sequel, this lemma will be helpful to design a supermartingale (\emph{i.e.}, to choose a relevant $f_m$ for any $m$) without further assumption.

\subsection{Contributions and outline}

By avoiding Markov, a key message of \citep{kuzborskij2019efron} is that, for learning problems with independent data, PAC-Bayes learning only requires the control of order 2 moment on losses to be used with convergence guarantees. This is strictly less restrictive than the classical subgaussian/subgamma assumptions appearing in the major part of the literature.

We successfully prove this fact remains even for non-independent data: we only need to control order 2 (conditional) moments to perform PAC-Bayes learning. We focus in this chapter on the PAC-Bayesian framework for martingales \citep{seldin2011pac,seldin2012bandit,seldin2012pac}.
We then provide a novel PAC-Bayesian bound holding for data-free priors and unbounded martingales. From this, we recover in PAC-Bayes bounds for unbounded losses and iid data as a significant particular case. We also propose an extension of \citet{seldin2012bandit}'s result for multi-armed bandits.

More precisely, \Cref{sec: main_result_mart} contains our novel PAC-Bayes bound for unbounded martingales and \Cref{sec: iid_case} contains an immediate corollary for learning theory with iid data. 
We eventually apply our main result for martingales in \Cref{sec: bandit} to the setting of multi-armed bandit. Doing so, we provably extend a result of \citet{seldin2012bandit} to the case of unbounded rewards.

\Cref{sec: pac_b_background} gathers more details on PAC-Bayes,
we draw in \Cref{sec: extensions} a detailed comparison between our new results and a few classical ones.  We show that adapting our bounds to the assumptions made in those papers allows recovering similar or improved bounds.
We defer to \Cref{sec: proofs} the proofs of \Cref{sec: iid_case,sec: bandit}.





\section{A PAC-Bayesian bound for unbounded martingales}

\subsection{Main result}
\label{sec: main_result_mart}
A line of work led by \citet{seldin2011pac,seldin2012bandit,seldin2012pac} provided PAC-Bayes bounds for almost surely bounded martingales. We provably extend the remits of their result to the case of unbounded martingales.

\paragraph{Framework}

Our framework is close to the one of \cite{seldin2012bandit}: we assume having access to a countable dataset $\S= (\z_i)_{i\geq 1} \in$ with no restriction on the distribution of $\S$ (in particular the $\z_i$ can depend on each others). We denote for any $m$, $\Sm:= (\z_i)_{i=1..m}$ the restriction of $\S$ to its $m$ first points.
$(\mathcal{F}_i)_{i\geq 0}$ is a filtration adapted to $\S$.  We denote for any $i\in\mathbb{N}$ $\mathbb{E}_{i-1} [.] := \mathbb{E} [. \mid \mathcal{F}_{i-1}]$.
We also precise the space $\mathcal{H}$ to be an index (or a hypothesis) space, possibly uncountably infinite.
Let $\left\{X_1(\S_1,h), X_2(\S_2,h), \cdots\right.$ : $h \in \mathcal{H}\}$ be martingale difference sequences, meaning that for any $m\geq 1,h\in\mathcal{H}$, $\mathbb{E}_{m-1}\left[X_m(\Sm,h) \right]=0$.

For any $h\in\mathcal{H}$, let $M_m(h)=\sum_{i=1}^m X_i(S_i,h)$ be martingales corresponding to the martingale difference sequences and we define, as in \citet{bercu2008exponential}, the following
$$[M]_{m}(h):=\sum_{i=1}^m X_i(\S_i,h)^2,$$
$$\langle M\rangle_{m}(h)=\sum_{i=1}^m\mathbb{E}_{i-1}\left[X_{i}(\S_i,h)^{2}\right].$$
For a distribution $\Q$ over $\mathcal{H}$ define weighted averages of the martingales with respect to $\Q$ as $M_m(\Q)= \mathbb{E}_{h\sim \Q}\left[M_m(h)\right]$ (similar definitions hold for $[M]_m(\Q),\langle M\rangle_m (\Q)$).

\textbf{Main result.} We now present the main result of this section where we succesfully avoid the boundedness assumption on martingales. This relaxation comes at the cost of additional variance terms $[M]_m,\langle M \rangle_m$.

\begin{theorem}[A PAC-Bayesian bound for unbounded martingales]
\label{th: main_thm}
For any data-free prior $\P\in \mathcal{M}(\mathcal{H})$, any $\lambda>0$, any collection of martingales $(M_m(h))_{m\geq 1}$ indexed by $h\in\mathcal{H}$, the following holds with probability $1-\delta$ over the sample $\S=(\z_i)_{i\in\mathbb{N}}$, for all $m\in\mathbb{N}/\{0\}$, $\Q\in\mathcal{M}(\mathcal{H})$:
\[|M_m(\Q)| \leq   \frac{\operatorname{KL}(\Q,\P) +\log(2/\delta)}{\lambda } + \frac{\lambda}{2}\left([M]_m(\Q) + \langle M\rangle_m(\Q) \right).  \]
\end{theorem}
Proof lies in \Cref{sec: proof_main_thm}.

\textbf{Analysis of the bound.} This theorem involves several terms. The change of measure inequality introduces the KL divergence term, the approximation term $\log(2/\delta)$ comes from Ville's inequality (instead of Markov in classical PAC-Bayes). Finally, the terms $[M]_m(\Q), \langle M\rangle_m(\Q)$ come from our choice of supermartingale as suggested by \citet{bercu2008exponential}. The term $[M]_m(\Q)$ can be interpreted as an empirical variance term while $\langle M\rangle_m(\Q)$ is its theoretical counterpart. Note that $\langle M\rangle_m(\Q)$ also appears in \citet[][Theorem 1]{seldin2012bandit}.

We recall that this general result stands with no assumption on the martingale difference sequence $(X_i)_{i\geq 1}$ and holds uniformly on all $m\geq 1$. Those two points are, to the best of our knowledge, new within the PAC-Bayes literature. We discuss
in \Cref{sec: iid_case,sec: extensions} more concrete instantiations.

\textbf{Comparison with literature.} The closest result from \cref{th: main_thm} is the PAC-Bayes Bernstein inequality of \cite{seldin2012bandit}. Our bound is a natural extension of theirs as their result only involves the variance term (not the empirical one), but requires two additional assumptions:
\begin{enumerate}
  \item Bounded variations of the martingale difference sequence: $\forall m, \exists C_m\in\mathbb{R}^2$ such that a.s. for all $h$ $| X_m(\Sm,h)| \leq C_m$.
  \item Restriction on the range of the $\lambda$: $\forall m, \lambda_m \leq 1/C_m $.
\end{enumerate}
 \citet{seldin2012bandit} need those assumptions to ensure the \emph{Bernstein assumption} which states that for any $h$,
 $\mathbb{E}[\exp(\lambda M_m(h) - \frac{\lambda^2}{2} \langle M \rangle_m(h))] \leq 1$.
 Our proof technique do not require the Bernstein assumption (and so none of the two conditions described above, which allow us to deal with unbounded martingales) as we exploit the supermartingale structure to obtain our results. More precisely, the price to pay to avoid the Bernstein assumption is to consider the empirical variance term $[M]_m(h)$ and to prove that $\left( \exp \left(\lambda  M_{m}-\frac{\lambda^{2}}{2}\left([M]_{m}+\langle M\rangle_{m}\right)\right)  \right)_{m\geq 1} $
 is a supermartingale using \Cref{l: ville_ineq} and \Cref{l: bercu_touati} (see \Cref{sec: proof_main_thm} for the complete proof).
 A broader discussion is detailed in \cref{sec: extensions}.

\subsection{Proof of \Cref{th: main_thm}}
\label{sec: proof_main_thm}

\begin{proof}[of \Cref{th: main_thm}]
We fix $\eta\in\mathbb{R}$, and we consider the function $f_m$ to be for all $(\S,h)$:
\begin{align*}
f_m(\S,h) & := \eta M_m(h) - \frac{\eta^2}{2} \left( [M]_m(h) + \langle M \rangle_m(h)   \right)\\
& = \sum_{i=1}^m \eta\Delta M_i(h)  - \frac{\eta^2}{2}(\Delta[M]_i(h) + \Delta \langle M\rangle_i(h)),
\end{align*}
where  $\Delta M_{i}(h)=X_i(\S_i,h), \quad\Delta[M]_{i}(h)=X_i(\S_i,h)^{2}, \quad\Delta\langle M\rangle_{i}(h)=\mathbb{E}_{i-1}\left[X_{i}(\S_i,h)^{2}\right]$.
For the sake of clarity, we dropped the dependency in $\S$ of $M_m$. Note that, given the definition of $M_m$, $M_m(h)$ is $\mathcal{F}_{m}$ measurable for any fixed $h$.

Let $\P$ a fixed data-free prior, we first apply the change of measure inequality to obtain $\forall m\in\mathbb{N},\forall \Q\in \mathcal{M}(\mathcal{H})$:
\[  \mathbb{E}_{h\sim \Q}[f_m(\S,h)] \leq \operatorname{KL}(\Q,\P) + \log\left(\underbrace{\mathbb{E}_{h\sim \P} \left[ \exp (f_m(\S,h))   \right] }_{:= Z_m} \right),   \]
with the convention $f_0= 0$.
We now have to show that $(Z_m)_m$ is a supermartingale with $Z_0=1$. To do so remark that for any $m$, because $\P$ is data free one has the following result.
\begin{lemma}
\label{l: cond_fubini}
For any data-free prior $\P$, any $\sigma$-algebra $\mathcal{F}$ belonging to the filtration $(\mathcal{F}_i)_{i\geq 0}$, any nonnegative function  $f$ taking as argument the sample $\S$ and a predictor $h$, one has almost surely: $$\mathbb{E}\left[\mathbb{E}_{h\sim \P}[f(\S,h)] \mid \mathcal{F}\right]  =  \mathbb{E}_{h\sim \P}\left[\mathbb{E}[f(\S,h) \mid \mathcal{F}] \right].$$
\end{lemma}
\begin{proof}[of \Cref{l: cond_fubini}]
Let $A$ be a $\mathcal{F}$-measurable event. We want to show that
\[ \mathbb{E}\left[\mathbb{E}_{h\sim \P} [f(\S,h)]\mathds{1}_A \right] = \mathbb{E}\left[\mathbb{E}_{h\sim \P} \left[ \mathbb{E}[f(\S,h)\mid \mathcal{F}] \right] \mathds{1}_A \right], \]
where the first expectation in each term is taken over $\S$. Note that it is possible to take this expectation thanks to the Kolomogorov's extension theorem \citep[see \emph{e.g.}][Thm 2.4.4]{tao2011introduction} which ensure the existence of a probability space for the discrete-time stochastic process $\S=(\z_i)_{i\geq 1}$.

Thus, this is enough to conclude that
\[ \mathbb{E}\left[\mathbb{E}_{h\sim \P} [f(\S,h)]\mid \mathcal{F} \right] = \mathbb{E}_{h\sim \P} \left[ \mathbb{E}[f(\S,h)\mid \mathcal{F}] \right],   \]
by definition of the conditional expectation.
To do so, notice that because $f(\S,h)\mathds{1}_A $ is a nonnegative function, and that $\P$ is data-free, we can apply the classical Fubini-Tonelli theorem.
\begin{align*}
\mathbb{E}\left[\mathbb{E}_{h\sim \P} [f(\S,h)]\mathds{1}_A \right]
&=  \mathbb{E}_{h\sim \P} \left[ \mathbb{E}\left[f(\S,h)\mathds{1}_A\right] \right]. \\
\intertext{One now conditions by $\mathcal{F}$ and use the fact that $\mathds{1}_A$ is $\mathcal{F}$-measurable:  }
&= \mathbb{E}_{h\sim \P} \left[ \mathbb{E}\left[\mathbb{E}\left[f(\S,h)\mid \mathcal{F}\right]\mathds{1}_A\right] \right]. \\
\intertext{We finally re-apply Fubini-Tonelli to re-intervert the expectations: }
&= \mathbb{E}\left[ \mathbb{E}_{h\sim \P}\left[\mathbb{E}\left[f(\S,h)\mid \mathcal{F}\right]\mathds{1}_A\right] \right].
\end{align*}
This concludes the proof of Lemma \ref{l: cond_fubini}.
\end{proof}
We then use Lemma \ref{l: cond_fubini} with $f=\exp(f_m)$ and $\mathcal{F}=\mathcal{F}_{m-1}$ to obtain:
\begin{align*}
\mathbb{E}_{m-1}[Z_m] & =  \mathbb{E}_{h\sim \P}\left[\mathbb{E}_{m-1}[(\exp (f_m(\S,h))] \right]\\
&= \mathbb{E}_{h\sim \P}\left[ \exp (f_{m-1}(\S,h)) \mathbb{E}_{m-1}\left[ \exp( \eta\Delta M_m(h)  - \frac{\eta^2}{2}(\Delta[M]_m
(h) + \Delta \langle M\rangle_m(h)) \right]   \right],
\end{align*}
with $f_{m-1}(\S,h) = \sum_{i=1}^{m-1} \eta(\Delta M_i(h) ) - \frac{\eta^2}{2}(\Delta[M]_i(h) + \Delta \langle M\rangle_i(h))$.
Using Lemma \ref{l: bercu_touati} ensures that for any $h$, $$\mathbb{E}_{m-1}[ \exp( \eta\Delta M_m(h)  - \frac{\eta^2}{2}(\Delta[M]_m(h) + \Delta \langle M\rangle_m(h)) ] \leq 1,$$
thus we have
\begin{align*}
\mathbb{E}_{m-1}[Z_m] & \leq  \mathbb{E}_{h\sim \P}\left[ \exp (f_{m-1}(\S,h))   \right]  = Z_{m-1}.
\end{align*}
Thus $(Z_m)_m$ is a nonnegative supermartingale with $Z_0=1$. We can use Ville's inequality (Lemma \ref{l: ville_ineq}) which states that
$$
\mathbb{P}_S\left(\exists m \geq 1: Z_{m}>\delta^{-1}\right) \leq \delta.
$$
Thus, with probability $1-\delta$ over $\S$, for all $m\in\mathbb{N}, Z_m \leq 1/\delta $.
We then have the following intermediary result. For all $\P$ a data-free prior, $\eta\in\mathbb{R}$, with probability $1-\delta$ over $\S$, for all $m>0, \Q\in\mathcal{M}(\mathcal{H})$
\begin{align}
\label{eq: intermediary_result_main}
\eta M_m(\Q)  \leq  \operatorname{KL}(\Q,\P) +\log(1/\delta) + \frac{\eta^2}{2}\left( [M]_m(\Q) + \langle M \rangle_m(\Q)  \right) ],
\end{align}
recalling that $M_m(\Q)= \mathbb{E}_{h\sim \Q}[M_m(h)]$, and that similar definitons hold for $[M]_m(\Q), \langle M \rangle_m(\Q)$.
Thus, applying the bound with $\eta= \pm\lambda$ ($\lambda>0$) and taking an union bound gives, with probability $1-\delta$ over $\S$, for any $m\in\mathbb{N}$, $\Q\in\mathcal{M}(\mathcal{H})$
\[ \lambda\left|M_m(\Q)\right| \leq  \operatorname{KL}(\Q,\P) + \log(2/\delta) + \frac{\lambda^2}{2}\left( [M]_m(\Q) + \langle M \rangle_m(\Q)  \right) ].  \]
Dividing by $\lambda$ concludes the proof.
\end{proof}


\subsection{A corollary: Batch learning with iid data and unbounded losses}
\label{sec: iid_case}

In this section, we instantiate \Cref{th: main_thm} onto a learning theory framework with iid data. We show that our bound encompasses several results of literature as particular cases.
\paragraph{Framework} We consider a \emph{learning problem} specified by a tuple $(\mathcal{H}, \mathcal{Z}, \ell)$ consisting of a set $\mathcal{H}$ of predictors, the data space $\mathcal{Z}$, and a loss function $\ell : \mathcal{H}\times \mathcal{Z} \rightarrow \mathbb{R}^{+} $.
We consider a countable dataset $\S=(\z_i)_{i\geq 1}\in\mathcal{Z}^{\mathbb{N}}$ and assume that sequence is \iid following the distribution $\D$. We also denote by $\mathcal{M}(\mathcal{H})$ is the set of probabilities on $\mathcal{H}$.
\paragraph{Definitions} Similarly to \Cref{chap:intro-pac-bayes}, the \emph{population risk} $\Risk$ of a predictor $h\in\mathcal{H}$ is $\forall h, \Risk(h)= \mathbb{E}_{\z\sim\D}[\ell(h,\z)]$, the \emph{empirical error} of $h$ is  $\forall h, \Riskhat_{\Sm}(h)= \frac{1}{m}\sum_{i=1}^m\ell(h,z_i)$
and finally the \emph{quadratic generalisation error} $V$ of $h$ is $\forall h, Quad(h)= \mathbb{E}_{\z\sim\D}[\ell(h,z)^2]$.
We also denote by \emph{generalisation gap} for any $h$ the quantity $\Risk(h)-\Riskhat_{\Sm}(h)$.
\medskip

\textbf{Main result.} We now state the main result of this section. This bound is a corollary of \Cref{th: main_thm} and fills the gap with learning theory.
\begin{theorem}[A PAC-Bayes bound for batch learning with heavy-tailed losses]
\label{th: main_thm_iid}
For any data-free prior $P\in \mathcal{M}(\mathcal{H})$, any $\lambda>0$ the following holds with probability $1-\delta$ over the sample $\S=(\z_i)_{i\in\mathbb{N}}$, for all $m\in\mathbb{N}/\{0\}$, $Q\in\mathcal{M}(\mathcal{H})$
\begin{multline*}\mathbb{E}_{h\sim \Q} [\Risk(h)] \leq   \mathbb{E}_{h\sim \Q} \left[\Riskhat_{\Sm}(h) + \frac{\lambda}{2m}\sum_{i=1}^m \ell(h,z_i)^2\right] \\
  + \frac{\operatorname{KL}(\Q,\P) +\log(2/\delta)}{\lambda m} + \frac{\lambda}{2}\mathbb{E}_{h\sim \Q} [\mathrm{Quad}(h)]. 
\end{multline*}
\end{theorem}

Proof is furnished in \Cref{sec: proofs}.
\paragraph{About the choice of $\lambda$.} A novelty in this theorem is that the bound holds \emph{simultaneously on all $m>0$} -- this is due to the use of Ville's inequality.
This sheds a new light on the choice of $\lambda$. Indeed, taking a localised $ \lambda$ depending on a given sample size (e.g. $\lambda_m= 1/\sqrt{m}$) ensures convergence guarantees for the expected generalisation gap. Doing so, our bound matches the usual PAC-Bayes literature (i.e. a bound holding with high probability for a single $m$). However the novelty brought by \Cref{th: main_thm_iid} is that our bound holds for unbounded losses for all times simultaneously. This suggests that taking a sample size-dependent $\lambda$ may not be the best answer. We detail an instance of this fact below when one thinks of $\lambda$ as a parameter of an optimisation objective.
Indeed, our bound suggests a new optimisation objective for unbounded losses which is for any $m>0$:
\begin{equation}
  \label{eq: optim_obj}
  \operatorname{argmin_{Q}}  \mathbb{E}_{h\sim \Q} \left[\frac{1}{m}\sum_{i=1}^m\left(\ell(h,z_i) + \frac{\lambda}{2} \ell(h,z_i)^2\right)\right] + \frac{\operatorname{KL}(\Q,\P)}{\lambda m}.
\end{equation}
\Cref{eq: optim_obj} differs from the classical objective of \citet[][Thm 1.2.6]{catoni2007pac} (described in \eqref{eq: alg-catoni}) on the additional quadratic term $\frac{\lambda}{2} \ell(h,z_i)^2$. Note that this objective implies a bound on the theoretical order 2 moment to be meaningful as we do not include it in our objective. Note that this constraint is less restrictive than Catoni's objective which requires a bounded loss.  This objective stresses the role of the parameter $\lambda$ as being involved in a new explicit trade-off between the KL term and the efficiency on training data.

Also, this optimisation objective is valid for any sample size $m$, this means that our $\lambda$ should not depend on certain dataset size but should be fixed in order to ensure a learning algorithm with generalisation guarantees at all time. This draws a parallel with Stochastic Gradient Descent with fixed learning step.


\textbf{About the underlying assumptions in this bound.} Our result  is empirical (all terms can be computer or approximated) at the exception of the term $\mathbb{E}_{h\sim \Q} [\mathrm{Quad}(h)]$. This invites to choose carefully the class of posteriors, in order to bound this second-order moment with minimal assumptions.
For instance, if we consider the particular case of the quadratic loss $\ell(h,z)= (h-z)^2$, then we only need to assume that our data have a finite variance if we restrict our posteriors to have both bounded means and variance. This assumption is strictly less restrictive than the classical subgaussian/subgamma assumption classically appearing in the literature.

\textbf{Comparison with literature.} Back to the bounded case, we note that instantiating the boundedness assumption in \cref{th: main_thm_iid} make us recover the result of \citet[][Theorem 4.1]{alquier2016properties} for the subgaussian case. We also remark that instantiating the HYPE condition \citet[][Theorem 3]{haddouche2021pac} allow us to improve their result as we transformed the control of an exponential moment into one on a second-order moment. More details are gathered in \Cref{sec: extensions}.
We also compare \Cref{th: main_thm_iid} to \citet[][Theorem 3]{kuzborskij2019efron} which is a PAC-Bayes bound for unbounded losses obtained through a concentration inequality from \citet{delapena2009self}. They arrived to what they denote as semi-empirical inequalities which also involve empirical and theoretical variance terms (and not an exponential moment). Their bound holds for independent data and a single posterior.
First, note that \Cref{th: main_thm_iid} holds for any posterior, which is strictly more general. Note also that our bound is a straightforward corollary of \Cref{th: main_thm} which holds for any martingale (thus for any data distribution in a learning theory framework) and so, exploits a different toolbox than \citet{kuzborskij2019efron} (control of a supermartingale vs. concentration bounds for independent data). We insist that a fundamental novelty in our work is to extend the conclusion of \cite{kuzborskij2019efron} to the case of non-independent data: it is possible to perform PAC-Bayes learning for unbounded losses at the expense of the control of second-order moments.
Note also that their bound is slightly tighter than ours as their result is \Cref{th: main_thm_iid} being optimised in $\lambda$ (which is something we cannot do as the resulting $\lambda$ would be data-dependent).







\section{Application to the multi-armed bandit problem}
\label{sec: bandit}

We exploit our main result in the context of the multi-armed bandit problem -- we adopt the framework of \citet{seldin2012bandit}.

\paragraph{Framework.} Let $\mathcal{A}$ be a set of actions of size $|\mathcal{A}|=K<+\infty$ and $a\in\mathcal{A}$ be an action. At each round $i$, the environment furnishes a reward function $R_i:\mathcal{A}\rightarrow \mathbb{R}$ which associate a reward $R_i(a)$ to the arm $a$. Assuming the $R_i$s are iid, we denote for any $a$, the \emph{expected reward for action $a$} to be $R(a)= \mathbb{E}_{R_1}[R_1(a)]$.
At each round $i$, the player executes an action $A_i$ according to a policy $\pi_i$. We then set the filtration $(\mathcal{F}_i)_{i\geq 1}$ to be $\mathcal{F}_i = \sigma\left( \{\pi_j,A_j,R_j \mid 1\leq j\leq m\}    \right)$.


\paragraph{Assumptions.} We suppose here that $(R_i)_{i\geq 1}$ is an iid sequence and that at each time $i$, $A_i$ and $R_i$ are independent and that $\pi_i$ is $\mathcal{F}_{i-1}$ measurable. This means that the player is not aware of the rewards at each round and performs its current move \wrt the past.

We also add two technical assumptions. First, the order two moment of the expected reward is uniformly bounded: $\sup_{a\in\mathcal{A}} \mathbb{E}_{R_1}[R_1(a)^2] \leq C$. This assumption is strictly less restrictive than the boundedness assumption made in \cite{seldin2012bandit}. Similarly to this work, we also assume that there exists a sequence $(\varepsilon_i)_{i\geq 1}$ such that $\inf_{a\in\mathcal{A}} \pi_i(a) \geq \varepsilon_i$.
We say that $(\pi_i)_{i\geq 1}$ is \emph{bounded from below by} $(\varepsilon_i)_{i\geq 1}$.


\paragraph{Definitions.}
For $i \geq 1$ and $a \in\{1, \ldots, K\}$, define a set of random variables $(R_i^a)_{i\geq 1}$ (\emph{the importance weighted samples}, \citealp{sutton2018reinforcement})
$$
R_i^a:=\left\{\begin{array}{cl}
\frac{1}{\pi_i(a)} R_i, & \text { if } A_i=a, \\
0, & \text { otherwise. }
\end{array}\right.
$$
We define for any time $m$:
$
\hat{R}_m(a)=\frac{1}{m} \sum_{i=1}^t R_i^a .
$
Observe that for all $i$, $\mathbb{E}\left[R_i^a \mid \mathcal{F}_{i-1}\right]=R(a)$ and $\mathbb{E}[\hat{R}_m(a)]=R(a)$.
Let $a^*$ be the "best" action (the action with the highest expected reward, if there are multiple "best" actions pick any of them). Define the \emph{expected and empirical per-round regrets} as
$$
\begin{aligned}
\Delta(a) =R\left(a^*\right)-R(a), \quad \hat{\Delta}_m(a) =\hat{R}_m\left(a^*\right)-\hat{R}_m(a) .
\end{aligned}
$$
Observe that $m\left(\hat{\Delta}_m(a)-\Delta(a)\right)$ forms a martingale. Let
$$
V_m(a)=\sum_{i=1}^m \mathbb{E}\left[\left(R_i^{a^*}-R_i^a-\left[R\left(a^*\right)-R(a)\right]\right)^2 \mid \mathcal{F}_{i-1}\right]
$$
be the cumulative variance of this martingale and
$$
\hat{V}_m(a)=\sum_{i=1}^m \left(R_i^{a^*}-R_i^a-\left[R\left(a^*\right)-R(a)\right]\right)^2
$$
its empirical counterpart. We denote for any distribution $\Q$ over $\mathcal{A}$, $\Delta(\Q) = \mathbb{E}_{a\sim \Q}[\Delta(a)]$, $V_m(\Q)= \mathbb{E}_{a\sim \Q}[V_m(a)]$, similar definitions hold for $\hat{\Delta}_m(\Q),\hat{V}_m(\Q)$.
We can now state the main result of this section -- its proof is deferred to \Cref{sec: proofs}.

\begin{theorem}[PAC-Bayes bounds for heavy-tailed rewards]
\label{th: bandits_bound}
For any $m\geq 1$, any history-dependent policy sequence $(\pi_i)_{i\geq 1}$ bounded from below by $(\varepsilon_i)_{i\geq 1}$, we have with probability $1-\delta$, for all posterior $\Q$
\begin{align*}
\left| \Delta(\Q) - \hat{\Delta}_m(\Q)  \right| & \leq 2\sqrt{\frac{\left(1+ \frac{2K}{\delta} \right)\left( \log(K) + \log(4/\delta) \right)}{ m \varepsilon_m}}.
\end{align*}
\end{theorem}
To the best of our knowledge, this result is the first PAC-Bayesian guarantees for multi-armed bandits with unbounded rewards. The proposed bound is as tight as Theorem 2.3 of \citet{seldin2012bandit}, up to a factor $(e-2)$ transformed into $\left(1+ \frac{2K}{\delta}\right)$ (which is a huge dependency in $K$) within the square root.
Note that our result comes at the price of the localisation: Theorem 2.3 of \citet{seldin2012bandit} proposes a bound holding uniformly for all time $m$ while our approach only holds for a single time $m$.


We believe there is room for improvement in \cref{th: bandits_bound}. Indeed, the current approach is naive as it consists in bounding crudely with high probability the empirical variance. Such a naive trick impeach us to consider all times simultaneously. Indeed, in its current form, taking an union bound on \Cref{th: bandits_bound} is costful as we have a dependency in $1/\delta$ in our result (instead of $\log(1/\delta)$ in \citealp{seldin2012bandit}): this would destroy the convergence rate. The question of dealing more subtly with the empirical variance term is left as an open question.

\section{Conclusion}

\textbf{A first step towards an optimisation perspective of PAC-Bayes} We showed that it is possible to generalise the PAC-Bayes toolbox to unbounded martingales and heavy-tailed losses (resp. learning problem with unbounded losses for batch/online learning), the solely implicit assumption being the existence of second order moments on the martingale difference sequence (resp. on the loss function) which is reasonable as many PAC-Bayes bound lies on assumptions on exponential moments (\emph{e.g.} the subgaussian assumption) to work.

\textbf{Current Limitations.} Doing so, we made a first step towards concrete optimisation perspective of PAC-Bayes by showing generalisation bounds are attainable with weak statistical assumptions and thus, compatible with many practical settings where optimisation is performed. However, \Cref{chap: pb-ht} still presents some strong links with the information-theoretic approach such as: \textit{(i)} the presence of a prior $\P$ in \Cref{th: main_thm_iid} which does no fit the optimisation views of the prior (see \Cref{fig: recap-optim}), and \textit{(ii)} the presence of a KL divergence, suggesting an information-theoretic perspective of learning. Point \textit{(i)} will be later developed in \Cref{chap:online-pb,chap:gen-flat-minima,chap: wpb-practical} when $\P$ is seen as an initialisation point and in \Cref{chap: wass-pb} when $\P$ is the learning objective. \textit{(ii)} will be later developed in \Cref{chap: wass-pb,chap: wpb-practical}. 

\textbf{Extensions of this work.} The supermartingale framework presented here are extracted from \citet{haddouche2023pac} and has inspired many follow-up works. \citet{chugg2023unified} extended the approach of this chapter to other supermartingales as well as reversed submartingales, allowing to recover a vast majority of existing PAC-Bayes literature, also, \citet{rodriguez2023more} tightened the theorems presented here by allowing the optimisation in $\lambda$. The tools presented in this work (\eg Ville's inequality) are also useful to obtain fast rate PAC-Bayes bounds based on the coin-betting approach \citet{jang2023tight,kuzborskij2024better}. The coin-betting approach originally in online learning \citep{orabona2016coin}. In \Cref{chap:online-pb}, we take a deeper focus on online learning, showing that an online approach of PAC-Bayes is possible, and allows to consider prior distribution as an initialisation point of a learning algorithm.
  
\chapter[Mitigating Initialisation Impact by Real-Time Control: Online PAC-Bayes Learning]{Mitigating Initialisation Impact by Real-Time Control: Online PAC-Bayes Learning}
\label{chap:online-pb}

\addchapterlof
\addchapterloa
\addchapterloe

\vspace{-1.0cm}
\begin{center}
\textbf{This chapter is based on the following papers}\\[0.1cm]
\end{center}
\printpublication{haddouche2022online}
\\
\printpublication{haddouche2023pac}
\\
\printpublication{viallard2023learning}

\vspace{0.2cm}
\minitoc

\begin{abstract}
Put OPB here. Precise in the intro that the martingale bounds allow to go beyond batch learning but that this has never been made for OL. Put the supermartingale OPB bound in a supplementary section and the Online WPB bound after the main results of OPB to reach heavy-tailed losses. 
\end{abstract}


\section{Online PAC-Bayes learning beyond bounded losses.}
\label{sec: main_result_onl}

Recently, an online learning framework has been designed in \citet{haddouche2022online}. This allowed the design of Online PAC-Bayes (OPB) algorithms which involved the use of history-dependent priors evolving at each time step of the learning procedure. The main contribution of this section is an OPB bound valid for unbounded losses.

\paragraph{Framework} We consider the same framework as in \Cref{sec: iid_case} except we do not make any assumption on the data distribution. Our goal is now to define a posterior sequence $(\Q_i)_{i\geq 1}$ from a prior sequence $(\P_i)_{i\geq 1}$. We also define a filtration $(\mathcal{F}_{i})_{i\geq 1}$ adapted to $(z_i)_{i\geq 1}$. We reuse the following definitions extracted from \cite{haddouche2022online}.

\paragraph{Definitions} For all $i$, we denote by $\mathbb{E}_{i}[.]$ the conditional expectation $\mathbb{E}[.\mid \mathcal{F}_i]$.

A \emph{stochastic kernel} from $\cup_{m=1}^\infty\mathcal{Z}^m$ to $\mathcal{H}$ is defined as a mapping $Q: \cup_{m=1}^\infty\mathcal{Z}^m\times \Sigma_{\mathcal{H}} \rightarrow [0,1]$ where
(i) For any $B\in \Sigma_{\mathcal{H}}$, the function  $S\mapsto Q(S,B)$ is measurable,  (ii) For any $\S$, the function $B\mapsto Q(S,B)$ is a probability measure over $\mathcal{H}$.


We also say that a sequence of stochastic kernels $(P_i)_{i\geq 1}$ is an \emph{online predictive sequence} if (i) for all $i\geq 1, S\in\cup_{m=1}^\infty\mathcal{Z}^m, P_i(S,.)$ is $\mathcal{F}_{i-1}$ measurable and (ii) for all $i \geq 2$, $P_i(S,.)\gg P_{1}(S,.)$.

\textbf{Main result.} We now state the main theorem of this section, which extends the remits of the Online PAC-Bayes framework to the case of unbounded losses.

\begin{theorem}
  \label{th: main_thm_onl}
  For any distribution over the (countable) dataset $\S$, any $\lambda>0$ and any online predictive sequence (used as priors) $(P_i)_{i\geq 1}$, we have with probability at least $1-\delta$ over the sample $S\sim\mu$, the following, holding for the data-dependent measures $P_{i,S}:= P_i(S,.)$ any posterior sequence $(Q_i)_{i\geq 1}$ and any $m\geq 1$:

  \begin{multline*}
     \sum_{i=1}^m \mathbb{E}_{h_i\sim Q_{i}}\left[ \mathbb{E}[\ell(h_i,z_i) \mid \mathcal{F}_{i-1}]    \right]  \leq \sum_{i=1}^m \mathbb{E}_{h_i\sim Q_{i}}\left[ \ell(h_i,z_i) \right] +\frac{\lambda}{2}\sum_{i=1}^m \mathbb{E}_{h_i\sim Q_i}\left[ \hat{V}_i(h_i,z_i) + V_i(h_i) \right] \\
     + \sum_{i=1}^m\frac{\operatorname{KL}(Q_{i}\| P_{i,S})}{\lambda}  + \frac{\log(1/\delta)}{\lambda}.
  \end{multline*}
  With for all $i$, $\hat{V}_i(h_i,z_i)= (\ell(h_i,z_i)-\mathbb{E}_{i-1}[\ell(h_i,z_i)])^2$ is the empirical variance at time $i$ and $V_i(h_i)= \mathbb{E}_{i-1}[\hat{V}(h_i,z_i)]$ is the true conditional variance.
\end{theorem}

Proof lies in \Cref{sec: proof_main_thm_online}.

\textbf{Analysis of the bound.} This bound is, to our knowledge, the first Online PAC-Bayes bound in literature holding for unbounded losses. It is semi-empirical as the variance and empirical variance terms have theoretical components. However, these terms can be controlled with assumptions on conditional second-order moments and not on exponential ones (as made in \citealp{haddouche2022online} where the bounded loss assumption was used to obtain conditional subgaussianity). To emphasise our point, we consider as in \Cref{sec: iid_case} the case of the quadratic loss $\ell(h,z)= (h-z)^2$. Here, we only need to assume that our data have a finite variance if we restrict our posteriors to have both bounded means and variance. Also the meaning of the online predictive sequence $P_i$ is that we must be able to design properly a sequence of priors before drawing our data, this can be for instance an online algorithm whihc generate a prior distribution from past data at each time step.

Finally, we note that if we assume being able to bound simultaneaously all condtional means and variance (which is strictly less restrictive than bounding the loss),then  \cref{th: main_thm_onl} suggests a new online learning objective which is an online counterpart to \Cref{eq: optim_obj}.

\begin{align}
    \forall i\geq1\; \hat{Q}_{i+1}&= \underset{Q\in\mathcal{M}^+_1(\mathcal{H})}{\mathrm{argmin}} \mathbb{E}_{h_i\sim Q} \; \left[\ell(h_i,z_i)+ \frac{\lambda}{2}\ell(h_i,z_i)^2\right] + \frac{\operatorname{KL}(Q\| P_{i,S})}{\lambda}
\end{align}

\textbf{Comparison with literature.} Our most natural comparison point is Theorem 2.3 of \cite{haddouche2022online} (re-stated in \cref{sec: pac_b_background}). We claim that \Cref{th: main_thm_onl} is a strict improvement of their result on various sides described below.

\begin{itemize}
  \item If we assume our loss to be bounded, then we can upper bound our empirical/theoretical variance terms to recover exactly \citet[][Theorem 2.3]{haddouche2022online}. Our bound can then be seen as a strict extension of theirs and shows that bounding order two moments is a sufficient condition to perform online PAC-Bayes: subgaussianity induced by boundedness is not necessary even when our data are non iid.
  \item Another crucial point lies on the range of our result which holds with high probability for any countable posterior sequence $(Q_i)_{i\geq 1}$, any time $m$ and the priors $(P_{i,S})_{i\geq 1}$.
  This is far much general than \citet[][Theorem 2.3]{haddouche2022online} which holds only for a single $m$ and a single posterior sequence $(Q_{i,S})_{i=1..m}$. This happens because in \citet{haddouche2022online}, the change of measure inequality has not been exploited: they used a preliminary theorem from \citet{rivasplata2020pac} which holds for a single (data-dependent) prior/posterior couple. This preliminary theorem already involved Markov's inequality which forced the authors to assume conditionnal subgaussianity to deal with an exponential moment. On the contrary, we exploited the fact that our online predictive sequence was history-dependent to use the change of measure inequality at any time step and control an exponential supermartingale through Ville's inequality.
  \item In \citet[Eq. 1]{haddouche2022online}, an OPB algorithm is given by their upper bound. This works because their associated learning objective admits a close form (Gibbs posterior) which matches the fact their bound hold for a single posterior sequence. Because our bound holds uniformly on all posteriors, it is now legitimate to restrict their algorithms to any parametric class of distributions and perform any optimisation algorithm to obtain a surrogate of the best candidate.
\end{itemize}

 Online PAC-Bayes as presented in \citet{haddouche2022online} relies on a conditional subgaussiannity assumption to control an exponential moment. They did not exploit a martingale-type structure to do so. Our supermartingale approach has proven to be well suited to Online PAC-Bayes as we provided atheorem valid for unbounded losses holding simultaneously on all posteriors: two points which have not been reached in \citet{haddouche2022online}.

\subsection{Proof of \Cref{th: main_thm_onl}}
\label{sec: proof_main_thm_online}
 \begin{proof}
   We fix $m\geq 1$, $\S$ a countable dataset and $(P_i)_{i\geq 1}$ an online predictive sequence. We aim to design a $m$-tuple of probabilities. Thus, our predictor set of interest is $\mathcal{H}_m:= \mathcal{H}^{\otimes m}$ and then, our predictor $h$ is a tuple $(h_1,..,h_m)\in\mathcal{H}$.

   Our goal is to apply the change of measure inequality on $\mathcal{H}_m$ to a specific function $f_m$ inspired from Lemma \ref{l: bercu_touati}. We define this function below, for any sample $\S$ and any predictor $h^m=(h_1,...,h_m)$

   \begin{align*}
   f_m(S,h^m) & := \sum_{i=1}^m \lambda X_i(h_i,z_i)  - \frac{\lambda^2}{2}\sum_{i=1}^m(\hat{V}_i(h_i,z_i) + V_i(h_i)),
   \end{align*}
   where $X_i(h_i,z_i)= \mathbb{E}_{i-1}[\ell(h_i,z_i)]- \ell(h_i,z_i)$. Notice that for fixed $h$, the sequence $(f_m(\S,h))_{m\geq 1}$ is a supermartingale according to Lemma \ref{l: bercu_touati}.

   Now for a given posterior tuple $Q_1,...Q_m$ we define $Q= Q_1 \otimes ...\otimes Q_m$ and also $P^m_S = P_{1,S}\otimes...\otimes P_{m,S}$. We can now properly apply the change of measure inequality for any $m$:
   \begin{align*}
    \sum_{i=1}^m \mathbb{E}_{h_i\sim Q_i}[\lambda X_i(h_i,z_i)  - \frac{\lambda^2}{2}(\hat{V}_i(h_i,z_i) + V_i(h_i))] & = \mathbb{E}_{h^m\sim Q}\left[ f_m(S,h^m) \right] \\
    & \leq \operatorname{KL}(Q,P^m_S) + \log \left( \mathbb{E}_{h^m\sim P^m_S}\exp(f_m(S,h^m))  \right).
   \end{align*}

   Noticing that $\operatorname{KL}(Q,P^m_S)= \sum_{i=1}^m \operatorname{KL}(Q_i,P_{i,S})$, the only remaining term to deal with is the exponential rv.

   To do so we prove the following lemma:

   \begin{lemma}
     The sequence $(M_m:=\mathbb{E}_{h^m\sim P^m_S}\exp(f_m(S,h^m))_{m\geq 1}$ is a non-negative supermartingale.
   \end{lemma}
   \begin{proof}
   We fix $m\geq 1$ and we recall that for any $i$, $P_{i,S}$ is $\mathcal{F}_{i-1}$-measurable. We show that $\mathbb{E}_{m-1}[M_m] \leq M_{m-1}$. We first recover $M_{m-1}$ from $\mathbb{E}_{m-1}[M_m]$.

     \begin{align*}
       \mathbb{E}_{m-1}[M_m]& =\mathbb{E}_{m-1}\left[\mathbb{E}_{h^m\sim P^m_S}\exp(f_m(S,h^m)\right] \\
       & = \mathbb{E}_{m-1}\left[\mathbb{E}_{h_1,..,h_m\sim P_{1,S}\otimes...\otimes P_{m,S}}\exp(f_m(S,h^m)\right] \\
       & = \mathbb{E}_{m-1}\left[\mathbb{E}_{h_1,..,h_m\sim P_{1,S}\otimes...\otimes P_{m,S}}\left[\Pi_{i=1}^m\exp\left(\lambda X_i(h_i,z_i)  - \frac{\lambda^2}{2}(\hat{V}_i(h_i,z_i) + V_i(h_i))\right)\right] \right] \\
        & = M_{m-1} \mathbb{E}_{m-1}\left[ \mathbb{E}_{h_m\sim P_{m,S}}\left[\exp\left(\lambda X_m(h_m,z_m)  - \frac{\lambda^2}{2}(\hat{V}_m(h_m,z_m) + V_m(h_m))\right) \right]\right].
     \end{align*}
 The last line holding because $P^{m-1}_S = P_{1,S}\otimes...\otimes P_{m-1,S}$ is $\mathcal{F}_{m-1}$ measurable.


   Now we exploit the fact that $P_{m,S}$ is $\mathcal{F}_{m-1}$ measurable to apply a conditional Fubini lemma stated in \citet[][Lemma D.3]{haddouche2022online}. We have:

   \begin{multline*}
     \mathbb{E}_{m-1}\left[ \mathbb{E}_{h_m\sim P_{m,S}}\left[\exp\left(\lambda X_m(h_m,z_m)  - \frac{\lambda^2}{2}(\hat{V}_m(h_m,z_m) + V_m(h_m))\right) \right]\right] \\ =  \mathbb{E}_{h_m\sim P_{m,S}}\left[\mathbb{E}_{m-1}\left[\exp\left(\lambda X_m(h_m,z_m)  - \frac{\lambda^2}{2}(\hat{V}_m(h_m,z_m) + V_m(h_m))\right) \right]\right].
   \end{multline*}

 Now we can apply Lemma \ref{l: bercu_touati} for any $h_m\in\mathcal{H}$ with $\Delta M_{m}=X_m(h_m,z_m), \Delta[M]_{m}=\hat{V}(h_m,z_m)$ and $\Delta\langle M\rangle_{m}= V_m(h_m)$. We then have for all $h_m\in\mathcal{H}$:

 \[ \mathbb{E}_{m-1}\left[\exp\left(\lambda X_m(h_m,z_m)  - \frac{\lambda^2}{2}(\hat{V}_m(h_m,z_m) + V_m(h_m))\right) \right] \leq 1.  \]

 Thus $\mathbb{E}_{m-1}[M_m] \leq M_{m-1}$, this concludes the lemma's proof.
   \end{proof}

 Now we can apply Ville's inequality which implies that with probability at least $1-\delta$, for any $m\geq 1$:

 \[ \mathbb{E}_{h^m\sim P^m_S}\exp(f_m(S,h^m)) \leq \frac{1}{\delta}. \]

 Thus we have with probability at least $1-\delta$, for any posterior sequence $(Q_i)_{i\geq 1}$, the data-dependent measures $P_{1,S},...,P_{m,S}$ and any $m\geq 1$:

 \begin{align*}
  \sum_{i=1}^m \mathbb{E}_{h_i\sim Q_i}\left[\lambda X_i(h_i,z_i)  - \frac{\lambda^2}{2}(\hat{V}_i(h_i,z_i) + V_i(h_i))\right] \leq \sum_{i=1}^m \operatorname{KL}(Q_i,P_{i,S}) + \log \left( \frac{1}{\delta}  \right).
 \end{align*}

 Re-organising the terms in this bound and dividing by $\lambda$ concludes the proof.

 \end{proof}

\newpage


% ----------------------------------------------------------------------------------------------- %

\part[Generalisation bounds for  Martingales and Online Learning allowing Heavy-Tailed Losses]{Generalisation bounds for  Martingales and Online Learning allowing Heavy-Tailed Losses}
\label{part:contrib-pac-bayes}

\chapter[Mitigating Initialisation Impact through Flat Minima: Fast Rates for Small Gradients]{Mitigating Initialisation Impact through Flat Minima: Fast Rates for Small Gradients}
\label{chap:gen-flat-minima}
\addchapterlof
\addchapterloa
\addchapterloe
 
\vspace{-2.0cm}
\begin{center}
\textbf{This chapter is based on the following paper}\\[-0.1cm]
\end{center}
\printpublication{haddouche2024pac}


\vspace{0.2cm}
\minitoc

\begin{abstract}
\vspace{-0.2cm}
In \Cref{chap:online-pb} we saw that a way to attenuate the impact of the prior, seen as an initialisation, in PAC-Bayes training is online learning, allowing the prior to evolve alongside the posterior through time. However, a legitimate question is to wonder whether the prior could be attenuated, even in the batch learning setting which is widely used in practice. Maintaining the vision of the prior as initialisation, we propose in this chapter to attenuate the impact of the prior in the batch setting through faster convergence rate. The proposed results hold when a flat minimum has been reached, \ie a minimum whose its neighbourhood nearly minimises the loss as well. Then, a sharper understanding of generalisation can be reached when exploiting the benefits of a successful optimisation process. Indeed, this study is particularly meaningful in the context of deep learning, where it has been shown that flat minimum (also known as sharpness) correlates to a good generalisation ability.  
\end{abstract}

\newpage

\section{Introduction}

Can we make the impact of the prior vanish at a faster rate than $1/\sqrt{m}$ in the context of batch learning? While this is desirable from an optimisation perspective, this is not what is proposed by classical PAC-Bayes bounds, considering all elements of $\Mcal(\Hcal)$ simultaneously. The challenge of this study is to obtain faster rates for a smaller class of posteriors. Doing so, we aim to attenuate the impact of the initialisation (seen as prior) for nonnegative heavy-tailed losses, potentially satisfying geometric assumptions such as gradient-lipschitz, making a promising step towards concrete optimisation settings. The practical way to do so is to obtain results holding only for posteriors distributions focusing on \emph{flat minima}, which can be seen, \eg in deep learning, as a benefit of a successful optimisation process.

Indeed, dating back to \citet{hochreiter1997flat}, it has been hypothesised that the notion of `flatness' (or sometimes equivalently referred to as `sharpness') has tight links with the generalisation error: among the minima (belonging to $\hat{\Risk}_{\S_m}$) that is found by the learning algorithm, the `flatter' the minimum is, the lower is the generalisation error.
While the initial flatness notion was (vaguely) defined through low Kolmogorov complexity, there is no single formal definition of `flatness'.
Hence, several flatness notions have been considered, which typically are based on the second-order derivatives of the empirical risk around the local minimum found by the algorithm, such as $\mathrm{trace}(\nabla^2 \hat{\Risk}_{\S_m}(h))$, see \eg, \citet{jastrzkebski2017three, wen2023sharpness}.


While there have been several attempts to link some form of flatness to generalisation in a mathematically rigorous way \citep{neyshabur2017explor,petzka2021relative,yue2023sharpness, andriushchenko2023modern}, mainly in the framework of `sharpness aware minimisation'~\citep{foret2020sharpness}, it has been recently shown that flat minima do not always imply good generalisation.
In fact, there exist scenarios such that the flattest minima achieve the worst generalisation performance compared to non-flat ones \citep{wen2023sharpness}. 


In this study, we aim at developing novel links between flatness and the generalisation error from a PAC-Bayesian perspective \citep[see \eg, ][]{guedj2019primer,hellstrom2023generalization,alquier2024user}.
Denoting by $\Q$, the probability distribution of the algorithm output $h$ (or the output of a learning algorithm), we identify sufficient conditions on $\Q$ such that flatness always implies good generalisation.
More precisely, we make the following contributions:
\begin{itemize}
    \item We show that, when $\Q$ satisfies the Poincaré inequality and a technical condition that we identify, we can obtain a `fast-rate' generalisation bound that diminishes with rate $\frac{1}{m}$ (rather than $\frac{1}{\sqrt{m}}$) and mainly contains two terms:  
    \begin{enumerate}[label=(\roman*)]
        \item The flatness term: $ \EE_{h\sim \Q}\LB \frac{1}{m}\sum_{i=1}^m \|\nabla_h\ell(h,\z_i)\|^2 \RB$.
        This term is directly linked to the Hessian of the loss $\ell$, due to the connection between the Fisher information and the Hessian of the loss \cite{bickel2015mathematical}.
        For instance, under certain conditions, it can be shown that $\mathrm{trace}(\nabla^2 \hat{\Risk}_{\S_m}(h)) = \frac{2}{m}\sum_{i=1}^m \|\nabla_h \ell(h,\z_i)\|^2$ \citep[Lemma 4.1]{wen2023sharpness}.  
        \item The classical PAC-Bayesian complexity term $\KL(\Q,\P)$, where $\KL$ denotes the Kullback-Leibler divergence and $\P$ is data-independent `prior' distribution. 
    \end{enumerate}
    \item We then further analyse the term $\KL(\Q,\P)$.
    We show that, when $\Q$ is a Gibbs distribution, \ie, $\Q(h)\propto \exp(- \gamma \hat{\Risk}_{\S_m}(h)) \P(h)$ for some $\gamma >0$ and $\P$ satisfies a log-Sobolev inequality, the generalisation error can be controlled \emph{solely} by the term: $\gamma^2 c_{LS}(\P)\EE_{h\sim \Q}[ \|\nabla_h \hat{\Risk}_{\S_m}(h) \|^2 ]$, where $c_{LS}(\P)$ denotes the log-Sobolev constant of the prior $\P$. 
    \item We finally go beyond the KL divergence to link flat minima to deterministic predictors (\ie, when $\Q$ is a Dirac distribution) through a novel Wasserstein-based generalisation bound for gradient Lipschitz loss functions. 
\end{itemize}
We provide a numerical assessment of the technical condition underlying our main result, suggesting that it is suitable in the case of neural networks on classification tasks, confirming the relevance of our bounds to better understand the generalisation ability of neural networks.
Our results shed further light on the impact of the flatness of the minima over the generalisation error: when the learning algorithm ensures a sufficiently regular distribution over the parameters, the generalisation error can be directly controlled by the flatness of the region found by the algorithm.  

\section{Preliminaries}

\textbf{Framework.}
We consider a predictor set $\Hcal\subseteq \mathbb{R}^d$ equipped with a norm $\|.\|$, a data space $\Zcal$ and the space of distributions over $\Hcal,\Mcal(\Hcal)$.
We also consider a loss function $\ell : \Hcal\times \Zcal \rightarrow \mathbb{R}$. 
We assume that we have access to a \iid dataset $\S=(\z_i)_{i\geq 1}\in\Zcal^\Nbb$  with associated distribution $\mathcal{D}$. For each $m\geq 1$, we define $\S_m:= \{\z_1,\cdots,\z_m\}$.
In PAC-Bayes learning, we construct a data-driven posterior distribution $\Q\in\Mcal(\Hcal)$ with respect to a prior distribution $\P$. 
To assess the generalisation ability of a predictor $h\in\Hcal$, we define  the \emph{population risk} to be $\Risk_{\D} (h) \defeq \EE_{\z\sim \mu}[\ell(h,\z)]$ and for each $m$, its empirical counterpart $\hat{\Risk}_{\S_m} (h) \defeq \frac{1}{m}\sum_{i=1}^{m} \ell(h,\z_{i})$. As PAC-Bayes focuses on elements of $\Mcal(\Hcal)$, we also define the expected risk and empirical risks for $\Q\in\Mcal(\Hcal)$ as $\Risk_{\D}(\Q):= \EE_{\h\sim Q}[ \Risk_{\D}(\h)]$ and $\hat{\Risk}_{\S_m}(\Q):= \EE_{\h\sim Q}[ \hat{\Risk}_{\S_m}(\h)]$.
PAC-Bayes bounds usually aim at controlling the \emph{expected generalisation error (or gap)} for each dataset size $m$, i.e.,  $\Delta_{\S_m}(\Q):=
\Risk_{\D}(\Q) - \hat{\Risk}_{\S_m}(\Q)$.

\paragraph{Background on Poincaré and log-Sobolev inequalities.} In this work, we exploit Poincaré and log-Sobolev inequalities in the PAC-Bayes framework.
We first recall the definition of Poincaré and log-Sobolev inequalities.
To do so, for a fixed distribution $\Q$, we define the \emph{Sobolev space of order $1$} on $\mathbb{R}^d$ as follows:
\[ \mathrm{H}^{1}(\Q) := \left\{ f\in \mathrm{L}^2(\Q)\cap \mathrm{D}_1(\mathbb{R}^d)\mid \|\nabla f\|\in \mathrm{L}^2(\Q) \right\}, \] 
where $\mathrm{D}_1(\mathbb{R}^d)$ denotes the set of derivable functions $f : \mathbb{R}^d \to \mathbb{R}$. 

\begin{definition}[Poincaré and Logarithmic Sobolev inequalites]
A measure $\Q$ satisfies a \emph{Poincaré inequality} with constant $c_{P}(\Q)$ if for all function $f\in \mathrm{H}^{1}(\Q)$ we have 
\begin{align*}
 \Var_\Q(f) \leq c_{P}(\Q) \EE_{h\sim \Q}\left[ \|\nabla f (h)\|^2 \right],
\end{align*}
where $\Var_\Q(f) = \EE_{h\sim\Q} \left[f(h) - \EE_{h\sim\Q}[f(h)]\right]^2$ is the \emph{variance} of $f$ \wrt $\Q$.
We then say that $\Q$ is Poincaré with constant $c_{P}(\Q)$, or that $\Q$ is $\Poinc(c_{P})$.
Also, $\Q$ satisfies a \emph{log-Sobolev inequality} with constant $c_{LS}(\Q)$ if for all function $f\in \mathrm{H}^{1}(\Q)$ we have 
\begin{align*}
\EE_{h\sim\Q}\LB f^2(h)\log\LP \frac{f^2(h)}{\mathbb{E}_{h\sim\Q}\LB f^2(h)\RB}\RP \RB \leq c_{LS}(\Q) \EE_{h\sim \Q}\left[ \|\nabla f (h)\|^2 \right],
\end{align*}
where the term on the left hand side is the \emph{entropy} of $f^2$, denoted as $\Ent_Q(f^2)$.
We then say that $\Q$ is log-Sobolev with constant $c_{LS}(\Q)$, or that $\Q$ is $\Lsob(c_{LS})$.
\end{definition}
The class of Gaussian distributions is an important particular case of distributions satisfying both Poincaré and log-Sobolev inequalities, this is the subject of Proposition \ref{prop:gaussian-inequalities}.
\begin{proposition}\label{prop:gaussian-inequalities}
For a given pair $(\mu,\Sigma)$ of mean and covariance matrix in $\mathbb{R}^d$, define $\Q= \Ncal(\mu, \Sigma)$.
Then we have, for any $f \in \mathrm{H}^{1}(\Q)$:
\begin{align*}
\Ent_\Q(f^2) \leq 2\mathbb{E}_{\Q}\left[ \left\langle \Sigma \nabla f, \nabla f\right\rangle \right],\;\text{and }  \Var_\Q(f^2) \leq \mathbb{E}_{\Q}\left[ \left\langle \Sigma \nabla f, \nabla f\right\rangle \right].
\end{align*}
Thus, $\Q$ is $\Lsob(c_{LS})$ with constant $c_{LS}(\Q)=2\|\Sigma\|_{op}$ and also  $\Poinc(c_{LS})$ with constant $c_{LS}(\Q)=\|\Sigma\|_{op}$, where $\|.\|_{op}$ denotes the operator norm. 
\end{proposition}
In \Cref{prop:gaussian-inequalities}, the first inequality can be derived from the classical log-Sobolev inequality for $\Ncal(\zerobf,\textrm{Id})$ stated in \citet{gross1975lsi}, with a change of variable. Similarly, the Poincaré inequality can be obtained through a change of variable from the Poincaré inequality for $\Ncal(\zerobf,\textrm{Id})$ which is a particular case of the Brascamp-Lieb inequality for log-concave probability measures \citep{brascamp1976extensions} and is stated explicitly in \citet[Theorem 1]{beckner1989new}. 


We now focus on specific posterior distributions called \emph{Gibbs posteriors, or Gibbs distributions}.
For a fixed loss $\ell$ and dataset $\S_m$, the Gibbs posterior, \wrt prior $\P\in\Mcal(\Hcal)$, risk $\hat{\Risk}_{\S_m}$ and \emph{inverse temperature} $\gamma>0$.is defined as $\P_{-\gamma \hat{\Risk}_{\S_m}}$ such that  $d\P_{-\gamma \hat{\Risk}_{\S_m}}(h)\propto \exp( - \gamma \hat{\Risk}_{\S_m}(h) ) d\P(h)$. 
Gibbs posteriors are a class of closed-form solutions for relaxation of \citet[Theorem 1.2.6]{catoni2007pac} stated, for instance, in \citet[Theorem 4.1]{alquier2016properties}. 
\Cref{prop:gibbs_logsob} shows that when the prior and the loss satisfies a few properties, then the associated Gibbs posterior is $\Lsob(c_{LS})$.

\begin{proposition}\label{prop:gibbs_logsob}
Assume that $\P$ is a probability measure on $\mathbb{R}^d$ such that $d\P(h) \propto \exp(-V(x))$ with $V$ a smooth function such that $Hess(V)\succeq \frac{2}{c_{LS}(\P)}\mathrm{Id}$.
Assume that $\ell= \ell_1 + \ell_2$ with $\ell_1$ convex, twice differentiable and $\ell_2$ bounded. 
Then for any $\gamma>0$, the Gibbs posterior $\Q= \P_{-\gamma\hat{\Risk}_{\S_m}}$ is $\Lsob(c_{LS})$ with constant $c_{LS}(\Q)= c_{LS}(\P)\exp\LP 4\|\ell_2\|_{\infty}\RP $.
\end{proposition}
\Cref{prop:gibbs_logsob} applies, \eg, when $\P$ is a Gaussian prior $\P=\Ncal(\mu_\P,\Sigma_\P)$. Notice that in this case $c_{LS}(\P)= 2\|\Sigma_\P\|_{op}$. This property is a straightforward application of \citet[Corollary 2.1]{chafai2004entropies} with \citet[Property 2.6]{guionnet2003lectures} and is stated in \Cref{sec: supp_background} for completeness.
Finally, notice that satisfying a log-Sobolev inequality is stronger than satisfying a Poincaré one. This is stated for instance in \citet[Proposition 2.1]{ledoux2006concentration} and properly recalled in \Cref{sec: supp_background}. 

\section{Reaching a flat minimum allows Poincaré posteriors to generalise well}
\label{sec:poincare_gauss}

\subsection{Fast rate PAC-Bayes bounds for heavy-tailed losses}  
\label{sec:fast_rates_gradient_h}

In order to obtain fast rates, \ie, bounds converging to zero faster than $\frac{1}{\sqrt{m}}$,  we exploit the notion of flat minimum (where the loss takes a small value in the neighbourhood of the minimum).
Indeed, in an overparametrised setting such as neural networks, it is likely to obtain such a minimum once the optimisation phase has been performed, as there are much more parameters than training data.
We exploit this flatness property within PAC-Bayes bounds through the gradient norm $\| \nabla_h \ell(.,\z)\|$ of the loss \wrt the predictor $h$ for any $\z$.
This is, to the best of our knowledge, the first attempt to do so as \citet{gat2022grad} focus on gradients with respect to the data $\nabla_{\z}\ell$ (one does not optimise on those, as the dataset is fixed in practice).
  
In this section, we consider posterior distributions $\Q$ being $\Poinc(c_P)$.
This assumption covers the important case of Gaussian measures (\Cref{prop:gaussian-inequalities}) as well as all measures satisfying a log-Sobolev inequality (\Cref{prop:ls_implies_poinc}).
We focus on PAC-Bayes bound holding for distributions $\Q$ satisfying a particular assumption involving the data distribution $\D$ (contrary to many PAC-Bayes bounds holding for all $\Q$).
We then define the \emph{error} of $\Q\in \Mcal(\Hcal)$ for any datum $\z\in\Zcal$ as $\Err(\ell,\Q,\z)\defeq \mathbb{E}_{h\sim\Q}[\ell(h,\z)]$ and identify Assumption \ref{hyp:relaxed_bounded} to later involve flat minima.

  
  \begin{assumption}
    \label{hyp:relaxed_bounded}
    We say that $\Q\in\Mcal(\Hcal)$ is \emph{quadratically self-bounded} with respect to $\ell$ and constant $C>0$ (namely $\texttt{QSB}(\ell,C)$) if
    \[ \mathbb{E}_{\z\sim \D}\LB \Err(\ell,\Q,\z)^2 \RB \leq C \Risk_\D(\Q) \LP = C\mathbb{E}_{\z\sim \D}\LB \Err(\ell,\Q,\z) \RB \RP   \]
  \end{assumption}
Assumption \ref{hyp:relaxed_bounded} is a relaxation of boundedness, as if $\ell\in[0,C]$ then it is $\texttt{QSB}(\ell,C)$.
It is an alternative to the bounded expected variance assumption in anytime-valid PAC-Bayes bounds as in \Cref{chap: pb-ht} and \citep{chugg2023unified}.
An issue with such boundedness assumption is that it has to hold for all posteriors, including those providing poor generalisation performances.
This is avoided by the $\texttt{QSB}$ assumption which intricate the properties of $\D,\ell$ and $\Q$.
Such a design is in line with the conclusions of the recent work of \citet{gastpar2023fantastic}, inviting to derive generalisation bounds valid for specific pairs $(\Q,\D)$ (and not uniformly valid for all such pairs).
Finally, we interpret $C$ as a contraction constant attenuating, on average, the local expansion (governed by variances of $\Q$, and $\D$) of the loss around the mean of $\Q$.
Exploiting the PAC-Bayes supermartingales bounds of \Cref{chap: pb-ht} and \citet{chugg2023unified} alongside Poincaré inequality leads to the following. 

\begin{theorem}\label{th:poincaré_gauss}
For any $C>0$, any $\frac{2}{C}>\lambda >0$, any data-free prior $\P$, any  $\ell\geq 0$ and any $\delta\in [0,1]$, we have, with probability at least $1-\delta$ over the sample $\S$, for any $m>0$, any $\Q$ being $\Poinc(c_P)$, $\texttt{QSB}(\ell,C)$ and $\ell(.,\z)\in \mathrm{H}^{1}(\Q)$ for all $\z$,
\begin{multline*}
\Risk_{\D}(\Q) \leq \frac{1}{1-\frac{\lambda C}{2}} \LP \hat{\Risk}_{\S_m}(\Q) + \frac{\KL(\Q,\P) +\log(1/\delta)}{\lambda m} \RP \\
+ \frac{\lambda}{2-\lambda C} c_{P}(\Q)\EE_{\z\sim\D} \LB \EE_{h\sim \Q}\LP \|\nabla_h \ell(h,\z)\|^2 \RP \RB.
\end{multline*}
\end{theorem}
This theorem shows that, for any posterior being \texttt{QSB} \wrt the distribution $\D$, fast rates are achievable as long as $\hat{\Risk}_{\S_m}\approx 0$, and expected gradients are vanishing.
While the first condition is often involved for deep neural networks in the overparametrised setting, the second holds if a flat minimum has been reached through the optimisation process.
Then, taking $\lambda = \frac{1}{C}$ ensures an anytime-valid PAC-Bayesian bound with a fast rate of $\frac{1}{m}$.
Otherwise, for a fixed $m$, taking $\lambda= \frac{m^{-\alpha}}{C}$, $\alpha \in \LB0;\frac{1}{2}\RB$ allows to adapt the convergence speed \wrt the behaviour of the gradients. 
In the case of constant gradients, we recover a convergence rate of $\frac{1}{\sqrt{m}}$, matching \citet[Theorem 4.1]{alquier2016properties}. 

  
\noindent\textbf{On the role of flat minima in PAC-Bayes learning.} 
Theorem \ref{th:poincaré_gauss} suggests that, in order to attain good generalisation ability, the mean of $\Q$ has to be close from two minima: 
\emph{(i)} on $\hat{\Risk}_{\S_m}$ in order to make $\hat{\Risk}_{\S_m}$ small, and \emph{(ii)} on $\Ebb_{\z\sim\D}[\|\nabla_h \ell(h,\z)\|^2]$  to make the gradients small. The variance of $\Q$ has to fit the flatness of those minima, the flatter they are, the larger the variance in order to shrink the expected terms on the right-hand-side of Theorem \ref{th:poincaré_gauss}. Finally, the KL term invites, \eg for Gaussian distributions, to consider high variances, hence flat minima to maintain a small value of the bound.
  %Indeed, we interpret $C$ as a contraction constant translating the sharpness of the minimum on $\Risk_\D$:  the flatter the minimum, (\eg $\Risk_\D(\mu)= 0$), the closer  $\Risk_\D(\Q)$ from 0, then $\Risk_\D(\Q)^2$ is even smaller, allowing smaller $C$ and translating contractions of the loss. \umut{I think this paragraph is very important in terms of the meaning of the bound because flatness is in the core of the paper. Could you rewrite it by giving more intuitive details? Right now it's not very clear to me what it's meant by sharpness/flatness and how explicitly it's linked to $C$. Also there are existing "flatness" notions, how does this relate to those? }
  
\noindent\textbf{A focus on $C$.}
Taking $\lambda= \frac{1}{C}$ in Theorem \ref{th:poincaré_gauss} attenuates the impact of the prior distribution and amplifies the gradient term. Then, a small $C$ is desirable when working with flat minima to attenuate an ill-designed prior. Having a small $C$ is reachable in practice: we show in \Cref{sec: expes}, for a classification task on MNIST, that the \texttt{QSB} assumption is verified with $C$ strictly smaller than $1$ when considering neural networks.
  
\noindent\textbf{High probability bounds with fast rates, a paradox?} \citet[page 7]{grunwald2021mac} showed that, for a trivial $\Hcal=\{h\}\subset \Rbb^d$, for any loss, any \iid dataset $\S_m$ with variance $\sigma^2$, we have asymptotically, with probability at least $\alpha$, for a constant $C_{\alpha}$ depending on $\alpha$ and $\Ncal(\zerobf,\textrm{Id})$, we have $\Risk_\D(h) \geq \hat{\Risk}_{\S_m}(h) + C_{\alpha}\frac{\sigma^2}{\sqrt{m}}$. Is it paradoxical with Theorem \ref{th:poincaré_gauss}? The answer is no: the bound in \citet{grunwald2021mac} gives an asymptotic lower bound on the convergence of $\hat{\Risk}_{\S_m}(h)$ to  $\Risk_\D(h)$. Theorem 
  \ref{th:poincaré_gauss} informs us on how $\Risk_\D$ is getting closer from $\frac{1}{1-\lambda/2}\hat{\Risk}_{\S_m}$ which converges to $\frac{1}{1-\lambda/2}\Risk_\D>\Risk_\D$ as the loss is non-negative.  
  Theorem \ref{th:poincaré_gauss} then show the existence of a `transition regime' involving a fast rate. Once  $\frac{1}{1-\lambda/2} \hat{\Risk}_{\S_m}$ is reached, the clower bound of \citet{grunwald2021mac} ensures an asymptotic regime with slow convergence rate. Note that such transition regimes already appeared in the literature in \citet{tolstikhin2013pac,mhammedi2019pac} at the cost of additional variance terms compared to Theorem \ref{th:poincaré_gauss}. However, such fast rates have never been linked before to flat minima (and optimisation in general), highlighting the potential of our bound to explain the ability of deep neural networks to generalise well in the overparametrised setting ($m$ far smaller than the dimension of $\Hcal$), where flat minima are likely to be reached, as studied, \eg, in \citet{dziugaite2020search}, showing correlations between flat minima and generalisation for various learning problems. 

  \begin{proof}[of Theorem \ref{th:poincaré_gauss}]
We start from \citet[Corollary 17]{chugg2023unified} instantiated with a single $\lambda$, \iid data and a prior $\P$. 
With probability at least $1-\delta$, for any $\Q\in\Mcal(\Hcal)$ and $m>0$:
\begin{align*}
\Risk_{\D}(\Q) \leq  \hat{\Risk}_{\S_m}(\Q) + \frac{\KL(\Q,\P) +\log(1/\delta)}{\lambda m} 
+ \frac{\lambda }{2}\left(   \EE_{h\sim \Q}\left[\mathbb{E}_{\z\sim \D}[\ell (h,\z)^2]  \right]  \right),
\end{align*} 
where $\z\sim \D$ is independent of $\S$. 
We study the last term on the right-hand side. First, applying Fubini's theorem gives: 
\begin{align*}
\EE_{h\sim \Q}\left[\mathbb{E}_{\z\sim \D}[\ell (h,\z)^2] \right] & = \mathbb{E}_{\z\sim \D}\left[ \EE_{h\sim \Q}[\ell (h,\z)^2] \right] \\
& = \EE_{\z\sim\D} \LB \Var_{h\sim \Q}\LP \ell(h,\z) \RP + \LP \EE_{h\sim\Q}[\ell(h,\z)] \RP^2 \RB .
\intertext{As for any $\z$, $\ell(.,\z)\in \mathrm{H}^{1}$, we apply Poincaré's inequality to obtain: }
& \leq  \EE_{\z\sim\D} \LB c_{P}(\Q)\EE_{h\sim \Q}\LP \|\nabla_h \ell(h,\z)\|^2 \RP + \LP \EE_{h\sim\Q}[\ell(h,\z)] \RP^2 \RB  .
\end{align*}
Using that $\Q$ is $\texttt{QSB}(\ell,C)$ and re-organising the terms gives: 
\begin{multline*}
\Risk_{\D}(\Q) \leq \frac{1}{1-\frac{\lambda C}{2}} \LP \hat{\Risk}_{\S_m}(\Q) + \frac{\KL(\Q,\P) +\log(1/\delta)}{\lambda m} \RP \\
+ \frac{\lambda}{2-\lambda C} c_{P}(\Q)\EE_{\z\sim\D} \LB \EE_{h\sim \Q}\LP \|\nabla_h \ell(h,\z)\|^2 \RP \RB. 
\end{multline*}
\end{proof}
\noindent{}It is possible to go beyond the $\texttt{QSB}$ assumption.
This comes at the cost of an upper bound on $\Risk_\D$ as well as a supplementary Poincaré assumption on $\D$.  

\begin{corollary}\label{cor:poincaré_pacb}
For any $C>0$, any $\delta\in (0,1)$ any $\frac{2}{C}>\lambda >0$, any data-free prior $\P$, any $\ell\geq 0$ such that, for any $\z\in\Zcal$, we have $\ell(.,\z)\in \mathrm{H}^{1}$ and for any $h$, the loss function $\ell(h,.)$ is $\Ccal^1$ almost everywhere on $\Zcal$.
If the data distribution $\D$ is $\Poinc(c_P)$, then with probability at least $1-\delta$ over the sample $\S$, for any $m>0$, any posterior $\Q$ being $\Poinc(c_P)$ with $\Risk_\D(\Q)\leq C$:

\begin{multline*}
\Risk_{\D}(\Q) \leq \frac{1}{1-\frac{\lambda C}{2}}\LP \hat{\Risk}_{\S_m}(\Q) + \frac{\KL(\Q,\P) +\log(1/\delta)}{\lambda m}\RP \\
+ \frac{\lambda}{2-\lambda C}\LP c_{P}(\Q)\EE_{\z\sim\D} \LB \EE_{h\sim \Q}\LP \|\nabla_h \ell(h,\z)\|^2 \RP \RB  + c_{P}(\D)\EE_{\z\sim\D} \LP \LM\EE_{h\sim\Q}[\nabla_z \ell(h,\z)] \RM^2\RP  \RP.
\end{multline*}     
\end{corollary}   
Proof is deferred to Section \ref{sec:proof_poincaré_pacb}.
Corollary \ref{cor:poincaré_pacb} states that, if $\Q$ reached a flat minimum (meaning $\|\nabla_h\ell\|$ is small), and this minimum is robust to the training dataset (meaning $\|\nabla_\z\ell\|$ is small), then a fast rate is attainable while only requiring an upper bound on $\Risk_\D(\Q)$.
This conclusion holds when $\D$ \Poinc, encompassing the case of Gaussian mixtures \citep{schlichting2019poinc}, which can approximate any smooth density \citep[as recalled in][]{gat2022grad}.
However, the Poincaré constant of a general mixture is not known, and the upper bound of \citet{schlichting2019poinc} scales with the number of components, involving potentially high $\chi^2$ divergences.\\

\noindent\textbf{Comparison with \citet{gat2022grad}}. We compare Corollary \ref{cor:poincaré_pacb} with \citet[Theorems 3.5, 3.6]{gat2022grad}.
First, our result holds with the assumption that $\D$ follows a Poincaré inequality, which is strictly less restrictive than assuming a log-Sobolev inequality (Proposition \ref{prop:ls_implies_poinc}).
Second, they assume a bounded loss and their result holds only for classification problem satisfying a technical assumption on the label repartition (see their Lemma 3.3) while ours holds for any learning problem at the sole assumption of a bounded $\Risk_\D(\Q)$, allowing $\ell$ to be non-negative. 
Moreover, note that to conclude their proof, \citet{gat2022grad} had to use a uniform bound on $\Ebb{\z}[\|\nabla_\z \ell\|]$ in their Theorem 3.5 to have a tractable bound, thus the benefits of gradient norm is unclear.
While they overcome this limitation in \citet[Theorem 3.6]{gat2022grad}, the explicit influence of the gradient norm appears within an exponential moment on the losses (attenuated by a logarithm).
However, a major limitation is that this exponential moment is averaged \wrt $\P$, being data-free.
Thus, the associated gradients have no apparent reason to be small, and their result cannot be linked to flat minima, contrary to Corollary \ref{cor:poincaré_pacb} involving expected gradients \wrt $\Q$, being the output of an optimisation process. 

\subsection{Towards fully empirical bound for gradient-Lipschitz functions.}
In this section,  we assume the loss $\ell$ is such that, for any $\z\in\Zcal$, the gradient $\nabla_h\ell(.,\z)$ is $G$-Lipschitz, which is often considered for convergence bounds in optimisation.
A large part of high-probability PAC-Bayes bounds are fully empirical: this has numerous advantages including in-training numerical evaluation of generalisation as well as novel PAC-Bayesian algorithms, minimising such empirical bounds; see \citep{dziugaite2017computing,perez2021progress,viallard2023learning} among others.
However, Theorem \ref{th:poincaré_gauss} and Corollary \ref{cor:poincaré_pacb} are not fully empirical and thus, do not have such desirable properties. We circumvent this issue in Theorem \ref{th:poincaré_grad_lpz}.
\begin{theorem}
  \label{th:poincaré_grad_lpz}
    For any $C_1,C_2,c>0$, any data-free prior $\P$, any $\ell\geq 0$ being $\mathcal{C}^2$ and any $\delta\in [0,1]$, we have, with probability at least $1-\delta$ over the sample $\S$, for any $m>0$, any $\Q$ being $\Poinc(c_P)$ with constant $c$, $\texttt{QSB}(\ell,C_1)$, $\texttt{QSB}\LP\|\nabla_h \ell\|^2,C_2\RP$ and $\ell(.,\z),\|\nabla_h \ell\|^2(.,\z)\in \mathrm{H}^{1}(\Q)$ for all $\z$, 
    \begin{multline*}
      \Risk_{\D}(\Q) \leq  2 \hat{\Risk}_{\S_m}(\Q) + \frac{2c}{C_1} \EE_{h\sim \Q}\LB \frac{1}{m}\sum_{i=1}^m \|\nabla_h\ell(h,\z_i)\|^2 \RB \\
       + 2\LP C_1 + c\frac{4cG^2 + C_2}{C_1} \RP\frac{\KL(\Q,\P) +\log(2/\delta)}{m}.
    \end{multline*}
\end{theorem}
Proof is deferred to Section \ref{sec: proof_poincaré_grad}.
Here, we showed that to attain fast rates, the \texttt{QSB} assumption has to be reached for both the loss and its gradient.
This suggests several things on the flat minimum that has to be reached by $\Q$ (designed from $\hat{\Risk}_\S$): first, it needs to be close from a flat minimum of $\Risk_\D$ to satisfy the \texttt{QSB} assumption.
Second, this minimum also ensures the contraction of the gradients.
We then are able to derive an empirical generalisation bound, involving both empirical loss and gradients.
Not only Theorem \ref{th:poincaré_grad_lpz} yields, to our knowledge, the first PAC-Bayesian algorithm involving gradient terms, but also can be translated to a generalisation metric in order to understand generalisation.
Such an idea has been exploited recently \citep{neyshabur2017explor,jiang2020fantastic,dziugaite2020search}.
In particular, from $\hat{\Risk}_\S(\Q)$, \citet{neyshabur2017explor} derived a notion of \emph{sharpness}, stated in \Cref{eq:flat_minima_neyshabur}, aiming to be informative on the flatness of the reached minima for any $\Q= \Ncal(\mu_Q,\sigma^2 \mathrm{Id})$.
This notion is defined by 
\begin{equation}
  \label{eq:flat_minima_neyshabur}
  \EE_{\nu\sim \Ncal(\mathbf{0},\sigma^2 \mathrm{Id})} \LB \hat{\Risk}_{\S_m}(\mu_\Q + \nu) -  \hat{\Risk}_{\S_m}(\mu_\Q) \RB.
\end{equation}
Theorem \ref{th:poincaré_grad_lpz} enhance this notion of sharpness by involving the empirical gradients when $\Q$ is $\texttt{QSB}(\ell,C_1)$: 
\begin{multline}
  \label{eq:flat_minima_us}
  \textrm{Sharp}_{\frac{\sigma^2}{C_1}}(\Q):=\\
   \EE_{\nu\sim \Ncal(\mathbf{0},\sigma^2 \mathrm{Id})} \LB \LP2\hat{\Risk}_{\S_m} + \frac{\sigma^2}{C_1}\textrm{G-}\hat{\Risk}_{\S_m}\RP(\mu_\Q + \nu)  -  \LP2\hat{\Risk}_{\S_m} + \frac{\sigma^2}{C_1}\textrm{G-}\hat{\Risk}_{\S_m}\RP(\mu_\Q) \RB ,
\end{multline}
where $\textrm{G-}\hat{\Risk}_{\S_m}(h) = \frac{1}{m}\sum_{i=1}^m\|\nabla_h \ell(h, \z_i)\|^2$.
This gradient term can be seen as an empirical Fisher information, linked to the second-order moment derivative.
Thus, \eqref{eq:flat_minima_us} involves a notion of flatness on both the loss and its gradient, contrary to \eqref{eq:flat_minima_neyshabur}.
For the sake of clarity, we particularise Theorem \ref{th:poincaré_grad_lpz} in Corollary \ref{cor:flatness_grad_lpz} with Gaussian distributions and this novel notion of sharpness.
\begin{corollary}\label{cor:flatness_grad_lpz}
For any $C_1,C_2>0$, any fixed variance $\sigma^2>0$, any data-free prior $\P=\Ncal(\mu_\P,\sigma^2 \mathrm{Id})$, any nonnegative loss $\ell$ being $\mathcal{C}^2$ and any $\delta\in [0,1]$, we have, with probability at least $1-\delta$ over the sample $\S$, for any $m>0$, any $\Q= \mathcal{N}(\mu_\Q,\sigma^2 \mathrm{Id})$ being $\texttt{QSB}(\ell,C_1)$, $\texttt{QSB}\LP\|\nabla_h \ell\|^2,C_2\RP$ and $\ell(.,\z),\|\nabla_h \ell\|^2(.,\z)\in \mathrm{H}^{1}(\Q)$ for all $\z$, 
\begin{align*}
\Risk_{\D}(\Q) \leq & 2 \hat{\Risk}_{\S_m}(\mu_\Q) + \textrm{G-}\hat{\Risk}_{\S_m}(\mu_\Q) + \textrm{Sharp}_{\frac{\sigma^2}{C_1}}(\Q) + \mathcal{O}\LP\frac{\KL(\Q,\P) +\log(2/\delta)}{ m}\RP.
\end{align*}
\end{corollary}
  

\section{Generalisation ability of Gibbs distributions with a log-Sobolev prior}
\label{sec:gibbs}

One limitation of the results given in \Cref{sec:poincare_gauss} is that the KL divergence term remains uncontrolled in general as its formulation depends on the nature of $\P$ and $\Q$.
A close form exists for Gaussian distributions for instance, but this class of distribution is limiting. 
Perpetrating the spirit of \citet{catoni2007pac}, we go beyond the Gaussian distributions to focus on the Gibbs posteriors which have naturally appeared in PAC-Bayes through the use of tools from statistical physics. We show that log-Sobolev inequalities allow us to control the KL divergence of such distributions \wrt their priors.

\textbf{Controlling the KL divergence when $\Q$ is a Gibbs posterior.}
\Cref{l: kl_bound} exploits the fact that the KL divergence can be formulated as an entropy \wrt the prior distribution $\P$. It then shows that the KL divergence of the Gibbs posterior $\P_{-\gamma \hat{\Risk}_{\S_m}}$ \wrt $\P$ is upper bounded by gradient terms as long as $\P$ satisfies a log-Sobolev inequality. 
\begin{lemma}
  \label{l: kl_bound}
  For any $m$, $\P$ being $\Lsob(c_{LS})$, any $\ell\geq 0$ such that for any $\z$, $\ell(.,\z) \in \mathrm{H}^{1}(\P)$, we have, for any $\gamma>0$:
  \[ \KL\LP \P_{-\gamma \hat{\Risk}_{\S_m}},\P\RP \leq \frac{\gamma^2 c_{LS}(\P)}{4} \EE_{h\sim \P_{-\gamma \hat{\Risk}_{\S_m}}}\LB \|\nabla_h \hat{\Risk}_{\S_m}(h) \|^2 \RB.\]
\end{lemma}
Proof is deferred to \Cref{sec: proof_kl_bound}. The crucial message of this lemma is that, a flat minimum of $\hat{\Risk}_\S$ allows controlling the KL divergence. This message is new and independent of \Cref{sec:poincare_gauss} which focus on flat minima reached for $\Risk_\D$.
Note that in this case, the KL divergence has an explicit formulation. However it involves to calculate the exponential moment $\mathbb{E}_{h\sim \P}[\exp(-\gamma \hat{\Risk}_{\S_m})]$ which is costly in practice. On the contrary, we only need to estimate a second-order moment over $\P_{-\gamma \hat{\Risk}_{\S_m}}$.

\noindent\textbf{Generalisation ability of Gibbs posteriors.}
When Gibbs posteriors are involved, KL divergence is controllable by a gradient term. An ideal way to conclude would be, as in Section \ref{sec:poincare_gauss} to involve Poincaré inequality. However, Gibbs posterior are not necessarily satisfying a Poincaré inequality as in Section \ref{sec:poincare_gauss}, we then need to make supplementary assumptions on the loss.


\begin{theorem}\label{th: gibbs_pacb}
For any $C>0$, any $\gamma>0$, any prior $\P$ being $\Lsob(c_{LS})$, any $\ell\geq 0$ and any $\delta\in [0,1]$, we have the following inequalities.
If $\ell\in [0,1]$, then with probability at least $1-\delta$ over the sample $\S$, for any $m>0$, and any $\Q\in\Mcal(\Hcal)$:
\begin{multline*}
    \Risk_{\D}(\P_{-\gamma \hat{\Risk}_{\S_m}}) \\ \leq 2 \LP \hat{\Risk}_{\S_m}(\P_{-\gamma \hat{\Risk}_{\S_m}}) + \frac{\gamma^2 c_{LS}(\P)}{4m} \EE_{h\sim \P_{-\gamma \hat{\Risk}_{\S_m}}}\LB \|\nabla_h \hat{\Risk}_{\S_m}(h) \|^2 \RB + \frac{\log(1/\delta)}{m} \RP .
\end{multline*}

If $\ell= \ell_1+\ell_2$ with $\ell_1$ convex, twice differentiable and $\ell_2$ bounded, assume that $\P$ satisfies the conditions of \Cref{prop:gibbs_logsob}. Then for any $\frac{2}{C}> \lambda>0$, with probability at least $1-\delta$ over the sample $\S$, for any $m>0$, such that $\Q$ is $\texttt{QSB}(\ell,C)$ and $\ell(.,\z)\in \mathrm{H}^{1}(\P_{-\gamma \hat{\Risk}_{\S_m}})$:

\begin{multline*}
    \Risk_{\D}(\P_{-\gamma \hat{\Risk}_{\S_m}}) \\ \leq 
     \frac{1}{1-\frac{\lambda C}{2}} \LP \hat{\Risk}_{\S_m}(\P_{-\gamma \hat{\Risk}_{\S_m}}) + \frac{\gamma^2 c_{LS}(\P)}{4\lambda m} \EE_{h\sim \P_{-\gamma \hat{\Risk}_{\S_m}}}\LB \|\nabla_h \hat{\Risk}_{\S_m}(h) \|^2 \RB + \frac{\log(1/\delta)}{\lambda m} \RP \\ 
    + \frac{\lambda e^{4\|\ell_2\|_{\infty}}c_{LS}(\P) }{4-2\lambda C} \EE_{\z\sim\D} \LB \EE_{h\sim \P_{-\gamma \hat{\Risk}_{\S_m}}}\LP \|\nabla_h \ell(h,\z)\|^2 \RP \RB .
\end{multline*} 
\end{theorem}
Proof is deferred to \Cref{sec:proof_gibbs_pacb}.
Note that we could have derived analogous to Corollary \ref{cor:poincaré_pacb} at the cost of a supplementary Poincaré assumption on $\D$.
The influence of the inverse temperature $\gamma$ is quadratic: this is the price to pay to fit the dataset and reduce the influence of the prior.
This dependency is therefore attenuated by a gradient term, small if a flat minimum on the empirical risk has been reached.
This suggests that in the case of Gibbs posteriors with log-Sobolev prior, reaching a flat minima on $\hat{\Risk}_{\S_m}$ controls not only $\hat{\Risk}_{\S_m}(\Q)$, but also the KL divergence and this last point is not reachable when considering Poincaré distributions.
The other gradient term comes from \Cref{sec:poincare_gauss} and requires to be close from a flat minimum on $\Risk_\D$ to attain fast rates. 

\section{On the benefits of the gradient norm in Wasserstein PAC-Bayes learning}

In \Cref{sec:poincare_gauss,sec:gibbs}, we provided various generalisation bounds, benefiting from flat minima.
However, our results involve a KL divergence, implying absolute continuity of $\Q$ \wrt $\P$, incompatible with the case of deterministic predictors (Dirac distributions).
To circumvent this issue, a recent line of work emerged, involving integral probability metrics, with a particular focus on the $1$-Wasserstein distance as in \Cref{chap: wass-pb,chap: wpb-practical} and \citet{amit2022integral}.
The idea behind these works is to replace the change of measure inequality  \citep{csizar1975divergence,donsker1976asymp} by the Kantorovich-Rubinstein duality \citep{villani2009optimal} to trade a KL for a Wasserstein.
We go even further here by obtaining the first PAC-Bayesian bound involving directly a 2-Wasserstein distance (see definition \ref{def: wasserstein-chap-4}), trading Lipschitz assumption for gradient-Lipschitz one (well-suited for optimisation). To do so, we first derive a novel change of measure inequality.

\begin{theorem}
  \label{th:2-wass}
  Assume $\Hcal$ to have a finite diameter $D>0$. Then for any function $f:\Hcal\rightarrow \Rbb$ with $G$-Lipschitz gradients, the following holds: for all distributions $\P,\Q\in\Mcal(\Hcal)^2$, 

  \[ \mathbb{E}_{h\sim \Q}[f(h)] \leq \frac{G}{2}\W_2^2(\Q,P) + \mathbb{E}_{h\sim \P}[f(h)] +D \mathbb{E}_{h\sim\Q}[\|\nabla f (h)\|].  \]
\end{theorem}
Proof is deferred to \Cref{sec: proof_2-wass}
Theorem \ref{th:2-wass} shows it is possible when gradients are Lipschitz, to obtain a duality formula involving the gradient of the considered function at the price of a linear dependency on the diameter of $\Hcal$.
Theorem \ref{th:2-wass} is also linked to the change of measure inequality when the prior distribution satisfies a log-Sobolev inequality. 

\begin{corollary}
  \label{cor:kl-change}
  Assume that $\P$ is such that $d\P \propto \exp(-V)dx$ with $V$ being $\mathcal{C}^2$ and $\P$ is $\Lsob(c_{LS})$. Then, for any $R>0$, any $f$ with gradients $G$-Lipschitz on $\Bcal(\zerobf,R)$, and any distributions $\P,\Q$, 

  \begin{multline*} \mathbb{E}_{h\sim \Q}\LB f\LP\Pcal_R(h)\RP \RB \\
    \leq \frac{Gc_{LS}(\P)}{4}\KL(\Q,\P) + \mathbb{E}_{h\sim \P}\LB f\LP\Pcal_R(h)\RP \RB + 2R\mathbb{E}_{h\sim \Q} \LB \LM\nabla f \LP\Pcal_R(h)  \RP \RM \RB,  
  \end{multline*}
  where $\Pcal_R$ denotes the Euclidean projection on $\Bcal(\zerobf,R)$.
\end{corollary}
Proof is deferred to \Cref{sec: proof_kl-change}.
Corollary \ref{cor:kl-change} involves a KL divergence and an Euclidean predictor space $\Hcal= \Rbb^d$. This comes at the cost of approximating $\Q,\P$ by $\Pcal_R\#\Q,\Pcal_R\#\P$. Thus, $R$ is now an hyperparameter which arbitrates a tradeoff between the quality of our approximations and the looseness of the bound (if the gradient norm is large). A notable strength is that the smoothness assumption is relaxed on smoothness over $\Bcal(\zerobf,R)$.
\\

\noindent{}From Theorem \ref{th:2-wass}, we now derive a novel generalisation bound allowing deterministic predictors.

\begin{theorem}
\label{th:wpb-grad}
Let $\delta\in(0,1)$ and $\P\in\Mcal(\Hcal)$ a data-free prior.
Assume $\Hcal$ has a finite diameter $D>0$, $\ell\geq 0$ and that for any $m$, the generalisation gap $\Delta_{\S_m}$ is $G$ gradient-Lipschitz.
Assume that $\mathbb{E}_{h\sim\P}\Ebb_{\z\sim\D}[\ell(h,z)^2] \leq \sigma^2$, then the following holds with probability at least $1-\delta$, for any $m>0$ and any $\Q$:
\begin{multline*}
    \Risk_D(\Q) \leq \hat{\Risk}_{\S_m}(\Q) + \frac{G}{2} \W_2^2(\Q,\P) + \sqrt{\frac{2\sigma^2\log\LP \frac{1}{\delta} \RP}{m}} \\
    + D \mathbb{E}_{h\sim \Q}\LP \left\| \nabla_h \Risk_\D(h) - \nabla_h \hat{\Risk}_{\S_m}(h) \right\| \RP
\end{multline*}
\end{theorem}
Proof is deferred to \Cref{sec:proof_wpb-grad}.  
\Cref{th:wpb-grad} is not the first generalisation bound to involve a 2-Wasserstein distance \citep{lugosi2022generalization,lugosi2023onlinetopac}.
However, those results involve infinitely smooth loss functions.
Also, results from \citet{amit2022integral}, \Cref{chap: wass-pb,chap: wpb-practical} using 1-Wasserstein can be directly relaxed on bounds involving the 2-Wasserstein, while still requiring a Lipschitz loss.
On the contrary, our result holds for any nonnegative gradient-Lipschitz $\Delta_{\S_m}$, which is well-suited for optimisation.
\Cref{th:wpb-grad} involves a slow rate of $\frac{1}{\sqrt{m}}$ as we have to control the generalisation gap \wrt to $\P$.
It is possible to make appear the gradients expected over $\P$ using the \texttt{QSB} assumption, but we have no reason to expect those gradients to be small, we then controlled this term uniformly by $\sigma^2$.
Another restriction of our result compared to previous ones is that it holds for $\H$ having a finite diameter, however, having a small expected $\| \nabla_h \Risk_\D-\nabla_h\hat{\Risk}_{\S_m}\|$ over $\Q$ (which is the case when flat minima on both empirical and true risks are reached) allows taking $D$ large, and thus, having good approximations of measures on a Euclidean space through orthogonal projections as in Corollary \ref{cor:kl-change}.

\section{An empirical study of Assumption \ref{hyp:relaxed_bounded} for neural networks}
\label{sec: expes}
In this section, we check empirically whether the \texttt{QSB} assumption is verified for neural networks.
This allows us to verify if Theorem \ref{th:poincaré_gauss} is useful to understand the generalisation ability of neural nets.\\

\noindent\textbf{Experimental protocol.} 
We consider classification tasks on two datasets: MNIST~\citep{lecun1998mnist} and FashionMNIST~\citep{xiao2017fashion}.
We kept the original training set $\Scal_m$ and the original test set denoted by $\Tcal_n$ (of size $n$).
We consider the convolutional neural network of \citet{springenberg2015striving} adapted for MNIST and FashionMNIST.
The model is composed of $4$ layers containing $10$ channels with a $5{\times}5$-kernel; we set the stride and the padding to $1$, except for the second layer, where it is fixed to $2$. 
Each of these (convolutional) layers is followed by a Leaky ReLU activation function.
Moreover, an average pooling with a $8{\times}8$-kernel is performed before the Softmax activation function.
To initialise the weights of the network, we use \citet{glorot2010understanding} uniform initializer, while the biases are initialised in $[-\frac{1}{\sqrt{250}}, +\frac{1}{1/\sqrt{250}}]$ uniformly (except the first layer, the interval is $[-\frac{1}{5}, +\frac{1}{5}]$).
Hence, in this case, $\Hcal$ is the set of neural networks with a fixed architecture, and parametrised with a vector $\wbf$.
while the posterior distribution $\Q$ is a Gaussian measure $\Ncal(\wbf, \sigma^2\textrm{Id})$ centered on the parameters $\wbf$ associated with the model; $\sigma$ is set to $10^{-4}$. 
Note that this distribution respects the $\Poinc(c_P)$ assumption; see \Cref{sec:fast_rates_gradient_h}.
We train the neural network with the (vanilla) stochastic gradient descent algorithm, where the batch size is equal to $512$, and the learning rate is fixed to $10^{-2}$.
We train for at least $10^{4}$ gradient steps and finish the current epoch when this number of iterations is reached.
Our loss $\loss$ is the bounded cross-entropy loss of~\citet[Section D]{dziugaite2017computing}.\\
In \Cref{fig:expe}, we report the evolution of three quantities: {\it (i)} the estimated value of $C$, {\it (ii)} the test risk $\hat{\Risk}_{\Tcal_n}(\Q)$ and {\it (iii)} the test risk with the 01-loss.
More precisely, for computational reasons, the risks and $C$ are estimated by sampling one hypothesis $h\sim\Q$ and by computing the values on a mini-batch of $\Tcal_n$ (with $512$ examples) at each iteration.
Then, \Cref{fig:expe} represents averaged values on 5 runs, each point of the curve representing the average on 100 iterations of the training process (for $10^{4}$ iterations we only plot $10^{2}$ averaged points for clarity).

\begin{figure}[!h]
    \centering
    \includegraphics[width=1.0\linewidth]{chapter_4/figure.pdf}
    \caption{Evolution of the test risks (with the $01$-loss and the bounded cross-entropy loss) and the value of $C$ during the training phase.}
    \label{fig:expe}
\end{figure}

\paragraph{Empirical findings.}
\Cref{fig:expe} illustrates that, when neural networks are involved for two classification tasks, $\Q$ evolves during the optimisation process while maintaining the \texttt{QSB} property with constant $C<1$.
For both MNIST and FashionMNIST, the constant $C$ decreases from approximately 0.55 to 0.45. We deduce two things from this: \emph{(i)} the learning phase, while optimising $\hat{\Risk}_{\S_m}$ also gain in generalisation ability, shrinking the averaged loss on new data which is translated by a smaller $C$; and $(ii)$, having a data-free $\P$ (0 iteration) being \texttt{QSB} with $C<1$ suggests that the architecture of our neural network also has an influence on the $\texttt{QSB}$ assumption. As precised in \Cref{sec:poincare_gauss}, having $C<1$ attenuates the impact of the KL term, thus $\P$. This is desirable as it allows the optimiser to deeply explore the predictor space when $\P$ yields poor performances. We also note that the generalisation ability of $\Q$ on the training loss nearly matches the performance on the 0-1 loss for MNIST but is deteriorated for FashionMNIST, this invites to study more deeply the design of such surrogates in future work. 

\noindent{}Finally, the take-home message of this study is that the \texttt{QSB} assumption is verified for neural networks on MNIST. Such an empirical confirmation is crucial as it is required for our main result (Theorem \ref{th:poincaré_gauss}) and thus confirms that, for neural networks, reaching flat minima during the optimisation phase translates in increased generalisation ability.



\section{Conclusion}

This chapter showed that it is possible to exploit the benefits of a successful optimisation process to obtain faster rates, making a promising step forward a better understanding of deep neural networks, whose generalisation ability correlates well to flat minima. However, while we exploited potential benefits of optimisation process to make PAC-Bayes in line with optimisaiton, we still do not know whether an optimisation algorithm will reach a flat minima. This is somewhat unconsistent as optimisation processes are often supported by deterministic convergence guarantees. To fill this gap we show in \Cref{chap: wass-pb} that it is possible to incorporate directly optimisation guarantees onto PAC-Bayes bounds, making a supplementary step towards optimisation.

%!TEX root = main.tex
\chapter[Wasserstein PAC-Bayes Learning: Exploiting Optimisation Guarantees to Explain Generalisation]{Wasserstein PAC-Bayes Learning: Exploiting Optimisation Guarantees to Explain Generalisation}
\label{chap: wass-pb}
\addchapterlof
\addchapterloa
\addchapterloe

\vspace{-1.6cm}
\begin{center}
\textbf{This chapter is based on the following papers}\\
\end{center}
\vspace{-0.3cm}
\printpublication{haddouche2023wasserstein}

\vspace{-0.3cm}
\minitoc

\vspace{-0.2cm}

\begin{abstract}
    To make PAC-Bayes consistent with practical optimisation which often considers deterministic predictors, we need to consider PAC-Bayes learning beyond the KL divergence term which has been a cornerstone of PAC-Bayes since its emergence. In this chapter, we develop PAC-Bayes learning with Wasserstein distances, allowing to trade statistical assumptions for geometric ones. We also develop an explicit bridge with optimisation by incorporating the convergence guarantees of the \emph{Bures-Wasserstein SGD} into a generalisation bound. This is possible when considering the prior distribution as the learning objective.    
\end{abstract}

\newpage

\section{Introduction and state-of-the-art results}
In \Cref{chap: pb-ht,chap:online-pb,chap:gen-flat-minima}, we challenged many information-theoretic visions of PAC-Bayes (\Cref{fig: recap-info}) by limiting statistical assumptions, attenuating the impact of the prior seen as initialisation in both batch and online settings. However, we always involved a KL divergence term as a complexity measure of the predictor space, and this term is strongly linked to information theory. Indeed, a KL divergence focuses on posteriors being absolutely continuous \wrt the prior, meaning that it is possible to transfer information to posteriors with similar shape to the prior (and thus a finite KL divergence). While this vision makes perfectly sense from an information theoretic perspective, it is harder to justify such a condition from an optimisation stance. Indeed, Dirac prior and posterior (\ie deterministic predictors) are often considered in practice and this makes the KL infinite. Furthermore, KL divergence suffers from limitations as it does not satisfy classical properties such as the triangle inequality or even symmetry: it is challenging to exploit geometric properties of the measure space and the loss function through it. Then we might ask whether it is possible to maintain the flexibility of PAC-Bayes by involving another complexity measure, more compatible with optimisation. In this chapter we will develop PAC-Bayes learning theory based on Wasserstein distances, issued from optimal transport and compatible with such considerations. 
\paragraph{PAC-Bayes learning with Wasserstein distances.}
A recent line of work led by \citet{amit2022integral} investigates PAC-Bayes generalisation bounds with a Wasserstein distance rather than the KL. This idea has been simultaneously developed by \citet{ohana2023shedding} for sliced adaptive Wasserstein distances.
Also the recent work of \citet{mbacke2023pacbayesian} provides PAC-Bayesian bounds for adversarial generative models where the quantity of interest is a Wasserstein distance (although the complexity measure remains a KL divergence).

In the present chapter, we propose a major development of the emerging \emph{Wasserstein PAC-Bayes} (WPB) theory.
\citet{amit2022integral} provided the first high-probability WPB bounds with explicit convergence rates (for bounded losses) only for finite predictor classes or for linear regression problems. We extend those results to a broader framework including uncountable predictor classes and unbounded losses. We first propose a novel  WPB bound valid on any compact for bounded lipschitz losses. From this, we demonstrate that the WPB framework allows to bypass both the compactness assumption on the predictor class and the bounded loss assumption: Wasserstein PAC-Bayes only requires Lipschitz or smooth functions to be used. We obtain explicit bounds for the case of prior and posterior distributions taken within a compact space of Gaussian measures.
We also extend those results to the case of data-dependent priors, which is of interest when one compares the output of an algorithmic procedure to its minimisation objective.


As Wasserstein distance recently appeared as complexity measure in expected generalisation bounds (see \emph{e.g.} \citealp{rodriguez2021tighter}), the high-probability Wasserstein PAC-Bayes bounds presented here investigate deeper this lead. We also go a step further by showing that Wasserstein PAC-Bayes allows to reap the benefits of optimisation guarantees within generalisation. To the best of our knowledge, no previous PAC-Bayes bound has achieved this goal.
More precisely, we focus on the Bures-Wasserstein SGD \citep{altschuler2021aver,lambert2022variational} and show that the output of this algorithm, with enough data, after enough optimisation steps, is able to generalise well, independently of the quality of the initialisation point. The take-home message is that if an optimisation method has convergence guarantees with respect to a Wasserstein distance, then WPB theory allow us to determine, before any training, whether the algorithmic output will generalise well.

\paragraph{Outline.}
The reminder of this chapter is structured as follows: we state in \Cref{sec: framework} the framework and notation. In \Cref{sec: intro_optim}, we describe how current PAC-Bayes procedures are designed and how their efficiency is evaluated, and we discuss current limitations.
In \Cref{sec: intro_contrib}, we describe our main contributions, showing how we establish a WPB theory (using techniques which differ from those in \citealp{amit2022integral}) in order to exploit the optimisation results of \citet{lambert2022variational}.

\Cref{sec: compact_space} gathers results for compact predictor spaces, \Cref{sec: wpb_gauss} gives WPB bounds for Gaussian prior and posterior, \Cref{sec: data_dep_priors} contains a WPB bound with a data-dependent prior for unbounded Lipschitz losses. \Cref{sec: gene_sgd} establishes a link between optimisation and generalisation by exploiting the results of \citet{lambert2022variational} to establish new generalisation guarantees for the Bures-Wasserstein SGD. We defer to \Cref{sec: background} additional background notes and to \Cref{sec: proofs_chap_5} proofs which are not essential to the understanding of our contributions.


\subsection{Framework}
\label{sec: framework}

\paragraph{Learning theory framework.}
We consider a \emph{learning problem} specified by a tuple $(\mathcal{H}, \mathcal{Z}, \ell)$ of a set $\mathcal{H}$ of predictors, a data space $\mathcal{Z}$, and a loss function $\ell : \mathcal{H}\times \mathcal{Z} \rightarrow \mathbb{R} $.
We consider a finite dataset $\Sm=(\z_i)_{i\in \{1..m\}}\in\mathcal{Z}^{m}$ and assume that sequence is \iid following the distribution $\D$. We always assume that $\mathcal{H}\subseteq\mathbb{R}^d$, we denote by $\Sigma_{\mathcal{H}}$ the associated Borel $\sigma$-algebra and we denote by $||.||$ the classical Euclidean norm. We denote by $\mathcal{M}(\mathcal{H})$ the set of probability measures on $\mathcal{H}$.
We denote by $\mathcal{P}_1(\mathcal{H})$ (resp. $\mathcal{P}_2(\mathcal{H})$) the subspace of $\mathcal{M}(\mathcal{H})$ of with finite order 1 (resp. order 2) moments \wrt $||.||$.

\paragraph{Definitions.}
The \emph{generalisation error} $\Risk_{\D}$ of any predictor $h\in\mathcal{H}$ is $\Risk_{\D}(h)= \mathbb{E}_{z\sim \D}[\ell(h,\z)]$, the \emph{empirical error} of $h$ is $\Riskhat_{\Sm}(h)= \frac{1}{m}\sum_{i=1}^m\ell(h,\z_i)$.
The \emph{generalisation gap} of any $h$ is the quantity $\Delta_{\Sm}(h)=\Risk_{\D}(h)-\Riskhat_{\Sm}(h)$ and, for any $\Q\in\mathcal{M}(H)$, $\Delta_{\Sm}(\Q)= \mathbb{E}_{h\sim \Q}[\Delta_{\Sm}(h)]$. In what follows, we let $\mathcal{B}(\x,r)$ (resp. $\bar{\mathcal{B}}(\x,r)$) denote the ball (resp. closed ball) centered in $\x\in\mathbb{R}^d$ of radius $r$.
We define the \emph{Gibbs posterior} associated to the prior $\P\in\mathcal{M}(\mathcal{H})$ as the measure $\P_{-\lambda \Riskhat_{\Sm}}$ such that $\mathrm{d}\P_{-\lambda \Riskhat_{\Sm}} \propto \exp(-\lambda \Riskhat_{\Sm}(.)) \mathrm{d}\P(.)$.



We denote by $\text{BW}(\mathbb{R}^d)\subset \mathcal{P}_2(\mathbb{R}^d)$ the set of non-degenerate Gaussian distributions, also known as the \emph{Bures-Wasserstein space}. For a measurable function $T:\mathbb{R}^d \rightarrow \mathbb{R}^d$, and a measure $\P\in\mathcal{P}_1(\mathbb{R}^d)$ we let $T\#\P$ denote the measure such that for any $B\in \Sigma_{\mathbb{R}^d}, T\#\P(B)= \P(T^{-1}(B))$.
For any $R>0$, we denote by $\mathcal{P}_{R}$ the projection over $\bar{\mathcal{B}}(0_{\mathbb{R}^d},R)$.
Finally, as we consider compact sets of $\text{BW}(\mathbb{R}^d)$, we define for any $0\leq \alpha\leq \beta, M\geq 0$ the set
\[ C_{\alpha,\beta,M} := \left\{ \mathcal{N}(m,\Sigma) \in \text{BW}(\mathbb{R}^d) \mid ||m||\leq M,\; \alpha \mathrm{Id} \preceq \Sigma \preceq \beta \mathrm{Id} \right\}.  \]

\subsection{PAC-Bayes and optimisation: limits and caveats}
\label{sec: intro_optim}
\paragraph{Optimisation in PAC-Bayes.}
PAC-Bayesian generalisation bounds are meant to control how well measures derived from a learning algorithm perform on novel data.  Those bounds involves a complexity term which is typically a Kullback Leibler (KL) divergence. A prototypic bound is as follows: with probability $1-\delta$, for all measure $\Q$,

$$\Delta_{\Sm}(\Q) \leq \sqrt{\frac{\operatorname{COMP}(\Q)}{m}}, $$
where $\operatorname{COMP}$ is a complexity term involving a data-free prior $\P$ and an approximation term $1-\delta$.
From an optimisation perspective, this upper bound can be seen as a learning objective, where $\operatorname{COMP}$ acts as a regulariser to avoid overfitting on the empirical risk:

$$ \Q^* := \underset{\Q\in\mathcal{M}(\mathcal{H})}{\operatorname{argmin}} \Riskhat_{\Sm}(\Q) + \sqrt{\frac{\operatorname{COMP}(\Q)}{m}}.$$
Such algorithms are build to ensure a candidate measure with a good generalisation ability.
However the convergence of the optimisation process remains unclear: as $\sqrt{\operatorname{COMP}}$ is not necessarily convex in $\Q$, it is unclear whether an optimisation procedure on the previous learning objective will lead to $\hat{\Q}$ (or a good approximation of it). A good introductory example is to optimise the PAC-Bayesian learning objective for the following complexity term, holding for a loss $\ell$ being in $[0,1]$:

$$ \sqrt{\frac{\operatorname{COMP}(\Q)}{m}} := \frac{\operatorname{KL}(\Q,\P)}{\lambda} + \frac{\lambda}{2m},$$
with $\lambda$ being usually fine-tuned over a countable grid.
This objective, linear in the KL divergence term is optimised by the Gibbs posterior:
$$ \mathrm{d}\Q^*(h)\propto \exp(-\lambda \Riskhat_{\Sm}(h)) \mathrm{d}\P(h).$$
This distribution, while being known analytically, may be hard to compute in practice. A class of methods dedicated to compute or approximate this posterior distribution are the Markov Chain Monte Carlo (MCMC) methods that rely on carefully constructed Markov chains which (approximately) converge to $\Q^*$. However, MCMC methods can be computationally costly and other methods were studied to obtain quickly surrogates of $\Q^*$. In particular, \emph{Variational Inference} (VI) has been developed as a time-efficient solution. VI algorithms aims to estimate a surrogate $\hat{\Q}$ of $\Q^*$, often chosen within a parametric class of measures such as Gaussian measures. For instance, in order to approximate $\Q^*$ it is natural to consider the following surrogate:
$$ \hat{\Q} = \underset{Q\in \mathcal{C}}{\operatorname{argmin}} \operatorname{KL}(\Q,\Q^*) ,$$
where $\mathcal{C}$ is a subset of $\mathcal{M}(\mathcal{H})$. When $\mathcal{C}$ is the set of Gaussian measures (also known as the \emph{Bures-Wasserstein} manifold), the convergence of the associated VI algorithm has been studied \citep{altschuler2021aver,lambert2022variational}.
This candidate $\hat{\Q}$ is approximated after $N$ optimisation steps by a measure $\hat{\Q}_N$ and is then used in McAllester's bound to assess its efficiency:
\begin{align}
\label{eq: eval_posterior}
\Delta_{\Sm}(\hat{\Q}_N) \leq \sqrt{\frac{\operatorname{KL}(\hat{\Q}_N,\P) + \log(m/\delta)}{2m}}.
\end{align}

\paragraph{Role of the prior $\P$.}
From an optimisation perspective, the conclusion of \eqref{eq: eval_posterior} is that if $\hat{\Q}_N$ is a good approximation of $\hat{\Q}$ and if the initialisation $\P$ is well-chosen, then the generalisation ability $\hat{\Q}$ is guaranteed to be high. Assuming such a condition on $\P$ may be unrealistic. Furthermore the term $\operatorname{KL}(\hat{\Q}_N,\P)$ acts as a blackbox as we do not have a theoretical control on how far $\hat{\Q}$ and $\hat{\Q}_N$ diverge from the prior.
In particular if the prior is ill-chosen, then we could have $\operatorname{KL}(\hat{\Q}_N,\P) = \mathcal{O}(m)$, making \eqref{eq: eval_posterior} vacuous.
%Thus, even if the classical PAC-Bayes theorems are somewhat implying a role of regulariser for $\P$ when using the bounds as learning objectives, the role of $\P$ in such results is meaningless as it is not a good comparison point to express the benefits of the optimisation procedure.


\paragraph{Data-dependent priors are not enough to explain the generalisation gain through optimisation.}
As shown above, in order to have a sound theoretical control on the generalisation ability of the algorithmic output $\hat{\Q}_N$, it is irrelevant to compare it to the initialisation $\P$. Thus, it is legitimate to wonder if the existing PAC-Bayesian techniques using data-dependent priors are enough to fill this gap. To do so, we identify two strategies.
\begin{enumerate}
  \item Taking $\Q^*$ as a 'prior' distribution (as advised by \citealp{dziugaite2017computing}) is, at first sight, a convincing answer. However, the use of KL divergence is problematic. Indeed, we cannot make $\hat{\Q}$ appear easily in \Cref{eq: eval_posterior} which is the relevant point of interest. Furthermore, to our knowledge, there is no VI algorithm which guarantees that $\operatorname{KL}(\hat{\Q}_N,\Q^*)$ is decreasing.
  \item The prior is obtained from an algorithmic method on a fraction of training data. Then, such a bound does not inform us whether the considered optimisation method has been able to reach an optimum during the training phase: similarly to a test bound, it mainly assesses the post-training efficiency of the output of the learning algorithm. A relevant example is Table 3 of \citet{perezortiz2021learning} which considers data-dependent priors obtained through SGD. Then as the performance of the prior and the posterior is roughly similar, it is hard to determine whether the associated theoretical guarantee is more meaningful than a test bound as the prior measure could have already converged near a local optimum.
\end{enumerate}

\paragraph{A strategy to replace \eqref{eq: eval_posterior}.}
In order to assess whether the output of a learning algorithm enjoys high generalisation, a PAC-Bayes bound should satisfy the following generic form:
\begin{align}
\label{eq: wanted_pattern}
\Delta_{\Sm}(\hat{\Q}_N) \leq \sqrt{\frac{f(N)\operatorname{D}(\P,\hat{\Q}) + \varepsilon+ \log(m/\delta)}{2m}},
\end{align}
where $f$ is a function decreasing to $0$ as $N$ goes to infinity, which comes from the optimisation procedure, $\operatorname{D}$ is the way to measure the discrepancy between $\P,\hat{\Q}$ (classically it would be the KL divergence) and $\varepsilon$ is a residual term which could contain for instance the discrepancy $\operatorname{KL}(\Q^*,\hat{\Q})$ between the approximation and the true minimiser.
Such a guarantee would give theoretical evidence that the generalisation ability of $\hat{\Q}_N$ is independent of the choice of the initialisation point $\P$ and tends to $\mathcal{O}\left( \sqrt{\frac{\varepsilon + \log(m/\delta)}{m}} \right)$.
To the best of our knowledge, there is no work proposing an optimisation procedure such that $\operatorname{KL}(\hat{\Q}_N,\hat{\Q}) \leq f(N)  \operatorname{KL}(\P,\hat{\Q})$.  This lack is unfortunate but not surprising as the $KL$ divergence is not a distance: it is not easy to incorporate optimisation guarantees, often based on geometric properties of the loss, into the KL divergence.

\paragraph{Our aims in this chapter.}
A legitimate question is then: is it possible to extend the PAC-Bayes theory beyond the KL divergence in order to explain before training, with a bound of the form of \eqref{eq: wanted_pattern}, whether the output of optimisation procedure have high generalisation ability? We structure the present chapter to provide a positive answer to this question. More precisely we develop a WPB bound of the form of \eqref{eq: wanted_pattern} for the output of the Bures-Wasserstein SGD \citep{lambert2022variational}.



\subsection{Summary of our contributions}
\label{sec: intro_contrib}
To make PAC-Bayes learning useful to explain the generalisation ability of minimisers reached by optimisation algorithms, we develop theoretical results built around Wasserstein distances whose definitions are recalled below.

\begin{definition}
\label{def: wasserstein}
The $1$-Wasserstein distance between $\P,\Q \in \mathcal{P}_1(\mathcal{H})$ is defined as
\[ \W_{1}(\Q,\P) = \inf_{\pi \in \Pi(\Q,\P)} \int_{\mathcal{H}^2} ||x-y||\mathrm{d}\pi(x,y). \]
where $\Pi(\Q,\P)$ denote the set of probability measures on $\mathcal{H}^2$ whose marginals are $\Q$ and $\P$.
We define the $2$-Wasserstein distance on $\mathcal{P}_2(\mathcal{H})$ as
\[ \W_{2}(\Q,\P) = \sqrt{\inf_{\pi \in \Pi(\Q,\P)} \int_{\mathcal{H}^2} ||x-y||^2\mathrm{d}\pi(x,y)}. \]
\end{definition}
\citet{amit2022integral} provided a preliminary WPB bound, being explicit for the case of finite predictor classes and linear regression problems. To do so, they exploited the Kantorovich-Rubinstein duality (see, \emph{e.g.}, Remark 6.5 in \citealp{villani2009optimal}) of the $1$-Wasserstein distance. We exploit another duality formula (Theorem 5.10 in \citealp{villani2009optimal}) valid for any cost function (in the framework of optimal transport). This leads to a WPB bound valid for \emph{uniformly Lipschitz} loss functions.
\begin{definition}
\label{def: unif_lpz}
We say that a function $\ell:\mathcal{H}\times\mathcal{Z}\rightarrow \mathbb{R}$ is \emph{uniformly $K$-Lipschitz} if for any $\z\in\mathcal{Z}$, $\ell(.,\z)$ is $K$-Lipschtiz. We also say that a function is \emph{uniformly L-smooth} (or simply smooth) if for any $\z\in\mathcal{Z}$, its gradient $\nabla \ell(.,\z)$ is $L$-Lipschitz.
\end{definition}

\paragraph{A WPB bound for compact predictor classes.}
We first extend the PAC-Bayes framework to the case where the discrepancy between measures is expressed through the $1$-Wasserstein distance. It is stated as follows: for uniformly $K$-lipschitz functions bounded in $[0,1]$ with  $\mathcal{H}\subseteq \mathcal{B}_R:= \bar{\mathcal{B}}(0_{\mathbb{R}^d},R)$, we have for any prior $\P\in\mathcal{M}(\mathcal{H})$, with probability at least $1-\delta$, for any posterior distribution $\Q\in\mathcal{M}(\mathcal{H})$
\[ |\Delta_{\Sm}(\Q)| \leq \mathcal{O}\left(\sqrt{2K(2K+1)\frac{2d\log\left(3\frac{1 +2Rm }{\delta}\right)}{m} \left(1+\W_{1}(\Q,\P)  \right) +\frac{\log\left( \frac{m}{\delta} \right)}{m} } \right). \]
This bound extends the WPB bound of \citet{amit2022integral} to the case of a compact space of predictors. The proof technique exploits covering number arguments to prove the Lipschitzness (with high probability) of a relevant functional. The duality theorem of \citet[Theorem 5.10]{villani2009optimal} allows us to generate a local change of measure inequality (see, \emph{e.g.}, \citealp{donsker1976asymp}) required to use PAC-Bayes learning.
This bound is stated in \Cref{th: compact_mcall} and further discussed in \Cref{sec: compact_space}. However, this result does not cover the celebrated case of PAC-Bayes with Gaussian priors and posteriors. We then develop the next result to address this important case.

\paragraph{WPB bounds with Gaussians measures for unbounded losses.} Through the calculus of the residuals of Euler's Gamma function we obtain in \Cref{th: main_gaussian_lpz}, stated in \Cref{sec: wpb_gauss}, the following result when $\mathcal{H}=\mathbb{R}^d$, for loss functions lying in $[0,1]$ being uniformly $K$-lipschitz: for any gaussian prior $\P$ in a compact $C_{\alpha,\beta,M}\subseteq \operatorname{BW}(\mathbb{R}^d)$, with probability at least $1-\delta$, for any posterior distribution $\Q\in \mathcal{C}$,
\begin{multline*} 
    |\Delta_{\Sm}(\Q)| \\ 
    \leq \mathcal{O}\left(\sqrt{2K(2K+1)\frac{2d\log\left(3\frac{1 +2Rm }{\delta}\right)}{m} \left(1+ \sqrt{\frac{d}{m}} + \W_{1}(\Q,\P)  \right) +\frac{\log\left( \frac{m}{\delta} \right)}{m}} \right), 
\end{multline*}
where $R= \mathcal{O}(\max \sqrt{d\log(d)},\sqrt{\log(m)})$.
This shows that, using $R$ as an hyperparameter, we are able to maintain nearly the same convergence rate than \Cref{th: compact_mcall} at the cost of an extra factor of $\sqrt{\log(dm)}$.
Interestingly, we are able to remove in \Cref{cor: unbounded_lpz} the boundedness assumption to obtain a WPB bound, valid for unbounded uniformly $K$-lipschitz function  with an additional boundedness assumption on $\sup_{z} \ell(0,\z)$. This bound is more sensitive to the dimension of the problem when few data points are available. However, the asymptotic dependency remains (nearly) unchanged, at the cost of an extra polynomial factor in $\log(dm)$:
\begin{align}
|\Delta_{\Sm}(\Q)|  \leq\Tilde{\mathcal{O}}\left( \sqrt{2K\frac{d}{m}\left(1 + \W_{1}(\Q,\P)\right)+(1+K^2\log(m))\frac{\log\left( \frac{m}{\delta} \right)}{m}}   \right).
\end{align}
$\Tilde{\mathcal{O}}$ hides a polynomial dependency in $(\log(d),\log(m))$. This result is further discussed ion \Cref{sec: wpb_gauss}. The underlying proof technique is general enough to deal with (possibly unbounded) convex smooth loss functions. More details are gathered in \Cref{th: main_gaussian_smooth,cor: unbounded_smooth}.

\paragraph{A WPB bound with data-dependent prior.}
As we aim to intricate optimisation guarantees with generalisation bounds, we have to overcome the Bayesian paradigm of data-free priors which sets the prior distribution as a comparison point. Here, it is necessary to compare the candidate posterior with the optimisation goal. To do so, we elaborate in \Cref{sec: data_dep_priors} on the idea of \citet{dziugaite2018data} who exploit differential privacy to obtain PAC-Bayesian bounds allowing to take data-dependent priors. We show that it is possible to maintain the asymptotic convergence rate of \Cref{cor: unbounded_lpz} when taking as 'prior' a Gibbs posterior.
We introduce the following theorem holding again when $\mathcal{H}=\mathbb{R}^d$. For any gaussian prior $\P$ living in $C_{\alpha,\beta,M}$, with probability at least $1-\delta$, for any posterior distribution $\Q\in C_{\alpha,\beta,M}$, we have the following asymptotic convergence rate
\begin{align*}
|\Delta_{\Sm}(\Q)|  \leq\Tilde{\mathcal{O}}\left( \sqrt{2K\frac{d}{m}\left(1 + \W_{1}(\Q,\P_{-\frac{\lambda}{2K}\Riskhat_{\Sm}})\right)+(1+K^2\log(m))\frac{\log\left( \frac{m}{\delta} \right)}{m}}   \right).
\end{align*}
We also study non-asymptotic regimes in \Cref{th: data_dep}. While \citet{dziugaite2018data} exploited differential privacy results for the Gibbs posterior when the loss is bounded, we successfully extended these results to (possibly unbounded) uniformly Lipschitz losses. This is not specific to the WPB framework and may be of independent interest.

\paragraph{PAC-Bayes provides generalisation guarantees for the Bures-Wasserstein SGD.}
While working on WPB theory, we notice a shift from classical assumptions due to the KL divergence. Indeed, statistical assumptions (such as subgaussiannity, bounded variances) are transformed into geometric assumptions such as Lipschitzness and convex smoothness when Wasserstein distances are involved. We exploit in \Cref{sec: gene_sgd} WPB theory to provide generalisation guarantees for the Bures-Wasserstein SGD (recalled in \Cref{alg: sgd}) which approximates the best Gaussian surrogate $\hat{\Q}$ of $\Q^*:= \P_{-\frac{\lambda}{2K}\Riskhat_{\Sm}}$ (in the sense of the KL divergence, see \Cref{sec: gene_sgd} for more details).
More precisely, we show that the KL divergence and Wasserstein distances are linked within the WPB framework: the (KL-based) PAC-Bayesian learning objective of \citet{catoni2007pac}, which outputs the Gibbs posterior $\Q^*$, can be approximated by $\hat{\Q}_N$, the output of the Bures-Wasserstein SGD after $N$ optimisation steps, which is provably close from $\hat{\Q}$ with respect to the $2$-Wasserstein distance (see \Cref{th: lambert}).
Within the WPB framework, this link is translated in \Cref{th: main_sgd} as a generalisation bound ensuring that asymptotically, the minima reached by the Bures-Wasserstein SGD has a strong generalisation ability.
\medskip

Concretely, for $N$ large enough, for uniformly $K$-lipschitz, convex, smooth loss functions we have the following asymptotic guarantee with probability $1-\delta$:
\begin{align*}
|\Delta_{\Sm}(\hat{\Q}_N)|  \leq\Tilde{\mathcal{O}}\left( \sqrt{2K\frac{d}{m}\left(1 + \W_{1}(\hat{\Q},\Q^*)\right)+ (1+K^2\log(m)) \frac{\log\left( \frac{m}{\delta} \right)}{m}} \right).
\end{align*}
Thus, the WPB framework is enough to provide an explicit convergence rate for the generalisation gap avoiding the comparison to an arbitrary prior. Instead, this bound shows that a (long enough) run of the Bures-Wasserstein SGD with enough data (or a Lipschitz constant small enough) leads to a minimiser with a high generalisation ability. Furthermore, \Cref{th: main_sgd} is a reformulation of \eqref{eq: complete_bound_sgd} which is, to our knowledge, the first PAC-Bayesian bound of the form \eqref{eq: wanted_pattern} with $D=\sqrt{d \W_{2}}$ and $$\varepsilon= \mathcal{O}(\sqrt{d\W_{1}(\hat{\Q},\Q^*)}).$$
This provides elements of answer to the question listed in \Cref{sec: intro_optim} and concludes this work.

\paragraph{Discussion about the assumptions}
For the sake of clarity, we provide in \Cref{fig: overview} the topography of our main results. We focus on the assumptions required to state each of the results and doing so, we aim to give to the reader a broader vision of when can these bounds be applied. We stress that the Lipschitzness assumption is at the core of all results, except \Cref{th: main_gaussian_smooth,cor: unbounded_smooth}.
Convexity is required to use differential privacy and to obtain \Cref{th: data_dep}. Finally, we note that while the results of \citet{lambert2022variational} are usable with only smoothness and convexity, we must add the uniform Lipschitz assumption to obtain \Cref{th: main_sgd}. The question of whether all these assumptions are minimal to perform WPB remains open.
\begin{figure}[ht]
\label[figure]{fig: overview}
\centering
\includegraphics[scale=0.5]{chapter_5/overview.png}
\caption{An overwiew of the assumptions required to obtain the main results. Assumptions are stated in blue, main results are in pink boxes and the proof technique exploited to obtain such results are within grey boxes.}
\end{figure}

\section{PAC-Bayesian bounds for compact predictor spaces}
\label{sec: compact_space}

Here we establish WPB bounds for bounded losses when the predictor space is a compact of $\mathbb{R}^d$. To intricate the $1$-Wasserstein distance within the PAC-Bayes proof, we design a surrogate of the change of measure inequality \citep{donsker1976asymp} by exploiting the uniform Lipschitz assumption on the loss. To do so we need to exploit the notion of \emph{covering number} recalled below as well as Kantorovich duality \citep[Theorem 5.10]{villani2009optimal}.
This notion of duality holds for any cost function (in an optimal transport framework) contrary to the Kantorovich-Rubinstein duality exploited by \citet{amit2022integral} which only holds when the cost function is a distance. This result is recalled in \Cref{sec: back_compact}.
\begin{definition}[Covering number]
Let $\mathcal{H}\subseteq \mathbb{R}^d$. An $\varepsilon$-covering of $\mathcal{H}$ is a subset $C$ of $\mathcal{H}$ such that $\mathcal{H} \subseteq \cup_{x\in C} \bar{\mathcal{B}}(x,\varepsilon)$. The $\varepsilon$-covering number of $\mathcal{H}$ is defined as
$$N(\mathcal{H},\varepsilon):= \min\{n\geq 1 \mid \exists \text{ an $\varepsilon$-covering of $\mathcal{H}$ of size $n$} \}.$$
\end{definition}
We also define the $\varepsilon,1$-Wasserstein to be $\W_{\varepsilon}(\Q,\P)= \varepsilon + \W_{1}(\Q,\P)$. This cost function is essential to the analysis.
We now state the main results of this section. Additional background is gathered in \Cref{sec: back_compact}.

\subsection{A Catoni-type bound}

We propose here a WPB bound analogous to a relaxation of \citet[Theorem 1.2.6]{catoni2007pac} stated for instance in \citet[Theorem 4.1]{alquier2016properties}.

\begin{theorem}
\label{th: compact_catoni}
For any $\varepsilon,\delta>0$, assume that $\ell\in [0,1]$ is uniformly $K$-Lipschitz and that $\mathcal{H}$ is a compact of $\mathbb{R}^d$ bounded by $R>0$. Let $\P\in \mathcal{P}_1(\mathcal{H})$ be a (data-free) prior distribution and assume we choose a parameter $\lambda$ such that
\[ 0< \lambda \leq  \frac{1}{K}\sqrt{\frac{2m}{2d\log(1+\frac{2R}{\varepsilon})+\log(\frac{2}{\delta})}}:= \lambda_{max}. \]
Then, with probability $1-\delta$ , for any posterior distribution $\Q\in\mathcal{P}_1(K)$,
\[ \Delta_{\Sm}(\Q) \leq 4K\varepsilon + \frac{\W_{1}(\Q,\P)+2\varepsilon +\log(2/\delta)}{\lambda} + \frac{\lambda}{2m}.   \]
\end{theorem}
Note that we assumed the loss to be bounded, although this can be relaxed to subgaussiannity at no cost.
In \Cref{th: compact_catoni}, the range of $\lambda$ is restricted and the loss required to be uniformly Lipschitz. Such restrictions do not exist in \citet[Theorem 4.1]{alquier2016properties} which recovers a similar result with a KL divergence coming from the change of measure inequality \citep{donsker1976asymp}. In WPB this is required to have a control on $\Delta_S$ which is exploited in Kantorovich duality (\Cref{th: kanto_dual}).
Furthermore, assuming Lipschitzness on a compact space is not restrictive as it covers, \emph{e.g.}, all $\mathcal{C}^1$ functions.
Note that the smaller the Lipschitz constant $K$ is, the larger $\lambda_{max}$.
This is not surprising as, from an optimisation point of view, $\lambda$ acts as a learning rate which determines the influence of data with respect to the regulariser $\W_{1}(\Q,\P)$.
A small $K$ says that huge variations between data have a small influence on the loss value, then we can give more influence to the training set without deteriorating much the generalisation ability of the posterior.
This bound also says that it is legitimate to consider a WPB learning objective analogous to the one derived from \citet[Theorem 4.1]{alquier2016properties} (which yields Gibbs posteriors):
$$\operatorname{argmin}_{Q\in\mathcal{P}_1(\mathbb{\mathcal{H}})}\frac{\W_{1}(\Q,\P)}{\lambda} + \frac{\lambda}{2m}.$$
\Cref{th: compact_catoni}'s proof is stated below and mixes up several arguments from optimal transport with PAC-Bayes learning through covering numbers.
\paragraph{Proof of \Cref{th: compact_catoni}}

\textit{Step 1: define a good data-dependent function.} We define, for any sample $\Sm$ and predictor $h\in \mathcal{H}$,
\[ f_{\Sm}(h) = \lambda \Delta_{\Sm}(h). \]
This function satisfies the following lemma:
\begin{lemma}
\label[lemma]{l: quasi_lpz_func}
Let $\varepsilon>0$ assume that $0<\lambda\leq \frac{1}{K}\sqrt{\frac{2m}{\log\left(\frac{N(\mathcal{H},\varepsilon)^2}{\delta}\right)}} $. We have, with probability $1-\delta$ for all $h,h'\in\mathcal{H}$, for any $\P$:

\[f_{\Sm}(h)-f_{\Sm}(h') \leq 2(1+2\lambda K)\varepsilon + ||h-h'||.  \]

\end{lemma}
\begin{proof}[of \Cref{l: quasi_lpz_func}]
We rename here $N:= N(\mathcal{H},\varepsilon)$. There exists an $\varepsilon$-covering $C:=\{h_1,...,h_N\}$ of $\mathcal{H}$ of size $N$.
Then for any $h,h'\in C^2$, we have:
\begin{align*}
f_{\Sm}(h)-f_{\Sm}(h')  & = \frac{\lambda}{m}\sum_{i=1}^m \mathbb{E}[\ell(h,\z)-\ell(h'\z)] - \left(\ell(h,\z_i)-\ell(h',\z_i)\right).
\end{align*}
We know that for any $h,h',z$, $|\ell(h,\z)-\ell(h',\z)| \leq \lambda K||h-h'||$. Then, applying Hoeffding's inequality for all pairs $h,h'\in C^2$ and performing an union bound gives that with probability at least $1-\delta$, for all pairs $(h,h')\in C^2$ :
\begin{align*}
f_{\Sm}(h)-f_{\Sm}(h')  \leq \sqrt{\frac{\log\left(\frac{N^2}{\delta}\right)}{2m}} \lambda K ||h-h'||.
\end{align*}
So for any $h,h'\in \mathcal{H}^2$ there exists $h_0,h'_0\in C^2$ such that $||h-h_0||\leq \varepsilon$ and $||h-h_0||\leq \varepsilon$. Thus, we have
\begin{align*}
f_{\Sm}(h)-f_{\Sm}(h')  & = f_{\Sm}(h)-f_{\Sm}(h_0) + f_{\Sm}(h_0)-f(h'_0)+ f_{\Sm}(h'_0)-f_{\Sm}(h') \\
& \leq 2\lambda K \left(||h-h_0|| + ||h'-h'_0||\right) + \sqrt{\frac{\log\left(\frac{N^2}{\delta}\right)}{2m}} \lambda K ||h_0-h_0'|| \\
& \leq 4\lambda K \varepsilon + \sqrt{\frac{\log\left(\frac{N^2}{\delta}\right)}{2m}} \lambda K ||h_0-h'_0||.
\end{align*}
By the triangle inequality, $||h_0-h'_0||\leq ||h-h'|| + 2\varepsilon$ so we finally have with probability at least $1-\delta$, for any $h,h'\in K^2$:
\[  f_{\Sm}(h)-f_{\Sm}(h') \leq 4\lambda K \varepsilon + \sqrt{\frac{\log\left(\frac{N^2}{\delta}\right)}{2m}} \lambda K \left( 2\varepsilon+ ||h-h'||\right).\]
Using $\lambda \leq \frac{1}{K}\sqrt{\frac{2m}{\log\left(\frac{N^2}{\delta}\right)}}$ and upper bounding concludes the proof.
\end{proof}

\textit{Step 2: A probabilistic change of measure inequality for $f_{\Sm}$.}
We do not have for the Wasserstein distance such a powerful tool than the change of measure inequality. However, we can generate a probabilistic surrogate on $\mathcal{P}_1(\mathcal{H})$ valid for the function $f_{\Sm}$.

\begin{lemma}
\label[lemma]{l: change_meas_chap_5}
For any $\epsilon>0$, any $\delta>0$, any $$0<\lambda\leq \frac{1}{K}\sqrt{\frac{2m}{\log\left(\frac{N(\mathcal{H},\varepsilon)^2}{\delta}\right)}},$$
we have with probability $1-\delta$ over the sample $\Sm$, for any $\P\in\mathcal{P}_1(K)$
\[ \left(\sup_{Q\in \mathcal{P}_1(K)} \mathbb{E}_{h\sim \Q}[ f_{\Sm}(h)] - 2(1+\lambda K)\varepsilon - \W_{1}(\Q,\P) \right) \leq \mathbb{E}_{h\sim \P}[ f_{\Sm}(h)].     \]
\end{lemma}

\begin{proof}[of \Cref{l: change_meas_chap_5}]
Firstly, we introduce the cost function $c_{\varepsilon}(x,y)= \varepsilon + ||x-y||$.
From this we notice that we can rewrite the $\varepsilon,1$- Wasserstein distance:
\[ \W_{\varepsilon}(\Q,\P)= \inf_{\pi \in \Pi(\Q,\P)} \int_{\mathcal{H}^2} c_{\varepsilon}(x,y)\mathrm{d}\pi(x,y).  \]
Remark that because $\W_{1}$ is a distance, then $W_\varepsilon$ is symmetric. Furthermore, if we fix $\mathcal{X}=\mathcal{Y}=\mathcal{H}$ and we notice that $c_{\varepsilon}\geq 0$, then the condition for Kantorovich duality is satisfied. Thus, we apply \Cref{th: kanto_dual} as follows: for all $\Q,\P\in \mathcal{P}_1(\mathcal{H})$:
\begin{align*}
W_\varepsilon(\Q,\P)= W_\varepsilon(\P,\Q) & =  \min _{\pi \in \Pi(\P, \Q)}  \int_{K^2} c_{\varepsilon}(h_1, h_2) d \pi(h_1, h_2)  \\
&=\sup_{\substack{(\psi, \phi) \in L^1(\Q) \times L^1(\P)\\ \psi-\phi \leq c_{\varepsilon}}}\left[\int_{K} \psi(h) \mathrm{d}\Q(h)- \int_{K} \phi(h) \mathrm{d}\P(h)\right] \\
& = \sup_{\substack{(\psi, \phi) \in L^1(\Q) \times L^1(\P)\\ \psi-\phi \leq c_{\varepsilon}}}\left[\mathbb{E}_{h\sim \Q}[ \psi(h)]- \mathbb{E}_{h\sim \P}[ \phi(h)]\right].
\end{align*}
A crucial point is that for a well-chosen $\lambda$ with high probability, the pair $(f_{\Sm},f_{\Sm})$ satisfies the condition stated under the last supremum. It is formalised in the following lemma.
\begin{lemma}
\label{lem:kk}
For any $\varepsilon>0$ any $\delta>0$, any $0<\lambda\leq \frac{1}{K}\sqrt{\frac{2m}{\log\left(\frac{N(\mathcal{H},\varepsilon)^2}{\delta}\right)}}$ , we have with probability at least $1-\delta$ over the sample $\Sm$ that, for all measures $\Q,\P\in\mathcal{P}_1(\mathcal{H})^2$:
\begin{itemize}
  \item $f_{\Sm}\in L_1(\Q),L_1(\P)$,
  \item for all $h,h' \in \mathcal{H}^2, f_{\Sm}(h)-f_{\Sm}(h') \leq c_{\varepsilon'}(h,h')$ with $\varepsilon':= 2(1+2\lambda K) \varepsilon$.
\end{itemize}
Thus, Kantorovich duality (\Cref{th: kanto_dual}) gives:
\[ \left(\sup_{Q\in \mathcal{P}_1(\mathcal{H})} \mathbb{E}_{h\sim \Q}[ f_{\Sm}(h)] -  \W_{\varepsilon'}(\Q,\P) \right)\leq \mathbb{E}_{h\sim \P}[ f_{\Sm}(h)],    \]
and using $\W_{\varepsilon'} = \varepsilon' + \W_{1}$ and the definition of $\varepsilon'$ concludes the proof.
\end{lemma}
\begin{proof}[Proof of \Cref{lem:kk}]
Because the space of predictors $\mathcal{H}$ is compact and that for any $\z\in\mathcal{Z}$, the loss function $\ell(.,\z)$ is $K$-Lipschitz on $\mathcal{H}$, then both the generalisation and empirical risk are continuous on $\mathcal{H}$. Thus $|f_{\Sm}|$ is also continuous and, by compacity, reaches its maximum $M_S$ on $\mathcal{H}$. Thus for any probability $\P$ on $\mathcal{H}, \mathbb{E}_{h\sim \P}[|f_{\Sm}(h)|] \leq M_S < +\infty$ almost surely. This proves the first statement.
We notice that the second statement, given the choice of $\lambda$, is the exact conclusion  of \Cref{l: quasi_lpz_func} with probability at least $1-\delta$.
So with probability at least $1-\delta$, Kantorovich duality gives us that for any $\P,\Q$ with $\varepsilon'= 2(1+ \lambda K) \varepsilon$,
\begin{align*}
\mathbb{E}_{h\sim \Q}[ f_{\Sm}(h)] - \mathbb{E}_{h\sim \P}[f_{\Sm}(h)] \leq \W_{\varepsilon'}(\Q,\P).
\end{align*}
Re-organising the terms and taking the supremum over $\Q$ concludes the proof.
\end{proof}
This concludes the proof of \Cref{l: change_meas_chap_5}.
\end{proof}

\textit{Step 3: The PAC-Bayes route of proof for the 1-Wasserstein distance.}

We start by exploiting \Cref{l: change_meas_chap_5}: for any prior $\P\in\mathcal{P}_1(K)$, for $$0<\lambda \leq  \frac{1}{K}\sqrt{\frac{2m}{\log\left(\frac{2N(K,\varepsilon)^2}{\delta}\right)}},$$
with probability at least $1-\delta/2$ we have
\[ \left(\sup_{Q\in \mathcal{P}_1(K)} \mathbb{E}_{h\sim \Q}[ f_{\Sm}(h)] - (2(1+2\lambda K)\varepsilon - \W_{1}(\Q,\P) \right) \leq \mathbb{E}_{h\sim \P}[ f_{\Sm}(h)]. \]
We then notice that by Jensen's inequality,  $$\mathbb{E}_{h\sim \P}[ f_{\Sm}(h)] \leq \log\left(\mathbb{E}_{h\sim \P}[ \exp(f_{\Sm}(h))]    \right).$$
Then, by Markov's inequality we have with probability $1-\delta/2$
\[ \mathbb{E}_{h\sim \P}[ f_{\Sm}(h)] \leq \log\left(\frac{2}{\delta}\right) + \log\left(\mathbb{E}_S\mathbb{E}_{h\sim \P}\left[ \exp(f_{\Sm}(h))   \right]\right).  \]
By Fubini and Hoeffding lemma applied $m$ times on the iid sample $\Sm$, we have
\[ \mathbb{E}_S\mathbb{E}_{h\sim \P}\left[ \exp(f_{\Sm}(h))\right] =  \mathbb{E}_{h\sim \P}\mathbb{E}_S\left[ \exp(f_{\Sm}(h))\right] \leq \frac{\lambda^2}{2m}. \]
Taking an union bound gives us with probability $1-\delta$, for any posterior $\Q$:
\[ \mathbb{E}_{h\sim \Q}[\Risk_{\D}(h)] \leq \mathbb{E}_{h\sim \Q}[\Risk_{\Sm}(h)] +   4K\varepsilon + \frac{\W_{1}(\Q,\P)+2\varepsilon +\log(2/\delta)}{\lambda} + \frac{\lambda}{2m}.   \]
Finally, we know that $\mathcal{H}$ is bounded by $R$ so by \Cref{prop: covering} we have $$N^2 = N(\bar{\mathcal{B}}(0,R), \varepsilon)^2 \leq \left(1+ 2mR \right)^{2d}.$$
Thus, we can take $\lambda$ equal to $$\frac{1}{K}\sqrt{\frac{2m}{2d\log(1+\frac{2R}{\varepsilon})+\log(\frac{2}{\delta})}}.$$
This concludes the proof.






\subsection{A McAllester-type bound}
We now move on to a McAllester-type bound, which can be tighter than \Cref{th: compact_catoni} for large values of the $1$-Wasserstein.

\begin{theorem}
\label{th: compact_mcall}
For any $\delta>0$, assume that $\ell\in[0,1]$ is uniformly $K$-Lipschitz and that $\mathcal{H}$ is a compact of $\mathbb{R}^d$. Let $\P\in \mathcal{P}_1(\mathcal{H})$ a (data-free) prior distribution.
Then, with probability $1-\delta$ , for any posterior distribution $\Q\in\mathcal{P}_1(\mathcal{H})$:
\[ |\Delta_{\Sm}(\Q)| \leq \sqrt{2K(2K+1)\frac{2d\log\left(3\frac{1 +2Rm }{\delta}\right)}{m} \left(\W_{1}(\Q,\P)+\varepsilon_m  \right) + \frac{\log\left( \frac{3m}{\delta} \right)}{m}  }, \]
with $\varepsilon_m = \frac{4}{\log(\frac{3}{\delta})} \left( 2 + \sqrt{\frac{\log\left(\frac{3}{\delta}\right) + 2d\log(1+2Rm)}{2m}}  \right) = \mathcal{O}\left(1 + \sqrt{\frac{d\log(Rm)}{m}}\right)$.
\end{theorem}
We deteriorate the bound of \cite{amit2022integral} by transforming a convergence rate of $$\sqrt{\frac{\W_{1}(\Q,\P)}{m}}$$ for finite predictor classes onto a  $\sqrt{\left(Kd \W_{1}(\Q,\P)+1\right)\frac{\log(m)}{m}}$ for compact classes. This deteriorated rate is the price to pay to consider a general WPB bound for an uncountable number of predictors. However, notice that the dimension dependency can be attenuated through the Lipschitz constant, with the limit rate of $$\mathcal{O}\left(\sqrt{\frac{\log\left( \frac{m}{\delta} \right)}{m}}\right)$$ which is dimension-free and is a consequence of the statistical component of PAC-Bayes learning. Furthermore, note that this proof technique allows us to recover the rate of \citet{amit2022integral} rate when considering finite classes.
The proof of \Cref{th: compact_mcall} involves similar arguments to the one of \Cref{th: compact_catoni}, therefore we defer it to \Cref{sec: proof_compact_mcall}.

\section{PAC-Bayesian bounds for Gaussian distributions}
\label{sec: wpb_gauss}

In this section we develop McAllester-type WPB bounds on an Euclidean predictor space. Indeed, in PAC-Bayes learning, considering this predictor space is common as PAC-Bayesian objective often focuses on Gaussian priors and posteriors (see, \emph{e.g.}, \citealp{dziugaite2017computing,amit2018meta}).
Those bounds build up on \Cref{th: compact_mcall} and the overall conclusion is the following: when considering functions with interesting geometric properties (\emph{i.e.}, Lipschitzness or smoothness) on $\mathbb{R}^d$, WPB bounds hold for Gaussian priors and posteriors over $\mathcal{H}= \mathbb{R}^d$ at the cost of negligible extra terms (\Cref{th: main_gaussian_lpz,th: main_gaussian_smooth}).
More importantly, we show that in this setup, the assumption of a bounded loss is not required anymore to perform WPB: only boundedness on a compact is needed. Thus, we propose WPB bounds for unbounded losses (\Cref{cor: unbounded_lpz,cor: unbounded_smooth}).

\paragraph{Two sets of assumption.} Previously, we assumed two assumptions on the losses: uniform Lipschitzness (\Cref{def: unif_lpz}) and boundedness (in $[0,1]$) on a compact of $\mathbb{R}^d$. We provide below to novel sets of hypotheses which encapsulates previous assumptions while allowing the loss to be unbounded on all $\mathbb{R}^d$.
\begin{itemize}
  \item \textbf{(A1)} $\ell$ is uniformly $K$-Lipschitz over  $\mathcal{H}$, and $\sup_{z\in\mathcal{Z}} || \ell(0,\z)|| =D< +\infty.$
  \item \textbf{(A2)} For any $\z\in\mathcal{Z}$, $\ell(.,\z)$ is continuously differentiable over $\mathcal{H}$, $\ell(.,\z)$ is also a convex $L$- smooth (\emph{i.e}, its gradient is $L$-Lipschitz) and $\sup_{z\in\mathcal{Z}} ||\nabla_h \ell(0,\z)|| =D< +\infty$.
\end{itemize}
\begin{example}
Recall that $\mathcal{H}=\mathbb{R}^d$ and let $\phi:\mathcal{H}\rightarrow \mathbb{R}^d$.
Also, let $\psi :\mathcal{Z}\rightarrow \mathbb{R}^d$ such that $\psi(\mathcal{Z})$ is bounded by $C_\phi >0$. We assume that both $\phi,\psi$ are continuously differentiable and that $\nabla\phi$ is $G$-Lipschitz.
Note that the $||\phi||$ is possibly unbounded on $\mathcal{H}$.
Then \textbf{(A2)} holds for the loss function $\ell(h,\z)= ||\phi(h)-\psi(z)||^2$
Indeed, $\nabla_h \ell(h,\z)= 2(\nabla\phi (h) -\psi(z))$ so on any compact $\mathcal{K}$ bounded by $R$, $\nabla_h \ell$ is uniformly $2$-Lipschitz. Also $\sup_{z\in\mathcal{Z}} ||\nabla_h \ell(0,\z)|| \leq 2C$.
Note that on $\mathbb{R}^d$, $\ell(.,\z)$ is not necessarily Lipschitz for any $\z$ (take the case $\phi= Id_{\mathbb{R}^d}$) so \textbf{(A1)} is not satisfied.
\end{example}


\paragraph{A brief summary of the proof technique.} To extend \Cref{th: compact_mcall} to the case $\mathcal{H}=\mathbb{R}^d$, we use the push-forward distribution $\mathcal{P}_{R} \# \P$ where $\P\in C_{\alpha,\beta,M}$ for fixed $\alpha,\beta,M$ (notation defined in \Cref{sec: framework}).
The interest of this is to use \Cref{th: compact_mcall} by considering projections of the Gaussian prior and posterior. When considering Gaussian distributions, the gap between projected distributions and original ones is explicitly controlled.
More precisely, for any $R>0$ large enough, for any $\P\in C_{\alpha,\beta, M}$, $\W_{1}(\P, \mathcal{P}_{R}\# \P)$ is upper bounded. This is the conclusion of an important technical lemma (\Cref{l: gaussian_tail}), stated with additional background in \Cref{sec: background_gaussian}.
We state below new WPB results with Gaussian distributions for Lipschitz functions in \Cref{sec: main_sec_gaussian_lpz} and for smooth functions in \Cref{sec: main_sec_gaussian_smooth}.

\subsection{PAC-Bayesian bounds for Lipschitz losses}
\label{sec: main_sec_gaussian_lpz}

This section focuses on the case of Lipschitz losses. We show that when the loss is uniformly Lipschitz, it is possible to maintain the tightness of \Cref{th: compact_mcall} on all $\mathbb{R}^d$ when the loss remains bounded. We also show that it is also possible to obtain a WPB bound when the loss function satisfies \textbf{(A1)} (\emph{i.e.} with an additional boundedness assumption on $\sup_{z} \ell(0,\z)$), while remaining unbounded (\Cref{cor: unbounded_lpz}).

\begin{theorem}
\label{th: main_gaussian_lpz}
Assume that $d\geq 3$, $\mathcal{H}= \mathbb{R}^d$ and that the loss is uniformly $K$-Lipschitz and lies in $[0,1]$ over $\mathcal{H}$ . For any $\delta>0, 0\leq \alpha\leq \beta, M\geq 0$, let $\P\in C_{\alpha,\beta,M}$ a (data-free) prior distribution.
Then, with probability $1-\delta$ , for any posterior distribution $\Q\in C_{\alpha,\beta,M}$:
\begin{multline*}
|\Delta_{\Sm}(\Q)|  \\ 
\leq  2\frac{\beta{\sqrt{\beta}}}{m} + \sqrt{2K(2K+1)\frac{2d\log\left(3\frac{1 +2Rm }{\delta}\right)}{m} \left(\W_{1}(\Q,\P)+\alpha_m \right) + \frac{\log\left( \frac{3m}{\delta} \right)}{m}  },
\end{multline*}
with $R=\mathcal{O}(\max \sqrt{d\log(d)},\sqrt{\log(m)})$ and $\alpha_m= 2(M+1)\frac{\beta\sqrt{\beta}}{m} + \varepsilon_m= \mathcal{O}\left(1 + \sqrt{\frac{d\log(Rm)}{m}}\right)$ with $\varepsilon_m$ defined in \Cref{th: compact_mcall}.
\end{theorem}
\Cref{th: main_gaussian_lpz} shows that, at the cost of additional residual terms, it is possible to maintain the convergence rate of \Cref{th: compact_mcall} when considering Gaussian prior and posterior within the compact $C_{\alpha,\beta,M}$. The influence of $\alpha,\beta,\gamma$ appear in the explicit value of $R$ described as it is always taken in this work as the smallest value satisfying the assumption \texttt{Rad} described in \Cref{sec: background_gaussian}.
As in \Cref{th: compact_mcall}, the idea that a small Lipschitz constant tightens the bound is still conveyed here and is of great importance for \Cref{cor: unbounded_lpz} which provides a WPB bound for unbounded losses with higher dimension dependency when few data is available.
\begin{proof}[ of \Cref{th: main_gaussian_lpz}]
We take a specific radius $R$ which is the smallest value satisfying \texttt{Rad}.
The proof starts with a straightforward application of \Cref{th: compact_mcall} on  the compact $\mathcal{B}(0,R)$, with the prior $\mathcal{P}_{R}\# \P$, and with high probability, for any posterior $\mathcal{P}_{R}\# \Q$ with $\Q\in C_{\alpha,\beta,M}$:
\begin{multline*}
|\Delta_{\Sm}(\mathcal{P}_{R}\# \Q)|  \\\leq  \sqrt{2K(2K+1)\frac{2d\log\left(3\frac{1 +2Rm }{\delta}\right)}{m} \left(\W_{1}(\mathcal{P}_{R}\# \P,\mathcal{P}_{R}\# \P) + \varepsilon_m \right) + \frac{\log\left( \frac{3m}{\delta} \right)}{m} }.
\end{multline*}
From this we control the left hand-side term as follows:
\begin{align*}
|\Delta_{\Sm}(\Q)| &\leq |\Delta_{\Sm}(\mathcal{P}_{R}\# \Q)| + |\Delta_{\Sm}(\Q) - \Delta_{\Sm}(\mathcal{P}_{R}\# \Q) |.
\intertext{And we also have}
|\Delta_{\Sm}(\Q) - \Delta_{\Sm}(\mathcal{P}_{R}\# \Q) | & \leq \mathbb{E}_{h\sim \Q}\left[|\Delta_{\Sm}(h) - \Delta_{\Sm}(\mathcal{P}_{R}(h))|\right] \\
& = \mathbb{E}_{h\sim \Q}\left[|\Delta_{\Sm}(h) - \Delta_{\Sm}(\mathcal{P}_{R}(h))| \mathds{1}(||h||>R)\right] \\
&\leq 2\Q(||h||>R) \leq 2 \frac{\beta\sqrt{2\beta}}{m},
\end{align*}
the last line holding thanks to \Cref{l: gaussian_tail} and because $\Delta_S\in[-1,1]$.
Also we have by the triangle inequality:
\[\W_{1}(\mathcal{P}_{R}\# \P,\mathcal{P}_{R}\# \P) \leq \W_{1}(\Q, (\mathcal{P}_{R}\# \Q)) + \W_{1}(\Q,\P)+ \W_{1}(\P, \mathcal{P}_{R}\# \P). \]
Because both $\Q,\P\in C_{\alpha,\beta,M}$, using again \Cref{l: gaussian_tail} gives:
\[\W_{1}(\mathcal{P}_{R}\# \P,\mathcal{P}_{R}\# \P) \leq  \W_{1}(\Q,\P)+2(M+1)\frac{\beta\sqrt{2\beta}}{m}. \]
We then have:
\begin{multline*}
|\Delta_{\Sm}(\Q)| \leq 2 \frac{\beta\sqrt{2\beta}}{m} \\
+ \sqrt{2K(2K+1)\frac{2d\log\left(3\frac{1 +2Rm }{\delta}\right)}{m} \left(\W_{1}(\Q,\P)+ \alpha_m \right) + \frac{\log\left( \frac{3m}{\delta} \right)}{m} },
\end{multline*}
with $\alpha_m= 2(M+1)\frac{\beta\sqrt{\beta}}{m} + \varepsilon_m= \mathcal{O}(1)$. This concludes the proof.
\end{proof}

\paragraph{A corollary for unbounded losses.} We provably extend \Cref{th: main_gaussian_lpz} to the case of unbounded Lipschitz losses.

\begin{corollary}
\label[corollary]{cor: unbounded_lpz}
Assume that $d\geq 3$, $\mathcal{H}= \mathbb{R}^d$ and that the (unbounded) loss satisfies \textbf{(A1)}. For any $\delta>0, 0\leq \alpha\leq \beta, M\geq 0$, let $\P\in C_{\alpha,\beta,M}$ a (data-free) prior distribution.
Then, with probability $1-\delta$, for any posterior distribution $\Q\in C_{\alpha,\beta,M}$, the three following bounds holds.

\noindent \textbf{Low-data regime} $(d\geq m)$
\begin{align*}
|\Delta_{\Sm}(\Q)|  \leq\Tilde{\mathcal{O}}\left( \sqrt{2K\frac{d^{\frac{3}{2}}}{m}\left(\sqrt{\frac{d}{m}}+ \W_{1}(\Q,\P)\right)+ (1+K^2d)\frac{\log\left( \frac{m}{\delta} \right)}{m}}   \right).
\end{align*}
\noindent \textbf{Transitory regime} ($m>d,\; d\log(d)\geq \log(m)$)

\begin{align*}
|\Delta_{\Sm}(\Q)|  \leq\Tilde{\mathcal{O}}\left( \sqrt{2K\frac{d^{\frac{3}{2}}}{m}\left(1+ \W_{1}(\Q,\P)\right) + (1+K^2d)\frac{\log\left( \frac{m}{\delta} \right)}{m}}   \right).
\end{align*}

\noindent \textbf{Asymptotic regime} ($d\log(d)< \log(m)$)
\begin{align*}
|\Delta_{\Sm}(\Q)|  \leq\Tilde{\mathcal{O}}\left( \sqrt{2K\frac{d}{m}\left(1 + \W_{1}(\Q,\P)\right)+ (1+K^2\log(m))\frac{\log\left( \frac{m}{\delta} \right)}{m}}   \right).
\end{align*}
In all these formulas, $\Tilde{\mathcal{O}}$ hides a polynomial dependency in $(\log(d),\log(m))$.
For an explicit formulation of the bounds, we refer to \eqref{eq: complete_lpz_unbounded}.
\end{corollary}
The message here is that in Wasserstein PAC-Bayes, the bounded loss assumption is not as important as in classical PAC-Bayes using KL divergence. Indeed, the geometric constraints of WPB forced us to consider compact classes of Gaussian distribution and Lipschitz losses. Having such geometric assumptions on the distribution space and the loss is enough to exploit the properties of the $1$-Wasserstein distance and to circumvent the boundedness assumption.
To avoid boundedness, we transformed the limit rate $$\mathcal{O}\left(\sqrt{\frac{\log\left( \frac{m}{\delta} \right)}{m}}\right)$$ of \Cref{th: compact_mcall} into $$\mathcal{O}\left(\sqrt{(1+K^2  d)\frac{\log\left( \frac{m}{\delta} \right)}{m}}\right)$$ for non-asymptotic regimes
and $$\mathcal{O}\left(\sqrt{(1+K^2\log(m))\frac{\log\left( \frac{m}{\delta} \right)}{m}}\right)$$ for the asymptotic one.
Thus, even when few data is available, a well constrained (unbounded) Lipschitz loss is able to control the impact of the dimension.
Note that, in the small data regime, we have the highest dimension dependency. Note also that the dimensionality of the learning problem is controlled by the Lipschitz constant with the limit rate of $$\mathcal{O}\left(\sqrt{\frac{\log\left( \frac{m}{\delta} \right)}{m}}\right)$$ which is dimension-free and is a consequence of the statistical component of PAC-Bayes learning.
To the best of our knowledge, our work is the first to exploit geometric properties of the loss to propose PAC-Bayes bounds for unbounded and heavy-tailed losses with explicit convergence rates. Indeed, the existing literature on unbounded losses exploits either general divergence properties \citep{alquier2018simpler,picard2022change}, functional properties for heavy-tailed distribution \citep{holland2019pac}, uniform boundedness assumption on the loss over the data space \citep{haddouche2021pac} or concentration inequalities as in \Cref{chap: pb-ht} or in \citet{kuzborskij2019efron,rivasplata2020pac,jang2023tight}.

\begin{proof}[of \Cref{cor: unbounded_lpz}]
First, we start from  \Cref{th: compact_mcall} which gives, with probability at least $1-\delta$:
\begin{multline}
\label{eq: tight_mcall}
|\Delta_{\Sm}(\Q)| \\
\leq \sqrt{ 2K(2K+1) \frac{\log(\frac{3}{\delta}) + 2d\log\left(1 +2Rm \right)}{m} \left(\W_{1}(\Q,\P)+\varepsilon_m \right) + \frac{\log\left( \frac{3m}{\delta} \right)}{m} }.
\end{multline}
This last bound holds for any uniformly Lipschitz function taking value on $[0,1]$ on a compact predictor space bounded by a certain $R$.
Let $\P\in C_{\alpha,\beta,M}$.
We now assume \textbf{(A1)} and consider $R$ to be the smallest value satisfying \texttt{Rad}.
Let $\ell'= \ell/(D+2KR)$. We note $D_R= D + 2KR$, then on the ball $\mathcal{B}(0,R)$, $\ell'$ takes value in $[0,1]$ (because the compact is bounded by $R$ and the loss is $K$-Lipschitz) and is $K/D_R$-Lipschitz.
Applying \Cref{eq: tight_mcall} with $\ell'$ on $\mathcal{B}(0,R)$ and multiplying by $D_R$ gives, with high probability, for any $\Q\in C_{\alpha,\beta,M}$:
\begin{multline*}
|\Delta_{\Sm}(\mathcal{P}_{R}\# \Q)| \\ \leq D_R\sqrt{ 2 \frac{K}{D_R} (2 \frac{K}{D_R} +1) \frac{\log(\frac{1}{\delta}) + 2d\log\left(1 +2Rm \right)}{m} \left(\W_{1}(\mathcal{P}_{R}\# \P,\mathcal{P}_{R}\# \P)+\varepsilon_m \right) + \frac{\log\left( \frac{m}{\delta} \right)}{m} } \\
=\sqrt{ 2 K(2 K+D_R) \frac{\log(\frac{1}{\delta}) + 2d\log\left(1 +2Rm \right)}{m} \left(\W_{1}(\mathcal{P}_{R}\# \P,\mathcal{P}_{R}\# \P)+\varepsilon_m \right) + D_R^2\frac{\log\left( \frac{m}{\delta} \right)}{m} },
\end{multline*}
where $\varepsilon_m = \mathcal{O}\left(1\right)$ defined in \Cref{th: compact_mcall}.
As in \Cref{th: main_gaussian_lpz}, we have:
\[\W_{1}(\mathcal{P}_{R}\# \P,\mathcal{P}_{R}\# \P) \leq  \W_{1}(\Q,\P)+2(M+1)\frac{\beta\sqrt{2\beta}}{m}. \]
We have
\begin{align*}
|\Delta_{\Sm}(\Q)| &\leq |\Delta_{\Sm}(\mathcal{P}_{R}\# \Q)| + |\Delta_{\Sm}(\Q) - \Delta_{\Sm}(\mathcal{P}_{R}\# \Q) |,
\intertext{And we have}
|\Delta_{\Sm}(\Q) - \Delta_{\Sm}(\mathcal{P}_{R}\# \Q) | & \leq \mathbb{E}_{h\sim \Q}\left[|\Delta_{\Sm}(h) - \Delta_{\Sm}(\mathcal{P}_{R}(h))|\right] \\
& = \mathbb{E}_{h\sim \Q}\left[|\Delta_{\Sm}(h) - \Delta_{\Sm}(\mathcal{P}_{R}(h))| \mathds{1}(||h||>R)\right] .
\intertext{And because $\ell$ is $K$-Lipschitz, $\Delta_S$ is $2K$-Lipschitz and we have:}
|\Delta_{\Sm}(\Q) - \Delta_{\Sm}(\mathcal{P}_{R}\# \Q) |&\leq 2K\mathbb{E}[||h-\mathcal{P}_{R}(h)||\mathds{1}(||h||>R)] \\
& \leq 2K\mathbb{E}[||h||\mathds{1}(||h||>R)].
\intertext{Finally, applying \Cref{l: gaussian_tail} gives:}
|\Delta_{\Sm}(\Q) - \Delta_{\Sm}(\mathcal{P}_{R}\# \Q) |&\leq 2K(M+1)\frac{\beta\sqrt{2\beta}}{m} = \mathcal{O}\left( \frac{1}{m} \right).
\end{align*}
Then we have:
\begin{multline}
\label{eq: complete_lpz_unbounded}
|\Delta_{\Sm}(\Q)| \leq 2K(M+1)\frac{\beta\sqrt{2\beta}}{m} + \\
\sqrt{ 2 K(2 K+D_R) \frac{\log(\frac{1}{\delta}) + 2d\log\left(1 +2Rm \right)}{m} \left(\W_{1}(\Q,\P)+\alpha_m \right) + D_R^2\frac{\log\left( \frac{3m}{\delta} \right)}{m} },
\end{multline}
where $\alpha_m= \mathcal{O}\left(1 + \sqrt{\frac{d\log(Rm)}{m}}\right)$ defined in \Cref{th: main_gaussian_lpz}.
Finally we exploit that $R= \mathcal{O}(\sqrt{d\log(d)},\sqrt{\log(m)})$ (cf. \Cref{rem: rad_rate}) and $D_R=\mathcal{O}(1+K^2R)$, to conclude the proof for all the three regimes.
\end{proof}

\subsection{PAC-Bayesian bounds for convex smooth functions}
\label{sec: main_sec_gaussian_smooth}

This section is focused on convex smooth loss functions, which are well suited for many optimisation objectives.
We show that under \textbf{(A2)}, it is possible to transform \Cref{th: compact_mcall} into a bound for smooth functions on all $\mathbb{R}^d$ when the loss remain bounded. We also show that it is possible to obtain a PAC-Bayesian bound for smooth unbounded loss functions.

\begin{theorem}
\label{th: main_gaussian_smooth}
Assume that $d\geq 3$, $\mathcal{H}= \mathbb{R}^d$ and that the loss satisfies \textbf{(A2)} and lies in $[0,1]$ over $\mathcal{H}$. For any $\delta>0, 0\leq \alpha\leq \beta, M\geq 0$, let $\P\in C_{\alpha,\beta,M}$ a (data-free) prior distribution. Then, with probability $1-\delta$, for any posterior distribution $\Q\in C_{\alpha,\beta,M}$:
\begin{multline*}
|\Delta_{\Sm}(\Q)| \leq 2 \frac{\beta\sqrt{2\beta}}{m} \\
+ \sqrt{2D_R(2D_R+1)\frac{2d\log\left(3\frac{1 +2Rm }{\delta}\right)}{m} \left(\W_{1}(\Q,\P)+ \alpha_m \right) + \frac{\log\left( \frac{3m}{\delta} \right)}{m} },
\end{multline*}
with $R= \mathcal{O}\left( \max \sqrt{d\log(d)}, \sqrt{\log(m)}   \right)$,  $D_R= D+LR$ and $\alpha_m= \mathcal{O}(1)$ is defined in \Cref{th: main_gaussian_lpz}.
\end{theorem}
The key idea of the proof is to state that on a compact space, a smooth function is also Lipschitz. Therefore, the proof follows the same route as the one of \Cref{th: main_gaussian_lpz}, with additional technical steps. We then defer it to \Cref{sec: proof_smooth}.
We note that, even for bounded losses, the price to pay to consider smooth functions instead of Lipschitz ones is an extra factor $D_R= \mathcal{O}(1+R)$ when $D>0$. Therefore, in the general case we lose the idea that a tight smooth function will change the convergence rate of the problem as in general the upper bound $D$ of  $\sup_{z}|\ell(0_{\mathbb{R}^d},z)|$ is greater than zero. However, we are able to obtain results still useful when enough data is available. We also show it is possible to obtain a WPB bound for unbounded convex smooth functions.

\begin{corollary}
\label[corollary]{cor: unbounded_smooth}
Assume that $d\geq 3$, $\mathcal{H}= \mathbb{R}^d$ and that the (unbounded) loss satisfies \textbf{(A2)}. For any $\delta>0, 0\leq \alpha\leq \beta, M\geq 0$, we assume that $R>0$ is the smallest value satisfying \texttt{Rad}.
We assume that $\sup_{z\in\mathcal{Z}} || \ell(0,\z)|| =D_\ell < +\infty$.
Let $\P\in C_{\alpha,\beta,M}$ a (data-free) prior distribution.
Then, with probability $1-\delta$ , for any posterior distribution $\Q\in C_{\alpha,\beta,M}$, the three following bounds holds.

\noindent \textbf{Low-data regime} $(d\geq m)$
\begin{align*}
|\Delta_{\Sm}(\Q)|  \leq\Tilde{\mathcal{O}}\left( \sqrt{\frac{d^{\frac{5}{2}}}{m}\left(\sqrt{\frac{d}{m}}+ \W_{1}(\Q,\P)\right)}   \right).
\end{align*}

\noindent \textbf{Transitory regime} $(d<m, d\log(d)\geq \log(m))$

\begin{align*}
|\Delta_{\Sm}(\Q)|  \leq\Tilde{\mathcal{O}}\left( \sqrt{\frac{d^{\frac{5}{2}}}{m}\left(1+ \W_{1}(\Q,\P)\right)}   \right).
\end{align*}

\noindent \textbf{Asymptotic regime} $(d\log(d)< \log(m))$
\begin{align*}
|\Delta_{\Sm}(\Q)|  \leq\Tilde{\mathcal{O}}\left( \sqrt{\frac{d}{m}}\left(1 + \W_{1}(\Q,\P)\right)  \right).
\end{align*}
In all these bounds, $\Tilde{\mathcal{O}}$ hides a polynomail factor in $(\log(d),\log(m))$.
For a complete formulation of the bounds, we refer to \eqref{eq: complete_smooth_unbounded}.
\end{corollary}
We remark that this theorem is particularly interesting in the transitory and asymptotic regime as, contrary to \Cref{cor: unbounded_lpz}, we do not have a Lipschitz constant to attenuate the impact of the dimension (indeed we have $D_R = D + LR$ and in general $D>0$). However, this bound remains of great interest when many data are available as the smoothness assumption is often used in optimisation.

\begin{proof}[of \Cref{cor: unbounded_smooth}]

Firstly, we use \Cref{th: compact_mcall} which state that for any prior on a compact, loss function $\ell\in  [0,1]$ being uniformly $K$-Lipschitz on this compact gives with probability at least $1-\delta$:
\begin{equation*}
|\Delta_{\Sm}(\Q)| \leq \sqrt{ 2K(2K+1) \frac{\log(\frac{3}{\delta}) + 2d\log\left(1 +2Rm \right)}{m} \left(\W_{1}(\Q,\P)+\varepsilon_m \right) + \frac{\log\left( \frac{3m}{\delta} \right)}{m} }.
\end{equation*}
Let $\P\in C_{\alpha,\beta,M}$. We fix $R$ to be the smallest value satisfying \texttt{Rad} and we assume \textbf{(A2)}.
On $\mathcal{B}(0,R)$, as seen in the proof of \Cref{th: main_gaussian_smooth}, $\ell$ is uniformly $D_R:= D+LR$-Lipschitz, so $\ell$ is bounded on this ball by $C_R:=D_\ell+RD_R= \mathcal{O}(1+R^2)$.
We apply \Cref{th: compact_mcall} on the loss function $\ell'= \ell/C_R$ and we multiply the resulting bound by $C_R$. Recall that $\ell'$ takes value in $[0,1]$  and is $D_R/C_R$-Lipschitz. We then have with high probability, for any $\Q\in C_{\alpha,\beta,M}$:
\begin{multline*}
|\Delta_{\Sm}(\mathcal{P}_{R}\# \Q)| \\ \leq \sqrt{ 2D_R(2D_R+C_R) \frac{\log(\frac{3}{\delta}) + 2d\log\left(1 +2Rm \right)}{m} \left(\W_{1}(\mathcal{P}_{R}\# \P,\mathcal{P}_{R}\# \P)+\varepsilon_m \right) + C_R^2\frac{\log\left( \frac{3m}{\delta} \right)}{m} },
\end{multline*}
where $\varepsilon_m = \mathcal{O}\left(1\right)$ defined in \Cref{th: compact_mcall}.
As in \Cref{th: main_gaussian_lpz}, we have:
\[\W_{1}(\mathcal{P}_{R}\# \P,\mathcal{P}_{R}\# \P) \leq  \W_{1}(\Q,\P)+2(M+1)\frac{\beta\sqrt{2\beta}}{m}. \]
We have:
\begin{align*}
|\Delta_{\Sm}(\Q)| &\leq |\Delta_{\Sm}(\mathcal{P}_{R}\# \Q)| + |\Delta_{\Sm}(\Q) - \Delta_{\Sm}(\mathcal{P}_{R}\# \Q) |,
\intertext{And we have:}
|\Delta_{\Sm}(\Q) - \Delta_{\Sm}(\mathcal{P}_{R}\# \Q) | & \leq \mathbb{E}_{h\sim \Q}\left[|\Delta_{\Sm}(h) - \Delta_{\Sm}(\mathcal{P}_{R}(h))|\right] \\
& = \mathbb{E}_{h\sim \Q}\left[|\Delta_{\Sm}(h) - \Delta_{\Sm}(\mathcal{P}_{R}(h))| \mathds{1}(||h||>R)\right] .
\intertext{We study the last gap more carefully:}
|\Delta_{\Sm}(h) - \Delta_{\Sm}(\mathcal{P}_{R}(h))|& = \mathbb{E}_z[|\ell(h,\z)- \ell(\mathcal{P}_{R}(h),\z)|] \\
& + \frac{1}{m}\sum_{i=1}^m |\ell(h,\z_i)- \ell(\mathcal{P}_{R}(h),z_i)|.
\intertext{And we know that for any $\z$, because $\ell$ is convex smooth:}
\ell(h,\z)- \ell(\mathcal{P}_{R}(h),\z) &\leq \nabla_h \ell(\mathcal{P}_{R}(h),\z)^T(h-\mathcal{P}_{R}(h)) + \frac{L}{2}||h-\mathcal{P}_{R}(h)||^2 || \\
& \leq D_R||h- \mathcal{P}_{R}(h)|| + \frac{L}{2}||h-\mathcal{P}_{R}(h)||^2 ||.
\intertext{We also have by convexity:}
\ell(\mathcal{P}_{R}(h),\z) - \ell(h,\z) &\leq \nabla_h\ell(\mathcal{P}_{R}(h),\z)^T(\mathcal{P}_{R}(h)-h) \\
& \leq D_R ||h-\mathcal{P}_{R}(h)||.
\intertext{In any case, we have for any $h,\z$: }
|\ell(h,\z)- \ell(\mathcal{P}_{R}(h),\z)| &\leq D_R||h-\mathcal{P}_{R}(h)|| + \frac{L}{2}||h-\mathcal{P}_{R}(h)||^2.
\intertext{Thus:}
|\Delta_{\Sm}(\Q) - \Delta_{\Sm}(\mathcal{P}_{R}\# \Q) | & \leq D_R\mathbb{E}_{h\sim \Q}\left[||h-\mathcal{P}_{R}(h)||\mathds{1}(||h||>R)\right] \\
& + \frac{L}{2}\mathbb{E}_{h\sim \Q}\left[||h-\mathcal{P}_{R}(h)||^2\mathds{1}(||h||>R)\right] \\
& \leq D_R\mathbb{E}_{h\sim \Q}\left[||h||\mathds{1}(||h||>R)\right] \\
& + \frac{L}{2}\mathbb{E}_{h\sim \Q}\left[||h||^2\mathds{1}(||h||>R)\right].
\intertext{And thanks to \Cref{l: gaussian_tail}, we finally have:}
|\Delta_{\Sm}(\Q) - \Delta_{\Sm}(\mathcal{P}_{R}\# \Q) | & \leq \left( D_R + \frac{L}{2}(M+1) \right)(M+1)\frac{\beta\sqrt{\beta}}{m}.
\end{align*}
Then we have:
\begin{multline}
\label{eq: complete_smooth_unbounded}
|\Delta_{\Sm}(\Q)| \leq \left( D_R + \frac{L}{2}(M+1) \right)(M+1)\frac{\beta\sqrt{\beta}}{m} + \\
\sqrt{ 2D_R(2D_R+C_R) \frac{\log(\frac{3}{\delta}) + 2d\log\left(1 +2Rm \right)}{m} \left(\W_{1}(\Q, \P)+\alpha_m \right) + C_R^2\frac{\log\left( \frac{3m}{\delta} \right)}{m} },
\end{multline}
where $\alpha_m= \mathcal{O}\left(1 + \sqrt{\frac{d\log(Rm)}{m}}\right)$ defined in \Cref{th: main_gaussian_lpz}.
Finally, we exploit that $R= \mathcal{O}(\sqrt{d\log(d)},\sqrt{\log(m)})$ (cf \Cref{rem: rad_rate}), that $D_R=\mathcal{O}(1+R)$ and $C_R=\mathcal{O}(1+R^2)$, to conclude the proof for all the three regimes.
\end{proof}

\section{Wasserstein PAC-Bayes with data-dependent priors}
\label{sec: data_dep_priors}

In PAC-Bayes learning, obtaining results holding with data-dependent priors is a widely studied topic. The reason behind that is that it is more meaningful to compare the posterior distribution, usually obtained via an optimisation procedure to a competitive one (classically the Gibbs posterior) to ensure tight generalisation bounds.
A classical way to do so is to use differential privacy as in \citet{dziugaite2018data}. However, their contribution relies on bounded losses to apply the \emph{exponential mechanism}, a useful tool to determine whether an algorithm is differentially private. We exploit new theorems from \citet{minami2016diff,rogers2016max} which allow us to exploit differentially private priors when the loss is unbounded, convex and Lipschitz. We recall in \Cref{sec: back_dp} elements of differential privacy.

\paragraph{A PAC-Bayesian bound for Lipschitz convex losses with data-dependent prior.} We now state a PAC-Bayes theorem valid for differentially private probability kernels. The proof elaborates on \citet[Theorem 4.2]{dziugaite2018data} and is based on the following bound, which is a minor modification of \eqref{eq: complete_lpz_unbounded}, making it valid for any prior (and not only Gaussian ones).

\begin{theorem}
\label{th: data_dep}
Assume that $d\geq 3$, $\mathcal{H}= \mathbb{R}^d$ and that the loss is convex and satisfies \textbf{(A1)}. Let $\beta_m= \mathcal{O}(\frac{1}{\sqrt{m}})$ and $\lambda \leq \sqrt{m}$.
Let $\P\in C_{\alpha,\beta,M}$ a (data-free) prior distribution. Then, for any $\beta_m<\delta<1$, with probability $1-\delta$ , for any posterior distribution $\Q\in C_{\alpha,\beta,M}$ and the Gibbs prior $\P_{-\frac{\lambda}{2K} \Riskhat_{\Sm}}$, the following bound holds.

\noindent\textbf{Low-data regime} $(d\geq m)$
\begin{multline*}
|\Delta_{\Sm}(\Q)|  \leq \\
\Tilde{\mathcal{O}}\left( \sqrt{2K\frac{d^{\frac{3}{2}}}{m}\left(\sqrt{\frac{d}{m}}+  \W_{1}(\Q,\P_{-\frac{\lambda}{2K}\Riskhat_{\Sm}}) +f_R\left(\P_{-\frac{\lambda}{2K}\Riskhat_{\Sm}}\right)\right)+(1+K^2d)\frac{\log\left( \frac{m}{\delta} \right)}{m}}   \right).
\end{multline*}
\textbf{Transitory regime} $(m>d,\; d\log(d)\geq \log(m))$

\begin{multline*}
|\Delta_{\Sm}(\Q)|  \leq\\ 
\Tilde{\mathcal{O}}\left( \sqrt{2K\frac{d^{\frac{3}{2}}}{m}\left(1+ \W_{1}(\Q,\P_{-\frac{\lambda}{2K}\Riskhat_{\Sm}})+ f_R\left(\P_{-\frac{\lambda}{2K}\Riskhat_{\Sm}}\right)\right)+(1+K^2d)\frac{\log\left( \frac{m}{\delta} \right)}{m}}   \right).
\end{multline*}
\textbf{Asymptotic regime} $(d\log(d)< \log(m))$
\begin{multline*}
|\Delta_{\Sm}(\Q)|  \leq\\
\Tilde{\mathcal{O}}\left( \sqrt{2K\frac{d}{m}\left(1 + \W_{1}(\Q,\P_{-\frac{\lambda}{2K}\Riskhat_{\Sm}})\right)+(1+K^2\log(m))\frac{\log\left( \frac{m}{\delta} \right)}{m}}   \right),
\end{multline*}
where $R= \mathcal{O}\left( \max \sqrt{d\log(d)}, \sqrt{\log(m)}   \right)$, $f_R(\P) := \W_{1}(\mathcal{P}_{R}\#\P,\P)$.
In the above $\Tilde{\mathcal{O}}$ hides a polynomial dependency in $(\log(d),\log(m))$. For an explicit formulation of the bounds, we refer to \eqref{eq: complete_data_dep}.
\end{theorem}
Note that in the asymptotic bound, the condition to get rid of $f_R(\P_{-\frac{\lambda}{2K}\Riskhat_{\Sm}})$ is that $\lambda$ is a fixed constant, in particular it does not depend on $m$. This is essential to apply the law of large numbers: a fixed learning rate in the Gibbs posterior is required for a bound with only explicit terms.
Furthermore, an important message is that Lipschitz functions are well suited to the PAC-Bayes framework through Wasserstein distances. Indeed, not only are we able to recover McAllester or Catoni-type WPB bounds, but we also obtain WPB with data-dependent priors using the same techniques than PAC-Bayes learning with KL divergences. Data-dependent WPB bounds have also an additional benefit as they provide guarantees for the Bures-Wasserstein SGD of \citet{lambert2022variational} as shown in \Cref{sec: gene_sgd}.

\begin{proof}[of \Cref{th: data_dep}]
Firstly, we start from a slightly modified version of \Cref{eq: complete_lpz_unbounded} which holds for any prior distribution (and not only Gaussian ones).
To obtain it we restart from the triangle inequality $\W_{1}(\mathcal{P}_{R}\# \Q, \mathcal{P}_{R}\# \P ) \leq \W_{1}(\mathcal{P}_{R}\# \Q,\Q) + \W_{1}(\Q,\P) + f_R(\P)$. where $f_R(\P) := \W_{1}(\mathcal{P}_{R}\#\P,\P)$ and we apply exactly the same route of proof than in \Cref{cor: unbounded_lpz}. We then obtain, for any data-free prior $\P$, with probability at least $1-\delta$, for any $\Q\in C_{\alpha,\beta,M}$:
\begin{multline*}
|\Delta_{\Sm}(\Q)| \leq 2K(M+1)\frac{\beta\sqrt{2\beta}}{m} + \\
\sqrt{ C_R \frac{\log(\frac{1}{\delta}) + 2d\log\left(1 +2Rm \right)}{m} \left(\W_{1}(\Q,\P)+\alpha_m  + f_R(\P)\right) + D_R^2\frac{\log\left( \frac{m}{\delta} \right)}{m} },
\end{multline*}
where $D_R= D+KR$ and $C_R= 2K(2K+D_R)$ ($D,K$ defined in \textbf{(A1)}).
We then denote by $\texttt{Bound}(\Sm,\P,\Q,\delta)$ the bound:
\begin{multline*}
|\Delta_{\Sm}(\Q)| > 2K(M+1)\frac{\beta\sqrt{2\beta}}{m} + \\
\sqrt{ C_R \frac{\log(\frac{1}{\delta}) + 2d\log\left(1 +2Rm \right)}{m} \left(\W_{1}(\Q,\P)+\alpha_m  + f_R(\P)\right) + D_R^2\frac{\log\left( \frac{m}{\delta} \right)}{m} }.
\end{multline*}
And for a given $\delta'$, let $$\texttt{Ev}(\P,\delta'):= \{\Sm \in \mathcal{Z}^m \mid  \exists \Q \in C_{\alpha,\beta,M} \text{ s.t. } \texttt{Bound}(\Sm,\P,\Q,\delta') \text{ holds} \}$$.
We know that for a data-free prior $\P$, $\mathbb{P}_{\Sm\in\D^m}(\Sm\in \texttt{Ev}(\P)) \leq \delta$.
To exploit the differential privacy framework, we first assume having a differentially private probability kernel $\mathcal{P}$. We fix $\beta>0$ and re-exploit the idea of \citet{dziugaite2018data}:
\begin{align}
\label{eq: pac_b_diff_priv}
\mathbb{P}_{\Sm \sim \D^m}\{\Sm \in \texttt{Ev}(\mathcal{P}(S),\delta')\} & \leq \mathrm{e}^{\mathbf{I}_{\infty}^\beta(\mathcal{P} ; m)} \underset{\left(S, \Sm^{\prime}\right) \sim \D^{2m} }{\mathbb{P}}\left\{\Sm \in \texttt{Ev}\left(\mathcal{P}\left(\Sm^{\prime}\right)\right)\right\}+\beta \\
& \leq \mathrm{e}^{\mathbf{I}_{\infty}^\beta(\mathcal{P} ; m)} \delta^{\prime}+\beta = \delta .
\end{align}
The last line holds for any $\delta > \beta$ by fixing $\delta^{\prime}=\mathrm{e}^{-I_{\infty}^\beta(\mathcal{P} ; m)}(\delta-\beta)$.
Note that $\log\left(\frac{1}{\delta'}   \right) = \log\left(\frac{1}{\delta - \beta}   \right) + I_{\infty}^\beta(\mathcal{P} ; m)$, this suggests to bound the $\beta$-approximate max-information. To do so, we need to give specific values for the pair $(\varepsilon,\gamma)$.
More concretely, let $\varepsilon= \sqrt{\frac{\log(m)}{m}}, \gamma= \frac{\varepsilon}{m^4}$.
Then thanks to \Cref{prop: rogers}, we know that for $\beta_m := \mathcal{O}(\frac{1}{m})$, we have:
\begin{equation}
\label{eq: rogers_bound}
I_{\infty}^\beta(\mathcal{P}, m)=O\left(\log(m) \right) .
\end{equation}
The last thing to do is to prove that the probability kernel $\mathcal{P}_0(\Sm):= P_{-\lambda' m \Riskhat_{\Sm}}$ is $(\varepsilon,\gamma)$ differentially private. This is true thanks to \Cref{prop: minami} which states that $\mathcal{P}_0$ satisfies differential privacy as long as $\lambda' \leq \lambda_{m}$ with:
\begin{equation}
\label{eq: inv_temp}
\lambda_m:= \frac{1}{2K}\sqrt{\frac{\alpha \log(m)}{m\left(1-2\log\log(m) +10\log(m) \right)}} = \mathcal{O}\left(\frac{1}{\sqrt{m}}\right).
\end{equation}
Note that $\alpha$ intervenes because for any prior $\P\in C_{\alpha,\beta,M}$, $-\log P(.)$ is $\alpha$-strongly convex.
From now we consider $\lambda'= \frac{\lambda}{2Km}$ where $\lambda \leq \sqrt{m}$. We then have $\lambda' \leq \lambda_m$.
We then know, thanks to \Cref{eq: pac_b_diff_priv} with $\beta=\beta_m$, that for any $\delta > \beta_m$,  $\mathbb{P}_{\Sm \sim \D^m}\{\Sm \in \texttt{Ev}(\mathcal{P}_0(\Sm),\delta') \leq \delta$
with $\delta^{\prime}=\mathrm{e}^{-I_{\infty}^\beta(\mathcal{P} ; m)}(\delta-\beta)$
Taking the complementary event and recalling that thanks to \Cref{eq: rogers_bound}, $\log\left(\frac{1}{\delta'}   \right) = \log\left(\frac{1}{\delta - \beta_m}   \right) + \mathcal{O}(\log(m))$ gives,
for any data-free Gaussian prior $\P$, for any $\delta > \beta_m$, with probability at least $1-\delta$, for any $\Q\in C_{\alpha,\beta,M}$:
\begin{multline}
\label{eq: complete_data_dep}
|\Delta_{\Sm}(\Q)| \leq 2K(M+1)\frac{\beta\sqrt{2\beta}}{m} + \\
\sqrt{ C_R \frac{\log(\frac{1}{\delta-\beta_m}) + \mathcal{O}(\log(m)) +2d\log\left(1 +2Rm \right)}{m} }  \\
\times\sqrt{\left(\W_{1}( \Q, \P_{-\frac{\lambda}{2K} \Riskhat_{\Sm}})+\alpha'_m + f_R(\P_{-\frac{\lambda}{2K} \Riskhat_{\Sm}}) \right)} \\
+\sqrt{ D_R^2\frac{\log\left( \frac{m}{\delta-\beta_m} \right) +\mathcal{O}(\log(m)) }{m}},
\end{multline}
where $\alpha'_m= \mathcal{O}(1+ \frac{d\log(m)}{m})$ has the same analytical expression than $\alpha_m$ (defined in \Cref{th: main_gaussian_lpz}) but where all the occurences of $\delta$ have been replaced by $\delta'$.
Note that in the last equation, we used $\sqrt{a+b} \leq \sqrt{a} + \sqrt{b}$ ($a,b>0$) for the sake of readability but we put everything within the same square root in our theorem as it is tighter.
Then, exploiting that $R= \mathcal{O}(\sqrt{d\log (d)}, \sqrt{\log(m)})$, gives us the results for the low-data and transitory regimes.
\medskip

\noindent Also, we are able to prove that asymptotically, because $R \sqrt{\log(m)}\rightarrow \infty$ when $m$ goes to infinity:
$$f_R(\P_{-\frac{\lambda}{2K} \Riskhat_{\Sm}}) \leq \mathbb{E}[||X-\mathcal{P}_{R}(X)||] \underset{m\rightarrow \infty}{\rightarrow} 0,    $$
where $X$ follows the Gibbs distribution $\P_{-\frac{\lambda}{2K} \Riskhat_{\Sm}}$. The convergence to zero comes from the dominated convergence theorem.
Indeed,
\[ \mathbb{E}[||X-\mathcal{P}_{R}(X)||] = \int_{\mathbb{R}^d} g_m(x) \mathrm{d}\P(x), \]
with $g_m(x)= ||x - \mathcal{P}_{R}(x)||\frac{\exp\left(-\lambda \Riskhat_{\Sm}(x) \right)}{\mathbb{E}_P[\exp\left(-\lambda \Riskhat_{\Sm}(x) \right)]}$. Thus, bounding crudely gives:
\[  \mathbb{E}[||X-\mathcal{P}_{R}(X)||] \leq \frac{1}{\inf_{m\geq 1} \mathbb{E}_P[\exp\left(-\lambda \Riskhat_{\Sm}(x) \right)]} \int_{\mathbb{R}^d} ||x - \mathcal{P}_{R}(x)|| \mathrm{d}\P(x).\]
We know that $\inf_{m\geq 1} \mathbb{E}_P[\exp\left(-\lambda \Riskhat_{\Sm}(x) \right)] := \inf_{m\geq 1} \mathbb{E}_P[f_m(x) ] >0$ because $f_m$ is $\lambda K$ - lipschitz ($x \rightarrow e^{- \lambda x}$ is $\lambda$-lipschitz and the loss $\ell$ is $K$-lipschitz)
and converges almost surely on $\mathbb{R^d}$ towards $x\rightarrow \exp{-\lambda R(x)}$. Indeed, thanks to the law of large numbers, we know that on $\mathbb{Q}^d$, $f_m \rightarrow f$ almost surely and using that all the sequence is $\lambda K$ lipschitz extends the result to all $\mathbb{R}^d$.
We also notice that for any $m$, $f_m \leq 1$ so we can use the dominated convergence theorem to conclude that $\mathbb{E}_P[f_m(x)] \rightarrow  \mathbb{E}_P[\exp(-\lambda R(X)) ] >0.$ So $\inf_{m\geq 1} \mathbb{E}_P[\exp\left(-\lambda \Riskhat_{\Sm}(x) \right)]>0$.
The last thing to do is to use \Cref{l: gaussian_tail} to ensure that $\int_{\mathbb{R}^d} ||x - \mathcal{P}_{R}(x)|| \mathrm{d}\P(x) \rightarrow 0$.
This allows us to get rid of $f_R$ for the asymptotic regime and then, conclude the proof.
\end{proof}



\section{Generalisation ability of the Bures-Wasserstein SGD}

\label{sec: gene_sgd}

For the sake of completeness, we recall (and precise) several elements already defined in \Cref{sec: intro_optim}.
In PAC-Bayes learning, the following learning algorithm can be derived from a relaxation of \citet[][Theorem 1.2.6]{catoni2007pac}, for any data-free prior $\P$ and inverse PAC-Bayesian temperature $\lambda>0$:

\[ \underset{Q\in \mathcal{M}(\mathcal{H})}{\operatorname{argmin}} \mathbb{E}_{h\sim \Q}[\Riskhat_{\Sm}(h)] + 2K\frac{\operatorname{KL}(\Q,\P)}{\lambda}.  \]

\noindent We considered the parameter $\frac{\lambda}{2K}$ as it was suggested by \Cref{th: data_dep}.
A closed form solution is given by the Gibbs posterior $\Q^*:= \P_{-\frac{\lambda}{2K}}$ such that $\mathrm{d}\Q^* \propto \exp(-V_{\Sm}(h))\mathrm{d}h$, with $V_{\Sm}(h) = \frac{\lambda}{2K}\Riskhat_{\Sm}(h) - \log(\mathrm{d}\P(h))$ and $\mathrm{d}h$ being the Lebesgue measure.
However, such a measure can be difficult to estimate in practice. Two solutions are available. We can estimate the Gibbs posterior through MCMC methods that rely on Markov chains which (approximately) converge to $\Q^*$. However, there is no clear stopping criterion to obtain a good approximate of the true posterior. Otherwise, we can exploit variational inference (VI) to produce rapidly a basic yet informative summary statistics on a subclass of $\mathcal{M}(\mathcal{H})$.
In this section, we focus on the VI approach. As $\Q^*$ is the result of an optimal trade-off between the empirical loss $\Riskhat_{\Sm}$ and the $KL$ divergence (weighed by $\lambda$) acting as a regulariser, we consider the closest measure of $\operatorname{BW}(\mathbb{R}^d)$ from $\Q^*$ with respect to the KL divergence:
\[ \hat{\Q} = \mathcal{N}(\hat{m},\hat{\Sigma}) := \underset{Q\in \operatorname{BW}(\mathbb{R}^d)}{\operatorname{argmin}} \operatorname{KL}(\Q,\Q^*). \]
At the cost of this approximation, can we have an optimisation algorithm with convergence guarantees which goes to $\hat{\Q}$? Furthermore, if enough data is available, does $\hat{\Q}$ possess a good generalisation ability?
We first state the assumptions holding throughout the whole section.

\textit{(A3):} We assume that $\mathcal{H}=\mathbb{R}^d$ and
\begin{itemize}
  \item There exists $M>0$ such that $||\hat{m}|| \leq M$ almost surely.
  \item $\ell$ is twice differentiable, and \textbf{(A1), (A2)} hold. In particular, $\ell$ is $L$-smooth, convex and uniformly $K$-Lipschitz over $\mathcal{H}$. We furthermore assume that $L=1$.
  \item The prior $\P$ used in the definition of $\Q^*$ is a Gaussian with mean $0$ and covariance matrix $\Sigma= \text{diag}(\gamma), 1\geq\gamma>0$. We assume $\lambda \leq 2K$ in the definition of $\Q^*$.
\end{itemize}
Note that under \textbf{(A3)}, we have $0 \prec \alpha I \preceq \nabla^2 V_{\Sm} \preceq I$.
The work of \citet[Theorem 4]{lambert2022variational} provides convergence guarantees for SGD over the Bures-Wasserstein space when \textbf{(A3)} holds (in particular, they do not even requires the uniformly Lipschitz assumption). We first state their algorithm in \Cref{alg: sgd}.
\begin{algorithm}[ht]
\SetAlgoLined
\SetKwInOut{Initialisation}{Initialisation}
\SetKwInOut{Parameter}{Parameters}
\Parameter{Strong convexity parameter $\alpha>0$, radius $M>0$; step size $\eta>0$, initial mean $m_0$, initial covariance $\Sigma_0$}
Set up $\hat{\Q}_0 = \mathcal{N}(m_0,\Sigma_0)$. \\
\For{$k= 0..N-1$}{
Draw a sample $X_k\sim \hat{\Q}_k$.\\
Set $m_k^+ = m_k - \eta \nabla V_{\Sm}(X_k)$.\\
Set $M_k= I- \eta (\nabla V^2(X_k) -\Sigma_k^{-1}) $.\\
Set $\Sigma_k^+ = M_k\Sigma_k M_k$.\\
Set $m_{k+1} = \mathcal{P}_{M}(m_k^+),\;\Sigma_{k+1} = \operatorname{clip}^{1/\alpha}\Sigma_k^+$.\\
Set $\hat{\Q}_{k+1}= \mathcal{N}(m_{k+1},\Sigma_{k+1})$
}
\textbf{Return} $(\hat{\Q}_k)_{k=1...N}$.
\caption{Bures-Wasserstein SGD.}
\label{alg: sgd}
\end{algorithm}
Note that \Cref{alg: sgd} is a slight adaptation of the work of \citet{lambert2022variational}. Indeed, we added a projection step $\mathcal{P}_M$  within the compact of radius $M$ in $\mathbb{R}^d$. This does not change the convergence guarantees stated in \Cref{th: lambert} as long as we assume \textbf{(A3)}.

\begin{theorem}
\label{th: lambert}
Assume \textbf{(A3)}. Also, assume that $\eta \leq \frac{\alpha^2}{60}$ and that we initialize \Cref{alg: sgd} at a matrix satisfying $\frac{\alpha}{9} I \preceq \Sigma_{0} \preceq \frac{1}{\alpha} I$. Then, for all $k \in \mathbb{N}$,
$$
\mathbb{E} \W_{2}^2\left(\hat{\Q}_k, \hat{\Q}\right) \leq \exp (-\alpha k \eta) \W_{2}^2\left(\hat{\Q}_0, \hat{\Q}\right)+\frac{36 d \eta}{\alpha^2} .
$$
In particular, we obtain $\mathbb{E} \W_{2}^2\left(\hat{\Q}_k, \hat{\Q}\right) \leq \varepsilon^2$ provided we set $\eta \asymp \frac{\alpha^2 \varepsilon^2}{d}$ and the number of iterations to be $k \gtrsim \frac{d}{\alpha^3 \varepsilon^2} \log \left(\W_{2}\left(\hat{\Q}_0, \hat{\Q}\right) / \varepsilon\right)$.
\end{theorem}
We want to incorporate \Cref{th: lambert} within \Cref{th: data_dep}. To do so, we need to make sure that the outputs of \Cref{alg: sgd} and $\hat{\Q}$ lie a compact of $BW(\mathbb{R}^d)$. To do so we exploit the following lemma, which sums up the work of \citet{lambert2022variational} (namely their Lemma 6 and the discussion in Section 3.3). 

\begin{lemma}
\label[lemma]{l: measures_in_compact}
Assume \textbf{(A3)} and the step-size $\eta$ of \Cref{alg: sgd} is lesser than $\frac{\alpha^2}{60}$. Also in \Cref{alg: sgd}, assume that $\frac{\alpha}{9} I \preceq \Sigma_k$.
Then $\frac{\alpha}{9} I \preceq\Sigma_{k}^+$, and so, $\frac{\alpha}{9}I \preceq\Sigma_{k+1} \preceq \frac{1}{\alpha} I$.
Furthermore, $I \preceq \hat{\Sigma}  \preceq \frac{1}{\alpha} I$.
Thus, if the initialisation of \Cref{alg: sgd} is such that $\frac{\alpha}{9} I \preceq \Sigma_{0} \preceq \frac{1}{\alpha} I$, then the sequence $(\hat{\Q}_k)_{k\geq 0}$ and $\hat{\Q}$ are in the compact $C_{\frac{\alpha}{9}, \frac{1}{\alpha}, M}$.
\end{lemma}
Using \Cref{l: measures_in_compact}, we now can apply \Cref{th: data_dep} and obtain the main result of this section.
\begin{theorem}
\label{th: main_sgd}
Assume \textbf{(A3)}, also assume that $d\geq 3$. Let $\beta_m= \mathcal{O}(\frac{1}{\sqrt{m}})$ and fix any $\beta_m<\delta<1$.
Assume that we perform \Cref{alg: sgd}, with step size $\eta \asymp \frac{\alpha^2 \delta}{d}$ and the number of iterations to be $N \gtrsim \frac{d}{\alpha^3 \delta} \log \left(\W_{2}\left(\Q_0, \hat{\Q}\right) / \delta\right)$.
We also set the initialisation such that $\frac{\alpha}{9} I \preceq \Sigma_{0} \preceq \frac{1}{\alpha} I$,
then we can upper bound the generalisation ability of $\hat{\Q}_N$, with probability $1-2\delta$:

\noindent \textbf{Asymptotic regime} $(d\log(d)< \log(m))$
\begin{align*}
|\Delta_{\Sm}(\hat{\Q}_N)|  \leq\Tilde{\mathcal{O}}\left( \sqrt{2K\frac{d}{m}\left(1 + \W_{1}(\hat{\Q},\Q^*)\right)+ (1+K^2\log(m)) \frac{\log\left( \frac{m}{\delta} \right)}{m}} \right),
\end{align*}
where $\Tilde{\mathcal{O}}$ hides a polynomial dependency in $(\log(d),\log(m))$. We refer to \eqref{eq: complete_bound_sgd} for a bound presenting the explicit influence of the Bures-Wasserstein SGD.
\end{theorem}
\Cref{th: main_sgd} is based on \Cref{eq: complete_bound_sgd} which answers the question stated in the 'Our aims in this chapter' paragraph of \Cref{sec: intro_optim}. We successfully designed a bound of the form of \eqref{eq: wanted_pattern} by incorporating the optimisation guarantees of \citet{lambert2022variational} onto a statistical framework.
As such, this bound is a bridge between optimisation and PAC-Bayes learning. To the best of our knowledge, it is the first time that PAC-Bayes is able to explain why the minimiser attained by an optimisation procedure on a measure space is also able to generalise well. Until now PAC-Bayes guarantees were used as a check-in procedure, which means that during the optimisation phase it is possible to see whether the candidate predictor is able to generalise well. On the contrary our bound higlights, before any training, that the output of the Bures-Wasserstein SGD will become better at generalising, with the limit rate of $\sqrt{\frac{Kd}{m}\W_{1}(\hat{\Q},\Q^*) + \frac{\log(m)}{m}}$.
\medskip

Let us analyse the bound: the convergence rate depends on the quality of the approximation $\hat{\Q}$ of $\Q^*$, this says that if Gaussian measures are not suited to approximate well the Gibbs posterior, then we sacrifice some generalisation ability. However this term is also controlled by the Lipschitz constant $K$: if $K$ is small, then the learning problem is easy enough to compensate both the curse of dimensionality and a possibly bad approximation $\hat{\Q}$ of $\Q^*$.
Again, the limit convergence rate is the statistical ersatz $\mathcal{O}\left( \sqrt{\frac{\log(m)}{m}} \right)$. This roughly says that we cannot hope to converge better than a Hoeffding test bound in this setting. Finally note also that the step $\eta$ of \Cref{alg: sgd} now depends on $\delta$: this suggests that the Bures-Wasserstein SGD needs to be tuned with a smaller step size to ensure not only convergence, but also a good generalisation ability.
\begin{proof}[Proof of \Cref{th: main_sgd}]
We start from \Cref{th: data_dep}, considering the asymptotic case. We have with probability $1-\delta$, for the posterior $\hat{\Q}_N$ obtained after $N$ steps of \Cref{alg: sgd} distribution $\Q\in C_{\alpha,\beta,M}$ and the prior $\Q^*$:
\begin{align*}
|\Delta_{\Sm}(\hat{\Q}_N)|  \leq\Tilde{\mathcal{O}}\left( \sqrt{2K\frac{d}{m}\left(1 + \W_{1}(\hat{\Q}_N,\Q^*)\right)+ (1+K^2\log(m))\frac{\log\left( \frac{m}{\delta} \right)}{m}}   \right).
\end{align*}
Then, the triangle inequality gives that $\W_{1}(\hat{\Q}_N,\Q^*) \leq \W_{1}(\hat{\Q}_N,\hat{\Q}) + \W_{1}(\hat{\Q},\Q^*)$.
Finally, we exploit \Cref{th: lambert} as follows:
\begin{align*}
\W_{1}(\hat{\Q}_N,\hat{\Q}) & \leq \sqrt{\W_{2}^2(\hat{\Q}_N,\hat{\Q})} & \text{by Jensen} \\
& \leq \sqrt{2\frac{\mathbb{E}[\W_{2}^2(\hat{\Q}_N,\hat{\Q})]}{\delta}} & \text{by Markov}\\
& \leq \sqrt{2\frac{\exp (-\alpha N \eta) \W_{2}^2\left(\hat{\Q}_0, \hat{\Q}\right)+\frac{36 d \eta}{\alpha^2}}{\delta}} & \text{by \Cref{th: lambert}.}
\end{align*}
Note that in the last line, we were able to apply \Cref{th: lambert} thanks to \Cref{l: measures_in_compact}.
This leads to the following bound:
\begin{multline}
\label{eq: complete_bound_sgd}
|\Delta_{\Sm}(\hat{\Q}_N)|  \\
\leq \Tilde{\mathcal{O}}\left( \sqrt{2K\frac{d}{m}\left( f(N,\eta)\sqrt{\W_{2}^2(\hat{\Q}_0,\hat{\Q})} + 1 +
\varepsilon \right)+ (1+K^2\log(m))\frac{\log\left( \frac{m}{\delta} \right)}{m}}   \right),
\end{multline}
where $f(N,\eta)= \sqrt{\frac{\exp (-\alpha N \eta) \W_{2}^2\left(\hat{\Q}_0, \hat{\Q}\right)}{\delta}}$ and $\varepsilon= \sqrt{\frac{36 d \eta}{\alpha^2\delta}} + \W_{1}(\hat{\Q},\Q^*)$.
Finally, using that with step size $\eta \asymp \frac{\alpha^2 \delta}{d}$ and the number of iterations to be $N \gtrsim \frac{d}{\alpha^3 \delta} \log \left(\W_{2}\left(\hat{\Q}_0, \hat{\Q}\right) / \delta\right)$ allows us to bound:
$\sqrt{2\frac{\exp (-\alpha k \eta) \W_{2}^2\left(\hat{\Q}_0, \hat{\Q}\right)+\frac{36 d \eta}{\alpha^2}}{\delta}} \leq 1$.
This concludes the proof.
\end{proof}


\section{Conclusion}

We extended the Wasserstein PAC-Bayes theory beyond the results of \citet{amit2022integral}. We exploited optimisation results to explain the generalisation ability of existing algorithms and we instantiated this for the Bures-Wasserstein algorithm of \citet{lambert2022variational}. We conclude by discussing avenues for future works.

\paragraph{Can we exploit WPB for neural networks?} As shown in \Cref{fig: overview}, we had to assume, Lipschitzness, smoothness and convexity to reach \Cref{th: main_sgd}. Those assumptions are necessary in the current framework and to obtain the results of \citet{lambert2022variational} and thus, do not cover the important case of neural networks.
Therefore, an interesting lead to investigate would be to first, avoid smoothness to reach convex neural networks \citet{bengio2005convex} and also avoid the convexity assumption to reach the broader subclass of Lipschitz neural networks (\emph{e.g} \citealp{gouk2021regularisation}).
The case of Lipschitz neural networks is particularly interesting as WPB theory shows that a small Lipschitz constant is enough to attenuate the impact of dimensionality. 

\paragraph{Are the classical PAC-Bayesian techniques suited to WPB?} In \Cref{th: compact_catoni,th: compact_mcall}, we exploited a surrogate of the change of measure inequality to then exploit the PAC-Bayesian theory. However, those techniques are developed around the control of an exponential moment which appears naturally through the change of measure inequality. The surrogate directly involves the true moment with respect to the prior: an interesting direction would be to check whether tighter concentration bounds (or other bounds exploiting weaker assumptions than a bounded loss) are accessible. Furthermore, we exploited covering numbers to state that, with high probability, the loss is close to a Lipschitz one. Those covering numbers, while crucial, involve explicitly the dimension of the problem. This is challenging as such a dependency do not appear explicitly in KL-based PAC-Bayes learning (although they play a role in the KL term). 

We provide elements of answer to those two questions in \Cref{chap: wpb-practical}, where we obtain tractable bounds for heavy-tailed losses, yielding sound learning algorithms for neural networks. Those benefits come at the cost of no convergence rate for the Wasserstein term but also does not involve explicitly the dimension of $\H$. This practical trade-off sacrifices theoretical understanding for new efficient algorithms.




% ----------------------------------------------------------------------------------------------- %

\part[Towards A Better Understanding of Generalisation through Optimisation]{Towards A Better Understanding of Generalisation through Optimisation}

\label{part:contrib-disintegrated}

%!TEX root = main.tex
\chapter[Wasserstein PAC-Bayes in Practice: Genrealisation-Driven Learning Algorithms for Deterministic Predictors]{Wasserstein PAC-Bayes in Practice: Genrealisation-Driven Learning Algorithms for Deterministic Predictors}
\label{chap: wpb-practical}
\addchapterlof
\addchapterloe

\vspace{-2.0cm}
\begin{center}
\textbf{This chapter is based on the following paper}\\[-0.1cm]
\end{center}
TODO
%\printpublication{ViallardGermainHabrardMorvant2022}

\vspace{0.2cm}
\minitoc 

\begin{abstract}
\vspace{-0.2cm}
After \Cref{chap: wass-pb} which proposed a theoretical study of PAC-Bayes learning with Wasserstein distances, building bridges with the exploiting of convergence guarantees in generalisation, we now focus on practical expansions of Wasserstein PAC-Bayes. The optimisation view of PAC-Bayes learning is deeply exploited here: we derive theory-driven batch and online algorithms (the online paradigm attenuates the impact of the prior) valid for deterministic predictors (and thus consistent with many practical optimisation algorithms) and are derived from bounds valid for heavy-tailed lipschitz losses (weak statistical assumption and a stronger geometric one to be in line with the optimisation literature). This chapter shows that the optimisation view of PAC-Bayes leads to efficient procedures, competing with classical methods.
\end{abstract}

\newpage
    
\section{Introduction}

\Cref{chap: wass-pb} introduced Wasserstein PAC-Bayes learning from a theoretical perspective. Indeed, the main goal there was to incorporate the convergence guarantees of existing algorithms onto a generalisation bound. On the contrary, we focus here on deriving novel learning algorithms from Wasserstein PAC-Bayes bounds, circumventing many classical limitations of KL-based PAC-Bayes, which is the major part of the literature. Indeed, the practical use of KL divergence comes with two main limitations: {\it (i)} as illustrated in the generative modeling literature, the KL divergence does not incorporate the underlying geometry or topology of the data space $\Z$, hence can behave in an erratic way \cite{arjovsky2017wasserstein},
{\it (ii)} the $\KL$ divergence and its variants require the posterior $\Q$ to be absolutely continuous with respect to the prior $\P$.
However, recent studies \citep{camuto2021fractal} have shown that, in stochastic optimisation, the distribution of the iterates, which is the natural choice for the posterior, can converge to a \emph{singular distribution}, which does not admit a density with respect to the Lebesgue measure.
Moreover, the structure of the singularity (\ie, the \emph{fractal dimension} of $\Q$) depends on the data sample $\S$ \citep{camuto2021fractal}. 
Hence, in such a case, it would not be possible to find a suitable prior $\P$ that can dominate $\Q$ for almost every $\S \sim \D^m$, which will trivially make $\KL(\Q\|\P) = +\infty$ and the generalisation bound vacuous. 

Some works have focused on replacing the Kullback-Leibler divergence with more general divergences in PAC-Bayes \citep{alquier2018simpler,ohnishi2021novel,picard2022change}, although the problems arising from the presence of the $\KL$ divergence in the generalisation bounds are actually not specific to PAC-Bayes: information-theoretic bounds \citep{goyal2017pac,xu2017info,russo2020how} also suffer from similar issues as they are based on a mutual information term, which is the $\KL$ divergence between two distributions.
In this context, as a remedy to these issues introduced by the $\KL$ divergence, \cite{zhang2018optimal,wang2019information,rodriguez2021tighter,lugosi2022generalization} proved analogous bounds that are based on the \emph{Wasserstein distance}, which arises from the theory of optimal transport~\cite{monge1781memoire}.
As the Wasserstein distance inherits the underlying geometry of the data space and does not require absolute continuity, it circumvents the problems introduced by the $\KL$ divergence.
Yet, these bounds hold only in expectation, \ie, none of these bounds is holding with high probability over the random choice of the learning sample $\S\sim\D^{m}$.

In the context of PAC-Bayesian learning, the recent works \cite{amit2022integral,chee2021learning} incorporated Wasserstein distances as a complexity measure and proved generalisation bounds based on the Wasserstein distance.
More precisely, \cite{amit2022integral} proved a high-probability generic PAC-Bayesian bound for bounded losses depending on an integral probability metric \citep{muller1997integral}, which contains the Wasserstein distance as a special case. 
On the other hand, \cite{chee2021learning} exploited PAC-Bayesian tools to obtain learning strategies with their associated regret bounds based on the Wasserstein distance for the \emph{online learning} setting while requiring a finite hypothesis space and do not deal with generalisation.

\textbf{Contributions.}
The theoretical understanding of the high-probability generalisation bounds based on the Wasserstein distance is still limited.
The aim of this paper is not only to prove generalisation bounds (for different learning settings) based on the optimal transport theory but also to propose new learning algorithms derived from our theoretical results.
\begin{enumerate}[label={\it (\roman*)}]
    \item Using the supermartingale toolbox introduced in \Cref{chap: pb-ht}, we prove in \Cref{sec:wasserstein-batch}, novel PAC-Bayesian bounds based on the Wasserstein distance for \iid data.
    While \cite{amit2022integral} proposed a McAllester-like bound for bounded losses, we propose a Catoni-like bound (see \eg, \citealp[Theorem 4.1]{alquier2016properties}) valid for heavy-tailed losses with bounded order 2 moments.
    This assumption is less restrictive than assuming subgaussian or bounded losses, which are at the core of many PAC-Bayes results.
    This assumption also covers distributions beyond subgaussian or subexponential ones (\eg, gamma distributions with a scale smaller than 1, which have an infinite exponential moment). 
    \item We provide in \Cref{sec:wasserstein-online} the first generalisation bounds based on Wasserstein distances for the online PAC-Bayes framework of \Cref{chap:online-pb}.
    Our results are, again, Catoni-like bounds and hold for heavy-tailed losses with bounded order 2 moments.
    Previous work \citep{chee2021learning} already provided online strategies mixing PAC-Bayes and Wasserstein distances.
    However, their contributions focus on the best deterministic strategy, regularised by a Wasserstein distance, with respect to the deterministic notion of regret.
    Our results differ significantly as we provide the best-regularised strategy (still in the sense of a Wasserstein term) with respect to the notion of generalisation, which is new.
    \item As our bounds are linear with respect to Wasserstein terms (contrary to those of \citealp{amit2022integral} and \Cref{chap: wass-pb}), they are well suited for optimisation procedures.
    Thus, we propose the first PAC-Bayesian learning algorithms based on Wasserstein distances instead of KL divergences.
    For the first time, we design PAC-Bayes algorithms able to output deterministic predictors (instead of distributions over all $\H$) designed from deterministic priors.
    This is due to the ability of the Wasserstein distance to measure the discrepancy between Dirac distributions.     
    We then instantiate those algorithms in \Cref{sec:experiments} on various datasets, paving the way to promising practical developments of PAC-Bayes learning. 
\end{enumerate}

To sum up, we highlight two benefits of PAC-Bayes learning with Wasserstein distance.
First, it ships with sound theoretical results exploiting the geometry of the predictor space, holding for heavy-tailed losses.
Such a weak assumption on the loss extends the usefulness of PAC-Bayes with Wasserstein distances to a wide range of learning problems, encompassing bounded losses.
Second, it allows us to consider deterministic algorithms (\ie, sampling from Dirac measures) designed with respect to the notion of generalisation: we showcase their performance in our experiments.

\textbf{Outline.} \Cref{sec:framework} describes our framework and background, \Cref{sec:wasserstein} contains our new theoretical results and \Cref{sec:experiments} gathers our experiments. 
\Cref{sec:discussion-supervised} gathers supplementary discussion, \Cref{sec:proofs} contains all proofs of our claims, and \Cref{sec:supplementary-expes} provides insights into our practical results as well as additional experiments.

\section{Our framework}
\label{sec:framework}

\textbf{Framework.} 
We consider a Polish predictor space $\H$ equipped with a distance $d$ and a $\sigma$-algebra $\Sigma_{\Hcal}$, a data space $\Z$, and a loss function $\loss : \H\times \Z \rightarrow \R$.
In this work, we consider Lipschitz functions with respect to $d$.
We also associate a filtration $(\Fcal_{i})_{i\geq 1}$ adapted to our data $(\z_i)_{i=1,\dots,m}$, and we assume that the dataset $\S$ follows the distribution $\Dcal_{\S}$.
In PAC-Bayes learning, we construct a data-driven posterior distribution $\Q\in\Mcal(\H)$ with respect to a prior distribution $\P$. 

\textbf{Definitions.} 
For all $i$, we denote by $\EE_{i}[\cdot]$ the conditional expectation $\EE[\ \cdot\mid \Fcal_i]$.
In this work, we consider data-dependent priors.
A stochastic kernel is a mapping $\P: \cup_{m=1}^\infty\Z^m\times \Sigma_{\Hcal} \rightarrow [0,1]$ where {\it (i)} for any $B\in \Sigma_{\Hcal}$, the function  $\S\mapsto \P(\S,B)$ is measurable, {\it (ii)} for any dataset $\S$, the function $B\mapsto \P(\S,B)$ is a probability measure over $\H$.

In what follows, we consider two different learning paradigms: \emph{batch learning}, where the dataset is directly available, and \emph{online learning}, where data streams arrive sequentially.

\textbf{Batch setting.} 
We assume the dataset $\Sm$ to be \iid, so there exists a distribution $\D$ over $\Z$ such that $\Dcal_{\Sm}=\D^m$.
We then define, for a given $h\in\H$, the \emph{risk} to be $\Risk_\D\defeq\EE_{\z\sim \D}[\loss(h,\z)]$ and its empirical counterpart $\Riskhat_{\Sm} \defeq \frac{1}{m}\sum_{i=1}^m \loss(h,\z_i)$. 
Our results aim to bound the \emph{expected generalisation gap} defined by $\EE_{h\sim\Q}[ \Risk_{\D}(h) - \Riskhat_{\Sm}(h)]$.
We assume that the dataset $\S$ is split into $K$ disjoint sets $\S_1,\dots,\S_K$.
We consider $K$ stochastic kernels  $\P_1,\dots,\P_K$ such that for any $\S$, the distribution $\P_{i}(\S,.)$ {\it does not} depend on $\S_i$.

\textbf{Online setting.} 
We adapt the online PAC-Bayes framework of \Cref{chap:online-pb}.
We assume that we have access to a stream of data $\S=(\z_i)_{i\geq1}$, arriving sequentially, with no assumption on $\Dcal_{\S}$.
In online PAC-Bayes, the goal is to define a posterior sequence $(\Q_i)_{i\geq 1}$ from a prior sequence $(\P_i)_{i\geq 1}$, which can be data-dependent.
We assume our sequence of stochastic kernels (used as priors) $(\P_i)_{i=1\cdots m}$ to satisfy: {\it (i)} for all $i$ and dataset $\S$, the distribution $\P_i(S,.)$ is $\Fcal_{i-1}$ measurable and {\it (ii)} there exists $\P_0$ such that for all $i \geq 1$, we have $\P_i(S,.)\ll \P_{0}$. Indeed, all those measures are uniformly continuous with respect to any Gaussian distribution.
This last condition covers, in particular, the case where $\H$ is an Euclidean space and for any $i$, the distribution $\P_{i,\S}$ is a Dirac mass. This is weaker than the condition \textit{(ii)} of the online predictive sequence in \Cref{chap:online-pb}, but enough to exploit the conditional Fubini lemma (\Cref{l: cond_fubini-chap3}).
 

\textbf{Wasserstein distance.}
We focus on the Wasserstein distance of order 1 introduced by \cite{kantorovich1960mathematical} in the optimal transport literature. 
Given a distance $d: \Acal\times\Acal \to \Rbb$ and a Polish space $(\Acal, d)$, for any probability measures $\alpha$ and $\beta$ on $\Acal$, the Wasserstein distance is defined by
\begin{align}
\W(\alpha, \beta) \defeq \inf_{\gamma\in \Gamma(\alpha, \beta)}\
\EE_{(a, b)\sim\gamma}d(a, b),\label{eq:wasserstein}
\end{align}
where $\Gamma(\alpha, \beta)$ is the set of joint probability measures $\gamma \in \Mcal(\Acal^2)$ such that the marginals are $\alpha$ and $\beta$.
The Wasserstein distance aims to find the probability measure $\gamma\in\Mcal(\Acal^2)$ minimising the expected cost $\EE_{(a, b)\sim\gamma}d(a, b)$.
We refer the reader to \cite{villani2009optimal,peyre2019computational} for an introduction to optimal transport.

\section{Wasserstein-based PAC-Bayesian generalisation bounds}
\label{sec:wasserstein}

We present novel high-probability PAC-Bayesian bounds involving Wasserstein distances instead of the classical Kullback-Leibler divergence. 
Our bounds hold for heavy-tailed losses (instead of classical subgaussian and subexponential assumptions), extending the remits of \cite[Theorem 11]{amit2022integral}.
We exploit the supermartingale toolbox, recently introduced in PAC-Bayes framework by \cite{haddouche2023pac,chugg2023unified,jang2023tight}, to derive bounds for both batch learning (\Cref{theorem:supervised-ht,theorem:supervised-nnl}) and online learning (\Cref{theorem:online-ht,theorem:online}).


\subsection{PAC-Bayes for batch learning with \iid data}
\label{sec:wasserstein-batch}

In this section, we use the batch setting described in \Cref{sec:framework}.
We state our first result, holding for heavy-tailed losses admitting order 2 moments.
Such an assumption is in line, for instance, with reinforcement learning with heavy-tailed reward (see, \eg, \citealp{liu2011multi,lu2019optimal,zhuang2021regret}).

\begin{restatable}{theorem}{theoremsupervisedht}\label{theorem:supervised-ht}
We assume the loss $\loss$ to be $L$-Lipschitz.
Then, for any $\delta\in(0,1]$, for any sequence of positive scalar $(\lambda_i)_{i\in \{1,\dots,K\}}$, with probability at least $1-\delta$ over the sample $\S$, the following holds for the distributions $\P_{i,\S}\defeq \P_i(\S,.)$ and for any $\Q\in\Mcal(\H)$: 
\begin{multline*}
        \EE_{h\sim\Q}\Big[ \Risk_{\D}(h) - \Riskhat_{\Sm}(h) \Big]  \\ \leq    \sum_{i=1}^{K} \frac{2|\S_i|L}{m}\W(\Q, \P_{i,\S}) + \frac{1}{m}\sum_{i=1}^{K}   \frac{\ln\left( \frac{K}{\delta}  \right)}{\lambda_i} + \frac{\lambda_i}{2}\left(\EE_{h\sim \P_{i,\S}}\LB\Vhat_{|\S_i|}(h) + V_{|\S_i|}(h) \RB\right), 
    \end{multline*}
where $\P_{i,\S}$ {\it does not} depend on $\S_i$. 
Also, for any $i,|S_i|$, we have $\Vhat_{|\S_i|}(h)= \sum_{\z\in \S_i} \left(\loss(h,\z) - R_\D(h)\right)^2$ and $V_{|\S_i|}(h) = \EE_{\S_i}\LB\Vhat_{|\S_i|}(h)\RB$.
\end{restatable}

The proof is deferred to \Cref{sec:proof-supervised-ht}.
While \Cref{theorem:supervised-ht} holds for losses taking values in $\R$, many learning problems rely in practice on more constrained losses.
This loss can be bounded as in the case of, \eg, supervised learning or the multi-armed bandit problem \citep{slivkins2019intro}, or simply non-negative as in regression problems involving the quadratic loss (studied, for instance, in \citealp{catoni2016pac,catoni2017dimension}).
Using again the supermartingale toolbox, we prove in \Cref{theorem:supervised-nnl} a tighter bound holding for heavy-tailed non-negative losses. 

\begin{restatable}{theorem}{theoremsupervisednnl}
\label{theorem:supervised-nnl}
We assume our loss $\loss$ to be non-negative and $L$-Lipschitz. We also assume that, for any $1\leq i\leq K$, for any dataset$\S$, we have $\EE_{h\sim \P_{i}(.,\S), z\sim \D}\LB \loss(h,z)^2 \RB \leq 1$ (\emph{bounded order 2 moments for priors}).
Then, for any $\delta\in(0,1]$, with probability at least $1-\delta$ over the sample $\S$, the following holds for the distributions $\P_{i,\S}\defeq \P_i(\S,.)$ and for any $\Q\in\Mcal(\H)$: 
\begin{align*}
\ \EE_{h\sim\Q}\Big[ \Risk_{\D}(h) - \Riskhat_{\Sm}(h) \Big] \le \sum_{i=1}^{K} \frac{2|\S_i|L}{m} \W(\Q, \P_{i,\S}) + \sum_{i=1}^{K} \sqrt{\frac{2|\S_i|\ln\frac{K}{\delta}}{m^2}}, 
\end{align*}
where $\P_{i,\S}$ {\it does not} depend on $\S_i$.
\end{restatable}

Note that when the loss function takes values in $[0,1]$, an alternative strategy allows tightening the last term of the bound by a factor $\frac{1}{2}$.
This result is rigorously stated in \Cref{theorem:supervised_tight} of \Cref{sec:alt-proof-supervised}. 

\textbf{High-level ideas of the proofs.}
\Cref{theorem:supervised-ht,theorem:supervised-nnl} are structured around two tools.
First, we exploit the Kantorovich-Rubinstein duality \cite[Remark 6.5]{villani2009optimal} to replace the change of measure inequality \cite{csizar1975divergence,donsker1976asymp}; this allows us to consider a Wasserstein distance instead of a KL term.
Then, we exploit the supermartingales used in \cite{haddouche2023pac,chugg2023unified} alongside Ville's inequality (instead of Markov's one) to obtain a high probability bound holding for heavy-tailed losses.
Combining those techniques provides our PAC-Bayesian bounds.

\textbf{Analysis of our bounds.} Our results hold for Lipschitz losses and allow us to consider heavy-tailed losses with bounded order 2 moments.
While such an assumption on the loss is more restrictive than in classical PAC-Bayes, allowing heavy-tailed losses is strictly less restrictive. 
While \Cref{theorem:supervised-ht} is our most general statement, \Cref{theorem:supervised-nnl} allows recovering a tighter result (without empirical variance terms) for non-negative heavy-tailed losses. 
An important point is that the variance terms are considered with respect to the prior distributions $\P_{i,\S}$  and not $\Q$ as in \cite{haddouche2023pac,chugg2023unified}. This is crucial as these papers rely on the implicit assumption of order 2 moments, holding uniformly for all $\Q\in\Mcal(\H)$, while we only require this assumption for the prior distributions $(\P_{i,\S})_{i=1,\dots,K}$.
Such an assumption is in line with the PAC-Bayesian literature, which often relies on bounding an averaged quantity with respect to the prior.
This strength is a consequence of the Kantorovich-Rubinstein duality.
To illustrate this, consider \iid data with distribution $\D$ admitting a finite variance bounded by $V$ and the loss $\loss(h,z)= |h-z|$ where both $h$ and $z$ lie in the real axis.
Notice that in this particular case, we can imagine that $z$ is a data point and $h$ is a hypothesis outputting the same scalar for all data.
To satisfy the assumption of \Cref{theorem:supervised-nnl}, it is enough, by Cauchy Schwarz, to satisfy  $\EE_{h\sim \P_{i,\S},z\sim\S}[\loss(h,z)^2] \le \EE[h^2] + 2V\EE[|h|] +V^2 \leq 1$ for all $\P_{i,\S}$.
On the contrary, \cite{haddouche2023pac,chugg2023unified} would require this condition to hold for all $\Q$, which is more restrictive.
Finally, an important point is that our bound allows us to consider Dirac distributions with disjoint support as priors and posteriors.
On the contrary, KL divergence forces us to consider a non-Dirac prior for our bound to be non-vacuous. 
This allows us to retrieve a uniform-convergence bound described in \Cref{corollary:supervised-nnl}.

\textbf{Role of data-dependent priors.} 
\Cref{theorem:supervised-ht,theorem:supervised-nnl} allow the use of prior distributions depending possibly on a fraction of data.
Such a dependency is crucial to control our sum of Wasserstein terms as we do not have an explicit convergence rate.
For instance, for a fixed $K$, consider a compact predictor space $\H$, a bounded loss and the \emph{Gibbs posterior} defined as $d\Q(h) \propto \exp\left(-\lambda \Riskhat_{\Sm}(h)\right)dh$ where $\lambda>0$.
Also define for any $i$ and $\S$, the distribution $d\P_{i,\S}(h) \propto \exp\left(-\lambda \Risk_{\S/\S_i}(h)\right)dh$. Then, by the law of large numbers, when $m$ goes to infinity, for any $h$, both $\Risk_S(h)$ and $(\Risk_{\S/\S_i}(h))_{i=1,\dots,m}$ converge to $\Risk_\D(h)$. 
This ensures, alongside with the dominated convergence theorem, that for any $i$, the Wasserstein distance $\W(\Q,\P_{i,\S})$ goes to zero as $m$ goes to infinity.  

\textbf{Comparison with the literature.} \cite[][Theorem 11]{amit2022integral} establishes a PAC-Bayes bound with Wasserstein distance valid for bounded losses being Lipschitz with high probability. While we circumvent the first assumption, the second one is less restrictive than actual Lipschitzness and can also be used in our setting. Also \cite[Theorem 12]{amit2022integral} proposes an explicit convergence for finite predictor classes. We show in \Cref{sec:discussion-supervised} that we are also able to recover such a convergence. 

\textbf{Towards new PAC-Bayesian algorithms.} From \Cref{theorem:supervised-nnl}, we derive a new PAC-Bayesian algorithm for Lipschitz non-negative losses:
\begin{equation}
    \label{eq:batch-alg}
    \argmin_{\Q\in\Mcal(\H)} \EE_{h\sim \Q}\left[\Riskhat_{\Sm}(h)\right] + \sum_{i=1}^{K} \frac{2|\S_i|L}{m} \W(\Q, \P_{i,\S}).
\end{equation}
\Cref{eq:batch-alg} uses Wasserstein distances as regularisers and allows the use of multiple priors. 
We compare ourselves to the classical PAC-Bayes algorithm derived from \cite[][Theorem 1.2.6]{catoni2007pac} (which leads to Gibbs posteriors):
\begin{equation}
    \label{eq:catoni-alg}
    \argmin_{\Q\in\Mcal(\H)} \EE_{h\sim \Q}\left[\Riskhat_{\Sm}(h)\right] +  \frac{\operatorname{KL}(\Q,\P)}{\lambda}.
\end{equation}
Considering a Wasserstein distance in \Cref{eq:batch-alg} makes our algorithm more flexible than in \Cref{eq:catoni-alg}, the KL divergence implies absolute continuity \wrt the prior $\P$.
Such an assumption is not required to use \Cref{eq:batch-alg} and covers the case of prior Dirac distributions.
Finally, \Cref{eq:batch-alg} relies on a fixed value $K$ whose value is discussed below.

\textbf{Role of $K$.} 
We study the cases $K=1$, $\sqrt{m}$, and $m$ in \Cref{theorem:supervised-nnl}. We refer to \Cref{sec:discussion-supervised} for a detailed treatment.
First of all, when $K=1$, we recover a classical batch learning setting where all data are collected at once.
In this case, we have a single Wasserstein with no convergence rate coupled with a statistical ersatz of $\sqrt{\frac{\ln(1/\delta)}{m}}$.
However, similarly to \cite[][Theorem 12]{amit2022integral}, in the case of a finite predictor class, we are able to recover an explicit convergence rate.
The case $K=\sqrt{m}$ provides a tradeoff between the number of points required to have good data-dependent priors (which may lead to a small $\sum_{i=1}^{\sqrt{m}}\W(\Q, \P_i)$) and the number of sets required to have an explicit convergence rate. 
Finally, the case $K=m$ leads to a vacuous bound as we have the incompressible term $\sqrt{\ln\left(\frac{m}{\delta}\right)}$, which makes the bound vacuous for large values of $m$.
This means that the batch setting is not fitted to deal with a data stream arriving sequentially. To mitigate that weakness, we propose in \Cref{sec:wasserstein-online} the first online PAC-Bayes bounds with Wasserstein distances.

\subsection{Wasserstein-based generalisation bounds for online learning}
\label{sec:wasserstein-online}
Here, we use the online setting described in \Cref{sec:framework} and derive the first online PAC-Bayes bounds involving Wasserstein distances in \Cref{theorem:online-ht,theorem:online}. 
Online PAC-Bayes bounds are meant to derive online counterparts of classical PAC-Bayesian algorithms \cite{haddouche2022online}, where the KL-divergence acts as a regulariser.
We show in  \Cref{theorem:online-ht,theorem:online} that it is possible to consider online PAC-Bayesian algorithms where the regulariser is a Wasserstein distance, which allows us to optimise on measure spaces without a restriction of absolute continuity.

\begin{restatable}{theorem}{theoremonlineht}\label{theorem:online-ht}
 We assume our loss $\loss$ to be $L$-Lipschitz.
Then, for any $\delta\in(0,1]$, with probability at least $1-\delta$ over the sample $\S$, the following holds for the distributions $\P_{i,\S}\defeq \P_i(\S,.)$ and for any sequence $(\Q_i)_{i=1\cdots m}\in\Mcal(\H)^m$:
\begin{multline*}
\sum_{i=1}^m \EE_{h_i\sim \Q_{i}} \Big[\EE[\loss(h_i,\z_i) \mid \Fcal_{i-1}] - \loss(h_i,\z_i) \Big]  \le 2L\sum_{i=1}^{m}\W(\Q_{i}, \P_{i,\S}) \\
+ \frac{\lambda}{2}\sum_{i=1}^m \EE_{h_i\sim \P_{i,\S}}\left[ \Vhat_i(h_i,\z_i) + V_i(h_i) \right]+\frac{\ln(1/\delta)}{\lambda}, 
\end{multline*}
where for all $i$, $\Vhat_i(h_i,\z_i)= (\loss(h_i,\z_i)-\EE_{i-1}[\loss(h_i,\z_i)])^2$ is the conditional empirical variance at time $i$ and $V_i(h_i)= \EE_{i-1}[\Vhat(h_i,\z_i)]$ is the true conditional variance.
\end{restatable}

The proof is deferred to \Cref{sec:proof-online-ht}.
We also provide the following bound, being an online analogous of \Cref{theorem:supervised-nnl}, valid for non-negative heavy-tailed losses.


\begin{restatable}{theorem}{theoremonline}
\label{theorem:online}
We assume our loss $\loss$ to be non-negative and $L$-Lipschitz.
We also assume that, for any $i,\S$, $\EE_{h\sim \P_{i}(.,\S)}\LB \EE_{i-1}[\loss(h,\z_i)^2] \RB \leq 1$ (\emph{bounded conditional order 2 moments for priors}).
Then, for any $\delta\in(0,1]$, with probability at least $1-\delta$ over the sample $\S$, any online predictive sequence (used as priors) $(\P_i)_{i\geq 1}$, we have with probability at least $1-\delta$ over the sample $S\sim\D$, the following, holding for the data-dependent measures $\P_{i,\S}\defeq \P_i(S,.)$ and any posterior sequence $(\Q_i)_{i\geq 1}$:
\begin{align*}
\frac{1}{m}\sum_{i=1}^m \EE_{h_i\sim \Q_{i}} \Big[\EE[\loss(h_i,\z_i) \mid \Fcal_{i-1}] - \loss(h_i,\z_i) \Big]  \le \frac{2L}{m}\sum_{i=1}^{m}\W(\Q_{i}, \P_{i,\S}) + \sqrt{\frac{2\ln\LP\frac{1}{\delta}\RP}{m}}.
\end{align*}
\end{restatable}
The proof is deferred to \Cref{sec:proof-online}.

\textbf{Analysis of our bounds.} 
\Cref{theorem:online-ht,theorem:online} are, to our knowledge, the first results involving Wasserstein distances for online PAC-Bayes learning.
They are the online counterpart of \Cref{theorem:supervised-ht,theorem:supervised-nnl}, and the discussion of \Cref{sec:wasserstein-batch} about the involved assumptions also apply here.
The sum of Wasserstein distances involved here is a consequence of the online setting and must grow sublinearly for the bound to be tight.
For instance, when $(\Q_i=\delta_{h_i})_{i\geq1}$ is the output of an online algorithm outputting Dirac measures and $\P_{i,\S}= \Q_{i-1}$, the sum of Wasserstein is exactly $\sum_{i=1}^m d(h_i,h_{i-1})$.
This sum has to be sublinear for the bound to be non-vacuous, and the tightness depends on the considered learning problem. 
An analogous of this sum can be found in dynamic online learning \cite{zinkevich2003online} where similar sums appear as \emph{path lengths} to evaluate the complexity of the problem.  

\textbf{Comparison with literature.}
We compare our results to existing PAC-Bayes bounds for martingales of \cite{seldin2012pac}. 
\cite[Theorem 4]{seldin2012pac} is a PAC-Bayes bound for martingales, which controls an average of martingales, similar to our \Cref{theorem:supervised-ht}.
Under a boundedness assumption, they recover a McAllester-typed bound, while \Cref{theorem:supervised-ht} is more of a Catoni-typed result.
Also, \cite[Theorem 7]{seldin2012pac} is a Catoni-typed bound involving a conditional variance, similar to our \Cref{theorem:online}.
They require to bound uniformly the variance on all the predictor sets, while we only assume averaged variance with respect to priors, which is what we required to perform \Cref{theorem:online}.

\textbf{A new online algorithm.}
\cite{haddouche2022online} derived from their main theorem, an online counterpart of \Cref{eq:catoni-alg}, proving it comes with guarantees.
Similarly, we exploit \Cref{theorem:online} to derive the online counterpart of \Cref{eq:batch-alg}, from the data-free initialisation $\Q_1$

\begin{equation}
    \label{eq:online-alg}
    \forall i \geq 1,\ \ \Q_i \in \argmin_{\Q\in\Mcal(\H)} \EE_{h\sim \Q}\left[\loss(h_i, \z_i)\right] + 2L\W(\Q, \P_{i,\S}).
\end{equation}

We highlight the merits of the algorithm defined by \Cref{eq:online-alg}, alongside with the one from \Cref{eq:batch-alg}, in \Cref{sec:experiments}.

\section{Learning via Wasserstein regularisation}
\label{sec:experiments}

\Cref{theorem:supervised-nnl,theorem:online} are designed to be informative on the generalisation ability of a single hypothesis even when Dirac distributions are considered.
In particular, our results involve Wasserstein distances acting as regularisers on $\H$. 
In this section, we show that a Wasserstein regularisation of the learning objective, which comes from our theoretical bounds, helps to better generalise in practice.
Inspired by \Cref{eq:batch-alg,eq:online-alg}, we derive new PAC-Bayesian algorithms for both batch and online learning involving a Wasserstein distance (see \Cref{sec:algo}), we describe our experimental framework in \Cref{sec:exp-fram} and we present some of the results in \Cref{sec:results}.
Additional details, experiments, and discussions are gathered in \Cref{sec:supplementary-expes} due to space constraints. 
All the experiments are reproducible with the source code provided on GitHub at \url{https://github.com/paulviallard/NeurIPS23-PB-Wasserstein}.

\subsection{Learning algorithms}
\label{sec:algo}

\textbf{Classification.} 
In the classification setting, we assume that the data space $\Z=\X{\times}\Y$ is composed of a $d$-dimensional \textit{input space} $\X=\{ \x \in \R^d \;|\; \|\x\|_2 \le 1 \}$ and a finite \textit{label space} $\Y=\{1,\dots, |\Y|\}$ with $|\Y|$ labels.
We aim to learn models $h_{\wbf}: \R^d\to \R^{|\Y|}$ parameterised by a weight vector $\wbf$ that outputs, given an input $\x\in\X$, a score $h_{\wbf}(\x)[y'] \in \R$ for each label $y'$. 
This score allows us to assign a label to $\x\in\X$; to check if $h_{\wbf}$ classifies correctly the example $(\x, y)$, we use the {\it classification loss} defined by $\loss^{c}(h_{\wbf}, (\x, y)) \defeq \mathds{1}\left[ h_{\wbf}(\x)[y] - \max_{y'\ne y} h_{\wbf}(\x)[y'] \le 0 \right]$, where $\mathds{1}$ denotes the indicator function.

\textbf{Batch algorithm.} In the batch setting, we aim to learn a parametrised hypothesis $h_{\wbf}\in\H$ that minimises the population classification risk $\Rfrak_{\D}(h_{\wbf}) = \EE_{(\x, y)\sim\D}\loss^{c}(h_{\wbf}, (\x, y))$ that we can only estimate through the empirical classification risk $\Rfrak_{\S}(h_{\wbf}) = \frac{1}{m}\sum_{i=1}^{m}\loss^{c}(h_{\wbf}, (\x_i, y_i))$.
To learn the hypothesis, we start from \Cref{eq:batch-alg}, when the distributions $\Q$ and $\P_1,\dots, \P_K$ are Dirac masses, localised at $h_{\wbf}, h_{\wbf_1},\dots h_{\wbf_K}\in\H$ respectively.
Indeed, in this case, $\W(\Q, \P_{i,\S}) = d(h_{\wbf}, h_{\wbf_i})$ for any $i$.
However, the loss $\loss^{c}(.,\z)$ is not Lipschitz and the derivatives are zero for all examples $\z\in\X\times\Y$, which prevents its use in practice to obtain such a hypothesis $h_{\wbf}$.
Instead, for the population risk $\Risk_{\D}(h)$ and the empirical risk $\Riskhat_{\Sm}(h)$ (in \Cref{theorem:supervised-nnl} and \Cref{eq:batch-alg}), we consider the loss $\loss(h, (\x, y)) = \frac{1}{|\Y|}\sum_{y'\ne y} \max(0, 1{-} \eta(h[y]{-}h[y']))$, which is $\eta$-Lipschitz \wrt the outputs $h[1],\dots, h[|\Y|]$.
This loss has subgradients everywhere, which is convenient in practice.
We go a step further by {\it (a)}  setting $L=\frac{1}{2}$ and {\it (b)} adding a parameter $\varepsilon>0$ to obtain the objective
\begin{align}
    \label{eq:batch-alg-exp}
    \argmin_{h_{\wbf}\in\Hcal}\left\{ \Riskhat_{\Sm}(h_{\wbf}) + \varepsilon\LB\sum_{i=1}^{K} \frac{|\S_i|}{m} d\LP h_\wbf,h_{\wbf_i}\RP\RB\right\{.
\end{align}
To (approximately) solve \Cref{eq:batch-alg-exp}, we propose a two-step algorithm.
First, \textsc{Priors Learning} learns $K$ hypotheses $h_{\wbf_1}, \dots, h_{\wbf_K} \in \H$ by minimising the empirical risk via stochastic gradient descent.
Second, \textsc{Posterior Learning} learns the hypothesis $h_{\wbf}\in\H$ by minimising the objective associated with \Cref{eq:batch-alg-exp}.
More precisely, \textsc{Priors Learning} outputs the hypotheses $h_{\wbf_1},\cdots,h_{\wbf_K}$, obtained by minimising the empirical risk through mini-batches.
Those batches are designed such that for any $i$, the hypothesis $h_{\wbf_i}$ does not depend on $\S_i$.
Then, given $h_{\wbf_1}, \dots, h_{\wbf_K} \in \H$, \textsc{Posterior Learning} minimises the objective in \Cref{eq:batch-alg-exp} with mini-batches.
Those algorithms are presented in \Cref{alg:batch} of \Cref{sec:supplementary-expes}.
While $\varepsilon$ is not suggested by \Cref{eq:batch-alg}, it helps to control the impact of the regularisation in practice.
\Cref{eq:batch-alg-exp} then optimises a tradeoff between the empirical risk and the regularisation term $\varepsilon\sum_{i=1}^{K} \frac{|\S_i|}{m} d(h_{\wbf}, h_{\wbf_i})$.

\textbf{Online algorithm.} Online algorithms output, at each time step $i \in \{1, \dots, m\}$, a new hypothesis $h_{\wbf_i}$. 
From \Cref{eq:online-alg}, particularised to a sequence of Dirac distributions (localised in $h_{\wbf_1},\cdots,h_{\wbf_K})$, we design a novel online PAC-Bayesian algorithm with a Wasserstein regulariser:
\begin{align}
    \label{eq:online-alg-exp}
    \forall i \geq 1,\ \ h_i\in \argmin_{h_{\wbf}\in \Hcal} \loss(h_{\wbf}, \z_i) +d\LP h_{\wbf},h_{\wbf_{i-1}}\RP \ \ \text{\emph{s.t.}} \ \ d\LP h_{\wbf},h_{\wbf_{i-1}}\RP \le 1.
\end{align}
According to \Cref{theorem:online}, such an algorithm aims to bound the {\it population cumulative classification loss} $\mathfrak{C}_{\D} = \sum_{i=1}^{m}\EE[\loss^{c}(h_{\wbf_i},\z_i) \mid \Fcal_{i-1}]$.
Note that we added the constraint $d\LP h_{\wbf},h_{\wbf_{i-1}}\RP \le 1$ compared to \Cref{eq:online-alg}. 
This constraint ensures that the new hypothesis $h_{\wbf_i}$ is not too far from $h_{\wbf_{i-1}}$ (in the sense of the distance $\|\cdot\|_2$).
Note that the constrained optimisation problem in \Cref{eq:online-alg-exp} can be rewritten in an unconstrained form (see \cite{boyd2004convex}) thanks to a barrier $B(\cdot)$ defined by $B(a)=0$ if $a\le 0$ and $B(a)=+\infty$ otherwise; we have 
\begin{align} 
    \label{eq:online-alg-exp-2}
    \forall i \geq 1,\ \ h_i\in \argmin_{h_{\wbf}\in \Hcal} \loss(h_{\wbf}, \z_i) + d\LP h_{\wbf},h_{\wbf_{i-1}}\RP+ B(d\LP h_{\wbf},h_{\wbf_{i-1}}\RP-1).
\end{align}
When solving the problem in \Cref{eq:online-alg-exp-2} is not feasible, we approximate it with a log barrier of \cite{kervadec2022constrained} (suitable in a stochastic gradient setting); given a parameter $t>0$, the log barrier extension is defined by $\Bhat(a) = -\frac{1}{t}\ln(-a)$ if $a\le -\frac{1}{t^2}$ and $\Bhat(a) = ta-\frac{1}{t}\ln(\frac{1}{t^2})+\frac{1}{t}$ otherwise.
We present in \Cref{sec:supplementary-expes} \Cref{alg:online} that aims to (approximately) solve \Cref{eq:online-alg-exp-2}.
To do so, for each new example $(\x_i, y_i)$, the algorithm runs several gradient descent steps to optimise \Cref{eq:online-alg-exp-2}.


\subsection{Experimental framework}
\label{sec:exp-fram}

In this part, we assimilate the predictor space $\H$ to the parameter space $\R^d$.
Thus, the distance $d$ is the Euclidean distance between two parameters: $d\LP h_{\wbf},h_{\wbf'}\RP = \|\wbf{-}\wbf'\|_2$.
This implies that the Lipschitzness of $\loss$ has to be taken \wrt $\wbf$ instead of $h_{\wbf}$.

\textbf{Models.} 
We consider that the models are either linear or neural networks (NN).
Linear models are defined by $h_{\wbf}(\x)=W\x+b$, where $W\in\R^{|\Y|\times d}$ is the weight matrix, $b\in\R^{|\Y|}$ is the bias, and $\wbf=\vect(\{W, b\})$ its vectorisation; the vector $\wbf$ with the zero vector.
Thanks to the definition of $\X$, we know from \Cref{lemma:lipschitz-linear} (and the composition of Lipschitz functions) that the loss is $\sqrt{2}\eta$-Lipschitz \wrt $\wbf$.
For neural networks, we consider fully connected ReLU neural networks with $L$ hidden layers and $D$ nodes, where the leaky ReLU activation function $\ReLU: \R^D\to\R^D$ applies elementwise $x\mapsto \max(x, 0.01x)$.
More precisely, the network is defined by $h_{\wbf}(\x) = Wh^{L}(\cdots h^{1}(\x))+b$ where $W\in \R^{|\Y|\times D}$, $b\in\R^{|\Y|}$. Each layer $h^{i}(\x)=\ReLU(W_i\x+b_i)$ has a weight matrix $W_i\in \R^{D\times D}$ and bias $b_i\in \R^{D}$ except for $i=1$ where we have $W_1\in \R^{D\times d}$. 
The weights $\wbf$ are also the vectorisation $\wbf=\vect(\{W, W_{L}, \dots, W_1, b, b_{L}, \dots, b_1\})$. 
We have precised in \Cref{l:lip-nn} that our loss is Lipschitz \wrt the weights $\wbf$.
We initialise the network similarly to \cite{dziugaite2017computing} by sampling the weights from a Gaussian distribution with zero mean and a standard deviation of $\sigma=0.04$; the weights are further clipped between $-2\sigma$ and $+2\sigma$.
Moreover, the values in the biases $b_1,\dots, b_L$ are set to 0.1, while the values for $b$ are set to $0$.
In the following, we consider $D=600$ and $L=2$; more experiments are considered in the appendix.

\textbf{Optimisation.} 
To perform the gradient steps, we use the COCOB-Backprop optimiser~\cite{orabona2017training} (with parameter $\alpha=10000$).\footnote{The parameter $\alpha$ in COCOB-Backprop can be seen as an initial learning rate; see \cite{orabona2017training}.}
This optimiser is flexible as the learning rate is adaptative and, thus, does not require hyperparameter tuning.
For \Cref{alg:batch}, which solves \Cref{eq:batch-alg-exp}, we fix a batch size of $100$, \ie, $|\mathcal{U}|=100$, and the number of epochs $T$ and $T'$ are fixed to perform at least $20000$ iterations.
Regarding \Cref{alg:online}, which solves \Cref{eq:online-alg-exp-2}, we set $t=100$ for the log barrier, which is enough to constrain the weights and the number of iterations to $T=10$.

\textbf{Datasets.} 
We study the performance of \Cref{alg:batch,alg:online} on UCI datasets~\citep{dua2017uci} along with MNIST~\citep{lecun1998mnist} and FashionMNIST~\citep{xiao2017fashion}.
We also split all the data (from the original training/test set) in two halves; the first part of the data serves in the algorithm (and is considered as a training set), while the second part is used to approximate the population risks $\Rfrak_{\D}(h)$ and $\mathfrak{C}_{\D}$ (and considered as a testing set).

\subsection{Results}
\label{sec:results}

We present in \Cref{tab:expe_1,tab:expe_2} the performance of \Cref{alg:batch,alg:online} compared to the Empirical Risk Minimisation (ERM) and the Online Gradient Descent (OGD) with the COCOB-Backprop optimiser.
\Cref{tab:linear_batch,tab:nn_batch} present the results of \Cref{alg:batch} for the \iid setting on linear and neural networks respectively, while \Cref{tab:linear_online,tab:nn_online} present the results of \Cref{alg:online} for the online case.

\begin{table}[ht]
    \caption{%
Performance of \Cref{alg:batch,alg:online} compared respectively to ERM and OGD on different datasets on linear models.
For the \iid setting, we consider $\varepsilon=\frac{1}{m}$ and $\varepsilon=\frac{1}{\sqrt{m}}$ and with $K=0.2\sqrt{m}$. 
For each method, we plot the empirical risk $\Rfrak_{\S}(h)$ or $\mathfrak{C}_{\S}$ with its associated test risk $\Rfrak_{\D}(h)$ or $\mathfrak{C}_{\D}$.
The risk in {\bf bold} corresponds to the lowest one among the ones considered.
For the online case, the two population risks are \underline{underlined} when the absolute difference is lower than 0.05.
    }
   \begin{subtable}{0.25\textwidth}
       \centering
       \caption{Linear model -- batch learning}
       \input{chapter_6/tables/paper_linear_batch}
      \label{tab:linear_batch}
   \end{subtable}
   \hfill
   \begin{subtable}{0.28\textwidth}
       \centering
       \caption{Linear model -- online learning}
       \input{chapter_6/tables/paper_linear_online}
       \label{tab:linear_online}
   \end{subtable}
   \label{tab:expe_1}
\end{table}

\begin{table}[ht]
    \caption{%
Performance of \Cref{alg:batch,alg:online} compared respectively to ERM and OGD on different datasets on neural network models.
For the \iid setting, we consider $\varepsilon=\frac{1}{m}$ and $\varepsilon=\frac{1}{\sqrt{m}}$ and with $K=0.2\sqrt{m}$. 
For each method, we plot the empirical risk $\Rfrak_{\S}(h)$ or $\mathfrak{C}_{\S}$ with its associated test risk $\Rfrak_{\D}(h)$ or $\mathfrak{C}_{\D}$.
The risk in {\bf bold} corresponds to the lowest one among the ones considered.
For the online case, the two population risks are \underline{underlined} when the absolute difference is lower than 0.05.
    }
   \begin{subtable}{0.25\textwidth}
       \centering
       \caption{NN model -- batch learning}
       \input{chapter_6/tables/paper_nn_batch}
      \label{tab:nn_batch}
   \end{subtable}
   \hfill
   \hspace{0.5cm}\begin{subtable}{0.28\textwidth}
       \centering
       \caption{NN model -- online learning}
       \input{chapter_6/tables/paper_nn_online}
       \label{tab:nn_online}
    \end{subtable}
    \label{tab:expe_2}
\end{table}

\textbf{Analysis of the results.} In batch learning, we note that the regularisation term brings generalisation improvements compared to the empirical risk minimisation.
Indeed, our batch algorithm (\Cref{alg:batch}) has a lower population risk $\Rfrak_{\D}(h)$ on 11 datasets for the linear models and 9 datasets for the neural networks. In particular, notice that NNs obtained from \Cref{alg:batch} are more efficient than the ones obtained from ERM on \textsc{MNIST} and \textsc{FashionMNIST}, which are the more challenging datasets.
This suggests that the regularisation term helps to generalise well.
For the online case, the performance of the linear models obtained from our algorithm (\Cref{alg:online}) and by OGD are comparable: we have a tighter population classification risk $\Rfrak_{\D}(h)$ on $5$ datasets over $13$. 
However, notice that the risk difference is less than $0.05$ on $6$ datasets.
The advantage of \Cref{alg:online} is more pronounced for neural networks: we improve the performance in all datasets except \textsc{ adult} and \textsc{ sensorless}.
Hence, this confirms that optimising the regularised loss $\loss(h_{\wbf}, \z_i) + \|\wbf{-}\wbf_{i-1}\|$ brings a good advantage compared to the loss $\loss(h_{\wbf}, \z_i)$ only. A possible explanation would be that OGD suffers from underfitting (with a high empirical risk $\mathfrak{C}_{\D}$) while we are able to control overfitting through a regularisation term.
Indeed, only one gradient descent step is done for each new datum $(\x_i, y_i)$, which might not be sufficient to decrease the loss.  
Instead, our method solves the problem associated with \Cref{eq:online-alg-exp-2} and constrains the descent with the norm $\|\wbf-\wbf_{i-1}\|$.

\section{Conclusion and Perspectives}

We derived novel generalisation bounds based on the Wasserstein distance, both for batch and online learning, allowing for the use of deterministic hypotheses through PAC-Bayes.
We derived new learning algorithms which are inspired by the bounds, with remarkable empirical performance on a number of datasets: we hope our work can pave the way to promising future developments (both theoretical and practical) of generalisation bounds based on the Wasserstein distance.
Given the mostly theoretical nature of our work, we do not foresee an immediate negative societal impact, although we hope a better theoretical understanding of generalisation will ultimately benefit practitioners of machine learning algorithms and encourage virtuous initiatives.
\chapter[Wasserstein PAC-Bayes Learning: Exploiting Optimisation Guarantees to Explain Generalisation]{Wasserstein PAC-Bayes Learning: Exploiting Optimisation Guarantees to Explain Generalisation}
\label{chap:dis-mu}
\addchapterlof
\addchapterloa
\addchapterloe

\vspace{-1.6cm}
\begin{center}
\textbf{This chapter is based on the following papers}\\
\end{center}
\vspace{-0.3cm}
\printpublication{haddouche2023wasserstein}
\\
\printpublication{viallard2023learning}

\vspace{-0.3cm}
\minitoc

\vspace{-0.2cm}

\begin{abstract}
    Put WPB here, precise that, when the prior is seen as the learning goal, it is possible for a certain optimisation algorithm to directly incorporate sound geometric optimisation guarantee into a generalisation bound, trading the hope to reach a flat minima with a sound convergence guarantees. However, this comes at the cost of the explicit impact of the dimension. Also put the paper with Paul(batch bounds) as a supplementary content.
\end{abstract}

\newpage

\section{Introduction}




\part{Conclusion and Perspectives}
\label{part:conclusion}

%\chapter*{Conclusion and Perspectives}
\addcontentsline{toc}{chapter}{Conclusion and Perspectives}
\label{chap:conclu}

\section*{Conclusion}
In this thesis we studied various interplays between PAC-Bayes learning and optimisation. Doing so, we challenged various prerequisites of PAC-Bayes bounds:

\begin{itemize}
    \item \emph{Strong statistical assumptions.} The main conclusion of \Cref{chap: pb-ht} is that, in order to perform PAC-Bayes, no assumption stronger than finite variance is required. In particular, classical bounded or subgaussian assumptions on the loss can be replaced by bounded variance at no additional cost. Note also that in \Cref{chap: wass-pb}, it is even possible to relax this assumption to boundedness over a compact alongisde lipschitz or gradient lipschitz assumption. This is consistent with  optimisation which often involves such geometric assumptions. Furthermore, the supermartingale toolbox allows bounds holding for all dataset size simultaneously, which is consistent with, \eg, online optimisation.
    \item \emph{The information-theoretic perspective of the prior.} A major contribution has been to formalise perspectives on the prior differing from the Bayesian view paradigm. Indeed, by considering the prior either as an initialisation point or a learning objective, we derived novel PAC-Bayesian bounds aiming to either reduce the impact of the prior (\Cref{chap:online-pb,chap:gen-flat-minima}) when seen as initialisation or highlight it (\Cref{chap: wass-pb}) when seen as a learning objective. This drove the emergence of Online PAC-Bayes learning and the introduction of gradient norm or convergence guarantees in PAC-Bayes.
    \item \emph{ PAC-Bayes is useful for stochastic predictors only.}. Following the spirit of \citet{amit2022integral}, we developed Wasserstein PAC-Bayes to incorporate deterministic predictors within PAC-Bayes bounds, as such predictors are often involved in optimisation algorithms. To obtain explicit convergence rates with such bounds we exploited duality results from optimal transport in \Cref{chap: wass-pb}. We also incorporated directly convergence guarantees of the Bures-Wasserstein SGD in a generalisation bound, at the price of an explicit impact of the dimension of the predictor space. It is then hard to tackle the case of deep neural networks, this is why we developed in \Cref{chap: wpb-practical}, another kind of Wasserstein PAC-Bayes bounds, with no explicit convergence rate, but yielding learning algorithm exploitable for neural nets.
\end{itemize}

\section*{Perspectives}

This thesis left unanswered many important questions on the interplays of optimisation and generalisation. 

\begin{itemize}
    \item \emph{Can we relax the finite variance assumption to obtain generalisation bounds?} As proven \Cref{chap: pb-ht} and \citet{chugg2023unified} it is possible to extend a large body of generalisation bounds to the case of finite variance. An interesting question is whether such an assumption is relaxed, this would be of interest, for instance, to understand the case of heavy-tailed SGD \citep{gurbuzbalaban2020heavy} which may be modelled by Lévy processes, often having infinite variance.
    \item \emph{Can we further exploit flat minima to understand generalisation?} \Cref{chap:gen-flat-minima} proposed the first PAC-Bayes generalisation bounds exploiting flat minima. However the \texttt{QSB} assumption is required to exploit those results. While we saw that such a condition is verified by small networks, whether this condition is verified for deep neural nets remains an open question. Furthermore, the only empirical bound we have implies gradient lipschitz loss, a condition possibly hard to reach for deep nets. Empirical evaluation of those results is then an interesting future lead.
    \item \emph{Can we reach Wasserstein PAC-Bayes bound as simple and efficient than a KL one?} As shown in \Cref{chap: wass-pb} and \Cref{chap: wpb-practical} we did not obtain Wasserstein PAC-Bayes bounds with explicit convergence rate and without the explicit impact of the dimension as in KL-based PAC-Bayes bounds. An open question is whether it is possible to obtain a Wasserstein PAC-Bayes with both these desirable properties simultaneously. 
    \item \emph{Investigating the links between Online Learning and PAC-Bayes}. \Cref{chap:online-pb} draws a link from PAC-Bayes toward online learning by deriving novel learning algorithms from Online PAC-Bayes bounds. Recently, the elegant work of \citet{lugosi2023onlinetopac} has taken the opposite perspective:  starting from an online game they retrieve various generalisation bound, including KL-based and Wasserstein-based ones. Given the direct connection between online learning and the supermartingale framework \citep{wintenberger2021stochastic}, obtaining a unifying framework encompassing Wasserstein and KL-based PAC-Bayes from online learning for heavy-tailed losses is a promising future lead. A first step in this direction has recently been made \citet{viallard2024tighter} but does not involve online learning and holds only for bounded losses. 
\end{itemize}

Investigating those leads, and then reaching a better understanding of the impact of optimisation on generalisation are exciting questions for future work\pokeball

% ----------------------------------------------------------------------------------------------- %

\appendix
\part{Appendix}

\chapter{Some Mathematical Tools}
\addtextlist{loe}{Appendix}

\section{\textsc{Jensen}'s Inequality}

\begin{theorem}[\textsc{Jensen}'s Inequality]
Let $X$ a random variable following a probability distribution $\nu$ with $f$ a real-valued measurable convex function, we have
\begin{align*}
    f\LP\EE_{X\sim\nu}\LB X\RB\RP \le \EE_{X\sim\nu}\Big[ f\LP X\RP \Big].
\end{align*}
\label{ap:tools:theorem:jensen}
\end{theorem}
\begin{noaddcontents}\begin{proof}
Since $f()$ is a convex function, the following inequality holds, \ie, we have
\begin{align*}
\forall X',\quad a\LP X' - \EE_{X\sim\nu}\LB X\RB\RP \le f(X') - f\LP\EE_{X\sim\nu}\LB X\RB\RP,
\end{align*}
where $a$ is the tangent's slope.
By taking the expectation to both sides of the inequality, we have
\begin{align*}
\underbrace{a\LP \EE_{X\sim\nu}\LB X\RB - \EE_{X\sim\nu}\LB X\RB\RP}_{\displaystyle = 0} \le \EE_{X\sim\nu}\LB f(X)\RB - f\LP\EE_{X\sim\nu}\LB X\RB\RP.
\end{align*}
Hence, by rearranging the terms, we prove the claimed result.
\end{proof}\end{noaddcontents}

\section{\textsc{Markov}'s Inequality}
\label{ap:tools:sec:markov}

\begin{theorem}[\textsc{Markov}'s Inequality] Let $X$ a non-negative random variable  following a probability distribution $\nu$ and $\delta>0$, we have
\begin{align*}
    \PP_{X\sim\nu}\LB X\ge \delta\RB \le \frac{\EE_{X\sim\nu}\LB X\RB}{\delta}.
\end{align*}
\label{ap:tools:theorem:first-markov}
\end{theorem}

\begin{noaddcontents}\begin{proof}
First of all, remark that we have the following inequality for any $X$
\begin{align}
    \delta\indic[X \ge \delta] \;\le\; X\indic[X \ge \delta] \;\le\; X.
    \label{ap:tools:eq:proof-general-markov}
\end{align}
Indeed, on the one hand, if $X<\delta$, $\indic[X \ge \delta]=0$, the inequality holds trivially.
On the other hand, if $X\ge\delta$, $\indic[X \ge \delta]=1$ and the inequality becomes $\delta\le X$, which is true.
By taking the expectation of \Cref{ap:tools:eq:proof-general-markov}, we have
\begin{align*}
    \EE_{X\sim\nu}\Big[\delta\indic[X \ge \delta]\Big] \le \EE_{X\sim\nu}\Big[ X\Big].
\end{align*}
From the fact that the expectation of a constant is the constant and by definition of the probability, we have
\begin{align*}
    \delta\PP_{X\sim\nu}\LB X \ge \delta\RB \le \EE_{X\sim\nu}\Big[ X\Big] \quad\iff\quad \PP_{X\sim\nu}\LB X \ge \delta\RB \le \frac{\EE_{X\sim\nu}\LB X\RB}{\delta},
\end{align*}
which is the desired result.
\end{proof}\end{noaddcontents}

\section{Ville's Inequality}



%%!TEX root = main.tex
\chapter{Appendix of Chapter~\ref{chap: pb-ht}}
\label{ap: pb-ht}

\begin{noaddcontents}
    

\section{Some PAC-Bayesian background}
\label{sec: pac_b_background}

We present below an immediate corollary of \citet[Thm 2.1]{seldin2012bandit} where we upper bounded the cumulative by an empirical quantity (the sum of squared upper bound of the martingale difference sequence).

\begin{theorem}[\citealp{seldin2012bandit}, Theorem 2.1]
\label{th: seldin_thm_mart}
Let $\left\{C_1, C_2, \ldots\right\}$ be an increasing sequence set in advance, such that $\left|X_i(\\S_i,h)\right| \leq C_i$ for all $\\S_i,h$ with probability 1.   Let $\left\{P_1, P_2, \ldots\right\}$ be a sequence of data-free prior distributions over $\mathcal{H}$. Let $(\lambda_i)_{i\geq 1}$ be a sequence of positive numbers such that
$$
\lambda_m \leq \frac{1}{C_m}.
$$
Then with probability $1-\delta$ over $\S=(\z_i)_{i\geq 1}$,
for all $m\geq 1$, any posterior $\Q$ over $\mathcal{H}$,
$$
\left|M_m\left(Q\right)\right| \leq \frac{\KL\left(\Q , \P_m\right)+2 \log (m+1)+\log \frac{2}{\delta}}{\lambda_m}+(e-2) \lambda_m V_m(\Q),
$$
where $V_m(\Q)$ is defined in \cref{subsec: comparison_seldin}.

Furthermore, if we bound the variance term, we would have:
$$
\left|M_m\left(\Q\right)\right| \leq \frac{\KL\left(\Q , \P_m\right)+2 \log (m+1)+\log \frac{2}{\delta}}{\lambda_m}+(e-2) \lambda_m \sum_{i=1}^m C_i^2.
$$
\end{theorem}
Below, we use the definitions introduced in \Cref{sec: iid_case}.
We study here a particular case of \cite{alquier2016properties} for bounded losses which are especially subgaussian thanks to Hoeffding's lemma.
\begin{theorem}[Adapted from  \citealp{alquier2016properties}, Theorem 4.1]
\label{th: naive_pac_bayes-chap3}
Let $m>0$,$\S_m=(\z_1,...,\z_m)$ be an \iid sample from the same law $\mu$.
For any data-free prior $\P$, for any loss function $\ell$ bounded by $K$, any $\lambda>0,\delta\in ]0;1[$, one has with probability $1-\delta$ for any posterior $Q\in\mathcal{M}_1(\mathcal{H})$
\[ \mathbb{E}_{h\sim \Q}[\Risk(h)] \leq  \mathbb{E}_{h\sim \Q}[\Riskhat_{\Sm}(h)] + \frac{\operatorname{KL}(\Q, \P) + \log(1/\delta)}{\lambda} + \frac{\lambda K^2}{2m}. \]
\end{theorem}

\begin{theorem}[\citealp{haddouche2021pac}, Theorem 3]
\label{th: haddouche_thm}
Let the loss $\ell$ be $\mathrm{HYPE}(K)$ compliant. For any $\P\in\mathcal{M}(\mathcal{H})$ with no data dependency, for any $\alpha\in\mathbb{R}$ and for any $\delta\in[0,1]$, we have with probability at least $1-\delta$ over size-$m$ samples S,
for any $\Q$% such that $Q \ll P$% and $P \ll Q$
\begin{align*}
\mathbb{E}_{h\sim \Q}\left[ \Risk(h)\right]
\leq \mathbb{E}_{h\sim \Q}\left[ \Riskhat_{\Sm}(h)\right] + \frac{\operatorname{KL}(\Q,\P) + \log\left(\frac{1}{\delta}\right)}{m^{\alpha}}
+\frac{1}{m^{\alpha}}\log\left(\mathbb{E}_{h\sim \P} \left[\exp\left( \frac{K(h)^2}{2m^{1-2\alpha}} \right) \right]\right).
\end{align*}
\end{theorem}


\section{Extensions of previous results}
\label{sec: extensions}

Here we gather several corollaries of our main result in order to show how our \Cref{th: main_thm} extends the validity of some classical results in the literature. More precisely we show that our result extends (up to numerical factors) the PAC-Bayes Bernstein inequality of \citet{seldin2012bandit}.
Then, going back to the bounded case, we generalise a result from \citet{catoni2007pac} reformulated in \citet{alquier2016properties} and we also show how our work strictly improves on the bound of \citet{haddouche2021pac}.

\subsection{Extension of the PAC-Bayes Bernstein inequality}
\label{subsec: comparison_seldin}

Here we rename two terms for consistency with Theorem 2.1 of \citet{seldin2012bandit} (see \Cref{th: seldin_thm_mart}). For a martingale $M_m(h)= \sum_{i=1}^m X_i(\S_i,h)$, we define, at time $m$, \emph{empirical cumulative variance } to be $\hat{V}_m(h)= [M]_m(h) = \sum_{i=1}^m X_i(\S_i,h)^2$ and
the \emph{cumulative variance} as $V_m(h)= \langle M\rangle_m(h) = \sum_{i=1}^m \mathbb{E}_{i-1}[X_i(\S_i,h)^2]$.

We provide below a corollary containing two bounds: the first one being a straightforward corollary of \cref{th: main_thm}, the second being valid for bounded martingales and formally close to Theorem 2.1 of \citet{seldin2012bandit}.
\begin{corollary}
\label{cor: bound_mart}
Let $\left\{\P_1, \P_2, \ldots\right\}$ be a sequence of data-free prior distributions over $\mathcal{H}$. Let $(\lambda_i)_{i\geq 1}$ be a sequence of positive numbers.
Then the following holds with probability $1-\delta$ over $\S=(\z_i)_{i\geq 1}$: for any tuple $(m,\lambda_k,\P_k)$ with $m,k\geq 1$, any posterior $\Q$ over $\mathcal{H}$,
\begin{align}
\label{eq: bound_mart_1}
\left|M_m\left(\Q\right)\right| \leq \frac{\KL\left(\Q, \P_k\right)+2 \log (k+1)+\log (2/\delta)}{\lambda_k}+ \frac{\lambda_k}{2}\left( \hat{V}_m(\Q) + V_m(\Q) \right),
\end{align}
with $\hat{V}_m(\Q)= \mathbb{E}_{h\sim \Q}[\hat{V}_m(h)], V_m(\Q)= \mathbb{E}_{h\sim \Q}[V_m(h)]$.
Furthermore, if we assume that for any $i$, there exists $C_i>0$ such that $|X_i(\S_i,h)|\leq C_i$ for all $\S_i,h$ then we have the following corollary: with probability $1-\delta$ over $S$, for any tuple $(m,\lambda_m,\P_m)$ $m\geq 1$, any posterior $\Q$,
\begin{align}
\label{eq: bound_mart_2}
\left|M_m\left(Q\right)\right| \leq \frac{\KL\left(\Q, \P_m\right)+2 \log (m+1)+\log (2/\delta)}{\lambda_m}+ \lambda_m\sum_{i=1}^m C_i^2.
\end{align}
\end{corollary}
The proof is deferred to \cref{sec: proofs}.
Note that \cref{eq: bound_mart_1} holds uniformly on all tuples $\{(\lambda_k,\P_k,m) \mid k\geq 1, m\geq 1\}$ while \cref{eq: bound_mart_2}, as well as Theorem 2.1 of \citet{seldin2012bandit} holds uniformly on the tuples
$\{(\lambda_m,\P_m,m) \mid m\geq 1\}$ which is a strictly smaller collection. Hence our approach gives guarantees for a larger event with the same confidence level.

Furthermore, Theorem 2.1 of \citet{seldin2012bandit} involves the cumulative variance $V_m(\Q)$ (and not its empirical counterpart). Because this term is theoretical, we bound it in \cref{th: seldin_thm_mart} by $\sum_{i=1}^m C_i^2$ which is supposedly empirical.
In this context, \cref{eq: bound_mart_2}, recovers nearly exactly the bound of \cite{seldin2012bandit} with the transformation of a factor $(e-2)$ into $1$.
Notice also that \cref{eq: bound_mart_2} stands with no assumption on the range of the $\lambda_i$, which is not the case in \cref{th: seldin_thm_mart}.

Finally, we stress two fundamental differences between our work and the one of \citet{seldin2012bandit}. First, we replace Markov's inequality by Ville's inequality; second, we exploited the exponential inequality of Lemma \ref{l: bercu_touati} instead of the Bernstein inequality. These allow for results for unbounded martingales for all $m$ simultaneously.


\subsection{Extensions of learning theory results}


\subsubsection{A general result for bounded losses}


We use definitions from \Cref{sec: iid_case} and provide a corollary of our main result when the loss is bounded by a positive constant $K>0$. We assume our data are iid.
\begin{corollary}
\label{cor: bounded_case}
For any data-free prior $P\in \mathcal{M}(\mathcal{H})$, any $\lambda>0$ the following holds with probability $1-\delta$ over the sample $S=(z_i)_{i\in\mathbb{N}}$, for all $m\in\mathbb{N}/\{0\}$, $\Q\in\mathcal{M}(\mathcal{H})$
\[ \left | \mathbb{E}_{h\sim \Q} [\Risk(h)] -  \mathbb{E}_{h\sim \Q} \left[\Riskhat_{\S_m}(h) \right] \right |  \leq \frac{\operatorname{KL}(\Q,\P) +\log(2/\delta)}{\lambda m } + \lambda K^2.  \]
We also have the local bound: for any $m\geq 1$, with probability $1-\delta$ over $S$, for all $\Q\in\mathcal{M}(\mathcal{H})$
\[ \mathbb{E}_{h\sim \Q} [\Risk(h)] \leq  \mathbb{E}_{h\sim \Q} \left[\Riskhat_{\S_m}(h) \right] + \frac{\operatorname{KL}(\Q,\P) +\log(2/\delta)}{\lambda} + \frac{\lambda K^2}{m}.  \]
\end{corollary}
The proof is deferred to \cref{sec: proofs}. Remark that the second bound of Corollary \ref{cor: bounded_case} is exactly the Catoni bound stated in \citet{alquier2016properties} (see \Cref{th: naive_pac_bayes-chap3} in \Cref{sec: pac_b_background}) up to a numerical factor of $2$.


The first bound is, to our knowledge, the first PAC-Bayesian bound for bounded losses holding uniformly (for a given parameter $\lambda$) on the choice of $Q,m$ and thus extends the scope of Catoni's bound which holds for a single $m$ with high probability.  Indeed, if we want for instance \Cref{th: naive_pac_bayes-chap3} to hold for any $i\in\{1..m\}$, we then have to take an union bound on $m$ events which turns the term $\log(1/\delta)$ into $\log(m/\delta)$ (but with the benefit of holding for $m$ parameters $\lambda_1,...,\lambda_m$). This point is common to the most classical PAC-Bayesian bounds
(including McAllester and Catoni's ones \eqref{eq: mcallester}, \eqref{eq: catoni})
and impeach us to have a bound uniformly on all $m\in\mathbb{N}/\{0\}$ as $\log(m)$ goes to infinity asymptotically.


\subsubsection{An extension of \citet{haddouche2021pac}}

We now focus on the work of \citet{haddouche2021pac} which provides general PAC-Bayesian bounds for unbounded losses. Their theorems hold for iid data and under the so-called \emph{HYPE} (for HYPothesis-dependent rangE) condition. It states that a loss function $\ell$ is \emph{HYPE}$(K)$ compliant if there exists a function $K:\mathcal{H} \rightarrow \mathbb{R}^+ $ (supposedly accessible)  such that $\forall z\in\mathcal{Z}, \ell(h,\z) \leq K(h)$.
We provide \Cref{cor: haddouche_comparison} to compare ourselves with their main result (stated in  \Cref{th: haddouche_thm} for convenience).
\begin{corollary}
\label{cor: haddouche_comparison}
For any data-free prior $\P\in \mathcal{M}(\mathcal{H})$, any loss function $\ell$ being \emph{HYPE}$(K)$ compliant, any $\alpha\in[0,1],m\geq 1$, the following holds with probability $1-\delta$ over the sample $\S=(\z_i)_{i\in\mathbb{N}}$, for all $\Q\in\mathcal{M}(\mathcal{H})$
\begin{multline*}
\mathbb{E}_{h\sim \Q} [\Risk(h)] \leq   \mathbb{E}_{h\sim \Q} \left[\frac{1}{m}\sum_{i=1}^m\left(\ell(h,\z_i) + \frac{1}{2m^{1-\alpha}} \ell(h,\z_i)^2\right)\right] \\
+ \frac{\operatorname{KL}(\Q,\P) +\log(1/\delta)}{m^{\alpha}}  + \frac{1}{2m^{1-\alpha}}\mathbb{E}_{h\sim \Q} [K^2(h)].
\end{multline*}
\end{corollary}

\begin{proof}
The proof is a straightforward application of \cref{th: main_thm_iid} by fixing $m\geq 1$ choosing $\lambda= m^{\alpha-1}$ (thus we localise \Cref{th: main_thm_iid} to a single $m$),  and bounding $\mathrm{Quad}(h)$ by $K^2(h)$.
\end{proof}
The main improvement of our bound over \Cref{th: haddouche_thm} is that we do not have to assume the convergence of an exponential moment to obtain a non-trivial bound. Indeed, we transformed the (implicit) assumption $\mathbb{E}_{h\sim \P} \left[\exp\left( \frac{K(h)^2}{2m^{1-2\alpha}} \right) \right] < +\infty $ onto $\mathbb{E}_{h\sim \Q}[K(h)^2] < +\infty$, which is significantly less restrictive.
Furthermore, \Cref{th: haddouche_thm} holds for a single choice of $m$ while ours still holds uniformly over all integers $m>0$.

Cor. \ref{cor: haddouche_comparison} also sheds new light on the \emph{HYPE} condition. Indeed, in \citet{haddouche2021pac}, $K$ only intervenes in an exponential moment involving the prior $\P$, while ours considers a second-order moment on $K$ implying the posterior $\Q$. The difference is major as $\mathbb{E}_{h\sim \Q}[K(h)^2] $ can be controlled by a wise choice of posterior. Thus it can be incorporated in our optimisation route, acting now as an optimisation constraint instead of an environment constraint.



\section{Proofs}
\label{sec: proofs}

\subsection{Proof of \cref{th: main_thm_iid}}

\begin{proof}
Let $\P$ a fixed data-free prior, set $(\mathcal{F}_i)_{i\geq 0}$ such that for all $i$, $\z_i$ is $\mathcal{F}_i$ measurable. We also set for any fixed $h\in\mathcal{H}, M_m(h):= \sum_{i=1}^m \ell(h,\z_i) - \Risk(h)$. Note that because data are \iid, for any fixed $h$, the sequence $(M_m(h))_m$ is indeed a martingale.
We set for any $m\geq 1, h\in\mathcal{H}$
$$[M]_m(h) = \sum_{i=1}^m \left(\ell(h,\z_i) - \Risk(h)\right)^2 $$ and
$$\langle M \rangle_m(h) =  \sum_{i=1}^m \mathbb{E}_{i-1}[\left(\ell(h,\z_i) - \Risk(h)\right)^2] = \sum_{i=1}^m \mathbb{E}_{\z\sim \D}[\left(\ell(h,\z) - \Risk(h)\right)^2].$$
The last equality holds because data is assumed iid. Thus, we can apply \cref{th: main_thm} to obtain with probability $1-\delta$
\[|M_m(\Q)| \leq   \frac{\operatorname{KL}(\Q,\P) +\log(2/\delta)}{\lambda } + \frac{\lambda}{2}\left([M]_m(Q)^2 + \langle M\rangle_m(Q)^2 \right) . \]
Now, we notice that $|M_m(\Q)| = m| \mathbb{E}_{h\sim \Q}[\Risk(h) - \Riskhat_{\Sm}(h)] |$  and that  for any $m,h$, because $\ell$ is nonnegative
\begin{align*}
[M]_m(h) +  \langle M\rangle_m(h) & = \sum_{i=1}^m (\ell(h,\z_i) - \Risk(h))^2 + \mathbb{E}_{\z\sim \D}[ (\ell(h,\z) - \Risk(h))^2] \\
& \leq  \sum_{i=1}^m \ell(h,\z_i)^2 + \Risk(h)^2 + \mathbb{E}_{\z\sim \D}[\ell(h,\z)^2] - \Risk(h)^2.\\
\intertext{ Thus integrating over $h$ gives: }
[M]_m(Q) +  \langle M\rangle_m(Q) & \leq \sum_{i=1}^m \mathbb{E}_{h\sim \Q} [\ell(h,\z_i)^2] + m\mathbb{E}_{h\sim \Q} [\mathrm{Quad}(h)].
\end{align*}
Then dividing by $m$ and applying the last inequality gives
\begin{multline*}
\mathbb{E}_{h\sim \Q} [\Risk(h)]  \leq  \mathbb{E}_{h\sim \Q} \left[\frac{1}{m}\sum_{i=1}^m\left(\ell(h,\z_i) + \frac{\lambda}{2} \ell(h,\z_i)^2\right)\right] \\
+ \frac{\operatorname{KL}(\Q,\P) +\log(2/\delta)}{\lambda m} + \frac{\lambda}{2}\mathbb{E}_{h\sim \Q} [\mathrm{Quad}(h)].
\end{multline*}
This concludes the proof.
\end{proof}






\subsection{Proof of \cref{th: bandits_bound}}

\begin{proof}
Let $(\lambda_m)_{i\geq 1}$ be a countable sequence of positive scalars.
As precised earlier $M_m(a):= m\left(\hat{\Delta}_m(a)-\Delta(a)\right)$ is a martingale.
We then apply \Cref{th: main_thm} with the uniform prior ($\forall a, P(a)= \frac{1}{K}$) and $\lambda= \lambda_m$  (depending possibly on $m$): with probability $1- \delta/2$, for any tuple $(m,\lambda_m)$ with $m\geq 1$, any posterior $\Q$,
\begin{align*}
\left|M_m\left(\Q\right)\right| \leq \frac{\operatorname{KL}\left(\Q, \P\right)+2 +\log (4/\delta)}{\lambda_m}+ \frac{\lambda_m}{2}\left( \hat{V}_m(\Q) + V_m(\Q) \right).
\end{align*}
Notice that for any $\Q$, $\operatorname{KL}(\Q,\P)\leq \log(K)$ by concavity of the log.
We now fix an horizon $M>0$, we then have in particular, with probability $1- \delta/2$: for any posterior $\Q$,
\begin{align*}
\left|M_m\left(Q\right)\right| \leq \frac{\log(K)+2 \log (k+1)+\log (4/\delta)}{\lambda_k}+ \frac{\lambda_m}{2}\left( \hat{V}_m(\Q) + V_m(\Q) \right).
\end{align*}
We now have to deal with $V_k(\Q), \hat{V}_k(\Q)$ for all $k\leq m$. To do so, we propose the two following lemmas.
\begin{lemma}
\label{l: bandit_lemma_1}
For all $m\geq 1$, $a\in\mathcal{A}$, $V_m(a)\leq \frac{2Cm}{\varepsilon_m}$.
Then, we have for any $m,Q$, $V_m(\Q)\leq \frac{2Cm}{\varepsilon_m}$.
\end{lemma}

\begin{proof} We have
\begin{align*}
V_t(a) &=\sum_{i=1}^m \mathbb{E}\left[\left(\left[R_i^{a^*}-R_i^a\right]-\Delta(a)\right)^2 \mid \mathcal{F}_{i-1}\right] \\
&=\sum_{i=1}^m \mathbb{E}\left[\left(R_i^{a^*}-R_i^a\right)^2 \mid \mathcal{F}_{i-1}\right]-m \Delta(a)^2\\
& \leq \sum_{i=1}^m \mathbb{E}\left[\left(R_i^{a^*}-R_i^a\right)^2 \mid \mathcal{F}_{i-1}\right]  \\
&  = \sum_{i=1}^m \mathbb{E}\left[\mathbb{E}_{A_i\sim \pi_i}\mathbb{E}_{R_i}\left[\frac{1}{\pi_i(a^*)^2} R_i(a^*)^2\mathds{1}(A_i=a^*) +\frac{1}{\pi_i(a)^2}R_i(a)^2\mathds{1}(A_i=a) \right] \mid \mathcal{F}_{i-1}\right].\\
\intertext{The last line holding because $R_i$ is independent of $\mathcal{F}_{i-1}$, $A_i$ is independent of $R_i$ and $\pi$ is $\mathcal{F}_{i-1}$ measurable.
We now use that for all $i,a$, $\mathbb{E}_{R_i}[R_i(a)^2] \leq C$ }
& = \sum_{i=1}^m \mathbb{E}\left[\mathbb{E}_{A_i\sim \pi_i}
\left[\frac{1}{\pi_i(a^*)^2} C\mathds{1}(A_i=a^*) +\frac{1}{\pi_i(a)^2}C\mathds{1}(A_i=a) \right] \mid \mathcal{F}_{i-1}\right]\\
& =\sum_{i=1}^m C\left(\frac{\pi_i(a)}{\pi_i(a)^2}+\frac{\pi_i\left(a^*\right)}{\pi_i\left(a^*\right)^2}\right) \\ &=\sum_{i=1}^m C\left(\frac{1}{\pi_i(a)}+\frac{1}{\pi_i\left(a^*\right)} \right) \\
& \leq \frac{2C m}{\varepsilon_m}.
\end{align*}
\end{proof}


\begin{lemma}
\label{l: bandit_lemma_2}
Let $m\geq 1$, with probability $1-\delta/2$, for any posterior $\Q$, we have
\[ \hat{V}_m(\Q) \leq \frac{4CKm}{\varepsilon_m\delta}. \]
\end{lemma}

\begin{proof}
Let $\Q$ a distribution over $\mathcal{A}$. Recall that
\begin{align*}
\hat{V}_m(\Q) & = \sum_{i=1}^m \left(R_i^{a^*}-R_i^a-\left[R\left(a^*\right)-R(a)\right]\right)^2 \\
& = \sum_{a\in\mathcal{A}} Q(a) \hat{V}_m(a).
\end{align*}
Notice that for any $a$, $(\hat{SM}_m^a)_m$ is a  nonnegative random variable. We then apply Markov's inequality for any $a$, with probability $1-\delta/2K$
\[ \hat{V}_m(a)\leq  \frac{2K\mathbb{E}[\hat{V}_m(a)]}{\delta} .  \]
Noticing that $\mathbb{E}[\hat{V}_m(a)] = \mathbb{E}[V_m(a)]$, we can apply \cref{l: bandit_lemma_1} to conclude that
$$\mathbb{E}[\hat{V}_m(a)] \leq \frac{2Cm}{\varepsilon_m}.$$
Finally, taking an union bound on thoser events for all $a\in\mathcal{A}$ gives us, with probability $1-\delta/2$, for any posterior $\Q$
\begin{align*}
V_m(\Q) & \leq \sum_{a\in\mathcal{A}} Q(a) \hat{V}_m(a) \\
& \leq \sum_{a\in\mathcal{A}} Q(a) \frac{4CKm}{\varepsilon_m\delta} \\
&= \frac{4CKm}{\varepsilon_m\delta}.
\end{align*}
This concludes the proof.
\end{proof}
To conclude, we apply \cref{l: bandit_lemma_1,l: bandit_lemma_2} to get that with probability $1-\delta$, for any posterior $\Q$
\begin{align*}
\left|M_m\left(\Q\right)\right| & \leq \frac{\operatorname{KL}\left(\Q, \P\right) +\log (4/\delta)}{\lambda_m}+ \frac{Cm\lambda_m}{\varepsilon_m} \left(1+ \frac{2K}{\delta}    \right).
\end{align*}
Dividing by $m$ and taking $$\lambda_m= \sqrt{\frac{\left(\log(K) +\log (4/\delta)\right) \varepsilon_m}{Cm\left(1+ \frac{2K}{\delta}    \right)}}$$ concludes the proof.

\end{proof}





\subsection{Proof of Cor. \ref{cor: bound_mart}}


\begin{proof}
Fix $\delta>0$. For any pair $(\lambda_k,P_k), k\geq 1$, we apply \Cref{th: main_thm} with $$\delta_k := \frac{\delta}{k(k+1)} \geq \frac{\delta}{(k+1)^2}. $$
Notice that we have $\sum_{k=1}^{+\infty} \delta_k = \delta$.
We then have with probability $1-\delta_k$ over $S$, for any $m\geq 1$, any posterior $\Q$,
\[ \left|M_m\left(\Q\right)\right| \leq \frac{\KL\left(\Q, \P_k\right)+2 \log (k+1)+\log (2/\delta)}{\lambda_k}+ \frac{\lambda_k}{2}\left( \hat{V}_m(\Q) + V_m(\Q) \right).\]
Taking an union bound on all those event, gives the final result, valid with probability $1-\delta$ over the sample $S$, for any any tuple $(m,\lambda_k,P_k)$ with $m,k\geq 1$, any posterior $\Q$ over $\mathcal{H}$. This gives \Cref{eq: bound_mart_1}.

To obtain \cref{eq: bound_mart_2}, we restrict the range of \cref{eq: bound_mart_1} to the tuples $(m,\lambda_m,P_m), m\geq 1$ (the restricted set of tuples where $k=m$) and we bound both $\hat{V}_m(\Q),V_m(\Q)$ by $\sum_{i=1}^m C_i^2$ to conclude.
\end{proof}

\subsection{Proof of Cor. \ref{cor: bounded_case}}


\begin{proof}
For the first bound we start from the intermediary result \cref{eq: intermediary_result_main} of  \cref{th: main_thm}. Using the same marrtingale as in \cref{th: main_thm_iid} gives, for any $\eta\in\mathbb{R}$, holding with probability $1-\delta$ for any $m>0,Q\in\mathcal{M}(\mathcal{H})$
\begin{multline*}
\eta \left(\sum_{i=1}^m \mathbb{E}_{h\sim \Q}[\ell(h,\z_i)]  -m \mathbb{E}_{h\sim \Q}[ \Risk(h) ] \right) \\
\leq  \operatorname{KL}(\Q,\P) +\log(1/\delta) + \frac{\eta^2}{2}\sum_{i=1}^m \mathbb{E}_{h\sim \Q}[\Delta[M]_i(h) + \Delta \langle M\rangle_i(h) ].
\end{multline*}
Taking $\eta= \pm\lambda$ with $\lambda>0$ gives
\begin{align}
\label{eq: temp_result}
\lambda m\left | \mathbb{E}_{h\sim \Q}[ \Risk(h) -\Riskhat_{\Sm}(h)] \right | & \leq \operatorname{KL}(\Q,\P) +\log(1/\delta) \\
& + \frac{\lambda^2}{2}\sum_{i=1}^m \mathbb{E}_{h\sim \Q}[\Delta[M]_i(h) + \Delta \langle M\rangle_i(h) ].
\end{align}
Finally, divide by $\lambda m$ and bound $\Delta[M]_i(h) + \Delta \langle M\rangle_i(h)$ by $2K^2$ to conclude.


For the second bound, we start from \Cref{eq: temp_result} again and for a fixed $m$, we now apply our result with $\lambda'=\lambda/m$. We then have for any $m$, with probability $1-\delta$, for any $\Q$
\[ \lambda\left |  \mathbb{E}_{h\sim \Q}[ \Risk(h) -\Riskhat_{\Sm}(h)] \right |  \leq \operatorname{KL}(\Q,\P) +\log(1/\delta) + \frac{\lambda^2}{2m^2}\sum_{i=1}^m \mathbb{E}_{h\sim \Q}[\Delta[M]_i(h) + \Delta \langle M\rangle_i(h) ].\]
Finally, dividing by $\lambda$, bounding $\Delta[M]_i(h) + \Delta \langle M\rangle_i(h)$ by $2K^2$ and rearranging the terms concludes the proof.
\end{proof}


\end{noaddcontents}
%%!TEX root = main.tex
\chapter{Appendix of Chapter~\ref{chap: pb-ht}}
\label{ap: pb-ht}

\begin{noaddcontents}
    

\section{Some PAC-Bayesian background}
\label{sec: pac_b_background}

We present below an immediate corollary of \citet[Thm 2.1]{seldin2012bandit} where we upper bounded the cumulative by an empirical quantity (the sum of squared upper bound of the martingale difference sequence).

\begin{theorem}[\citealp{seldin2012bandit}, Theorem 2.1]
\label{th: seldin_thm_mart}
Let $\left\{C_1, C_2, \ldots\right\}$ be an increasing sequence set in advance, such that $\left|X_i(\\S_i,h)\right| \leq C_i$ for all $\\S_i,h$ with probability 1.   Let $\left\{P_1, P_2, \ldots\right\}$ be a sequence of data-free prior distributions over $\mathcal{H}$. Let $(\lambda_i)_{i\geq 1}$ be a sequence of positive numbers such that
$$
\lambda_m \leq \frac{1}{C_m}.
$$
Then with probability $1-\delta$ over $\S=(\z_i)_{i\geq 1}$,
for all $m\geq 1$, any posterior $\Q$ over $\mathcal{H}$,
$$
\left|M_m\left(Q\right)\right| \leq \frac{\KL\left(\Q , \P_m\right)+2 \log (m+1)+\log \frac{2}{\delta}}{\lambda_m}+(e-2) \lambda_m V_m(\Q),
$$
where $V_m(\Q)$ is defined in \cref{subsec: comparison_seldin}.

Furthermore, if we bound the variance term, we would have:
$$
\left|M_m\left(\Q\right)\right| \leq \frac{\KL\left(\Q , \P_m\right)+2 \log (m+1)+\log \frac{2}{\delta}}{\lambda_m}+(e-2) \lambda_m \sum_{i=1}^m C_i^2.
$$
\end{theorem}
Below, we use the definitions introduced in \Cref{sec: iid_case}.
We study here a particular case of \cite{alquier2016properties} for bounded losses which are especially subgaussian thanks to Hoeffding's lemma.
\begin{theorem}[Adapted from  \citealp{alquier2016properties}, Theorem 4.1]
\label{th: naive_pac_bayes-chap3}
Let $m>0$,$\S_m=(\z_1,...,\z_m)$ be an \iid sample from the same law $\mu$.
For any data-free prior $\P$, for any loss function $\ell$ bounded by $K$, any $\lambda>0,\delta\in ]0;1[$, one has with probability $1-\delta$ for any posterior $Q\in\mathcal{M}_1(\mathcal{H})$
\[ \mathbb{E}_{h\sim \Q}[\Risk(h)] \leq  \mathbb{E}_{h\sim \Q}[\Riskhat_{\Sm}(h)] + \frac{\operatorname{KL}(\Q, \P) + \log(1/\delta)}{\lambda} + \frac{\lambda K^2}{2m}. \]
\end{theorem}

\begin{theorem}[\citealp{haddouche2021pac}, Theorem 3]
\label{th: haddouche_thm}
Let the loss $\ell$ be $\mathrm{HYPE}(K)$ compliant. For any $\P\in\mathcal{M}(\mathcal{H})$ with no data dependency, for any $\alpha\in\mathbb{R}$ and for any $\delta\in[0,1]$, we have with probability at least $1-\delta$ over size-$m$ samples S,
for any $\Q$% such that $Q \ll P$% and $P \ll Q$
\begin{align*}
\mathbb{E}_{h\sim \Q}\left[ \Risk(h)\right]
\leq \mathbb{E}_{h\sim \Q}\left[ \Riskhat_{\Sm}(h)\right] + \frac{\operatorname{KL}(\Q,\P) + \log\left(\frac{1}{\delta}\right)}{m^{\alpha}}
+\frac{1}{m^{\alpha}}\log\left(\mathbb{E}_{h\sim \P} \left[\exp\left( \frac{K(h)^2}{2m^{1-2\alpha}} \right) \right]\right).
\end{align*}
\end{theorem}


\section{Extensions of previous results}
\label{sec: extensions}

Here we gather several corollaries of our main result in order to show how our \Cref{th: main_thm} extends the validity of some classical results in the literature. More precisely we show that our result extends (up to numerical factors) the PAC-Bayes Bernstein inequality of \citet{seldin2012bandit}.
Then, going back to the bounded case, we generalise a result from \citet{catoni2007pac} reformulated in \citet{alquier2016properties} and we also show how our work strictly improves on the bound of \citet{haddouche2021pac}.

\subsection{Extension of the PAC-Bayes Bernstein inequality}
\label{subsec: comparison_seldin}

Here we rename two terms for consistency with Theorem 2.1 of \citet{seldin2012bandit} (see \Cref{th: seldin_thm_mart}). For a martingale $M_m(h)= \sum_{i=1}^m X_i(\S_i,h)$, we define, at time $m$, \emph{empirical cumulative variance } to be $\hat{V}_m(h)= [M]_m(h) = \sum_{i=1}^m X_i(\S_i,h)^2$ and
the \emph{cumulative variance} as $V_m(h)= \langle M\rangle_m(h) = \sum_{i=1}^m \mathbb{E}_{i-1}[X_i(\S_i,h)^2]$.

We provide below a corollary containing two bounds: the first one being a straightforward corollary of \cref{th: main_thm}, the second being valid for bounded martingales and formally close to Theorem 2.1 of \citet{seldin2012bandit}.
\begin{corollary}
\label{cor: bound_mart}
Let $\left\{\P_1, \P_2, \ldots\right\}$ be a sequence of data-free prior distributions over $\mathcal{H}$. Let $(\lambda_i)_{i\geq 1}$ be a sequence of positive numbers.
Then the following holds with probability $1-\delta$ over $\S=(\z_i)_{i\geq 1}$: for any tuple $(m,\lambda_k,\P_k)$ with $m,k\geq 1$, any posterior $\Q$ over $\mathcal{H}$,
\begin{align}
\label{eq: bound_mart_1}
\left|M_m\left(\Q\right)\right| \leq \frac{\KL\left(\Q, \P_k\right)+2 \log (k+1)+\log (2/\delta)}{\lambda_k}+ \frac{\lambda_k}{2}\left( \hat{V}_m(\Q) + V_m(\Q) \right),
\end{align}
with $\hat{V}_m(\Q)= \mathbb{E}_{h\sim \Q}[\hat{V}_m(h)], V_m(\Q)= \mathbb{E}_{h\sim \Q}[V_m(h)]$.
Furthermore, if we assume that for any $i$, there exists $C_i>0$ such that $|X_i(\S_i,h)|\leq C_i$ for all $\S_i,h$ then we have the following corollary: with probability $1-\delta$ over $S$, for any tuple $(m,\lambda_m,\P_m)$ $m\geq 1$, any posterior $\Q$,
\begin{align}
\label{eq: bound_mart_2}
\left|M_m\left(Q\right)\right| \leq \frac{\KL\left(\Q, \P_m\right)+2 \log (m+1)+\log (2/\delta)}{\lambda_m}+ \lambda_m\sum_{i=1}^m C_i^2.
\end{align}
\end{corollary}
The proof is deferred to \cref{sec: proofs}.
Note that \cref{eq: bound_mart_1} holds uniformly on all tuples $\{(\lambda_k,\P_k,m) \mid k\geq 1, m\geq 1\}$ while \cref{eq: bound_mart_2}, as well as Theorem 2.1 of \citet{seldin2012bandit} holds uniformly on the tuples
$\{(\lambda_m,\P_m,m) \mid m\geq 1\}$ which is a strictly smaller collection. Hence our approach gives guarantees for a larger event with the same confidence level.

Furthermore, Theorem 2.1 of \citet{seldin2012bandit} involves the cumulative variance $V_m(\Q)$ (and not its empirical counterpart). Because this term is theoretical, we bound it in \cref{th: seldin_thm_mart} by $\sum_{i=1}^m C_i^2$ which is supposedly empirical.
In this context, \cref{eq: bound_mart_2}, recovers nearly exactly the bound of \cite{seldin2012bandit} with the transformation of a factor $(e-2)$ into $1$.
Notice also that \cref{eq: bound_mart_2} stands with no assumption on the range of the $\lambda_i$, which is not the case in \cref{th: seldin_thm_mart}.

Finally, we stress two fundamental differences between our work and the one of \citet{seldin2012bandit}. First, we replace Markov's inequality by Ville's inequality; second, we exploited the exponential inequality of Lemma \ref{l: bercu_touati} instead of the Bernstein inequality. These allow for results for unbounded martingales for all $m$ simultaneously.


\subsection{Extensions of learning theory results}


\subsubsection{A general result for bounded losses}


We use definitions from \Cref{sec: iid_case} and provide a corollary of our main result when the loss is bounded by a positive constant $K>0$. We assume our data are iid.
\begin{corollary}
\label{cor: bounded_case}
For any data-free prior $P\in \mathcal{M}(\mathcal{H})$, any $\lambda>0$ the following holds with probability $1-\delta$ over the sample $S=(z_i)_{i\in\mathbb{N}}$, for all $m\in\mathbb{N}/\{0\}$, $\Q\in\mathcal{M}(\mathcal{H})$
\[ \left | \mathbb{E}_{h\sim \Q} [\Risk(h)] -  \mathbb{E}_{h\sim \Q} \left[\Riskhat_{\S_m}(h) \right] \right |  \leq \frac{\operatorname{KL}(\Q,\P) +\log(2/\delta)}{\lambda m } + \lambda K^2.  \]
We also have the local bound: for any $m\geq 1$, with probability $1-\delta$ over $S$, for all $\Q\in\mathcal{M}(\mathcal{H})$
\[ \mathbb{E}_{h\sim \Q} [\Risk(h)] \leq  \mathbb{E}_{h\sim \Q} \left[\Riskhat_{\S_m}(h) \right] + \frac{\operatorname{KL}(\Q,\P) +\log(2/\delta)}{\lambda} + \frac{\lambda K^2}{m}.  \]
\end{corollary}
The proof is deferred to \cref{sec: proofs}. Remark that the second bound of Corollary \ref{cor: bounded_case} is exactly the Catoni bound stated in \citet{alquier2016properties} (see \Cref{th: naive_pac_bayes-chap3} in \Cref{sec: pac_b_background}) up to a numerical factor of $2$.


The first bound is, to our knowledge, the first PAC-Bayesian bound for bounded losses holding uniformly (for a given parameter $\lambda$) on the choice of $Q,m$ and thus extends the scope of Catoni's bound which holds for a single $m$ with high probability.  Indeed, if we want for instance \Cref{th: naive_pac_bayes-chap3} to hold for any $i\in\{1..m\}$, we then have to take an union bound on $m$ events which turns the term $\log(1/\delta)$ into $\log(m/\delta)$ (but with the benefit of holding for $m$ parameters $\lambda_1,...,\lambda_m$). This point is common to the most classical PAC-Bayesian bounds
(including McAllester and Catoni's ones \eqref{eq: mcallester}, \eqref{eq: catoni})
and impeach us to have a bound uniformly on all $m\in\mathbb{N}/\{0\}$ as $\log(m)$ goes to infinity asymptotically.


\subsubsection{An extension of \citet{haddouche2021pac}}

We now focus on the work of \citet{haddouche2021pac} which provides general PAC-Bayesian bounds for unbounded losses. Their theorems hold for iid data and under the so-called \emph{HYPE} (for HYPothesis-dependent rangE) condition. It states that a loss function $\ell$ is \emph{HYPE}$(K)$ compliant if there exists a function $K:\mathcal{H} \rightarrow \mathbb{R}^+ $ (supposedly accessible)  such that $\forall z\in\mathcal{Z}, \ell(h,\z) \leq K(h)$.
We provide \Cref{cor: haddouche_comparison} to compare ourselves with their main result (stated in  \Cref{th: haddouche_thm} for convenience).
\begin{corollary}
\label{cor: haddouche_comparison}
For any data-free prior $\P\in \mathcal{M}(\mathcal{H})$, any loss function $\ell$ being \emph{HYPE}$(K)$ compliant, any $\alpha\in[0,1],m\geq 1$, the following holds with probability $1-\delta$ over the sample $\S=(\z_i)_{i\in\mathbb{N}}$, for all $\Q\in\mathcal{M}(\mathcal{H})$
\begin{multline*}
\mathbb{E}_{h\sim \Q} [\Risk(h)] \leq   \mathbb{E}_{h\sim \Q} \left[\frac{1}{m}\sum_{i=1}^m\left(\ell(h,\z_i) + \frac{1}{2m^{1-\alpha}} \ell(h,\z_i)^2\right)\right] \\
+ \frac{\operatorname{KL}(\Q,\P) +\log(1/\delta)}{m^{\alpha}}  + \frac{1}{2m^{1-\alpha}}\mathbb{E}_{h\sim \Q} [K^2(h)].
\end{multline*}
\end{corollary}

\begin{proof}
The proof is a straightforward application of \cref{th: main_thm_iid} by fixing $m\geq 1$ choosing $\lambda= m^{\alpha-1}$ (thus we localise \Cref{th: main_thm_iid} to a single $m$),  and bounding $\mathrm{Quad}(h)$ by $K^2(h)$.
\end{proof}
The main improvement of our bound over \Cref{th: haddouche_thm} is that we do not have to assume the convergence of an exponential moment to obtain a non-trivial bound. Indeed, we transformed the (implicit) assumption $\mathbb{E}_{h\sim \P} \left[\exp\left( \frac{K(h)^2}{2m^{1-2\alpha}} \right) \right] < +\infty $ onto $\mathbb{E}_{h\sim \Q}[K(h)^2] < +\infty$, which is significantly less restrictive.
Furthermore, \Cref{th: haddouche_thm} holds for a single choice of $m$ while ours still holds uniformly over all integers $m>0$.

Cor. \ref{cor: haddouche_comparison} also sheds new light on the \emph{HYPE} condition. Indeed, in \citet{haddouche2021pac}, $K$ only intervenes in an exponential moment involving the prior $\P$, while ours considers a second-order moment on $K$ implying the posterior $\Q$. The difference is major as $\mathbb{E}_{h\sim \Q}[K(h)^2] $ can be controlled by a wise choice of posterior. Thus it can be incorporated in our optimisation route, acting now as an optimisation constraint instead of an environment constraint.



\section{Proofs}
\label{sec: proofs}

\subsection{Proof of \cref{th: main_thm_iid}}

\begin{proof}
Let $\P$ a fixed data-free prior, set $(\mathcal{F}_i)_{i\geq 0}$ such that for all $i$, $\z_i$ is $\mathcal{F}_i$ measurable. We also set for any fixed $h\in\mathcal{H}, M_m(h):= \sum_{i=1}^m \ell(h,\z_i) - \Risk(h)$. Note that because data are \iid, for any fixed $h$, the sequence $(M_m(h))_m$ is indeed a martingale.
We set for any $m\geq 1, h\in\mathcal{H}$
$$[M]_m(h) = \sum_{i=1}^m \left(\ell(h,\z_i) - \Risk(h)\right)^2 $$ and
$$\langle M \rangle_m(h) =  \sum_{i=1}^m \mathbb{E}_{i-1}[\left(\ell(h,\z_i) - \Risk(h)\right)^2] = \sum_{i=1}^m \mathbb{E}_{\z\sim \D}[\left(\ell(h,\z) - \Risk(h)\right)^2].$$
The last equality holds because data is assumed iid. Thus, we can apply \cref{th: main_thm} to obtain with probability $1-\delta$
\[|M_m(\Q)| \leq   \frac{\operatorname{KL}(\Q,\P) +\log(2/\delta)}{\lambda } + \frac{\lambda}{2}\left([M]_m(Q)^2 + \langle M\rangle_m(Q)^2 \right) . \]
Now, we notice that $|M_m(\Q)| = m| \mathbb{E}_{h\sim \Q}[\Risk(h) - \Riskhat_{\Sm}(h)] |$  and that  for any $m,h$, because $\ell$ is nonnegative
\begin{align*}
[M]_m(h) +  \langle M\rangle_m(h) & = \sum_{i=1}^m (\ell(h,\z_i) - \Risk(h))^2 + \mathbb{E}_{\z\sim \D}[ (\ell(h,\z) - \Risk(h))^2] \\
& \leq  \sum_{i=1}^m \ell(h,\z_i)^2 + \Risk(h)^2 + \mathbb{E}_{\z\sim \D}[\ell(h,\z)^2] - \Risk(h)^2.\\
\intertext{ Thus integrating over $h$ gives: }
[M]_m(Q) +  \langle M\rangle_m(Q) & \leq \sum_{i=1}^m \mathbb{E}_{h\sim \Q} [\ell(h,\z_i)^2] + m\mathbb{E}_{h\sim \Q} [\mathrm{Quad}(h)].
\end{align*}
Then dividing by $m$ and applying the last inequality gives
\begin{multline*}
\mathbb{E}_{h\sim \Q} [\Risk(h)]  \leq  \mathbb{E}_{h\sim \Q} \left[\frac{1}{m}\sum_{i=1}^m\left(\ell(h,\z_i) + \frac{\lambda}{2} \ell(h,\z_i)^2\right)\right] \\
+ \frac{\operatorname{KL}(\Q,\P) +\log(2/\delta)}{\lambda m} + \frac{\lambda}{2}\mathbb{E}_{h\sim \Q} [\mathrm{Quad}(h)].
\end{multline*}
This concludes the proof.
\end{proof}






\subsection{Proof of \cref{th: bandits_bound}}

\begin{proof}
Let $(\lambda_m)_{i\geq 1}$ be a countable sequence of positive scalars.
As precised earlier $M_m(a):= m\left(\hat{\Delta}_m(a)-\Delta(a)\right)$ is a martingale.
We then apply \Cref{th: main_thm} with the uniform prior ($\forall a, P(a)= \frac{1}{K}$) and $\lambda= \lambda_m$  (depending possibly on $m$): with probability $1- \delta/2$, for any tuple $(m,\lambda_m)$ with $m\geq 1$, any posterior $\Q$,
\begin{align*}
\left|M_m\left(\Q\right)\right| \leq \frac{\operatorname{KL}\left(\Q, \P\right)+2 +\log (4/\delta)}{\lambda_m}+ \frac{\lambda_m}{2}\left( \hat{V}_m(\Q) + V_m(\Q) \right).
\end{align*}
Notice that for any $\Q$, $\operatorname{KL}(\Q,\P)\leq \log(K)$ by concavity of the log.
We now fix an horizon $M>0$, we then have in particular, with probability $1- \delta/2$: for any posterior $\Q$,
\begin{align*}
\left|M_m\left(Q\right)\right| \leq \frac{\log(K)+2 \log (k+1)+\log (4/\delta)}{\lambda_k}+ \frac{\lambda_m}{2}\left( \hat{V}_m(\Q) + V_m(\Q) \right).
\end{align*}
We now have to deal with $V_k(\Q), \hat{V}_k(\Q)$ for all $k\leq m$. To do so, we propose the two following lemmas.
\begin{lemma}
\label{l: bandit_lemma_1}
For all $m\geq 1$, $a\in\mathcal{A}$, $V_m(a)\leq \frac{2Cm}{\varepsilon_m}$.
Then, we have for any $m,Q$, $V_m(\Q)\leq \frac{2Cm}{\varepsilon_m}$.
\end{lemma}

\begin{proof} We have
\begin{align*}
V_t(a) &=\sum_{i=1}^m \mathbb{E}\left[\left(\left[R_i^{a^*}-R_i^a\right]-\Delta(a)\right)^2 \mid \mathcal{F}_{i-1}\right] \\
&=\sum_{i=1}^m \mathbb{E}\left[\left(R_i^{a^*}-R_i^a\right)^2 \mid \mathcal{F}_{i-1}\right]-m \Delta(a)^2\\
& \leq \sum_{i=1}^m \mathbb{E}\left[\left(R_i^{a^*}-R_i^a\right)^2 \mid \mathcal{F}_{i-1}\right]  \\
&  = \sum_{i=1}^m \mathbb{E}\left[\mathbb{E}_{A_i\sim \pi_i}\mathbb{E}_{R_i}\left[\frac{1}{\pi_i(a^*)^2} R_i(a^*)^2\mathds{1}(A_i=a^*) +\frac{1}{\pi_i(a)^2}R_i(a)^2\mathds{1}(A_i=a) \right] \mid \mathcal{F}_{i-1}\right].\\
\intertext{The last line holding because $R_i$ is independent of $\mathcal{F}_{i-1}$, $A_i$ is independent of $R_i$ and $\pi$ is $\mathcal{F}_{i-1}$ measurable.
We now use that for all $i,a$, $\mathbb{E}_{R_i}[R_i(a)^2] \leq C$ }
& = \sum_{i=1}^m \mathbb{E}\left[\mathbb{E}_{A_i\sim \pi_i}
\left[\frac{1}{\pi_i(a^*)^2} C\mathds{1}(A_i=a^*) +\frac{1}{\pi_i(a)^2}C\mathds{1}(A_i=a) \right] \mid \mathcal{F}_{i-1}\right]\\
& =\sum_{i=1}^m C\left(\frac{\pi_i(a)}{\pi_i(a)^2}+\frac{\pi_i\left(a^*\right)}{\pi_i\left(a^*\right)^2}\right) \\ &=\sum_{i=1}^m C\left(\frac{1}{\pi_i(a)}+\frac{1}{\pi_i\left(a^*\right)} \right) \\
& \leq \frac{2C m}{\varepsilon_m}.
\end{align*}
\end{proof}


\begin{lemma}
\label{l: bandit_lemma_2}
Let $m\geq 1$, with probability $1-\delta/2$, for any posterior $\Q$, we have
\[ \hat{V}_m(\Q) \leq \frac{4CKm}{\varepsilon_m\delta}. \]
\end{lemma}

\begin{proof}
Let $\Q$ a distribution over $\mathcal{A}$. Recall that
\begin{align*}
\hat{V}_m(\Q) & = \sum_{i=1}^m \left(R_i^{a^*}-R_i^a-\left[R\left(a^*\right)-R(a)\right]\right)^2 \\
& = \sum_{a\in\mathcal{A}} Q(a) \hat{V}_m(a).
\end{align*}
Notice that for any $a$, $(\hat{SM}_m^a)_m$ is a  nonnegative random variable. We then apply Markov's inequality for any $a$, with probability $1-\delta/2K$
\[ \hat{V}_m(a)\leq  \frac{2K\mathbb{E}[\hat{V}_m(a)]}{\delta} .  \]
Noticing that $\mathbb{E}[\hat{V}_m(a)] = \mathbb{E}[V_m(a)]$, we can apply \cref{l: bandit_lemma_1} to conclude that
$$\mathbb{E}[\hat{V}_m(a)] \leq \frac{2Cm}{\varepsilon_m}.$$
Finally, taking an union bound on thoser events for all $a\in\mathcal{A}$ gives us, with probability $1-\delta/2$, for any posterior $\Q$
\begin{align*}
V_m(\Q) & \leq \sum_{a\in\mathcal{A}} Q(a) \hat{V}_m(a) \\
& \leq \sum_{a\in\mathcal{A}} Q(a) \frac{4CKm}{\varepsilon_m\delta} \\
&= \frac{4CKm}{\varepsilon_m\delta}.
\end{align*}
This concludes the proof.
\end{proof}
To conclude, we apply \cref{l: bandit_lemma_1,l: bandit_lemma_2} to get that with probability $1-\delta$, for any posterior $\Q$
\begin{align*}
\left|M_m\left(\Q\right)\right| & \leq \frac{\operatorname{KL}\left(\Q, \P\right) +\log (4/\delta)}{\lambda_m}+ \frac{Cm\lambda_m}{\varepsilon_m} \left(1+ \frac{2K}{\delta}    \right).
\end{align*}
Dividing by $m$ and taking $$\lambda_m= \sqrt{\frac{\left(\log(K) +\log (4/\delta)\right) \varepsilon_m}{Cm\left(1+ \frac{2K}{\delta}    \right)}}$$ concludes the proof.

\end{proof}





\subsection{Proof of Cor. \ref{cor: bound_mart}}


\begin{proof}
Fix $\delta>0$. For any pair $(\lambda_k,P_k), k\geq 1$, we apply \Cref{th: main_thm} with $$\delta_k := \frac{\delta}{k(k+1)} \geq \frac{\delta}{(k+1)^2}. $$
Notice that we have $\sum_{k=1}^{+\infty} \delta_k = \delta$.
We then have with probability $1-\delta_k$ over $S$, for any $m\geq 1$, any posterior $\Q$,
\[ \left|M_m\left(\Q\right)\right| \leq \frac{\KL\left(\Q, \P_k\right)+2 \log (k+1)+\log (2/\delta)}{\lambda_k}+ \frac{\lambda_k}{2}\left( \hat{V}_m(\Q) + V_m(\Q) \right).\]
Taking an union bound on all those event, gives the final result, valid with probability $1-\delta$ over the sample $S$, for any any tuple $(m,\lambda_k,P_k)$ with $m,k\geq 1$, any posterior $\Q$ over $\mathcal{H}$. This gives \Cref{eq: bound_mart_1}.

To obtain \cref{eq: bound_mart_2}, we restrict the range of \cref{eq: bound_mart_1} to the tuples $(m,\lambda_m,P_m), m\geq 1$ (the restricted set of tuples where $k=m$) and we bound both $\hat{V}_m(\Q),V_m(\Q)$ by $\sum_{i=1}^m C_i^2$ to conclude.
\end{proof}

\subsection{Proof of Cor. \ref{cor: bounded_case}}


\begin{proof}
For the first bound we start from the intermediary result \cref{eq: intermediary_result_main} of  \cref{th: main_thm}. Using the same marrtingale as in \cref{th: main_thm_iid} gives, for any $\eta\in\mathbb{R}$, holding with probability $1-\delta$ for any $m>0,Q\in\mathcal{M}(\mathcal{H})$
\begin{multline*}
\eta \left(\sum_{i=1}^m \mathbb{E}_{h\sim \Q}[\ell(h,\z_i)]  -m \mathbb{E}_{h\sim \Q}[ \Risk(h) ] \right) \\
\leq  \operatorname{KL}(\Q,\P) +\log(1/\delta) + \frac{\eta^2}{2}\sum_{i=1}^m \mathbb{E}_{h\sim \Q}[\Delta[M]_i(h) + \Delta \langle M\rangle_i(h) ].
\end{multline*}
Taking $\eta= \pm\lambda$ with $\lambda>0$ gives
\begin{align}
\label{eq: temp_result}
\lambda m\left | \mathbb{E}_{h\sim \Q}[ \Risk(h) -\Riskhat_{\Sm}(h)] \right | & \leq \operatorname{KL}(\Q,\P) +\log(1/\delta) \\
& + \frac{\lambda^2}{2}\sum_{i=1}^m \mathbb{E}_{h\sim \Q}[\Delta[M]_i(h) + \Delta \langle M\rangle_i(h) ].
\end{align}
Finally, divide by $\lambda m$ and bound $\Delta[M]_i(h) + \Delta \langle M\rangle_i(h)$ by $2K^2$ to conclude.


For the second bound, we start from \Cref{eq: temp_result} again and for a fixed $m$, we now apply our result with $\lambda'=\lambda/m$. We then have for any $m$, with probability $1-\delta$, for any $\Q$
\[ \lambda\left |  \mathbb{E}_{h\sim \Q}[ \Risk(h) -\Riskhat_{\Sm}(h)] \right |  \leq \operatorname{KL}(\Q,\P) +\log(1/\delta) + \frac{\lambda^2}{2m^2}\sum_{i=1}^m \mathbb{E}_{h\sim \Q}[\Delta[M]_i(h) + \Delta \langle M\rangle_i(h) ].\]
Finally, dividing by $\lambda$, bounding $\Delta[M]_i(h) + \Delta \langle M\rangle_i(h)$ by $2K^2$ and rearranging the terms concludes the proof.
\end{proof}


\end{noaddcontents}
%%!TEX root = main.tex
\chapter{Appendix of Chapter~\ref{chap: pb-ht}}
\label{ap: pb-ht}

\begin{noaddcontents}
    

\section{Some PAC-Bayesian background}
\label{sec: pac_b_background}

We present below an immediate corollary of \citet[Thm 2.1]{seldin2012bandit} where we upper bounded the cumulative by an empirical quantity (the sum of squared upper bound of the martingale difference sequence).

\begin{theorem}[\citealp{seldin2012bandit}, Theorem 2.1]
\label{th: seldin_thm_mart}
Let $\left\{C_1, C_2, \ldots\right\}$ be an increasing sequence set in advance, such that $\left|X_i(\\S_i,h)\right| \leq C_i$ for all $\\S_i,h$ with probability 1.   Let $\left\{P_1, P_2, \ldots\right\}$ be a sequence of data-free prior distributions over $\mathcal{H}$. Let $(\lambda_i)_{i\geq 1}$ be a sequence of positive numbers such that
$$
\lambda_m \leq \frac{1}{C_m}.
$$
Then with probability $1-\delta$ over $\S=(\z_i)_{i\geq 1}$,
for all $m\geq 1$, any posterior $\Q$ over $\mathcal{H}$,
$$
\left|M_m\left(Q\right)\right| \leq \frac{\KL\left(\Q , \P_m\right)+2 \log (m+1)+\log \frac{2}{\delta}}{\lambda_m}+(e-2) \lambda_m V_m(\Q),
$$
where $V_m(\Q)$ is defined in \cref{subsec: comparison_seldin}.

Furthermore, if we bound the variance term, we would have:
$$
\left|M_m\left(\Q\right)\right| \leq \frac{\KL\left(\Q , \P_m\right)+2 \log (m+1)+\log \frac{2}{\delta}}{\lambda_m}+(e-2) \lambda_m \sum_{i=1}^m C_i^2.
$$
\end{theorem}
Below, we use the definitions introduced in \Cref{sec: iid_case}.
We study here a particular case of \cite{alquier2016properties} for bounded losses which are especially subgaussian thanks to Hoeffding's lemma.
\begin{theorem}[Adapted from  \citealp{alquier2016properties}, Theorem 4.1]
\label{th: naive_pac_bayes-chap3}
Let $m>0$,$\S_m=(\z_1,...,\z_m)$ be an \iid sample from the same law $\mu$.
For any data-free prior $\P$, for any loss function $\ell$ bounded by $K$, any $\lambda>0,\delta\in ]0;1[$, one has with probability $1-\delta$ for any posterior $Q\in\mathcal{M}_1(\mathcal{H})$
\[ \mathbb{E}_{h\sim \Q}[\Risk(h)] \leq  \mathbb{E}_{h\sim \Q}[\Riskhat_{\Sm}(h)] + \frac{\operatorname{KL}(\Q, \P) + \log(1/\delta)}{\lambda} + \frac{\lambda K^2}{2m}. \]
\end{theorem}

\begin{theorem}[\citealp{haddouche2021pac}, Theorem 3]
\label{th: haddouche_thm}
Let the loss $\ell$ be $\mathrm{HYPE}(K)$ compliant. For any $\P\in\mathcal{M}(\mathcal{H})$ with no data dependency, for any $\alpha\in\mathbb{R}$ and for any $\delta\in[0,1]$, we have with probability at least $1-\delta$ over size-$m$ samples S,
for any $\Q$% such that $Q \ll P$% and $P \ll Q$
\begin{align*}
\mathbb{E}_{h\sim \Q}\left[ \Risk(h)\right]
\leq \mathbb{E}_{h\sim \Q}\left[ \Riskhat_{\Sm}(h)\right] + \frac{\operatorname{KL}(\Q,\P) + \log\left(\frac{1}{\delta}\right)}{m^{\alpha}}
+\frac{1}{m^{\alpha}}\log\left(\mathbb{E}_{h\sim \P} \left[\exp\left( \frac{K(h)^2}{2m^{1-2\alpha}} \right) \right]\right).
\end{align*}
\end{theorem}


\section{Extensions of previous results}
\label{sec: extensions}

Here we gather several corollaries of our main result in order to show how our \Cref{th: main_thm} extends the validity of some classical results in the literature. More precisely we show that our result extends (up to numerical factors) the PAC-Bayes Bernstein inequality of \citet{seldin2012bandit}.
Then, going back to the bounded case, we generalise a result from \citet{catoni2007pac} reformulated in \citet{alquier2016properties} and we also show how our work strictly improves on the bound of \citet{haddouche2021pac}.

\subsection{Extension of the PAC-Bayes Bernstein inequality}
\label{subsec: comparison_seldin}

Here we rename two terms for consistency with Theorem 2.1 of \citet{seldin2012bandit} (see \Cref{th: seldin_thm_mart}). For a martingale $M_m(h)= \sum_{i=1}^m X_i(\S_i,h)$, we define, at time $m$, \emph{empirical cumulative variance } to be $\hat{V}_m(h)= [M]_m(h) = \sum_{i=1}^m X_i(\S_i,h)^2$ and
the \emph{cumulative variance} as $V_m(h)= \langle M\rangle_m(h) = \sum_{i=1}^m \mathbb{E}_{i-1}[X_i(\S_i,h)^2]$.

We provide below a corollary containing two bounds: the first one being a straightforward corollary of \cref{th: main_thm}, the second being valid for bounded martingales and formally close to Theorem 2.1 of \citet{seldin2012bandit}.
\begin{corollary}
\label{cor: bound_mart}
Let $\left\{\P_1, \P_2, \ldots\right\}$ be a sequence of data-free prior distributions over $\mathcal{H}$. Let $(\lambda_i)_{i\geq 1}$ be a sequence of positive numbers.
Then the following holds with probability $1-\delta$ over $\S=(\z_i)_{i\geq 1}$: for any tuple $(m,\lambda_k,\P_k)$ with $m,k\geq 1$, any posterior $\Q$ over $\mathcal{H}$,
\begin{align}
\label{eq: bound_mart_1}
\left|M_m\left(\Q\right)\right| \leq \frac{\KL\left(\Q, \P_k\right)+2 \log (k+1)+\log (2/\delta)}{\lambda_k}+ \frac{\lambda_k}{2}\left( \hat{V}_m(\Q) + V_m(\Q) \right),
\end{align}
with $\hat{V}_m(\Q)= \mathbb{E}_{h\sim \Q}[\hat{V}_m(h)], V_m(\Q)= \mathbb{E}_{h\sim \Q}[V_m(h)]$.
Furthermore, if we assume that for any $i$, there exists $C_i>0$ such that $|X_i(\S_i,h)|\leq C_i$ for all $\S_i,h$ then we have the following corollary: with probability $1-\delta$ over $S$, for any tuple $(m,\lambda_m,\P_m)$ $m\geq 1$, any posterior $\Q$,
\begin{align}
\label{eq: bound_mart_2}
\left|M_m\left(Q\right)\right| \leq \frac{\KL\left(\Q, \P_m\right)+2 \log (m+1)+\log (2/\delta)}{\lambda_m}+ \lambda_m\sum_{i=1}^m C_i^2.
\end{align}
\end{corollary}
The proof is deferred to \cref{sec: proofs}.
Note that \cref{eq: bound_mart_1} holds uniformly on all tuples $\{(\lambda_k,\P_k,m) \mid k\geq 1, m\geq 1\}$ while \cref{eq: bound_mart_2}, as well as Theorem 2.1 of \citet{seldin2012bandit} holds uniformly on the tuples
$\{(\lambda_m,\P_m,m) \mid m\geq 1\}$ which is a strictly smaller collection. Hence our approach gives guarantees for a larger event with the same confidence level.

Furthermore, Theorem 2.1 of \citet{seldin2012bandit} involves the cumulative variance $V_m(\Q)$ (and not its empirical counterpart). Because this term is theoretical, we bound it in \cref{th: seldin_thm_mart} by $\sum_{i=1}^m C_i^2$ which is supposedly empirical.
In this context, \cref{eq: bound_mart_2}, recovers nearly exactly the bound of \cite{seldin2012bandit} with the transformation of a factor $(e-2)$ into $1$.
Notice also that \cref{eq: bound_mart_2} stands with no assumption on the range of the $\lambda_i$, which is not the case in \cref{th: seldin_thm_mart}.

Finally, we stress two fundamental differences between our work and the one of \citet{seldin2012bandit}. First, we replace Markov's inequality by Ville's inequality; second, we exploited the exponential inequality of Lemma \ref{l: bercu_touati} instead of the Bernstein inequality. These allow for results for unbounded martingales for all $m$ simultaneously.


\subsection{Extensions of learning theory results}


\subsubsection{A general result for bounded losses}


We use definitions from \Cref{sec: iid_case} and provide a corollary of our main result when the loss is bounded by a positive constant $K>0$. We assume our data are iid.
\begin{corollary}
\label{cor: bounded_case}
For any data-free prior $P\in \mathcal{M}(\mathcal{H})$, any $\lambda>0$ the following holds with probability $1-\delta$ over the sample $S=(z_i)_{i\in\mathbb{N}}$, for all $m\in\mathbb{N}/\{0\}$, $\Q\in\mathcal{M}(\mathcal{H})$
\[ \left | \mathbb{E}_{h\sim \Q} [\Risk(h)] -  \mathbb{E}_{h\sim \Q} \left[\Riskhat_{\S_m}(h) \right] \right |  \leq \frac{\operatorname{KL}(\Q,\P) +\log(2/\delta)}{\lambda m } + \lambda K^2.  \]
We also have the local bound: for any $m\geq 1$, with probability $1-\delta$ over $S$, for all $\Q\in\mathcal{M}(\mathcal{H})$
\[ \mathbb{E}_{h\sim \Q} [\Risk(h)] \leq  \mathbb{E}_{h\sim \Q} \left[\Riskhat_{\S_m}(h) \right] + \frac{\operatorname{KL}(\Q,\P) +\log(2/\delta)}{\lambda} + \frac{\lambda K^2}{m}.  \]
\end{corollary}
The proof is deferred to \cref{sec: proofs}. Remark that the second bound of Corollary \ref{cor: bounded_case} is exactly the Catoni bound stated in \citet{alquier2016properties} (see \Cref{th: naive_pac_bayes-chap3} in \Cref{sec: pac_b_background}) up to a numerical factor of $2$.


The first bound is, to our knowledge, the first PAC-Bayesian bound for bounded losses holding uniformly (for a given parameter $\lambda$) on the choice of $Q,m$ and thus extends the scope of Catoni's bound which holds for a single $m$ with high probability.  Indeed, if we want for instance \Cref{th: naive_pac_bayes-chap3} to hold for any $i\in\{1..m\}$, we then have to take an union bound on $m$ events which turns the term $\log(1/\delta)$ into $\log(m/\delta)$ (but with the benefit of holding for $m$ parameters $\lambda_1,...,\lambda_m$). This point is common to the most classical PAC-Bayesian bounds
(including McAllester and Catoni's ones \eqref{eq: mcallester}, \eqref{eq: catoni})
and impeach us to have a bound uniformly on all $m\in\mathbb{N}/\{0\}$ as $\log(m)$ goes to infinity asymptotically.


\subsubsection{An extension of \citet{haddouche2021pac}}

We now focus on the work of \citet{haddouche2021pac} which provides general PAC-Bayesian bounds for unbounded losses. Their theorems hold for iid data and under the so-called \emph{HYPE} (for HYPothesis-dependent rangE) condition. It states that a loss function $\ell$ is \emph{HYPE}$(K)$ compliant if there exists a function $K:\mathcal{H} \rightarrow \mathbb{R}^+ $ (supposedly accessible)  such that $\forall z\in\mathcal{Z}, \ell(h,\z) \leq K(h)$.
We provide \Cref{cor: haddouche_comparison} to compare ourselves with their main result (stated in  \Cref{th: haddouche_thm} for convenience).
\begin{corollary}
\label{cor: haddouche_comparison}
For any data-free prior $\P\in \mathcal{M}(\mathcal{H})$, any loss function $\ell$ being \emph{HYPE}$(K)$ compliant, any $\alpha\in[0,1],m\geq 1$, the following holds with probability $1-\delta$ over the sample $\S=(\z_i)_{i\in\mathbb{N}}$, for all $\Q\in\mathcal{M}(\mathcal{H})$
\begin{multline*}
\mathbb{E}_{h\sim \Q} [\Risk(h)] \leq   \mathbb{E}_{h\sim \Q} \left[\frac{1}{m}\sum_{i=1}^m\left(\ell(h,\z_i) + \frac{1}{2m^{1-\alpha}} \ell(h,\z_i)^2\right)\right] \\
+ \frac{\operatorname{KL}(\Q,\P) +\log(1/\delta)}{m^{\alpha}}  + \frac{1}{2m^{1-\alpha}}\mathbb{E}_{h\sim \Q} [K^2(h)].
\end{multline*}
\end{corollary}

\begin{proof}
The proof is a straightforward application of \cref{th: main_thm_iid} by fixing $m\geq 1$ choosing $\lambda= m^{\alpha-1}$ (thus we localise \Cref{th: main_thm_iid} to a single $m$),  and bounding $\mathrm{Quad}(h)$ by $K^2(h)$.
\end{proof}
The main improvement of our bound over \Cref{th: haddouche_thm} is that we do not have to assume the convergence of an exponential moment to obtain a non-trivial bound. Indeed, we transformed the (implicit) assumption $\mathbb{E}_{h\sim \P} \left[\exp\left( \frac{K(h)^2}{2m^{1-2\alpha}} \right) \right] < +\infty $ onto $\mathbb{E}_{h\sim \Q}[K(h)^2] < +\infty$, which is significantly less restrictive.
Furthermore, \Cref{th: haddouche_thm} holds for a single choice of $m$ while ours still holds uniformly over all integers $m>0$.

Cor. \ref{cor: haddouche_comparison} also sheds new light on the \emph{HYPE} condition. Indeed, in \citet{haddouche2021pac}, $K$ only intervenes in an exponential moment involving the prior $\P$, while ours considers a second-order moment on $K$ implying the posterior $\Q$. The difference is major as $\mathbb{E}_{h\sim \Q}[K(h)^2] $ can be controlled by a wise choice of posterior. Thus it can be incorporated in our optimisation route, acting now as an optimisation constraint instead of an environment constraint.



\section{Proofs}
\label{sec: proofs}

\subsection{Proof of \cref{th: main_thm_iid}}

\begin{proof}
Let $\P$ a fixed data-free prior, set $(\mathcal{F}_i)_{i\geq 0}$ such that for all $i$, $\z_i$ is $\mathcal{F}_i$ measurable. We also set for any fixed $h\in\mathcal{H}, M_m(h):= \sum_{i=1}^m \ell(h,\z_i) - \Risk(h)$. Note that because data are \iid, for any fixed $h$, the sequence $(M_m(h))_m$ is indeed a martingale.
We set for any $m\geq 1, h\in\mathcal{H}$
$$[M]_m(h) = \sum_{i=1}^m \left(\ell(h,\z_i) - \Risk(h)\right)^2 $$ and
$$\langle M \rangle_m(h) =  \sum_{i=1}^m \mathbb{E}_{i-1}[\left(\ell(h,\z_i) - \Risk(h)\right)^2] = \sum_{i=1}^m \mathbb{E}_{\z\sim \D}[\left(\ell(h,\z) - \Risk(h)\right)^2].$$
The last equality holds because data is assumed iid. Thus, we can apply \cref{th: main_thm} to obtain with probability $1-\delta$
\[|M_m(\Q)| \leq   \frac{\operatorname{KL}(\Q,\P) +\log(2/\delta)}{\lambda } + \frac{\lambda}{2}\left([M]_m(Q)^2 + \langle M\rangle_m(Q)^2 \right) . \]
Now, we notice that $|M_m(\Q)| = m| \mathbb{E}_{h\sim \Q}[\Risk(h) - \Riskhat_{\Sm}(h)] |$  and that  for any $m,h$, because $\ell$ is nonnegative
\begin{align*}
[M]_m(h) +  \langle M\rangle_m(h) & = \sum_{i=1}^m (\ell(h,\z_i) - \Risk(h))^2 + \mathbb{E}_{\z\sim \D}[ (\ell(h,\z) - \Risk(h))^2] \\
& \leq  \sum_{i=1}^m \ell(h,\z_i)^2 + \Risk(h)^2 + \mathbb{E}_{\z\sim \D}[\ell(h,\z)^2] - \Risk(h)^2.\\
\intertext{ Thus integrating over $h$ gives: }
[M]_m(Q) +  \langle M\rangle_m(Q) & \leq \sum_{i=1}^m \mathbb{E}_{h\sim \Q} [\ell(h,\z_i)^2] + m\mathbb{E}_{h\sim \Q} [\mathrm{Quad}(h)].
\end{align*}
Then dividing by $m$ and applying the last inequality gives
\begin{multline*}
\mathbb{E}_{h\sim \Q} [\Risk(h)]  \leq  \mathbb{E}_{h\sim \Q} \left[\frac{1}{m}\sum_{i=1}^m\left(\ell(h,\z_i) + \frac{\lambda}{2} \ell(h,\z_i)^2\right)\right] \\
+ \frac{\operatorname{KL}(\Q,\P) +\log(2/\delta)}{\lambda m} + \frac{\lambda}{2}\mathbb{E}_{h\sim \Q} [\mathrm{Quad}(h)].
\end{multline*}
This concludes the proof.
\end{proof}






\subsection{Proof of \cref{th: bandits_bound}}

\begin{proof}
Let $(\lambda_m)_{i\geq 1}$ be a countable sequence of positive scalars.
As precised earlier $M_m(a):= m\left(\hat{\Delta}_m(a)-\Delta(a)\right)$ is a martingale.
We then apply \Cref{th: main_thm} with the uniform prior ($\forall a, P(a)= \frac{1}{K}$) and $\lambda= \lambda_m$  (depending possibly on $m$): with probability $1- \delta/2$, for any tuple $(m,\lambda_m)$ with $m\geq 1$, any posterior $\Q$,
\begin{align*}
\left|M_m\left(\Q\right)\right| \leq \frac{\operatorname{KL}\left(\Q, \P\right)+2 +\log (4/\delta)}{\lambda_m}+ \frac{\lambda_m}{2}\left( \hat{V}_m(\Q) + V_m(\Q) \right).
\end{align*}
Notice that for any $\Q$, $\operatorname{KL}(\Q,\P)\leq \log(K)$ by concavity of the log.
We now fix an horizon $M>0$, we then have in particular, with probability $1- \delta/2$: for any posterior $\Q$,
\begin{align*}
\left|M_m\left(Q\right)\right| \leq \frac{\log(K)+2 \log (k+1)+\log (4/\delta)}{\lambda_k}+ \frac{\lambda_m}{2}\left( \hat{V}_m(\Q) + V_m(\Q) \right).
\end{align*}
We now have to deal with $V_k(\Q), \hat{V}_k(\Q)$ for all $k\leq m$. To do so, we propose the two following lemmas.
\begin{lemma}
\label{l: bandit_lemma_1}
For all $m\geq 1$, $a\in\mathcal{A}$, $V_m(a)\leq \frac{2Cm}{\varepsilon_m}$.
Then, we have for any $m,Q$, $V_m(\Q)\leq \frac{2Cm}{\varepsilon_m}$.
\end{lemma}

\begin{proof} We have
\begin{align*}
V_t(a) &=\sum_{i=1}^m \mathbb{E}\left[\left(\left[R_i^{a^*}-R_i^a\right]-\Delta(a)\right)^2 \mid \mathcal{F}_{i-1}\right] \\
&=\sum_{i=1}^m \mathbb{E}\left[\left(R_i^{a^*}-R_i^a\right)^2 \mid \mathcal{F}_{i-1}\right]-m \Delta(a)^2\\
& \leq \sum_{i=1}^m \mathbb{E}\left[\left(R_i^{a^*}-R_i^a\right)^2 \mid \mathcal{F}_{i-1}\right]  \\
&  = \sum_{i=1}^m \mathbb{E}\left[\mathbb{E}_{A_i\sim \pi_i}\mathbb{E}_{R_i}\left[\frac{1}{\pi_i(a^*)^2} R_i(a^*)^2\mathds{1}(A_i=a^*) +\frac{1}{\pi_i(a)^2}R_i(a)^2\mathds{1}(A_i=a) \right] \mid \mathcal{F}_{i-1}\right].\\
\intertext{The last line holding because $R_i$ is independent of $\mathcal{F}_{i-1}$, $A_i$ is independent of $R_i$ and $\pi$ is $\mathcal{F}_{i-1}$ measurable.
We now use that for all $i,a$, $\mathbb{E}_{R_i}[R_i(a)^2] \leq C$ }
& = \sum_{i=1}^m \mathbb{E}\left[\mathbb{E}_{A_i\sim \pi_i}
\left[\frac{1}{\pi_i(a^*)^2} C\mathds{1}(A_i=a^*) +\frac{1}{\pi_i(a)^2}C\mathds{1}(A_i=a) \right] \mid \mathcal{F}_{i-1}\right]\\
& =\sum_{i=1}^m C\left(\frac{\pi_i(a)}{\pi_i(a)^2}+\frac{\pi_i\left(a^*\right)}{\pi_i\left(a^*\right)^2}\right) \\ &=\sum_{i=1}^m C\left(\frac{1}{\pi_i(a)}+\frac{1}{\pi_i\left(a^*\right)} \right) \\
& \leq \frac{2C m}{\varepsilon_m}.
\end{align*}
\end{proof}


\begin{lemma}
\label{l: bandit_lemma_2}
Let $m\geq 1$, with probability $1-\delta/2$, for any posterior $\Q$, we have
\[ \hat{V}_m(\Q) \leq \frac{4CKm}{\varepsilon_m\delta}. \]
\end{lemma}

\begin{proof}
Let $\Q$ a distribution over $\mathcal{A}$. Recall that
\begin{align*}
\hat{V}_m(\Q) & = \sum_{i=1}^m \left(R_i^{a^*}-R_i^a-\left[R\left(a^*\right)-R(a)\right]\right)^2 \\
& = \sum_{a\in\mathcal{A}} Q(a) \hat{V}_m(a).
\end{align*}
Notice that for any $a$, $(\hat{SM}_m^a)_m$ is a  nonnegative random variable. We then apply Markov's inequality for any $a$, with probability $1-\delta/2K$
\[ \hat{V}_m(a)\leq  \frac{2K\mathbb{E}[\hat{V}_m(a)]}{\delta} .  \]
Noticing that $\mathbb{E}[\hat{V}_m(a)] = \mathbb{E}[V_m(a)]$, we can apply \cref{l: bandit_lemma_1} to conclude that
$$\mathbb{E}[\hat{V}_m(a)] \leq \frac{2Cm}{\varepsilon_m}.$$
Finally, taking an union bound on thoser events for all $a\in\mathcal{A}$ gives us, with probability $1-\delta/2$, for any posterior $\Q$
\begin{align*}
V_m(\Q) & \leq \sum_{a\in\mathcal{A}} Q(a) \hat{V}_m(a) \\
& \leq \sum_{a\in\mathcal{A}} Q(a) \frac{4CKm}{\varepsilon_m\delta} \\
&= \frac{4CKm}{\varepsilon_m\delta}.
\end{align*}
This concludes the proof.
\end{proof}
To conclude, we apply \cref{l: bandit_lemma_1,l: bandit_lemma_2} to get that with probability $1-\delta$, for any posterior $\Q$
\begin{align*}
\left|M_m\left(\Q\right)\right| & \leq \frac{\operatorname{KL}\left(\Q, \P\right) +\log (4/\delta)}{\lambda_m}+ \frac{Cm\lambda_m}{\varepsilon_m} \left(1+ \frac{2K}{\delta}    \right).
\end{align*}
Dividing by $m$ and taking $$\lambda_m= \sqrt{\frac{\left(\log(K) +\log (4/\delta)\right) \varepsilon_m}{Cm\left(1+ \frac{2K}{\delta}    \right)}}$$ concludes the proof.

\end{proof}





\subsection{Proof of Cor. \ref{cor: bound_mart}}


\begin{proof}
Fix $\delta>0$. For any pair $(\lambda_k,P_k), k\geq 1$, we apply \Cref{th: main_thm} with $$\delta_k := \frac{\delta}{k(k+1)} \geq \frac{\delta}{(k+1)^2}. $$
Notice that we have $\sum_{k=1}^{+\infty} \delta_k = \delta$.
We then have with probability $1-\delta_k$ over $S$, for any $m\geq 1$, any posterior $\Q$,
\[ \left|M_m\left(\Q\right)\right| \leq \frac{\KL\left(\Q, \P_k\right)+2 \log (k+1)+\log (2/\delta)}{\lambda_k}+ \frac{\lambda_k}{2}\left( \hat{V}_m(\Q) + V_m(\Q) \right).\]
Taking an union bound on all those event, gives the final result, valid with probability $1-\delta$ over the sample $S$, for any any tuple $(m,\lambda_k,P_k)$ with $m,k\geq 1$, any posterior $\Q$ over $\mathcal{H}$. This gives \Cref{eq: bound_mart_1}.

To obtain \cref{eq: bound_mart_2}, we restrict the range of \cref{eq: bound_mart_1} to the tuples $(m,\lambda_m,P_m), m\geq 1$ (the restricted set of tuples where $k=m$) and we bound both $\hat{V}_m(\Q),V_m(\Q)$ by $\sum_{i=1}^m C_i^2$ to conclude.
\end{proof}

\subsection{Proof of Cor. \ref{cor: bounded_case}}


\begin{proof}
For the first bound we start from the intermediary result \cref{eq: intermediary_result_main} of  \cref{th: main_thm}. Using the same marrtingale as in \cref{th: main_thm_iid} gives, for any $\eta\in\mathbb{R}$, holding with probability $1-\delta$ for any $m>0,Q\in\mathcal{M}(\mathcal{H})$
\begin{multline*}
\eta \left(\sum_{i=1}^m \mathbb{E}_{h\sim \Q}[\ell(h,\z_i)]  -m \mathbb{E}_{h\sim \Q}[ \Risk(h) ] \right) \\
\leq  \operatorname{KL}(\Q,\P) +\log(1/\delta) + \frac{\eta^2}{2}\sum_{i=1}^m \mathbb{E}_{h\sim \Q}[\Delta[M]_i(h) + \Delta \langle M\rangle_i(h) ].
\end{multline*}
Taking $\eta= \pm\lambda$ with $\lambda>0$ gives
\begin{align}
\label{eq: temp_result}
\lambda m\left | \mathbb{E}_{h\sim \Q}[ \Risk(h) -\Riskhat_{\Sm}(h)] \right | & \leq \operatorname{KL}(\Q,\P) +\log(1/\delta) \\
& + \frac{\lambda^2}{2}\sum_{i=1}^m \mathbb{E}_{h\sim \Q}[\Delta[M]_i(h) + \Delta \langle M\rangle_i(h) ].
\end{align}
Finally, divide by $\lambda m$ and bound $\Delta[M]_i(h) + \Delta \langle M\rangle_i(h)$ by $2K^2$ to conclude.


For the second bound, we start from \Cref{eq: temp_result} again and for a fixed $m$, we now apply our result with $\lambda'=\lambda/m$. We then have for any $m$, with probability $1-\delta$, for any $\Q$
\[ \lambda\left |  \mathbb{E}_{h\sim \Q}[ \Risk(h) -\Riskhat_{\Sm}(h)] \right |  \leq \operatorname{KL}(\Q,\P) +\log(1/\delta) + \frac{\lambda^2}{2m^2}\sum_{i=1}^m \mathbb{E}_{h\sim \Q}[\Delta[M]_i(h) + \Delta \langle M\rangle_i(h) ].\]
Finally, dividing by $\lambda$, bounding $\Delta[M]_i(h) + \Delta \langle M\rangle_i(h)$ by $2K^2$ and rearranging the terms concludes the proof.
\end{proof}


\end{noaddcontents}
%%!TEX root = main.tex
\chapter{Appendix of Chapter~\ref{chap: pb-ht}}
\label{ap: pb-ht}

\begin{noaddcontents}
    

\section{Some PAC-Bayesian background}
\label{sec: pac_b_background}

We present below an immediate corollary of \citet[Thm 2.1]{seldin2012bandit} where we upper bounded the cumulative by an empirical quantity (the sum of squared upper bound of the martingale difference sequence).

\begin{theorem}[\citealp{seldin2012bandit}, Theorem 2.1]
\label{th: seldin_thm_mart}
Let $\left\{C_1, C_2, \ldots\right\}$ be an increasing sequence set in advance, such that $\left|X_i(\\S_i,h)\right| \leq C_i$ for all $\\S_i,h$ with probability 1.   Let $\left\{P_1, P_2, \ldots\right\}$ be a sequence of data-free prior distributions over $\mathcal{H}$. Let $(\lambda_i)_{i\geq 1}$ be a sequence of positive numbers such that
$$
\lambda_m \leq \frac{1}{C_m}.
$$
Then with probability $1-\delta$ over $\S=(\z_i)_{i\geq 1}$,
for all $m\geq 1$, any posterior $\Q$ over $\mathcal{H}$,
$$
\left|M_m\left(Q\right)\right| \leq \frac{\KL\left(\Q , \P_m\right)+2 \log (m+1)+\log \frac{2}{\delta}}{\lambda_m}+(e-2) \lambda_m V_m(\Q),
$$
where $V_m(\Q)$ is defined in \cref{subsec: comparison_seldin}.

Furthermore, if we bound the variance term, we would have:
$$
\left|M_m\left(\Q\right)\right| \leq \frac{\KL\left(\Q , \P_m\right)+2 \log (m+1)+\log \frac{2}{\delta}}{\lambda_m}+(e-2) \lambda_m \sum_{i=1}^m C_i^2.
$$
\end{theorem}
Below, we use the definitions introduced in \Cref{sec: iid_case}.
We study here a particular case of \cite{alquier2016properties} for bounded losses which are especially subgaussian thanks to Hoeffding's lemma.
\begin{theorem}[Adapted from  \citealp{alquier2016properties}, Theorem 4.1]
\label{th: naive_pac_bayes-chap3}
Let $m>0$,$\S_m=(\z_1,...,\z_m)$ be an \iid sample from the same law $\mu$.
For any data-free prior $\P$, for any loss function $\ell$ bounded by $K$, any $\lambda>0,\delta\in ]0;1[$, one has with probability $1-\delta$ for any posterior $Q\in\mathcal{M}_1(\mathcal{H})$
\[ \mathbb{E}_{h\sim \Q}[\Risk(h)] \leq  \mathbb{E}_{h\sim \Q}[\Riskhat_{\Sm}(h)] + \frac{\operatorname{KL}(\Q, \P) + \log(1/\delta)}{\lambda} + \frac{\lambda K^2}{2m}. \]
\end{theorem}

\begin{theorem}[\citealp{haddouche2021pac}, Theorem 3]
\label{th: haddouche_thm}
Let the loss $\ell$ be $\mathrm{HYPE}(K)$ compliant. For any $\P\in\mathcal{M}(\mathcal{H})$ with no data dependency, for any $\alpha\in\mathbb{R}$ and for any $\delta\in[0,1]$, we have with probability at least $1-\delta$ over size-$m$ samples S,
for any $\Q$% such that $Q \ll P$% and $P \ll Q$
\begin{align*}
\mathbb{E}_{h\sim \Q}\left[ \Risk(h)\right]
\leq \mathbb{E}_{h\sim \Q}\left[ \Riskhat_{\Sm}(h)\right] + \frac{\operatorname{KL}(\Q,\P) + \log\left(\frac{1}{\delta}\right)}{m^{\alpha}}
+\frac{1}{m^{\alpha}}\log\left(\mathbb{E}_{h\sim \P} \left[\exp\left( \frac{K(h)^2}{2m^{1-2\alpha}} \right) \right]\right).
\end{align*}
\end{theorem}


\section{Extensions of previous results}
\label{sec: extensions}

Here we gather several corollaries of our main result in order to show how our \Cref{th: main_thm} extends the validity of some classical results in the literature. More precisely we show that our result extends (up to numerical factors) the PAC-Bayes Bernstein inequality of \citet{seldin2012bandit}.
Then, going back to the bounded case, we generalise a result from \citet{catoni2007pac} reformulated in \citet{alquier2016properties} and we also show how our work strictly improves on the bound of \citet{haddouche2021pac}.

\subsection{Extension of the PAC-Bayes Bernstein inequality}
\label{subsec: comparison_seldin}

Here we rename two terms for consistency with Theorem 2.1 of \citet{seldin2012bandit} (see \Cref{th: seldin_thm_mart}). For a martingale $M_m(h)= \sum_{i=1}^m X_i(\S_i,h)$, we define, at time $m$, \emph{empirical cumulative variance } to be $\hat{V}_m(h)= [M]_m(h) = \sum_{i=1}^m X_i(\S_i,h)^2$ and
the \emph{cumulative variance} as $V_m(h)= \langle M\rangle_m(h) = \sum_{i=1}^m \mathbb{E}_{i-1}[X_i(\S_i,h)^2]$.

We provide below a corollary containing two bounds: the first one being a straightforward corollary of \cref{th: main_thm}, the second being valid for bounded martingales and formally close to Theorem 2.1 of \citet{seldin2012bandit}.
\begin{corollary}
\label{cor: bound_mart}
Let $\left\{\P_1, \P_2, \ldots\right\}$ be a sequence of data-free prior distributions over $\mathcal{H}$. Let $(\lambda_i)_{i\geq 1}$ be a sequence of positive numbers.
Then the following holds with probability $1-\delta$ over $\S=(\z_i)_{i\geq 1}$: for any tuple $(m,\lambda_k,\P_k)$ with $m,k\geq 1$, any posterior $\Q$ over $\mathcal{H}$,
\begin{align}
\label{eq: bound_mart_1}
\left|M_m\left(\Q\right)\right| \leq \frac{\KL\left(\Q, \P_k\right)+2 \log (k+1)+\log (2/\delta)}{\lambda_k}+ \frac{\lambda_k}{2}\left( \hat{V}_m(\Q) + V_m(\Q) \right),
\end{align}
with $\hat{V}_m(\Q)= \mathbb{E}_{h\sim \Q}[\hat{V}_m(h)], V_m(\Q)= \mathbb{E}_{h\sim \Q}[V_m(h)]$.
Furthermore, if we assume that for any $i$, there exists $C_i>0$ such that $|X_i(\S_i,h)|\leq C_i$ for all $\S_i,h$ then we have the following corollary: with probability $1-\delta$ over $S$, for any tuple $(m,\lambda_m,\P_m)$ $m\geq 1$, any posterior $\Q$,
\begin{align}
\label{eq: bound_mart_2}
\left|M_m\left(Q\right)\right| \leq \frac{\KL\left(\Q, \P_m\right)+2 \log (m+1)+\log (2/\delta)}{\lambda_m}+ \lambda_m\sum_{i=1}^m C_i^2.
\end{align}
\end{corollary}
The proof is deferred to \cref{sec: proofs}.
Note that \cref{eq: bound_mart_1} holds uniformly on all tuples $\{(\lambda_k,\P_k,m) \mid k\geq 1, m\geq 1\}$ while \cref{eq: bound_mart_2}, as well as Theorem 2.1 of \citet{seldin2012bandit} holds uniformly on the tuples
$\{(\lambda_m,\P_m,m) \mid m\geq 1\}$ which is a strictly smaller collection. Hence our approach gives guarantees for a larger event with the same confidence level.

Furthermore, Theorem 2.1 of \citet{seldin2012bandit} involves the cumulative variance $V_m(\Q)$ (and not its empirical counterpart). Because this term is theoretical, we bound it in \cref{th: seldin_thm_mart} by $\sum_{i=1}^m C_i^2$ which is supposedly empirical.
In this context, \cref{eq: bound_mart_2}, recovers nearly exactly the bound of \cite{seldin2012bandit} with the transformation of a factor $(e-2)$ into $1$.
Notice also that \cref{eq: bound_mart_2} stands with no assumption on the range of the $\lambda_i$, which is not the case in \cref{th: seldin_thm_mart}.

Finally, we stress two fundamental differences between our work and the one of \citet{seldin2012bandit}. First, we replace Markov's inequality by Ville's inequality; second, we exploited the exponential inequality of Lemma \ref{l: bercu_touati} instead of the Bernstein inequality. These allow for results for unbounded martingales for all $m$ simultaneously.


\subsection{Extensions of learning theory results}


\subsubsection{A general result for bounded losses}


We use definitions from \Cref{sec: iid_case} and provide a corollary of our main result when the loss is bounded by a positive constant $K>0$. We assume our data are iid.
\begin{corollary}
\label{cor: bounded_case}
For any data-free prior $P\in \mathcal{M}(\mathcal{H})$, any $\lambda>0$ the following holds with probability $1-\delta$ over the sample $S=(z_i)_{i\in\mathbb{N}}$, for all $m\in\mathbb{N}/\{0\}$, $\Q\in\mathcal{M}(\mathcal{H})$
\[ \left | \mathbb{E}_{h\sim \Q} [\Risk(h)] -  \mathbb{E}_{h\sim \Q} \left[\Riskhat_{\S_m}(h) \right] \right |  \leq \frac{\operatorname{KL}(\Q,\P) +\log(2/\delta)}{\lambda m } + \lambda K^2.  \]
We also have the local bound: for any $m\geq 1$, with probability $1-\delta$ over $S$, for all $\Q\in\mathcal{M}(\mathcal{H})$
\[ \mathbb{E}_{h\sim \Q} [\Risk(h)] \leq  \mathbb{E}_{h\sim \Q} \left[\Riskhat_{\S_m}(h) \right] + \frac{\operatorname{KL}(\Q,\P) +\log(2/\delta)}{\lambda} + \frac{\lambda K^2}{m}.  \]
\end{corollary}
The proof is deferred to \cref{sec: proofs}. Remark that the second bound of Corollary \ref{cor: bounded_case} is exactly the Catoni bound stated in \citet{alquier2016properties} (see \Cref{th: naive_pac_bayes-chap3} in \Cref{sec: pac_b_background}) up to a numerical factor of $2$.


The first bound is, to our knowledge, the first PAC-Bayesian bound for bounded losses holding uniformly (for a given parameter $\lambda$) on the choice of $Q,m$ and thus extends the scope of Catoni's bound which holds for a single $m$ with high probability.  Indeed, if we want for instance \Cref{th: naive_pac_bayes-chap3} to hold for any $i\in\{1..m\}$, we then have to take an union bound on $m$ events which turns the term $\log(1/\delta)$ into $\log(m/\delta)$ (but with the benefit of holding for $m$ parameters $\lambda_1,...,\lambda_m$). This point is common to the most classical PAC-Bayesian bounds
(including McAllester and Catoni's ones \eqref{eq: mcallester}, \eqref{eq: catoni})
and impeach us to have a bound uniformly on all $m\in\mathbb{N}/\{0\}$ as $\log(m)$ goes to infinity asymptotically.


\subsubsection{An extension of \citet{haddouche2021pac}}

We now focus on the work of \citet{haddouche2021pac} which provides general PAC-Bayesian bounds for unbounded losses. Their theorems hold for iid data and under the so-called \emph{HYPE} (for HYPothesis-dependent rangE) condition. It states that a loss function $\ell$ is \emph{HYPE}$(K)$ compliant if there exists a function $K:\mathcal{H} \rightarrow \mathbb{R}^+ $ (supposedly accessible)  such that $\forall z\in\mathcal{Z}, \ell(h,\z) \leq K(h)$.
We provide \Cref{cor: haddouche_comparison} to compare ourselves with their main result (stated in  \Cref{th: haddouche_thm} for convenience).
\begin{corollary}
\label{cor: haddouche_comparison}
For any data-free prior $\P\in \mathcal{M}(\mathcal{H})$, any loss function $\ell$ being \emph{HYPE}$(K)$ compliant, any $\alpha\in[0,1],m\geq 1$, the following holds with probability $1-\delta$ over the sample $\S=(\z_i)_{i\in\mathbb{N}}$, for all $\Q\in\mathcal{M}(\mathcal{H})$
\begin{multline*}
\mathbb{E}_{h\sim \Q} [\Risk(h)] \leq   \mathbb{E}_{h\sim \Q} \left[\frac{1}{m}\sum_{i=1}^m\left(\ell(h,\z_i) + \frac{1}{2m^{1-\alpha}} \ell(h,\z_i)^2\right)\right] \\
+ \frac{\operatorname{KL}(\Q,\P) +\log(1/\delta)}{m^{\alpha}}  + \frac{1}{2m^{1-\alpha}}\mathbb{E}_{h\sim \Q} [K^2(h)].
\end{multline*}
\end{corollary}

\begin{proof}
The proof is a straightforward application of \cref{th: main_thm_iid} by fixing $m\geq 1$ choosing $\lambda= m^{\alpha-1}$ (thus we localise \Cref{th: main_thm_iid} to a single $m$),  and bounding $\mathrm{Quad}(h)$ by $K^2(h)$.
\end{proof}
The main improvement of our bound over \Cref{th: haddouche_thm} is that we do not have to assume the convergence of an exponential moment to obtain a non-trivial bound. Indeed, we transformed the (implicit) assumption $\mathbb{E}_{h\sim \P} \left[\exp\left( \frac{K(h)^2}{2m^{1-2\alpha}} \right) \right] < +\infty $ onto $\mathbb{E}_{h\sim \Q}[K(h)^2] < +\infty$, which is significantly less restrictive.
Furthermore, \Cref{th: haddouche_thm} holds for a single choice of $m$ while ours still holds uniformly over all integers $m>0$.

Cor. \ref{cor: haddouche_comparison} also sheds new light on the \emph{HYPE} condition. Indeed, in \citet{haddouche2021pac}, $K$ only intervenes in an exponential moment involving the prior $\P$, while ours considers a second-order moment on $K$ implying the posterior $\Q$. The difference is major as $\mathbb{E}_{h\sim \Q}[K(h)^2] $ can be controlled by a wise choice of posterior. Thus it can be incorporated in our optimisation route, acting now as an optimisation constraint instead of an environment constraint.



\section{Proofs}
\label{sec: proofs}

\subsection{Proof of \cref{th: main_thm_iid}}

\begin{proof}
Let $\P$ a fixed data-free prior, set $(\mathcal{F}_i)_{i\geq 0}$ such that for all $i$, $\z_i$ is $\mathcal{F}_i$ measurable. We also set for any fixed $h\in\mathcal{H}, M_m(h):= \sum_{i=1}^m \ell(h,\z_i) - \Risk(h)$. Note that because data are \iid, for any fixed $h$, the sequence $(M_m(h))_m$ is indeed a martingale.
We set for any $m\geq 1, h\in\mathcal{H}$
$$[M]_m(h) = \sum_{i=1}^m \left(\ell(h,\z_i) - \Risk(h)\right)^2 $$ and
$$\langle M \rangle_m(h) =  \sum_{i=1}^m \mathbb{E}_{i-1}[\left(\ell(h,\z_i) - \Risk(h)\right)^2] = \sum_{i=1}^m \mathbb{E}_{\z\sim \D}[\left(\ell(h,\z) - \Risk(h)\right)^2].$$
The last equality holds because data is assumed iid. Thus, we can apply \cref{th: main_thm} to obtain with probability $1-\delta$
\[|M_m(\Q)| \leq   \frac{\operatorname{KL}(\Q,\P) +\log(2/\delta)}{\lambda } + \frac{\lambda}{2}\left([M]_m(Q)^2 + \langle M\rangle_m(Q)^2 \right) . \]
Now, we notice that $|M_m(\Q)| = m| \mathbb{E}_{h\sim \Q}[\Risk(h) - \Riskhat_{\Sm}(h)] |$  and that  for any $m,h$, because $\ell$ is nonnegative
\begin{align*}
[M]_m(h) +  \langle M\rangle_m(h) & = \sum_{i=1}^m (\ell(h,\z_i) - \Risk(h))^2 + \mathbb{E}_{\z\sim \D}[ (\ell(h,\z) - \Risk(h))^2] \\
& \leq  \sum_{i=1}^m \ell(h,\z_i)^2 + \Risk(h)^2 + \mathbb{E}_{\z\sim \D}[\ell(h,\z)^2] - \Risk(h)^2.\\
\intertext{ Thus integrating over $h$ gives: }
[M]_m(Q) +  \langle M\rangle_m(Q) & \leq \sum_{i=1}^m \mathbb{E}_{h\sim \Q} [\ell(h,\z_i)^2] + m\mathbb{E}_{h\sim \Q} [\mathrm{Quad}(h)].
\end{align*}
Then dividing by $m$ and applying the last inequality gives
\begin{multline*}
\mathbb{E}_{h\sim \Q} [\Risk(h)]  \leq  \mathbb{E}_{h\sim \Q} \left[\frac{1}{m}\sum_{i=1}^m\left(\ell(h,\z_i) + \frac{\lambda}{2} \ell(h,\z_i)^2\right)\right] \\
+ \frac{\operatorname{KL}(\Q,\P) +\log(2/\delta)}{\lambda m} + \frac{\lambda}{2}\mathbb{E}_{h\sim \Q} [\mathrm{Quad}(h)].
\end{multline*}
This concludes the proof.
\end{proof}






\subsection{Proof of \cref{th: bandits_bound}}

\begin{proof}
Let $(\lambda_m)_{i\geq 1}$ be a countable sequence of positive scalars.
As precised earlier $M_m(a):= m\left(\hat{\Delta}_m(a)-\Delta(a)\right)$ is a martingale.
We then apply \Cref{th: main_thm} with the uniform prior ($\forall a, P(a)= \frac{1}{K}$) and $\lambda= \lambda_m$  (depending possibly on $m$): with probability $1- \delta/2$, for any tuple $(m,\lambda_m)$ with $m\geq 1$, any posterior $\Q$,
\begin{align*}
\left|M_m\left(\Q\right)\right| \leq \frac{\operatorname{KL}\left(\Q, \P\right)+2 +\log (4/\delta)}{\lambda_m}+ \frac{\lambda_m}{2}\left( \hat{V}_m(\Q) + V_m(\Q) \right).
\end{align*}
Notice that for any $\Q$, $\operatorname{KL}(\Q,\P)\leq \log(K)$ by concavity of the log.
We now fix an horizon $M>0$, we then have in particular, with probability $1- \delta/2$: for any posterior $\Q$,
\begin{align*}
\left|M_m\left(Q\right)\right| \leq \frac{\log(K)+2 \log (k+1)+\log (4/\delta)}{\lambda_k}+ \frac{\lambda_m}{2}\left( \hat{V}_m(\Q) + V_m(\Q) \right).
\end{align*}
We now have to deal with $V_k(\Q), \hat{V}_k(\Q)$ for all $k\leq m$. To do so, we propose the two following lemmas.
\begin{lemma}
\label{l: bandit_lemma_1}
For all $m\geq 1$, $a\in\mathcal{A}$, $V_m(a)\leq \frac{2Cm}{\varepsilon_m}$.
Then, we have for any $m,Q$, $V_m(\Q)\leq \frac{2Cm}{\varepsilon_m}$.
\end{lemma}

\begin{proof} We have
\begin{align*}
V_t(a) &=\sum_{i=1}^m \mathbb{E}\left[\left(\left[R_i^{a^*}-R_i^a\right]-\Delta(a)\right)^2 \mid \mathcal{F}_{i-1}\right] \\
&=\sum_{i=1}^m \mathbb{E}\left[\left(R_i^{a^*}-R_i^a\right)^2 \mid \mathcal{F}_{i-1}\right]-m \Delta(a)^2\\
& \leq \sum_{i=1}^m \mathbb{E}\left[\left(R_i^{a^*}-R_i^a\right)^2 \mid \mathcal{F}_{i-1}\right]  \\
&  = \sum_{i=1}^m \mathbb{E}\left[\mathbb{E}_{A_i\sim \pi_i}\mathbb{E}_{R_i}\left[\frac{1}{\pi_i(a^*)^2} R_i(a^*)^2\mathds{1}(A_i=a^*) +\frac{1}{\pi_i(a)^2}R_i(a)^2\mathds{1}(A_i=a) \right] \mid \mathcal{F}_{i-1}\right].\\
\intertext{The last line holding because $R_i$ is independent of $\mathcal{F}_{i-1}$, $A_i$ is independent of $R_i$ and $\pi$ is $\mathcal{F}_{i-1}$ measurable.
We now use that for all $i,a$, $\mathbb{E}_{R_i}[R_i(a)^2] \leq C$ }
& = \sum_{i=1}^m \mathbb{E}\left[\mathbb{E}_{A_i\sim \pi_i}
\left[\frac{1}{\pi_i(a^*)^2} C\mathds{1}(A_i=a^*) +\frac{1}{\pi_i(a)^2}C\mathds{1}(A_i=a) \right] \mid \mathcal{F}_{i-1}\right]\\
& =\sum_{i=1}^m C\left(\frac{\pi_i(a)}{\pi_i(a)^2}+\frac{\pi_i\left(a^*\right)}{\pi_i\left(a^*\right)^2}\right) \\ &=\sum_{i=1}^m C\left(\frac{1}{\pi_i(a)}+\frac{1}{\pi_i\left(a^*\right)} \right) \\
& \leq \frac{2C m}{\varepsilon_m}.
\end{align*}
\end{proof}


\begin{lemma}
\label{l: bandit_lemma_2}
Let $m\geq 1$, with probability $1-\delta/2$, for any posterior $\Q$, we have
\[ \hat{V}_m(\Q) \leq \frac{4CKm}{\varepsilon_m\delta}. \]
\end{lemma}

\begin{proof}
Let $\Q$ a distribution over $\mathcal{A}$. Recall that
\begin{align*}
\hat{V}_m(\Q) & = \sum_{i=1}^m \left(R_i^{a^*}-R_i^a-\left[R\left(a^*\right)-R(a)\right]\right)^2 \\
& = \sum_{a\in\mathcal{A}} Q(a) \hat{V}_m(a).
\end{align*}
Notice that for any $a$, $(\hat{SM}_m^a)_m$ is a  nonnegative random variable. We then apply Markov's inequality for any $a$, with probability $1-\delta/2K$
\[ \hat{V}_m(a)\leq  \frac{2K\mathbb{E}[\hat{V}_m(a)]}{\delta} .  \]
Noticing that $\mathbb{E}[\hat{V}_m(a)] = \mathbb{E}[V_m(a)]$, we can apply \cref{l: bandit_lemma_1} to conclude that
$$\mathbb{E}[\hat{V}_m(a)] \leq \frac{2Cm}{\varepsilon_m}.$$
Finally, taking an union bound on thoser events for all $a\in\mathcal{A}$ gives us, with probability $1-\delta/2$, for any posterior $\Q$
\begin{align*}
V_m(\Q) & \leq \sum_{a\in\mathcal{A}} Q(a) \hat{V}_m(a) \\
& \leq \sum_{a\in\mathcal{A}} Q(a) \frac{4CKm}{\varepsilon_m\delta} \\
&= \frac{4CKm}{\varepsilon_m\delta}.
\end{align*}
This concludes the proof.
\end{proof}
To conclude, we apply \cref{l: bandit_lemma_1,l: bandit_lemma_2} to get that with probability $1-\delta$, for any posterior $\Q$
\begin{align*}
\left|M_m\left(\Q\right)\right| & \leq \frac{\operatorname{KL}\left(\Q, \P\right) +\log (4/\delta)}{\lambda_m}+ \frac{Cm\lambda_m}{\varepsilon_m} \left(1+ \frac{2K}{\delta}    \right).
\end{align*}
Dividing by $m$ and taking $$\lambda_m= \sqrt{\frac{\left(\log(K) +\log (4/\delta)\right) \varepsilon_m}{Cm\left(1+ \frac{2K}{\delta}    \right)}}$$ concludes the proof.

\end{proof}





\subsection{Proof of Cor. \ref{cor: bound_mart}}


\begin{proof}
Fix $\delta>0$. For any pair $(\lambda_k,P_k), k\geq 1$, we apply \Cref{th: main_thm} with $$\delta_k := \frac{\delta}{k(k+1)} \geq \frac{\delta}{(k+1)^2}. $$
Notice that we have $\sum_{k=1}^{+\infty} \delta_k = \delta$.
We then have with probability $1-\delta_k$ over $S$, for any $m\geq 1$, any posterior $\Q$,
\[ \left|M_m\left(\Q\right)\right| \leq \frac{\KL\left(\Q, \P_k\right)+2 \log (k+1)+\log (2/\delta)}{\lambda_k}+ \frac{\lambda_k}{2}\left( \hat{V}_m(\Q) + V_m(\Q) \right).\]
Taking an union bound on all those event, gives the final result, valid with probability $1-\delta$ over the sample $S$, for any any tuple $(m,\lambda_k,P_k)$ with $m,k\geq 1$, any posterior $\Q$ over $\mathcal{H}$. This gives \Cref{eq: bound_mart_1}.

To obtain \cref{eq: bound_mart_2}, we restrict the range of \cref{eq: bound_mart_1} to the tuples $(m,\lambda_m,P_m), m\geq 1$ (the restricted set of tuples where $k=m$) and we bound both $\hat{V}_m(\Q),V_m(\Q)$ by $\sum_{i=1}^m C_i^2$ to conclude.
\end{proof}

\subsection{Proof of Cor. \ref{cor: bounded_case}}


\begin{proof}
For the first bound we start from the intermediary result \cref{eq: intermediary_result_main} of  \cref{th: main_thm}. Using the same marrtingale as in \cref{th: main_thm_iid} gives, for any $\eta\in\mathbb{R}$, holding with probability $1-\delta$ for any $m>0,Q\in\mathcal{M}(\mathcal{H})$
\begin{multline*}
\eta \left(\sum_{i=1}^m \mathbb{E}_{h\sim \Q}[\ell(h,\z_i)]  -m \mathbb{E}_{h\sim \Q}[ \Risk(h) ] \right) \\
\leq  \operatorname{KL}(\Q,\P) +\log(1/\delta) + \frac{\eta^2}{2}\sum_{i=1}^m \mathbb{E}_{h\sim \Q}[\Delta[M]_i(h) + \Delta \langle M\rangle_i(h) ].
\end{multline*}
Taking $\eta= \pm\lambda$ with $\lambda>0$ gives
\begin{align}
\label{eq: temp_result}
\lambda m\left | \mathbb{E}_{h\sim \Q}[ \Risk(h) -\Riskhat_{\Sm}(h)] \right | & \leq \operatorname{KL}(\Q,\P) +\log(1/\delta) \\
& + \frac{\lambda^2}{2}\sum_{i=1}^m \mathbb{E}_{h\sim \Q}[\Delta[M]_i(h) + \Delta \langle M\rangle_i(h) ].
\end{align}
Finally, divide by $\lambda m$ and bound $\Delta[M]_i(h) + \Delta \langle M\rangle_i(h)$ by $2K^2$ to conclude.


For the second bound, we start from \Cref{eq: temp_result} again and for a fixed $m$, we now apply our result with $\lambda'=\lambda/m$. We then have for any $m$, with probability $1-\delta$, for any $\Q$
\[ \lambda\left |  \mathbb{E}_{h\sim \Q}[ \Risk(h) -\Riskhat_{\Sm}(h)] \right |  \leq \operatorname{KL}(\Q,\P) +\log(1/\delta) + \frac{\lambda^2}{2m^2}\sum_{i=1}^m \mathbb{E}_{h\sim \Q}[\Delta[M]_i(h) + \Delta \langle M\rangle_i(h) ].\]
Finally, dividing by $\lambda$, bounding $\Delta[M]_i(h) + \Delta \langle M\rangle_i(h)$ by $2K^2$ and rearranging the terms concludes the proof.
\end{proof}


\end{noaddcontents}
%%!TEX root = main.tex
\chapter{Appendix of Chapter~\ref{chap: pb-ht}}
\label{ap: pb-ht}

\begin{noaddcontents}
    

\section{Some PAC-Bayesian background}
\label{sec: pac_b_background}

We present below an immediate corollary of \citet[Thm 2.1]{seldin2012bandit} where we upper bounded the cumulative by an empirical quantity (the sum of squared upper bound of the martingale difference sequence).

\begin{theorem}[\citealp{seldin2012bandit}, Theorem 2.1]
\label{th: seldin_thm_mart}
Let $\left\{C_1, C_2, \ldots\right\}$ be an increasing sequence set in advance, such that $\left|X_i(\\S_i,h)\right| \leq C_i$ for all $\\S_i,h$ with probability 1.   Let $\left\{P_1, P_2, \ldots\right\}$ be a sequence of data-free prior distributions over $\mathcal{H}$. Let $(\lambda_i)_{i\geq 1}$ be a sequence of positive numbers such that
$$
\lambda_m \leq \frac{1}{C_m}.
$$
Then with probability $1-\delta$ over $\S=(\z_i)_{i\geq 1}$,
for all $m\geq 1$, any posterior $\Q$ over $\mathcal{H}$,
$$
\left|M_m\left(Q\right)\right| \leq \frac{\KL\left(\Q , \P_m\right)+2 \log (m+1)+\log \frac{2}{\delta}}{\lambda_m}+(e-2) \lambda_m V_m(\Q),
$$
where $V_m(\Q)$ is defined in \cref{subsec: comparison_seldin}.

Furthermore, if we bound the variance term, we would have:
$$
\left|M_m\left(\Q\right)\right| \leq \frac{\KL\left(\Q , \P_m\right)+2 \log (m+1)+\log \frac{2}{\delta}}{\lambda_m}+(e-2) \lambda_m \sum_{i=1}^m C_i^2.
$$
\end{theorem}
Below, we use the definitions introduced in \Cref{sec: iid_case}.
We study here a particular case of \cite{alquier2016properties} for bounded losses which are especially subgaussian thanks to Hoeffding's lemma.
\begin{theorem}[Adapted from  \citealp{alquier2016properties}, Theorem 4.1]
\label{th: naive_pac_bayes-chap3}
Let $m>0$,$\S_m=(\z_1,...,\z_m)$ be an \iid sample from the same law $\mu$.
For any data-free prior $\P$, for any loss function $\ell$ bounded by $K$, any $\lambda>0,\delta\in ]0;1[$, one has with probability $1-\delta$ for any posterior $Q\in\mathcal{M}_1(\mathcal{H})$
\[ \mathbb{E}_{h\sim \Q}[\Risk(h)] \leq  \mathbb{E}_{h\sim \Q}[\Riskhat_{\Sm}(h)] + \frac{\operatorname{KL}(\Q, \P) + \log(1/\delta)}{\lambda} + \frac{\lambda K^2}{2m}. \]
\end{theorem}

\begin{theorem}[\citealp{haddouche2021pac}, Theorem 3]
\label{th: haddouche_thm}
Let the loss $\ell$ be $\mathrm{HYPE}(K)$ compliant. For any $\P\in\mathcal{M}(\mathcal{H})$ with no data dependency, for any $\alpha\in\mathbb{R}$ and for any $\delta\in[0,1]$, we have with probability at least $1-\delta$ over size-$m$ samples S,
for any $\Q$% such that $Q \ll P$% and $P \ll Q$
\begin{align*}
\mathbb{E}_{h\sim \Q}\left[ \Risk(h)\right]
\leq \mathbb{E}_{h\sim \Q}\left[ \Riskhat_{\Sm}(h)\right] + \frac{\operatorname{KL}(\Q,\P) + \log\left(\frac{1}{\delta}\right)}{m^{\alpha}}
+\frac{1}{m^{\alpha}}\log\left(\mathbb{E}_{h\sim \P} \left[\exp\left( \frac{K(h)^2}{2m^{1-2\alpha}} \right) \right]\right).
\end{align*}
\end{theorem}


\section{Extensions of previous results}
\label{sec: extensions}

Here we gather several corollaries of our main result in order to show how our \Cref{th: main_thm} extends the validity of some classical results in the literature. More precisely we show that our result extends (up to numerical factors) the PAC-Bayes Bernstein inequality of \citet{seldin2012bandit}.
Then, going back to the bounded case, we generalise a result from \citet{catoni2007pac} reformulated in \citet{alquier2016properties} and we also show how our work strictly improves on the bound of \citet{haddouche2021pac}.

\subsection{Extension of the PAC-Bayes Bernstein inequality}
\label{subsec: comparison_seldin}

Here we rename two terms for consistency with Theorem 2.1 of \citet{seldin2012bandit} (see \Cref{th: seldin_thm_mart}). For a martingale $M_m(h)= \sum_{i=1}^m X_i(\S_i,h)$, we define, at time $m$, \emph{empirical cumulative variance } to be $\hat{V}_m(h)= [M]_m(h) = \sum_{i=1}^m X_i(\S_i,h)^2$ and
the \emph{cumulative variance} as $V_m(h)= \langle M\rangle_m(h) = \sum_{i=1}^m \mathbb{E}_{i-1}[X_i(\S_i,h)^2]$.

We provide below a corollary containing two bounds: the first one being a straightforward corollary of \cref{th: main_thm}, the second being valid for bounded martingales and formally close to Theorem 2.1 of \citet{seldin2012bandit}.
\begin{corollary}
\label{cor: bound_mart}
Let $\left\{\P_1, \P_2, \ldots\right\}$ be a sequence of data-free prior distributions over $\mathcal{H}$. Let $(\lambda_i)_{i\geq 1}$ be a sequence of positive numbers.
Then the following holds with probability $1-\delta$ over $\S=(\z_i)_{i\geq 1}$: for any tuple $(m,\lambda_k,\P_k)$ with $m,k\geq 1$, any posterior $\Q$ over $\mathcal{H}$,
\begin{align}
\label{eq: bound_mart_1}
\left|M_m\left(\Q\right)\right| \leq \frac{\KL\left(\Q, \P_k\right)+2 \log (k+1)+\log (2/\delta)}{\lambda_k}+ \frac{\lambda_k}{2}\left( \hat{V}_m(\Q) + V_m(\Q) \right),
\end{align}
with $\hat{V}_m(\Q)= \mathbb{E}_{h\sim \Q}[\hat{V}_m(h)], V_m(\Q)= \mathbb{E}_{h\sim \Q}[V_m(h)]$.
Furthermore, if we assume that for any $i$, there exists $C_i>0$ such that $|X_i(\S_i,h)|\leq C_i$ for all $\S_i,h$ then we have the following corollary: with probability $1-\delta$ over $S$, for any tuple $(m,\lambda_m,\P_m)$ $m\geq 1$, any posterior $\Q$,
\begin{align}
\label{eq: bound_mart_2}
\left|M_m\left(Q\right)\right| \leq \frac{\KL\left(\Q, \P_m\right)+2 \log (m+1)+\log (2/\delta)}{\lambda_m}+ \lambda_m\sum_{i=1}^m C_i^2.
\end{align}
\end{corollary}
The proof is deferred to \cref{sec: proofs}.
Note that \cref{eq: bound_mart_1} holds uniformly on all tuples $\{(\lambda_k,\P_k,m) \mid k\geq 1, m\geq 1\}$ while \cref{eq: bound_mart_2}, as well as Theorem 2.1 of \citet{seldin2012bandit} holds uniformly on the tuples
$\{(\lambda_m,\P_m,m) \mid m\geq 1\}$ which is a strictly smaller collection. Hence our approach gives guarantees for a larger event with the same confidence level.

Furthermore, Theorem 2.1 of \citet{seldin2012bandit} involves the cumulative variance $V_m(\Q)$ (and not its empirical counterpart). Because this term is theoretical, we bound it in \cref{th: seldin_thm_mart} by $\sum_{i=1}^m C_i^2$ which is supposedly empirical.
In this context, \cref{eq: bound_mart_2}, recovers nearly exactly the bound of \cite{seldin2012bandit} with the transformation of a factor $(e-2)$ into $1$.
Notice also that \cref{eq: bound_mart_2} stands with no assumption on the range of the $\lambda_i$, which is not the case in \cref{th: seldin_thm_mart}.

Finally, we stress two fundamental differences between our work and the one of \citet{seldin2012bandit}. First, we replace Markov's inequality by Ville's inequality; second, we exploited the exponential inequality of Lemma \ref{l: bercu_touati} instead of the Bernstein inequality. These allow for results for unbounded martingales for all $m$ simultaneously.


\subsection{Extensions of learning theory results}


\subsubsection{A general result for bounded losses}


We use definitions from \Cref{sec: iid_case} and provide a corollary of our main result when the loss is bounded by a positive constant $K>0$. We assume our data are iid.
\begin{corollary}
\label{cor: bounded_case}
For any data-free prior $P\in \mathcal{M}(\mathcal{H})$, any $\lambda>0$ the following holds with probability $1-\delta$ over the sample $S=(z_i)_{i\in\mathbb{N}}$, for all $m\in\mathbb{N}/\{0\}$, $\Q\in\mathcal{M}(\mathcal{H})$
\[ \left | \mathbb{E}_{h\sim \Q} [\Risk(h)] -  \mathbb{E}_{h\sim \Q} \left[\Riskhat_{\S_m}(h) \right] \right |  \leq \frac{\operatorname{KL}(\Q,\P) +\log(2/\delta)}{\lambda m } + \lambda K^2.  \]
We also have the local bound: for any $m\geq 1$, with probability $1-\delta$ over $S$, for all $\Q\in\mathcal{M}(\mathcal{H})$
\[ \mathbb{E}_{h\sim \Q} [\Risk(h)] \leq  \mathbb{E}_{h\sim \Q} \left[\Riskhat_{\S_m}(h) \right] + \frac{\operatorname{KL}(\Q,\P) +\log(2/\delta)}{\lambda} + \frac{\lambda K^2}{m}.  \]
\end{corollary}
The proof is deferred to \cref{sec: proofs}. Remark that the second bound of Corollary \ref{cor: bounded_case} is exactly the Catoni bound stated in \citet{alquier2016properties} (see \Cref{th: naive_pac_bayes-chap3} in \Cref{sec: pac_b_background}) up to a numerical factor of $2$.


The first bound is, to our knowledge, the first PAC-Bayesian bound for bounded losses holding uniformly (for a given parameter $\lambda$) on the choice of $Q,m$ and thus extends the scope of Catoni's bound which holds for a single $m$ with high probability.  Indeed, if we want for instance \Cref{th: naive_pac_bayes-chap3} to hold for any $i\in\{1..m\}$, we then have to take an union bound on $m$ events which turns the term $\log(1/\delta)$ into $\log(m/\delta)$ (but with the benefit of holding for $m$ parameters $\lambda_1,...,\lambda_m$). This point is common to the most classical PAC-Bayesian bounds
(including McAllester and Catoni's ones \eqref{eq: mcallester}, \eqref{eq: catoni})
and impeach us to have a bound uniformly on all $m\in\mathbb{N}/\{0\}$ as $\log(m)$ goes to infinity asymptotically.


\subsubsection{An extension of \citet{haddouche2021pac}}

We now focus on the work of \citet{haddouche2021pac} which provides general PAC-Bayesian bounds for unbounded losses. Their theorems hold for iid data and under the so-called \emph{HYPE} (for HYPothesis-dependent rangE) condition. It states that a loss function $\ell$ is \emph{HYPE}$(K)$ compliant if there exists a function $K:\mathcal{H} \rightarrow \mathbb{R}^+ $ (supposedly accessible)  such that $\forall z\in\mathcal{Z}, \ell(h,\z) \leq K(h)$.
We provide \Cref{cor: haddouche_comparison} to compare ourselves with their main result (stated in  \Cref{th: haddouche_thm} for convenience).
\begin{corollary}
\label{cor: haddouche_comparison}
For any data-free prior $\P\in \mathcal{M}(\mathcal{H})$, any loss function $\ell$ being \emph{HYPE}$(K)$ compliant, any $\alpha\in[0,1],m\geq 1$, the following holds with probability $1-\delta$ over the sample $\S=(\z_i)_{i\in\mathbb{N}}$, for all $\Q\in\mathcal{M}(\mathcal{H})$
\begin{multline*}
\mathbb{E}_{h\sim \Q} [\Risk(h)] \leq   \mathbb{E}_{h\sim \Q} \left[\frac{1}{m}\sum_{i=1}^m\left(\ell(h,\z_i) + \frac{1}{2m^{1-\alpha}} \ell(h,\z_i)^2\right)\right] \\
+ \frac{\operatorname{KL}(\Q,\P) +\log(1/\delta)}{m^{\alpha}}  + \frac{1}{2m^{1-\alpha}}\mathbb{E}_{h\sim \Q} [K^2(h)].
\end{multline*}
\end{corollary}

\begin{proof}
The proof is a straightforward application of \cref{th: main_thm_iid} by fixing $m\geq 1$ choosing $\lambda= m^{\alpha-1}$ (thus we localise \Cref{th: main_thm_iid} to a single $m$),  and bounding $\mathrm{Quad}(h)$ by $K^2(h)$.
\end{proof}
The main improvement of our bound over \Cref{th: haddouche_thm} is that we do not have to assume the convergence of an exponential moment to obtain a non-trivial bound. Indeed, we transformed the (implicit) assumption $\mathbb{E}_{h\sim \P} \left[\exp\left( \frac{K(h)^2}{2m^{1-2\alpha}} \right) \right] < +\infty $ onto $\mathbb{E}_{h\sim \Q}[K(h)^2] < +\infty$, which is significantly less restrictive.
Furthermore, \Cref{th: haddouche_thm} holds for a single choice of $m$ while ours still holds uniformly over all integers $m>0$.

Cor. \ref{cor: haddouche_comparison} also sheds new light on the \emph{HYPE} condition. Indeed, in \citet{haddouche2021pac}, $K$ only intervenes in an exponential moment involving the prior $\P$, while ours considers a second-order moment on $K$ implying the posterior $\Q$. The difference is major as $\mathbb{E}_{h\sim \Q}[K(h)^2] $ can be controlled by a wise choice of posterior. Thus it can be incorporated in our optimisation route, acting now as an optimisation constraint instead of an environment constraint.



\section{Proofs}
\label{sec: proofs}

\subsection{Proof of \cref{th: main_thm_iid}}

\begin{proof}
Let $\P$ a fixed data-free prior, set $(\mathcal{F}_i)_{i\geq 0}$ such that for all $i$, $\z_i$ is $\mathcal{F}_i$ measurable. We also set for any fixed $h\in\mathcal{H}, M_m(h):= \sum_{i=1}^m \ell(h,\z_i) - \Risk(h)$. Note that because data are \iid, for any fixed $h$, the sequence $(M_m(h))_m$ is indeed a martingale.
We set for any $m\geq 1, h\in\mathcal{H}$
$$[M]_m(h) = \sum_{i=1}^m \left(\ell(h,\z_i) - \Risk(h)\right)^2 $$ and
$$\langle M \rangle_m(h) =  \sum_{i=1}^m \mathbb{E}_{i-1}[\left(\ell(h,\z_i) - \Risk(h)\right)^2] = \sum_{i=1}^m \mathbb{E}_{\z\sim \D}[\left(\ell(h,\z) - \Risk(h)\right)^2].$$
The last equality holds because data is assumed iid. Thus, we can apply \cref{th: main_thm} to obtain with probability $1-\delta$
\[|M_m(\Q)| \leq   \frac{\operatorname{KL}(\Q,\P) +\log(2/\delta)}{\lambda } + \frac{\lambda}{2}\left([M]_m(Q)^2 + \langle M\rangle_m(Q)^2 \right) . \]
Now, we notice that $|M_m(\Q)| = m| \mathbb{E}_{h\sim \Q}[\Risk(h) - \Riskhat_{\Sm}(h)] |$  and that  for any $m,h$, because $\ell$ is nonnegative
\begin{align*}
[M]_m(h) +  \langle M\rangle_m(h) & = \sum_{i=1}^m (\ell(h,\z_i) - \Risk(h))^2 + \mathbb{E}_{\z\sim \D}[ (\ell(h,\z) - \Risk(h))^2] \\
& \leq  \sum_{i=1}^m \ell(h,\z_i)^2 + \Risk(h)^2 + \mathbb{E}_{\z\sim \D}[\ell(h,\z)^2] - \Risk(h)^2.\\
\intertext{ Thus integrating over $h$ gives: }
[M]_m(Q) +  \langle M\rangle_m(Q) & \leq \sum_{i=1}^m \mathbb{E}_{h\sim \Q} [\ell(h,\z_i)^2] + m\mathbb{E}_{h\sim \Q} [\mathrm{Quad}(h)].
\end{align*}
Then dividing by $m$ and applying the last inequality gives
\begin{multline*}
\mathbb{E}_{h\sim \Q} [\Risk(h)]  \leq  \mathbb{E}_{h\sim \Q} \left[\frac{1}{m}\sum_{i=1}^m\left(\ell(h,\z_i) + \frac{\lambda}{2} \ell(h,\z_i)^2\right)\right] \\
+ \frac{\operatorname{KL}(\Q,\P) +\log(2/\delta)}{\lambda m} + \frac{\lambda}{2}\mathbb{E}_{h\sim \Q} [\mathrm{Quad}(h)].
\end{multline*}
This concludes the proof.
\end{proof}






\subsection{Proof of \cref{th: bandits_bound}}

\begin{proof}
Let $(\lambda_m)_{i\geq 1}$ be a countable sequence of positive scalars.
As precised earlier $M_m(a):= m\left(\hat{\Delta}_m(a)-\Delta(a)\right)$ is a martingale.
We then apply \Cref{th: main_thm} with the uniform prior ($\forall a, P(a)= \frac{1}{K}$) and $\lambda= \lambda_m$  (depending possibly on $m$): with probability $1- \delta/2$, for any tuple $(m,\lambda_m)$ with $m\geq 1$, any posterior $\Q$,
\begin{align*}
\left|M_m\left(\Q\right)\right| \leq \frac{\operatorname{KL}\left(\Q, \P\right)+2 +\log (4/\delta)}{\lambda_m}+ \frac{\lambda_m}{2}\left( \hat{V}_m(\Q) + V_m(\Q) \right).
\end{align*}
Notice that for any $\Q$, $\operatorname{KL}(\Q,\P)\leq \log(K)$ by concavity of the log.
We now fix an horizon $M>0$, we then have in particular, with probability $1- \delta/2$: for any posterior $\Q$,
\begin{align*}
\left|M_m\left(Q\right)\right| \leq \frac{\log(K)+2 \log (k+1)+\log (4/\delta)}{\lambda_k}+ \frac{\lambda_m}{2}\left( \hat{V}_m(\Q) + V_m(\Q) \right).
\end{align*}
We now have to deal with $V_k(\Q), \hat{V}_k(\Q)$ for all $k\leq m$. To do so, we propose the two following lemmas.
\begin{lemma}
\label{l: bandit_lemma_1}
For all $m\geq 1$, $a\in\mathcal{A}$, $V_m(a)\leq \frac{2Cm}{\varepsilon_m}$.
Then, we have for any $m,Q$, $V_m(\Q)\leq \frac{2Cm}{\varepsilon_m}$.
\end{lemma}

\begin{proof} We have
\begin{align*}
V_t(a) &=\sum_{i=1}^m \mathbb{E}\left[\left(\left[R_i^{a^*}-R_i^a\right]-\Delta(a)\right)^2 \mid \mathcal{F}_{i-1}\right] \\
&=\sum_{i=1}^m \mathbb{E}\left[\left(R_i^{a^*}-R_i^a\right)^2 \mid \mathcal{F}_{i-1}\right]-m \Delta(a)^2\\
& \leq \sum_{i=1}^m \mathbb{E}\left[\left(R_i^{a^*}-R_i^a\right)^2 \mid \mathcal{F}_{i-1}\right]  \\
&  = \sum_{i=1}^m \mathbb{E}\left[\mathbb{E}_{A_i\sim \pi_i}\mathbb{E}_{R_i}\left[\frac{1}{\pi_i(a^*)^2} R_i(a^*)^2\mathds{1}(A_i=a^*) +\frac{1}{\pi_i(a)^2}R_i(a)^2\mathds{1}(A_i=a) \right] \mid \mathcal{F}_{i-1}\right].\\
\intertext{The last line holding because $R_i$ is independent of $\mathcal{F}_{i-1}$, $A_i$ is independent of $R_i$ and $\pi$ is $\mathcal{F}_{i-1}$ measurable.
We now use that for all $i,a$, $\mathbb{E}_{R_i}[R_i(a)^2] \leq C$ }
& = \sum_{i=1}^m \mathbb{E}\left[\mathbb{E}_{A_i\sim \pi_i}
\left[\frac{1}{\pi_i(a^*)^2} C\mathds{1}(A_i=a^*) +\frac{1}{\pi_i(a)^2}C\mathds{1}(A_i=a) \right] \mid \mathcal{F}_{i-1}\right]\\
& =\sum_{i=1}^m C\left(\frac{\pi_i(a)}{\pi_i(a)^2}+\frac{\pi_i\left(a^*\right)}{\pi_i\left(a^*\right)^2}\right) \\ &=\sum_{i=1}^m C\left(\frac{1}{\pi_i(a)}+\frac{1}{\pi_i\left(a^*\right)} \right) \\
& \leq \frac{2C m}{\varepsilon_m}.
\end{align*}
\end{proof}


\begin{lemma}
\label{l: bandit_lemma_2}
Let $m\geq 1$, with probability $1-\delta/2$, for any posterior $\Q$, we have
\[ \hat{V}_m(\Q) \leq \frac{4CKm}{\varepsilon_m\delta}. \]
\end{lemma}

\begin{proof}
Let $\Q$ a distribution over $\mathcal{A}$. Recall that
\begin{align*}
\hat{V}_m(\Q) & = \sum_{i=1}^m \left(R_i^{a^*}-R_i^a-\left[R\left(a^*\right)-R(a)\right]\right)^2 \\
& = \sum_{a\in\mathcal{A}} Q(a) \hat{V}_m(a).
\end{align*}
Notice that for any $a$, $(\hat{SM}_m^a)_m$ is a  nonnegative random variable. We then apply Markov's inequality for any $a$, with probability $1-\delta/2K$
\[ \hat{V}_m(a)\leq  \frac{2K\mathbb{E}[\hat{V}_m(a)]}{\delta} .  \]
Noticing that $\mathbb{E}[\hat{V}_m(a)] = \mathbb{E}[V_m(a)]$, we can apply \cref{l: bandit_lemma_1} to conclude that
$$\mathbb{E}[\hat{V}_m(a)] \leq \frac{2Cm}{\varepsilon_m}.$$
Finally, taking an union bound on thoser events for all $a\in\mathcal{A}$ gives us, with probability $1-\delta/2$, for any posterior $\Q$
\begin{align*}
V_m(\Q) & \leq \sum_{a\in\mathcal{A}} Q(a) \hat{V}_m(a) \\
& \leq \sum_{a\in\mathcal{A}} Q(a) \frac{4CKm}{\varepsilon_m\delta} \\
&= \frac{4CKm}{\varepsilon_m\delta}.
\end{align*}
This concludes the proof.
\end{proof}
To conclude, we apply \cref{l: bandit_lemma_1,l: bandit_lemma_2} to get that with probability $1-\delta$, for any posterior $\Q$
\begin{align*}
\left|M_m\left(\Q\right)\right| & \leq \frac{\operatorname{KL}\left(\Q, \P\right) +\log (4/\delta)}{\lambda_m}+ \frac{Cm\lambda_m}{\varepsilon_m} \left(1+ \frac{2K}{\delta}    \right).
\end{align*}
Dividing by $m$ and taking $$\lambda_m= \sqrt{\frac{\left(\log(K) +\log (4/\delta)\right) \varepsilon_m}{Cm\left(1+ \frac{2K}{\delta}    \right)}}$$ concludes the proof.

\end{proof}





\subsection{Proof of Cor. \ref{cor: bound_mart}}


\begin{proof}
Fix $\delta>0$. For any pair $(\lambda_k,P_k), k\geq 1$, we apply \Cref{th: main_thm} with $$\delta_k := \frac{\delta}{k(k+1)} \geq \frac{\delta}{(k+1)^2}. $$
Notice that we have $\sum_{k=1}^{+\infty} \delta_k = \delta$.
We then have with probability $1-\delta_k$ over $S$, for any $m\geq 1$, any posterior $\Q$,
\[ \left|M_m\left(\Q\right)\right| \leq \frac{\KL\left(\Q, \P_k\right)+2 \log (k+1)+\log (2/\delta)}{\lambda_k}+ \frac{\lambda_k}{2}\left( \hat{V}_m(\Q) + V_m(\Q) \right).\]
Taking an union bound on all those event, gives the final result, valid with probability $1-\delta$ over the sample $S$, for any any tuple $(m,\lambda_k,P_k)$ with $m,k\geq 1$, any posterior $\Q$ over $\mathcal{H}$. This gives \Cref{eq: bound_mart_1}.

To obtain \cref{eq: bound_mart_2}, we restrict the range of \cref{eq: bound_mart_1} to the tuples $(m,\lambda_m,P_m), m\geq 1$ (the restricted set of tuples where $k=m$) and we bound both $\hat{V}_m(\Q),V_m(\Q)$ by $\sum_{i=1}^m C_i^2$ to conclude.
\end{proof}

\subsection{Proof of Cor. \ref{cor: bounded_case}}


\begin{proof}
For the first bound we start from the intermediary result \cref{eq: intermediary_result_main} of  \cref{th: main_thm}. Using the same marrtingale as in \cref{th: main_thm_iid} gives, for any $\eta\in\mathbb{R}$, holding with probability $1-\delta$ for any $m>0,Q\in\mathcal{M}(\mathcal{H})$
\begin{multline*}
\eta \left(\sum_{i=1}^m \mathbb{E}_{h\sim \Q}[\ell(h,\z_i)]  -m \mathbb{E}_{h\sim \Q}[ \Risk(h) ] \right) \\
\leq  \operatorname{KL}(\Q,\P) +\log(1/\delta) + \frac{\eta^2}{2}\sum_{i=1}^m \mathbb{E}_{h\sim \Q}[\Delta[M]_i(h) + \Delta \langle M\rangle_i(h) ].
\end{multline*}
Taking $\eta= \pm\lambda$ with $\lambda>0$ gives
\begin{align}
\label{eq: temp_result}
\lambda m\left | \mathbb{E}_{h\sim \Q}[ \Risk(h) -\Riskhat_{\Sm}(h)] \right | & \leq \operatorname{KL}(\Q,\P) +\log(1/\delta) \\
& + \frac{\lambda^2}{2}\sum_{i=1}^m \mathbb{E}_{h\sim \Q}[\Delta[M]_i(h) + \Delta \langle M\rangle_i(h) ].
\end{align}
Finally, divide by $\lambda m$ and bound $\Delta[M]_i(h) + \Delta \langle M\rangle_i(h)$ by $2K^2$ to conclude.


For the second bound, we start from \Cref{eq: temp_result} again and for a fixed $m$, we now apply our result with $\lambda'=\lambda/m$. We then have for any $m$, with probability $1-\delta$, for any $\Q$
\[ \lambda\left |  \mathbb{E}_{h\sim \Q}[ \Risk(h) -\Riskhat_{\Sm}(h)] \right |  \leq \operatorname{KL}(\Q,\P) +\log(1/\delta) + \frac{\lambda^2}{2m^2}\sum_{i=1}^m \mathbb{E}_{h\sim \Q}[\Delta[M]_i(h) + \Delta \langle M\rangle_i(h) ].\]
Finally, dividing by $\lambda$, bounding $\Delta[M]_i(h) + \Delta \langle M\rangle_i(h)$ by $2K^2$ and rearranging the terms concludes the proof.
\end{proof}


\end{noaddcontents}
%%!TEX root = main.tex
\chapter{Appendix of Chapter~\ref{chap: pb-ht}}
\label{ap: pb-ht}

\begin{noaddcontents}
    

\section{Some PAC-Bayesian background}
\label{sec: pac_b_background}

We present below an immediate corollary of \citet[Thm 2.1]{seldin2012bandit} where we upper bounded the cumulative by an empirical quantity (the sum of squared upper bound of the martingale difference sequence).

\begin{theorem}[\citealp{seldin2012bandit}, Theorem 2.1]
\label{th: seldin_thm_mart}
Let $\left\{C_1, C_2, \ldots\right\}$ be an increasing sequence set in advance, such that $\left|X_i(\\S_i,h)\right| \leq C_i$ for all $\\S_i,h$ with probability 1.   Let $\left\{P_1, P_2, \ldots\right\}$ be a sequence of data-free prior distributions over $\mathcal{H}$. Let $(\lambda_i)_{i\geq 1}$ be a sequence of positive numbers such that
$$
\lambda_m \leq \frac{1}{C_m}.
$$
Then with probability $1-\delta$ over $\S=(\z_i)_{i\geq 1}$,
for all $m\geq 1$, any posterior $\Q$ over $\mathcal{H}$,
$$
\left|M_m\left(Q\right)\right| \leq \frac{\KL\left(\Q , \P_m\right)+2 \log (m+1)+\log \frac{2}{\delta}}{\lambda_m}+(e-2) \lambda_m V_m(\Q),
$$
where $V_m(\Q)$ is defined in \cref{subsec: comparison_seldin}.

Furthermore, if we bound the variance term, we would have:
$$
\left|M_m\left(\Q\right)\right| \leq \frac{\KL\left(\Q , \P_m\right)+2 \log (m+1)+\log \frac{2}{\delta}}{\lambda_m}+(e-2) \lambda_m \sum_{i=1}^m C_i^2.
$$
\end{theorem}
Below, we use the definitions introduced in \Cref{sec: iid_case}.
We study here a particular case of \cite{alquier2016properties} for bounded losses which are especially subgaussian thanks to Hoeffding's lemma.
\begin{theorem}[Adapted from  \citealp{alquier2016properties}, Theorem 4.1]
\label{th: naive_pac_bayes-chap3}
Let $m>0$,$\S_m=(\z_1,...,\z_m)$ be an \iid sample from the same law $\mu$.
For any data-free prior $\P$, for any loss function $\ell$ bounded by $K$, any $\lambda>0,\delta\in ]0;1[$, one has with probability $1-\delta$ for any posterior $Q\in\mathcal{M}_1(\mathcal{H})$
\[ \mathbb{E}_{h\sim \Q}[\Risk(h)] \leq  \mathbb{E}_{h\sim \Q}[\Riskhat_{\Sm}(h)] + \frac{\operatorname{KL}(\Q, \P) + \log(1/\delta)}{\lambda} + \frac{\lambda K^2}{2m}. \]
\end{theorem}

\begin{theorem}[\citealp{haddouche2021pac}, Theorem 3]
\label{th: haddouche_thm}
Let the loss $\ell$ be $\mathrm{HYPE}(K)$ compliant. For any $\P\in\mathcal{M}(\mathcal{H})$ with no data dependency, for any $\alpha\in\mathbb{R}$ and for any $\delta\in[0,1]$, we have with probability at least $1-\delta$ over size-$m$ samples S,
for any $\Q$% such that $Q \ll P$% and $P \ll Q$
\begin{align*}
\mathbb{E}_{h\sim \Q}\left[ \Risk(h)\right]
\leq \mathbb{E}_{h\sim \Q}\left[ \Riskhat_{\Sm}(h)\right] + \frac{\operatorname{KL}(\Q,\P) + \log\left(\frac{1}{\delta}\right)}{m^{\alpha}}
+\frac{1}{m^{\alpha}}\log\left(\mathbb{E}_{h\sim \P} \left[\exp\left( \frac{K(h)^2}{2m^{1-2\alpha}} \right) \right]\right).
\end{align*}
\end{theorem}


\section{Extensions of previous results}
\label{sec: extensions}

Here we gather several corollaries of our main result in order to show how our \Cref{th: main_thm} extends the validity of some classical results in the literature. More precisely we show that our result extends (up to numerical factors) the PAC-Bayes Bernstein inequality of \citet{seldin2012bandit}.
Then, going back to the bounded case, we generalise a result from \citet{catoni2007pac} reformulated in \citet{alquier2016properties} and we also show how our work strictly improves on the bound of \citet{haddouche2021pac}.

\subsection{Extension of the PAC-Bayes Bernstein inequality}
\label{subsec: comparison_seldin}

Here we rename two terms for consistency with Theorem 2.1 of \citet{seldin2012bandit} (see \Cref{th: seldin_thm_mart}). For a martingale $M_m(h)= \sum_{i=1}^m X_i(\S_i,h)$, we define, at time $m$, \emph{empirical cumulative variance } to be $\hat{V}_m(h)= [M]_m(h) = \sum_{i=1}^m X_i(\S_i,h)^2$ and
the \emph{cumulative variance} as $V_m(h)= \langle M\rangle_m(h) = \sum_{i=1}^m \mathbb{E}_{i-1}[X_i(\S_i,h)^2]$.

We provide below a corollary containing two bounds: the first one being a straightforward corollary of \cref{th: main_thm}, the second being valid for bounded martingales and formally close to Theorem 2.1 of \citet{seldin2012bandit}.
\begin{corollary}
\label{cor: bound_mart}
Let $\left\{\P_1, \P_2, \ldots\right\}$ be a sequence of data-free prior distributions over $\mathcal{H}$. Let $(\lambda_i)_{i\geq 1}$ be a sequence of positive numbers.
Then the following holds with probability $1-\delta$ over $\S=(\z_i)_{i\geq 1}$: for any tuple $(m,\lambda_k,\P_k)$ with $m,k\geq 1$, any posterior $\Q$ over $\mathcal{H}$,
\begin{align}
\label{eq: bound_mart_1}
\left|M_m\left(\Q\right)\right| \leq \frac{\KL\left(\Q, \P_k\right)+2 \log (k+1)+\log (2/\delta)}{\lambda_k}+ \frac{\lambda_k}{2}\left( \hat{V}_m(\Q) + V_m(\Q) \right),
\end{align}
with $\hat{V}_m(\Q)= \mathbb{E}_{h\sim \Q}[\hat{V}_m(h)], V_m(\Q)= \mathbb{E}_{h\sim \Q}[V_m(h)]$.
Furthermore, if we assume that for any $i$, there exists $C_i>0$ such that $|X_i(\S_i,h)|\leq C_i$ for all $\S_i,h$ then we have the following corollary: with probability $1-\delta$ over $S$, for any tuple $(m,\lambda_m,\P_m)$ $m\geq 1$, any posterior $\Q$,
\begin{align}
\label{eq: bound_mart_2}
\left|M_m\left(Q\right)\right| \leq \frac{\KL\left(\Q, \P_m\right)+2 \log (m+1)+\log (2/\delta)}{\lambda_m}+ \lambda_m\sum_{i=1}^m C_i^2.
\end{align}
\end{corollary}
The proof is deferred to \cref{sec: proofs}.
Note that \cref{eq: bound_mart_1} holds uniformly on all tuples $\{(\lambda_k,\P_k,m) \mid k\geq 1, m\geq 1\}$ while \cref{eq: bound_mart_2}, as well as Theorem 2.1 of \citet{seldin2012bandit} holds uniformly on the tuples
$\{(\lambda_m,\P_m,m) \mid m\geq 1\}$ which is a strictly smaller collection. Hence our approach gives guarantees for a larger event with the same confidence level.

Furthermore, Theorem 2.1 of \citet{seldin2012bandit} involves the cumulative variance $V_m(\Q)$ (and not its empirical counterpart). Because this term is theoretical, we bound it in \cref{th: seldin_thm_mart} by $\sum_{i=1}^m C_i^2$ which is supposedly empirical.
In this context, \cref{eq: bound_mart_2}, recovers nearly exactly the bound of \cite{seldin2012bandit} with the transformation of a factor $(e-2)$ into $1$.
Notice also that \cref{eq: bound_mart_2} stands with no assumption on the range of the $\lambda_i$, which is not the case in \cref{th: seldin_thm_mart}.

Finally, we stress two fundamental differences between our work and the one of \citet{seldin2012bandit}. First, we replace Markov's inequality by Ville's inequality; second, we exploited the exponential inequality of Lemma \ref{l: bercu_touati} instead of the Bernstein inequality. These allow for results for unbounded martingales for all $m$ simultaneously.


\subsection{Extensions of learning theory results}


\subsubsection{A general result for bounded losses}


We use definitions from \Cref{sec: iid_case} and provide a corollary of our main result when the loss is bounded by a positive constant $K>0$. We assume our data are iid.
\begin{corollary}
\label{cor: bounded_case}
For any data-free prior $P\in \mathcal{M}(\mathcal{H})$, any $\lambda>0$ the following holds with probability $1-\delta$ over the sample $S=(z_i)_{i\in\mathbb{N}}$, for all $m\in\mathbb{N}/\{0\}$, $\Q\in\mathcal{M}(\mathcal{H})$
\[ \left | \mathbb{E}_{h\sim \Q} [\Risk(h)] -  \mathbb{E}_{h\sim \Q} \left[\Riskhat_{\S_m}(h) \right] \right |  \leq \frac{\operatorname{KL}(\Q,\P) +\log(2/\delta)}{\lambda m } + \lambda K^2.  \]
We also have the local bound: for any $m\geq 1$, with probability $1-\delta$ over $S$, for all $\Q\in\mathcal{M}(\mathcal{H})$
\[ \mathbb{E}_{h\sim \Q} [\Risk(h)] \leq  \mathbb{E}_{h\sim \Q} \left[\Riskhat_{\S_m}(h) \right] + \frac{\operatorname{KL}(\Q,\P) +\log(2/\delta)}{\lambda} + \frac{\lambda K^2}{m}.  \]
\end{corollary}
The proof is deferred to \cref{sec: proofs}. Remark that the second bound of Corollary \ref{cor: bounded_case} is exactly the Catoni bound stated in \citet{alquier2016properties} (see \Cref{th: naive_pac_bayes-chap3} in \Cref{sec: pac_b_background}) up to a numerical factor of $2$.


The first bound is, to our knowledge, the first PAC-Bayesian bound for bounded losses holding uniformly (for a given parameter $\lambda$) on the choice of $Q,m$ and thus extends the scope of Catoni's bound which holds for a single $m$ with high probability.  Indeed, if we want for instance \Cref{th: naive_pac_bayes-chap3} to hold for any $i\in\{1..m\}$, we then have to take an union bound on $m$ events which turns the term $\log(1/\delta)$ into $\log(m/\delta)$ (but with the benefit of holding for $m$ parameters $\lambda_1,...,\lambda_m$). This point is common to the most classical PAC-Bayesian bounds
(including McAllester and Catoni's ones \eqref{eq: mcallester}, \eqref{eq: catoni})
and impeach us to have a bound uniformly on all $m\in\mathbb{N}/\{0\}$ as $\log(m)$ goes to infinity asymptotically.


\subsubsection{An extension of \citet{haddouche2021pac}}

We now focus on the work of \citet{haddouche2021pac} which provides general PAC-Bayesian bounds for unbounded losses. Their theorems hold for iid data and under the so-called \emph{HYPE} (for HYPothesis-dependent rangE) condition. It states that a loss function $\ell$ is \emph{HYPE}$(K)$ compliant if there exists a function $K:\mathcal{H} \rightarrow \mathbb{R}^+ $ (supposedly accessible)  such that $\forall z\in\mathcal{Z}, \ell(h,\z) \leq K(h)$.
We provide \Cref{cor: haddouche_comparison} to compare ourselves with their main result (stated in  \Cref{th: haddouche_thm} for convenience).
\begin{corollary}
\label{cor: haddouche_comparison}
For any data-free prior $\P\in \mathcal{M}(\mathcal{H})$, any loss function $\ell$ being \emph{HYPE}$(K)$ compliant, any $\alpha\in[0,1],m\geq 1$, the following holds with probability $1-\delta$ over the sample $\S=(\z_i)_{i\in\mathbb{N}}$, for all $\Q\in\mathcal{M}(\mathcal{H})$
\begin{multline*}
\mathbb{E}_{h\sim \Q} [\Risk(h)] \leq   \mathbb{E}_{h\sim \Q} \left[\frac{1}{m}\sum_{i=1}^m\left(\ell(h,\z_i) + \frac{1}{2m^{1-\alpha}} \ell(h,\z_i)^2\right)\right] \\
+ \frac{\operatorname{KL}(\Q,\P) +\log(1/\delta)}{m^{\alpha}}  + \frac{1}{2m^{1-\alpha}}\mathbb{E}_{h\sim \Q} [K^2(h)].
\end{multline*}
\end{corollary}

\begin{proof}
The proof is a straightforward application of \cref{th: main_thm_iid} by fixing $m\geq 1$ choosing $\lambda= m^{\alpha-1}$ (thus we localise \Cref{th: main_thm_iid} to a single $m$),  and bounding $\mathrm{Quad}(h)$ by $K^2(h)$.
\end{proof}
The main improvement of our bound over \Cref{th: haddouche_thm} is that we do not have to assume the convergence of an exponential moment to obtain a non-trivial bound. Indeed, we transformed the (implicit) assumption $\mathbb{E}_{h\sim \P} \left[\exp\left( \frac{K(h)^2}{2m^{1-2\alpha}} \right) \right] < +\infty $ onto $\mathbb{E}_{h\sim \Q}[K(h)^2] < +\infty$, which is significantly less restrictive.
Furthermore, \Cref{th: haddouche_thm} holds for a single choice of $m$ while ours still holds uniformly over all integers $m>0$.

Cor. \ref{cor: haddouche_comparison} also sheds new light on the \emph{HYPE} condition. Indeed, in \citet{haddouche2021pac}, $K$ only intervenes in an exponential moment involving the prior $\P$, while ours considers a second-order moment on $K$ implying the posterior $\Q$. The difference is major as $\mathbb{E}_{h\sim \Q}[K(h)^2] $ can be controlled by a wise choice of posterior. Thus it can be incorporated in our optimisation route, acting now as an optimisation constraint instead of an environment constraint.



\section{Proofs}
\label{sec: proofs}

\subsection{Proof of \cref{th: main_thm_iid}}

\begin{proof}
Let $\P$ a fixed data-free prior, set $(\mathcal{F}_i)_{i\geq 0}$ such that for all $i$, $\z_i$ is $\mathcal{F}_i$ measurable. We also set for any fixed $h\in\mathcal{H}, M_m(h):= \sum_{i=1}^m \ell(h,\z_i) - \Risk(h)$. Note that because data are \iid, for any fixed $h$, the sequence $(M_m(h))_m$ is indeed a martingale.
We set for any $m\geq 1, h\in\mathcal{H}$
$$[M]_m(h) = \sum_{i=1}^m \left(\ell(h,\z_i) - \Risk(h)\right)^2 $$ and
$$\langle M \rangle_m(h) =  \sum_{i=1}^m \mathbb{E}_{i-1}[\left(\ell(h,\z_i) - \Risk(h)\right)^2] = \sum_{i=1}^m \mathbb{E}_{\z\sim \D}[\left(\ell(h,\z) - \Risk(h)\right)^2].$$
The last equality holds because data is assumed iid. Thus, we can apply \cref{th: main_thm} to obtain with probability $1-\delta$
\[|M_m(\Q)| \leq   \frac{\operatorname{KL}(\Q,\P) +\log(2/\delta)}{\lambda } + \frac{\lambda}{2}\left([M]_m(Q)^2 + \langle M\rangle_m(Q)^2 \right) . \]
Now, we notice that $|M_m(\Q)| = m| \mathbb{E}_{h\sim \Q}[\Risk(h) - \Riskhat_{\Sm}(h)] |$  and that  for any $m,h$, because $\ell$ is nonnegative
\begin{align*}
[M]_m(h) +  \langle M\rangle_m(h) & = \sum_{i=1}^m (\ell(h,\z_i) - \Risk(h))^2 + \mathbb{E}_{\z\sim \D}[ (\ell(h,\z) - \Risk(h))^2] \\
& \leq  \sum_{i=1}^m \ell(h,\z_i)^2 + \Risk(h)^2 + \mathbb{E}_{\z\sim \D}[\ell(h,\z)^2] - \Risk(h)^2.\\
\intertext{ Thus integrating over $h$ gives: }
[M]_m(Q) +  \langle M\rangle_m(Q) & \leq \sum_{i=1}^m \mathbb{E}_{h\sim \Q} [\ell(h,\z_i)^2] + m\mathbb{E}_{h\sim \Q} [\mathrm{Quad}(h)].
\end{align*}
Then dividing by $m$ and applying the last inequality gives
\begin{multline*}
\mathbb{E}_{h\sim \Q} [\Risk(h)]  \leq  \mathbb{E}_{h\sim \Q} \left[\frac{1}{m}\sum_{i=1}^m\left(\ell(h,\z_i) + \frac{\lambda}{2} \ell(h,\z_i)^2\right)\right] \\
+ \frac{\operatorname{KL}(\Q,\P) +\log(2/\delta)}{\lambda m} + \frac{\lambda}{2}\mathbb{E}_{h\sim \Q} [\mathrm{Quad}(h)].
\end{multline*}
This concludes the proof.
\end{proof}






\subsection{Proof of \cref{th: bandits_bound}}

\begin{proof}
Let $(\lambda_m)_{i\geq 1}$ be a countable sequence of positive scalars.
As precised earlier $M_m(a):= m\left(\hat{\Delta}_m(a)-\Delta(a)\right)$ is a martingale.
We then apply \Cref{th: main_thm} with the uniform prior ($\forall a, P(a)= \frac{1}{K}$) and $\lambda= \lambda_m$  (depending possibly on $m$): with probability $1- \delta/2$, for any tuple $(m,\lambda_m)$ with $m\geq 1$, any posterior $\Q$,
\begin{align*}
\left|M_m\left(\Q\right)\right| \leq \frac{\operatorname{KL}\left(\Q, \P\right)+2 +\log (4/\delta)}{\lambda_m}+ \frac{\lambda_m}{2}\left( \hat{V}_m(\Q) + V_m(\Q) \right).
\end{align*}
Notice that for any $\Q$, $\operatorname{KL}(\Q,\P)\leq \log(K)$ by concavity of the log.
We now fix an horizon $M>0$, we then have in particular, with probability $1- \delta/2$: for any posterior $\Q$,
\begin{align*}
\left|M_m\left(Q\right)\right| \leq \frac{\log(K)+2 \log (k+1)+\log (4/\delta)}{\lambda_k}+ \frac{\lambda_m}{2}\left( \hat{V}_m(\Q) + V_m(\Q) \right).
\end{align*}
We now have to deal with $V_k(\Q), \hat{V}_k(\Q)$ for all $k\leq m$. To do so, we propose the two following lemmas.
\begin{lemma}
\label{l: bandit_lemma_1}
For all $m\geq 1$, $a\in\mathcal{A}$, $V_m(a)\leq \frac{2Cm}{\varepsilon_m}$.
Then, we have for any $m,Q$, $V_m(\Q)\leq \frac{2Cm}{\varepsilon_m}$.
\end{lemma}

\begin{proof} We have
\begin{align*}
V_t(a) &=\sum_{i=1}^m \mathbb{E}\left[\left(\left[R_i^{a^*}-R_i^a\right]-\Delta(a)\right)^2 \mid \mathcal{F}_{i-1}\right] \\
&=\sum_{i=1}^m \mathbb{E}\left[\left(R_i^{a^*}-R_i^a\right)^2 \mid \mathcal{F}_{i-1}\right]-m \Delta(a)^2\\
& \leq \sum_{i=1}^m \mathbb{E}\left[\left(R_i^{a^*}-R_i^a\right)^2 \mid \mathcal{F}_{i-1}\right]  \\
&  = \sum_{i=1}^m \mathbb{E}\left[\mathbb{E}_{A_i\sim \pi_i}\mathbb{E}_{R_i}\left[\frac{1}{\pi_i(a^*)^2} R_i(a^*)^2\mathds{1}(A_i=a^*) +\frac{1}{\pi_i(a)^2}R_i(a)^2\mathds{1}(A_i=a) \right] \mid \mathcal{F}_{i-1}\right].\\
\intertext{The last line holding because $R_i$ is independent of $\mathcal{F}_{i-1}$, $A_i$ is independent of $R_i$ and $\pi$ is $\mathcal{F}_{i-1}$ measurable.
We now use that for all $i,a$, $\mathbb{E}_{R_i}[R_i(a)^2] \leq C$ }
& = \sum_{i=1}^m \mathbb{E}\left[\mathbb{E}_{A_i\sim \pi_i}
\left[\frac{1}{\pi_i(a^*)^2} C\mathds{1}(A_i=a^*) +\frac{1}{\pi_i(a)^2}C\mathds{1}(A_i=a) \right] \mid \mathcal{F}_{i-1}\right]\\
& =\sum_{i=1}^m C\left(\frac{\pi_i(a)}{\pi_i(a)^2}+\frac{\pi_i\left(a^*\right)}{\pi_i\left(a^*\right)^2}\right) \\ &=\sum_{i=1}^m C\left(\frac{1}{\pi_i(a)}+\frac{1}{\pi_i\left(a^*\right)} \right) \\
& \leq \frac{2C m}{\varepsilon_m}.
\end{align*}
\end{proof}


\begin{lemma}
\label{l: bandit_lemma_2}
Let $m\geq 1$, with probability $1-\delta/2$, for any posterior $\Q$, we have
\[ \hat{V}_m(\Q) \leq \frac{4CKm}{\varepsilon_m\delta}. \]
\end{lemma}

\begin{proof}
Let $\Q$ a distribution over $\mathcal{A}$. Recall that
\begin{align*}
\hat{V}_m(\Q) & = \sum_{i=1}^m \left(R_i^{a^*}-R_i^a-\left[R\left(a^*\right)-R(a)\right]\right)^2 \\
& = \sum_{a\in\mathcal{A}} Q(a) \hat{V}_m(a).
\end{align*}
Notice that for any $a$, $(\hat{SM}_m^a)_m$ is a  nonnegative random variable. We then apply Markov's inequality for any $a$, with probability $1-\delta/2K$
\[ \hat{V}_m(a)\leq  \frac{2K\mathbb{E}[\hat{V}_m(a)]}{\delta} .  \]
Noticing that $\mathbb{E}[\hat{V}_m(a)] = \mathbb{E}[V_m(a)]$, we can apply \cref{l: bandit_lemma_1} to conclude that
$$\mathbb{E}[\hat{V}_m(a)] \leq \frac{2Cm}{\varepsilon_m}.$$
Finally, taking an union bound on thoser events for all $a\in\mathcal{A}$ gives us, with probability $1-\delta/2$, for any posterior $\Q$
\begin{align*}
V_m(\Q) & \leq \sum_{a\in\mathcal{A}} Q(a) \hat{V}_m(a) \\
& \leq \sum_{a\in\mathcal{A}} Q(a) \frac{4CKm}{\varepsilon_m\delta} \\
&= \frac{4CKm}{\varepsilon_m\delta}.
\end{align*}
This concludes the proof.
\end{proof}
To conclude, we apply \cref{l: bandit_lemma_1,l: bandit_lemma_2} to get that with probability $1-\delta$, for any posterior $\Q$
\begin{align*}
\left|M_m\left(\Q\right)\right| & \leq \frac{\operatorname{KL}\left(\Q, \P\right) +\log (4/\delta)}{\lambda_m}+ \frac{Cm\lambda_m}{\varepsilon_m} \left(1+ \frac{2K}{\delta}    \right).
\end{align*}
Dividing by $m$ and taking $$\lambda_m= \sqrt{\frac{\left(\log(K) +\log (4/\delta)\right) \varepsilon_m}{Cm\left(1+ \frac{2K}{\delta}    \right)}}$$ concludes the proof.

\end{proof}





\subsection{Proof of Cor. \ref{cor: bound_mart}}


\begin{proof}
Fix $\delta>0$. For any pair $(\lambda_k,P_k), k\geq 1$, we apply \Cref{th: main_thm} with $$\delta_k := \frac{\delta}{k(k+1)} \geq \frac{\delta}{(k+1)^2}. $$
Notice that we have $\sum_{k=1}^{+\infty} \delta_k = \delta$.
We then have with probability $1-\delta_k$ over $S$, for any $m\geq 1$, any posterior $\Q$,
\[ \left|M_m\left(\Q\right)\right| \leq \frac{\KL\left(\Q, \P_k\right)+2 \log (k+1)+\log (2/\delta)}{\lambda_k}+ \frac{\lambda_k}{2}\left( \hat{V}_m(\Q) + V_m(\Q) \right).\]
Taking an union bound on all those event, gives the final result, valid with probability $1-\delta$ over the sample $S$, for any any tuple $(m,\lambda_k,P_k)$ with $m,k\geq 1$, any posterior $\Q$ over $\mathcal{H}$. This gives \Cref{eq: bound_mart_1}.

To obtain \cref{eq: bound_mart_2}, we restrict the range of \cref{eq: bound_mart_1} to the tuples $(m,\lambda_m,P_m), m\geq 1$ (the restricted set of tuples where $k=m$) and we bound both $\hat{V}_m(\Q),V_m(\Q)$ by $\sum_{i=1}^m C_i^2$ to conclude.
\end{proof}

\subsection{Proof of Cor. \ref{cor: bounded_case}}


\begin{proof}
For the first bound we start from the intermediary result \cref{eq: intermediary_result_main} of  \cref{th: main_thm}. Using the same marrtingale as in \cref{th: main_thm_iid} gives, for any $\eta\in\mathbb{R}$, holding with probability $1-\delta$ for any $m>0,Q\in\mathcal{M}(\mathcal{H})$
\begin{multline*}
\eta \left(\sum_{i=1}^m \mathbb{E}_{h\sim \Q}[\ell(h,\z_i)]  -m \mathbb{E}_{h\sim \Q}[ \Risk(h) ] \right) \\
\leq  \operatorname{KL}(\Q,\P) +\log(1/\delta) + \frac{\eta^2}{2}\sum_{i=1}^m \mathbb{E}_{h\sim \Q}[\Delta[M]_i(h) + \Delta \langle M\rangle_i(h) ].
\end{multline*}
Taking $\eta= \pm\lambda$ with $\lambda>0$ gives
\begin{align}
\label{eq: temp_result}
\lambda m\left | \mathbb{E}_{h\sim \Q}[ \Risk(h) -\Riskhat_{\Sm}(h)] \right | & \leq \operatorname{KL}(\Q,\P) +\log(1/\delta) \\
& + \frac{\lambda^2}{2}\sum_{i=1}^m \mathbb{E}_{h\sim \Q}[\Delta[M]_i(h) + \Delta \langle M\rangle_i(h) ].
\end{align}
Finally, divide by $\lambda m$ and bound $\Delta[M]_i(h) + \Delta \langle M\rangle_i(h)$ by $2K^2$ to conclude.


For the second bound, we start from \Cref{eq: temp_result} again and for a fixed $m$, we now apply our result with $\lambda'=\lambda/m$. We then have for any $m$, with probability $1-\delta$, for any $\Q$
\[ \lambda\left |  \mathbb{E}_{h\sim \Q}[ \Risk(h) -\Riskhat_{\Sm}(h)] \right |  \leq \operatorname{KL}(\Q,\P) +\log(1/\delta) + \frac{\lambda^2}{2m^2}\sum_{i=1}^m \mathbb{E}_{h\sim \Q}[\Delta[M]_i(h) + \Delta \langle M\rangle_i(h) ].\]
Finally, dividing by $\lambda$, bounding $\Delta[M]_i(h) + \Delta \langle M\rangle_i(h)$ by $2K^2$ and rearranging the terms concludes the proof.
\end{proof}


\end{noaddcontents}

\newrefcontext[sorting=nyt]
\printbibliography[filter={references}, title={References}]

% ----------------------------------------------------------------------------------------------- %

\newgeometry{left=3cm,right=3cm,bottom=3cm,top=3cm}
\newpage
\pagestyle{empty}
\strictpagecheck
\checkoddpage
\ifoddpage
~\newpage
\fi
\input{main/end}
\end{document}