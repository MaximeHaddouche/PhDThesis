\chapter[PAC-Bayes Learning, a field of many paradigms]{PAC-Bayes Learning, a field of many paradigms}
\label{chap:intro-pac-bayes}

\minitoc

\addchapterlof
\addchapterloa
\addchapterloe

\section{A brief introduction to statistical learning}
\label{sec: intro-stat-learning}
Statistical learning \citep{vapnik1999overview,james2013introduction} quantifies and identifies how learning algorithms, trained on a specific task using a finite training dataset, generalise to novel, unseen datum. More precisely, a learning agent has to learn how to answer a question, formalised as a \emph{learning problem} being a tuple $(\H,\Z,\ell)$ composed of a \emph{predictor space} on which evolves the agent during the learning process, a \emph{data space} $\Z$ and a \emph{loss function} being the mathematical formulation of the question. Such a minimalistic structure is convenient to encompass a broad range of real-life learning scenarii. To learn, the agent has access to a \emph{training dataset} $\S_m= (\z_i)_{i=1\cdots m}$. The most classical way to learn from $\Sm$ is the empirical risk minimisation (ERM), minimising the \emph{empirical risk} $\Riskhat_{\Sm}:= \frac{1}{m} \sum_{i=1}^m \ell(h,\z_i)$. In this setting, when $\S_m$ is \iid (following the distribution $\D$), two facets of generalisation are commonly studied in statistical learning for a given agent $h\in\H$.

\begin{itemize}
    \item First, the \emph{population risk} $\Risk_\D(h):= \mathbb{E}_{\z\sim\D}[\ell(h,\z)]$ focus on the average performance of our learning agent \wrt any new situation $\z\in\D$, independent of $\S_m$, possibly faced by the agent. A small population risk ensure then efficient generalisation.   
    \item Second, the \emph{generalisation gap}  $\Delta_{\Sm}(h):= \Risk_D(h) - \Riskhat_{\Sm}(h)$ evaluate the coherence between the empirical risk and the population one. Having a small generalisation gap ensure that the generalisation ability of the agent has the same magnitude than its training performance. 
\end{itemize}
Note that the population risk is a stronger notion of generalisation than the generalisation gap. However, a small generalisation gap (in absolute value) as well as a small empirical risk is enough to ensure a good population risk. Given that modern optimisation algorithm are often enough to ensure a small empirical risk, the generalisation gap has received a particular attention in statistical learning. 

\paragraph{Generalisation bounds.} Generalisation bounds are inequalities often controlling the generalisation gap by various quantities depending either on $\H,\Z$ or $\S_m$. We propose below general patterns usually involved in generalisation bounds for an agent $h_{\Sm}\in\H$ depending on $\S_m$ (for instance the output of the ERM). 

\textbf{Expected generalisation bound.} For any training set $\S_m$:

\begin{align} 
  \label{eq: expected-bound}
  \mathbb{E}_{\S_m}\LB \Delta_{\Sm}(h_{\Sm}) \RB \leq f\LP \textsc{Complexity}, \frac{1}{m}\RP. 
\end{align}

\textbf{High-probability generalisation bounds.} For any training set $\S_m$, with probability $1-\delta$ pver the draw of $\Sm$:
\begin{align}
  \label{eq: hp-bound}
  \Delta_{\Sm}(h_{\Sm}) \leq f\LP \textsc{Complexity}, \frac{1}{m}, \log\frac{1}{\delta}\RP.
\end{align}
The nature of $f$ and the \textsc{Complexity} term depend on the facet of the complexity of the learning problem we aim to focus. Celebrated examples are for instance the dimension of $\H$, if euclidean, the VC dimension of $\H$ \citep{vapnik2000learning}, the Rademacher complexity \citep{bartlett2001rademacher,bartlett2002rademacher},  the stability parameter of a learning algorithm \citep{bousquet2000algo} or the subgaussian diameter of $\Z$ \citep{kontorovich2014conc}. 
Another approach relies on the Bayesian learning paradigm, deriving \emph{posterior} knowledge from data and prior modelling of the environment. 
Then, the \textsc{Complexity} can be borrowed from information theory \citep{cover2001elements}, \eg mutual information \citep{neal2012bayesian}, or from optimal transport, \eg Wasserstein distances \citep{wang2019information,rodriguez2021tighter}. 

Those two approaches have various benefits. A notable strength of expected bounds is that they may reach fast convergence rates (\ie faster than $\frac{1}{\sqrt{m}}$) contrary to high-probability one, even when $\H$ is a singleton thanks to the central limit theorem \citep{grunwald2021mac}. However, expected bounds often involves a theoretical \textsc{Complexity} which cannot be estimated in practice and may be hard to interpret while high probability bounds may be fully empirical and can be considered with small confidence parameter $\delta$ as it is attenuated by a logarithm.

\paragraph{How to choose the complexity term ? An introductory example.} There is no evidence proving that a certain notion of complexity is preferrable to another. The choice of \textsc{Complexity} may however be driven by practical considerations, emerging from the learning problem of interest. To illustrate this point, let us focus on the following example, providing two learning problems which differs only from the predictor space $\H$ and which have very different interactions with the VC dimension.
\begin{example}
  \label{ex: neural_net}
  Consider a supervised learning problem where $\Z = \Rbb^k\times\Ycal$ with $\Ycal=\{0,1\}$, $k$ smaller than $m$ and with loss $\ell(h,(x,y))= \mathds{1}\{h(x) \neq y\}$.  First, assume that $\Hcal$ is the set of linear classifiers; \ie $\H_1:= \left\{ h_{\theta}(x)= sgn(\langle \theta,x\rangle)  \right\}$, where $sgn(a)$ denotes the sign of $a$. In this case, using the VC dimension may lead to non-vacuous generalisation bounds \citep{vapnik2000learning}. 

However, in modern machine learning, deep neural networks are often considered, let us first define a celebrated class of deep neural networks. 

\begin{definition}
    \label{def:mlp}
    A multlayer perceptron with depth $K$ and architecture $\{N_1,\cdots,N_K\}$, denoted as $h_{\wbf}(\x) \defeq Wh^{K}(\cdots h^{1}(\x))+b$, is composed of $K$ layers $h^1(\cdot),\dots,h^K(\cdot)$.
 $W\in\R^{|\Ycal|\times N_K}$ and $b\in\R^{N_K}$ are the weight matrix and the bias of the last layer, and the $i$-th layer $h^{i}$, composed of $N_i$ nodes, is defined by $h^{i}(\xbf)\defeq\sigma_i(W_i \xbf + b_i)$, where $W_i\in\R^{N_i\times N_{i-1}}$ and the bias $b_i\in \R^{N_i}$ are its weight matrix and bias respectively; $\sigma_i : \R^{N_i} \to \R^{N_i} $ is an activation function.
The weights $\wbf=\vect(\{W, W_{K}, \dots, W_1, b, b_{K}, \dots, b_1\})$ represent the vectorisation of all parameters of the network.
\end{definition}
Now, consider the learning problem with the same $\Z,\ell$ as above, but with $\H_2$ being the set of multilayer perceptrons \wrt a fixed depth $K$ and architecture $\{N_1,\cdots,N_K\}$. To be consistent with modern practice, assume also that we are in the \emph{overparametrised setting}, meaning that the space $\H_2$ has a dimension $d$ far greater than $m$.
In this case, VC dimension fails to explain the good generalisation ability (seen in practice) of multilayer perceptrons \citep{bartlett2003vapnik}.
\end{example}

Understanding the generalisation ability of deep neural networks remains nowadays a major challenge and in what follows, we focus on a modern branch of learning theory which provided non-vacuous bounds of the generalisation ability of deep neural networks: PAC-Bayes learning.

\section{PAC-Bayes learning from an information-theoretic perspective}

PAC-Bayes learning is a recent branch of learning theory which emerged in the late 90s via the seminal work of \citep{shawe1997pac,mcallester1998some,mcallester1999pac,mcallester2003pac} and later pursued by \citep{catoni2003pac,catoni2007pac}. Modern surveys recently emerged to describe the various advances in the field \citep{guedj2019primer,hellstrom2023generalization,alquier2024user}.  Similarly to the various subfields of statistical learning described in \Cref{sec: intro-stat-learning}, PAC-Bayes theory is designed top provide generalisation bounds involving a \textsc{Complexity} term apprehending a facet of the complexity of the learning problem. In PAC-Bayes, this term is inspired from the Bayesian learning paradigm of designing a \emph{posterior} knowledge of the learning problem based on the positive impact of data onto a \emph{prior} knowledge of the considered situation. A concrete example of Bayesian learning would be an explorer mapping an ill-known territory. The explorer has to adapt the existing maps at its disposal before exploration to its discoveries, generating a new map imbricating the benefits of both the prior one alongside its findings. From a mathematical perspective, the Bayes approach relies on the Bayes formula, providing an update recipe from a prior distribution $\P\in\Mcal(\Hcal)$ over the predictor space $\H$ to a posterior $\Q\in\Mcal(\Hcal)$ through a likelihood. On the contrary, PAC-Bayes, while inspired from the Bayesian philosophy, does not relies on the Bayes formula but instead on tools from information theory. This general approach benefits from additional flexibility as PAC-Bayes can be linked and applied to Bayesian learning (see \citealp{guedj2019primer}) but also blurs the notion of prior and posterior distributions, now independent of the fundamental Bayes formula. We further develop those points through two celebrated bounds: the McAllester and Catoni ones. 

\subsection*{Two fundamental results}
    The McAllester's bound \citep{mcallester2003pac} enriched with Maurer's trick \citep{maurer2004note} and Catoni's bound (\citealp[Theorem 4.1]{alquier2016properties}, being a relaxation of \citealp[Theorem 1.2.6]{catoni2007pac}) are probably the most known high-probability PAC-Bayes bounds. We recall them in \Cref{prop: mcall-catoni}.

    \begin{proposition}
        \label{prop: mcall-catoni}
        Assume $\Sm$ to be \iid.\\
    \textbf{McAllester's bound, \citep[Theorem 5]{maurer2004note}.}  For any $\delta\in(0,1),\ell\in[0,1]$, any data-free prior $\P\in\Mcal(\Hcal)$, with probability at least $1-\delta$, for any posterior $\Q\in\Mcal(\H)$,
    \begin{align}
    \label{eq: mcallester}
     \Delta_{\Sm}(\Q) \le \sqrt{\frac{\KL(\Q,\P) + \ln{\frac{2\sqrt{m}}{\delta}}}{2m}}.
    \end{align}

    \textbf{Catoni's bound, \citep[Theorem 4.1]{alquier2016properties}.}
    For any $\lambda\in\mathbb{R}/\{0\},\delta\in(0,1),\ell$ being $\sigma^2$-subgaussian and a data-free prior $\P$, with probability at least $1-\delta$ over $\S$, for any $\Q\in \Mcal(\Hcal)$,

  \begin{align}
    \label{eq: catoni}
    \Delta_{\Sm}(\Q) \leq  \frac{\KL(\Q,\P) + \log(1/\delta)}{\lambda} + \frac{\lambda\sigma^2}{2m}.  
  \end{align}

    For both results, $\Delta_{\Sm}(\Q)$ denotes the expected generalisation gap \wrt $\Q$ and $\KL$ denotes the Kullback-Leibler divergence.
    \end{proposition}

Recall that a random variable $X$ is $\sigma^2$-subgaussian if for any $\lambda\in\Rbb$, $\Ebb[\exp(\lambda(X-\Ebb[X]))] \leq \exp\LP \frac{\lambda^2 \sigma^2}{2} \RP$ and that any loss $\loss\in[0,C]$ is $C$-subgaussian. Both McAllester and Catoni bounds fit the general shape of \eqref{eq: hp-bound}. In both cases, $\textsc{Complexity}= \KL(\Q,\P)$ and $f$ varies. The immediate link with the Bayesian philosophy of learning is that the prior has to be data-free. However, \eqref{eq: mcallester} and \eqref{eq: catoni} are both valid simultaneously for any posterior, which is strictly more general than considering the Bayesian posterior. Note that if $\lambda$ is optimised, then Catoni's bound would boil down to an upgraded McAllester bound without the $\log(\sqrt{m})$ term, but such an optimisation is not feasible as $\lambda$ has to be chosen independently of the dataset $\Sm$. 
Note that this gap has been recently filled by \citet[Theorem 33]{dupuis2024generalization}. While the theoretical links between those two bounds are clear, they involve two different toolboxes: McAllester's bound heavily relies on the KL divergence between Bernoullis alongisde calculation tricks exploiting the boundedness of the loss while the original Catoni's bound \citep[Theorem 1.2.6]{catoni2007pac} exploits tools from statistical physics. 
The relaxation \eqref{eq: catoni} proposed here is reachable by a few key arguments, involved in a vast majority of PAC-Bayes proofs. We propose it below for pedagogical purpose. 

\begin{proof}[of \Cref{eq: catoni}]
Note that the first part of the proof holds for a large part of PAC-Bayes literature.
  \textbf{A generic pattern for PAC-Bayes bounds.}
    This part is designed upon two cornerstones, retrievable in many existing results: the change of measure inequality (\citealp{csizar1975divergence,donsker1976asymp} -- see also \citealp{banerjee2006bayesian,guedj2019primer} for a proof) and Markov's inequality.

  \begin{lemma}[Change of measure inequality]
    \label{l: change_meas} 
    For any measurable function $\psi :\mathcal{H}\rightarrow \mathbb{R}$ and any distributions $\Q,\P$ on $\mathcal{H}$:
    
    \[ \mathbb{E}_{h\sim \Q}[\psi (h)] \leq \operatorname{KL}(\Q,\P) + \log\left( \mathbb{E}_{h\sim \P}[\exp(\psi(h))]  \right).  \]
    \end{lemma}
For a given $\lambda>0$, the change of measure inequality is then applied to a certain function $f_m: \Hcal \mathbb{R}$, possibly involving $\Sm$: for all posteriors $Q$,
\begin{align}
\label{eq: change_meas_pacb}
\mathbb{E}_{h\sim \Q}[f_m(h)] \leq \operatorname{KL}(\Q,\P) + \log\left( \mathbb{E}_{h\sim \P}[\exp(f_m(h))]  \right).
\end{align}
To deal with the random variable  $X(\Sm):=\mathbb{E}_{h\sim \P}[\exp(f_m(h))] $, our second building block is Markov's inequality $\left(\mathbb{P}(X>a) \leq \frac{\mathbb{E}[X]}{a}\right)$ which we apply for a fixed $\delta\in (0,1)$ on $X(\Sm)$ with $a= \mathbb{E}_{\Sm}[X(\Sm)]/\delta$.
Taking the complementary event gives that for any $m$, with probability at least $1-\delta$ over the sample $\Sm$, $X(\Sm)\leq \mathbb{E}_{\Sm}[X(\Sm)]/\delta$, thus:


\begin{equation}
  \label{eq: prelim_pb_bound}
  \mathbb{E}_{h\sim \Q}[f_m(h)] \leq \operatorname{KL}(\Q,\P) + \log(1/\delta) + \log\left( \mathbb{E}_{h\sim P}\mathbb{E}_{\Sm}[\exp(f_m(h))]  \right).
\end{equation}

Note that in \eqref{eq: prelim_pb_bound}, we swapped the two expectations in the last term thanks to Fubini's theorem and the fact that $\P$ is data-free.

\textbf{Proving Catoni's bound.}
Now, we take $f_m(h)= \lambda\Delta_{\Sm}$ and consider for any $h\in\Hcal, A(h)= \mathbb{E}_{\Sm}[\exp(f_m(h))]$. 

Note that, given $\Sm$ is iid, 
\begin{align*}
  A(h) &= \prod_{i=1}^m \mathbb{E}_{\Sm}\LB\exp\LP\frac{\lambda}{m}(\Risk_\D(h)-\ell(h,\z_i))\RP\RB, 
  \intertext{and thanks to Heoffding's lemma alongside $\ell$ being $\sigma^2$-subgaussian,}
  A(h) &\leq \prod_{i=1}^m \exp\LP \frac{\lambda^2 \sigma^2}{2m^2} \RP = \exp\LP \frac{\lambda^2 \sigma^2}{2m} \RP.
\end{align*}

Plugging this upper bound in \eqref{eq: prelim_pb_bound} and dividing by $\lambda$ concludes the proof.
\end{proof}

The generic pattern \eqref{eq: prelim_pb_bound}, allows to retrieve many PAC-Bayes bounds, starting with McAllester's one, where $f_m= kl(\Risk_\D(h),\Riskhat_{\Sm}(h)), kl$ being the KL divergence between Bernoullis and completing with the subtle calculations of \citet{maurer2004note}. This pattern is also valid, for instance, for the results of \citet{germain2009pac}, the Bernstein PAC-Bayesian bounds of \citet{tolstikhin2013pac,mhammedi2019pac} and many other results, \eg \citet{thiemann2017strongly,guedj2018pac,holland2019pac,wu2022split}. This then pins two major points for a large part of PAC-Bayes literature: 

\begin{enumerate}
  \item Interpreting PAC-Bayes from a Bayesian perspective is legitimated by the change of measure inequality, yet the KL divergence. More generally, this property allows interpreting PAC-Bayes under a more general information theoretic paradigm, where information from the prior is partially transferred to the posterior (here by absolute continuity to keep the KL finite). This vision encompasses the Bayesian one, while being less restrictive.
  \item The statistical properties of the learning problem are linked to the exponential moment coming from the change of measure inequality, this often implies the strong assumptions of \Cref{prop: mcall-catoni}: data-free prior, bounded or subgaussian losses (sometimes attenuated to subexponentiality \citealp{catoni2004statistical}).
\end{enumerate}

\paragraph{A theory suited for \Cref{ex: neural_net}?}
The two previous points show that \Cref{prop: mcall-catoni} holds for learning problem with light-tailed losses (often bounded), \iid data, encompassing classification tasks for instance. Then, PAC-Bayes learning seems suited to understand, on such problems, the McAllester and Catoni bounds are suited to the learning problem $(\Hcal_2,\Z,\ell)$ of \Cref{ex: neural_net}. 

However, the question of their tightness is unsolved as we do not know the behavior of the KL term in practice. Furthermore the question of which distribution $\Q$ should be taken in \Cref{prop: mcall-catoni} remains open.

% Rework this paragraph, do not talk about dziugaite now, just say that the  theoretical framework fits the learning problem of example 1.1.1. But it still needs to be verified that this complexity measure will be small for neural nets, and thus, non-vacuous bound can be attained in this case.

\section{From theory to learning algorithms}

\subsection*{Algorithms associated to McAllester and Catoni bounds}
A shared particularity of McAllester and Catoni bounds is that they are both fully empirical. Then it is possible to minimise them in practice and thus, deriving new theory-driven learning algorithms which are expected to have at worse, a small generalisation gap and at best, a small population risk. More precisely, learning algorithms associated to \Cref{prop: mcall-catoni} are stated below: 

\begin{align}
  \label{eq: alg-mcall}
  \Q_{M}:= \underset{\Q\in\mathcal{C}}{\operatorname{argmin}}\; \Riskhat_{\Sm}(\Q) + \sqrt{\frac{\KL(\Q,\P)}{2m}}.
  \intertext{For any $\lambda>0$,}
  \label{eq: alg-catoni}
  \Q_{C}:= \underset{\Q\in\mathcal{C}}{\operatorname{argmin}}\; \Riskhat_{\Sm}(\Q) +\frac{\KL(\Q,\P)}{\lambda}.
\end{align}
In both cases, $\mathcal{C}\subseteq \Mcal(\Hcal)$ is the class of distributions on which we optimise. The choice of $\mathcal{C}$ can come from a priori knowledge of the problem or from an optimisation perspective to make the KL divergence tractable.  

Knowing Catoni's bound is a relaxation of McAllester's one, it seems more natural to consider $\Q_{M}$ over $\Q_{C}$. However, the presence of a square root in \eqref{eq: alg-mcall} can be challenging from an optimisation perspective. We illustrate this below.
\begin{example}
  \label{ex: gaussian-kl}
  Consider the case where, for a given $\sigma>0$, $\mathcal{C}= \{\Ncal(\mu,\sigma^2 \mathrm{Id}) \mid \mu\in\Rbb^d\}$. Then the for any $\P= \Ncal(\mu_1,\sigma^2 \mathrm{Id}), \Q= \Ncal(\mu_2,\sigma^2 \mathrm{Id})$, $ \KL(\Q,\P)= \frac{\|\mu_1-\mu_2\|^2}{2\sigma^2}$. 
  Then, optimising \eqref{eq: alg-mcall} in this case implies to lose the strong convexity of the KL divergence while it is retained for \eqref{eq: alg-catoni}.
\end{example}
  
Another practical advantage of \eqref{eq: alg-catoni} over \eqref{eq: alg-mcall} emerges when $\mathcal{C}= \Mcal(\Hcal)$. In this case, Catoni's bound admits a closed form solution, while McAllester's one should be numerically optimised on all the space of distributions, which is not feasible. This closed form, extracted from \citet[Section 5.1]{catoni2003pac}, is recalled below.

\begin{align}
  \label{eq: catoni-gibbs}
  \text{When}\; \mathcal{C}= \Mcal(\Hcal),\; d\Q_{C}(h) = \frac{\exp(-\lambda \Riskhat_{\Sm}(h))}{\Ebb_{h\sim\P}[\exp(-\lambda \Riskhat_{\Sm}(h))]} d\P(h)
\end{align}

Then, $\Q_{C} = \P_{-\lambda \Riskhat_{\Sm}}$ is the \emph{Gibbs posterior} associated to $\P,\lambda\Riskhat_{\Sm}$. By introducing Gibbs posterior in statistical learning, \citet{catoni2007pac} draws a theoretical link between statistical physics and learning theory. Unfortunately, Gibbs posteriors often requires Monte Carlo methods to be implemented, which can be in practice time-consuming. Below we then focus on PAC-Bayes algorithms working on a subset $\mathcal{C}$ of $\Mcal(\Hcal)$. 

\subsection*{Instantiation and efficiency of PAC-Bayesian algorithms}

\paragraph{A general pattern for PAC-Bayesian algorithms}

The introductory examples \eqref{eq: alg-mcall},\eqref{eq: alg-catoni} unveil a general design for any KL-based PAC-Bayesian algorithm. satisfy a tradeoff between \textit{(i)} the empirical risk, showing that the learner has to fit the training dataset, and \textit{(ii)} a \emph{regulariser} being a function of $\KL(\Q,\P)$. This regulariser ensures that, during training, the learner will not overfit on training data. This training ensures a good generalisation ability as long as the associated generalisation bound is small. 

We then understand better the ins and outs of PAC-Bayes algorithms from a conceptual perspective, two unanswered questions remains: 

\begin{enumerate}
  \item How are those algorithms instantiated in practice?
  \item Are these algorithms efficient and do they come with non-vacuous theoretical guarantees? 
\end{enumerate}

\paragraph{Instantiating a PAC-Bayes algorithm}
  In practice, using a single prior $\P$ usually does not work but it remains theoretically possible to consider a finite set of priors. Indeed, if one wants to consider $k$ priors, then it is possible to consider $k$ PAC-Bayes bounds holding for each of those priors with probability at least $1-\frac{\delta}{k}$ and then consider a union bound, such a set of priors is called a grid. This method has been widely used in many PAC-Bayes work with clever grids, deteriorating initial bounds at the cost of supplementary $\log(n)$ or $\log\log(n)$ (divided by $m$), see \eg \citet{alquier2024user}. 
  Similarly, for Catoni-typed algorithms where a parameter $\lambda$ is involved, it is also possible to consider grids for it. In both cases, considering grids allows to optimise on both the prior, the posterior (and possibly $\lambda$ when involved) and then taking the closest value of those optimised parameters on the grid to still obtain theoretical guarantees. Another technique to ensure a good prior is to sacrifice a part of the training set to pre-train $\P$. 
  Doing so, the prior is then data-dependent and yields tighter bounds alongside increased performance \citep{perez2021tighter}.

\paragraph{Efficiency of PAC-Bayes algorithms on supervised learning problems.}
The work of \citet{dziugaite2017computing} showed that optimising \eqref{eq: alg-mcall} when $\mathcal{C}$ is a class of Gaussian measures for the weights of a deep neural network yields non-vacuous generalisation bound, meaning that the benefits of PAC-Bayesian training on deep nets can be ensured from a theoretical perspective to understand the generalisation ability of neural networks. In their work, authors used the instantiation toolbox described above, alongside a preliminary use of Stochastic Gradient Descent (SGD) to update $\Q$ before the PAC-Bayes training algorithm. This promising work paved the way to various extensions, providing non-vacuous guarantees for a wide range PAC-Bayes algorithms \citep{rivasplata2019pac,letarte2019dichotomize,perezortiz2021learning,perez2021progress,perez2021tighter,dziugaite2021role,biggs2022non,biggs2023tighter}, showing that the PAC-Bayes toolbox provides elements of answer to understand the generalisation ability of neural networks. Beyond generalisation guarantees, PAC-Bayes bounds may also be useful to propose original training methods, even if the associated guarantees are vacuous \citep{biggs2021differentiable,biggs2022margin}. Another important empirical use is to exploit PAC-Bayes bounds as correlation measures, to see whether a decrease of the bound is related to an increased generalisation ability of the learner. For instance \citet{neyshabur2017explor} used McAllester's bound \eqref{eq: mcallester} as a 'flatness' measure and showed that it correlates well with a good generalisation ability for a few learning problems. This conclusion has been extended to a wider range of problems in \citet{jiang2020fantastic,dziugaite2020search}. 
\\
\paragraph{PAC-Bayes algorithms beyond supervised learning.} While supervised learning is a widely used setting to perform experiments in PAC-Bayes learning (often involving celebrated datasets such as MNIST or CIFAR-10), the McAllester bound holds for any learning problem with bounded loss, going beyond this setting. This theoretical flexibility has been exploited to derive PAC-Bayesian algorithm for various learning settings reinforcement learning \citep{fard2010pac}, multi-armed bandits \citep{seldin2011pac,seldin2012pac,sakhi2023pac}, meta-learning \citep{amit2018meta,farid2021generalization,rothfuss2021pacoh,rothfuss2022pac,ding2021bridging} to name but a few. 
\\
\paragraph{Is \Cref{ex: neural_net} tackled now?}
\citep{dziugaite2017computing} and following works have provided a positive answer by obtaining non-vacuous guarantees (sometimes tight) for $(\Hcal_2,\Zcal,\ell)$ of \Cref{ex: neural_net} for various $\Z$ (being, \eg, set of images for MNIST CIFAR-10 etc...). To obtain such guarantees, a PAC-Bayesian training needs to be performed to minimise its associated theoretical bound. That being said, several questions then legitimately emerge.

\begin{itemize}
  \item Modern machine learning often implies learning problems where assumptions such as bounded (or subgaussian) losses or \iid data do not hold. Is PAC-Bayes theory extendable beyond those assumptions? 
  \item As shown in \citet{dziugaite2017computing}, the PAC-Bayesian training is often combined to another procedure (\eg ERM) to yield non-vacuous bounds. However, PAC-Bayes bounds do not bring the theoretical understanding of such additional methods, often outputting deterministic predictors (\ie Dirac distributions). This kind of predictor is not allowed in \eqref{eq: mcallester}, \eqref{eq: catoni}. Is it possible to obtain PAC-Bayes bounds valid for such methods? 
\end{itemize}




\section{Modern extensions of PAC-Bayes}

%After this, goes to the modern extension part with many other pac bounds beyond classical paradigm, reuse the paragraph of supermartingale to help the bibliography part. 

\section{An optimisation perspective on PAC-Bayes learning}

\newpage
Detail broadly what generalisation is, to what kind of structures it is applied (neural nets or linear classfier eg). Details on the other hand what optimisation is doing (ERM eg) and explain that interestingly in various methods, reaching minimisers of empirical objectives is enough to ensure a good generalisation ability. From this, discuss about the current limitations of generalisation: either not going so often beyond light-tailed assumptions or noticing that the interplays between generalisation (statistical arguments) and optimisation (geometric ones) remains uncharted for a vast range of cases.

Vision: after generic paragraphs on generalisation an optimisation, do a broader paragraph on PAC-Bayes and details the problem of exisiting PAC-Bayes approach: 

Says that PAC-Bayes spontaneously offer a clear link from generalisation to optimisation by providing new learning algorithms: this implicitly suggests assumptions on the loss (eg convex) or on the regulariser (KL between gaussians to get a strongly convex function) to make sure the minimisation goes well and thus build a bridge with optimisation.

TODO look if there are links from optimiastion to PAC-Bayes (must have been some with dziugaite, neu with SGD).

Here, we are studying the interplays on both directions. First, we take the opposite perspective and, starting from optimisation benefits/perspectives, we want to understand generalisation, to do so we have several routes within PAC-Bayes. Second, we investigate deeper on the influence of generalisation bounds to derive novel learning algorithms

Thinking the role of the prior in PAC-Bayes: in a similar manner than initialisation/ goal to attain in optimisation: if we target a data-free posterior (eg GIbbs catoni) then ok: we target the learning objective. Otherwise, it is common to compare to a random initialisation point of a learning procedure: meaningless. Answer: Online PAC-Bayes which allows, among other, to make the prior evolve through time. (note that there is also either the data-dependent prior: its bad and the differential privacy approach, which is nice)

Switching from statistical to geometric assumptions on the loss. Most of the bounds holds for data-free light-tailed losses: do not necessarily fit the reality of losses involved in optimisation, often unbounded, and either convex, gradient lipschitz or smooth. Answer: PAC-Bayes for heavy-tailed martingales and flat minima.

Can the convergence properties of optimisation procedure play a role in generalisation? Direct answer: Wasserstein PAC-Bayes, Indirect one: flat minima. 

Can we derive generalisation-based learning algorithms beyond Gaussian or Gibbs distributions? Answer: Yes as a byproduct: both in Online PAC-Bayes, Flat Minima and Paper with Paul  

Precise the structure of the document: see it as a natural flow where one question implies another one: 

Light-tailed losses? --> supermartingales! 

But then : what should we do of the prior? $\rightarrow$  if you see it as an initialisation: Online PAC-Bayes with novel online algorithms or Flat Minima to attenuate the impact of the prior through fast rates.

What if it should be the optimisation goal? --> Wasserstein PAC-Bayes



 


CHALLENGE HERE: being very rigorous on the lit review.
